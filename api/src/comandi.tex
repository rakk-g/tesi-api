% AMBIENTI PER TEOREMI, DEFINIZIONI ET CETERA (DA AMSTHM)
\theoremstyle{plain}
\newtheorem{teorema}{Teorema}
\newtheorem{problema}{Problema} % usato generalmente nella phorma input - output richiesto
\newtheorem{lemma}{Lemma}
\newtheorem{corollario}{Corollario}
\newmdtheoremenv{postulato}{Postulato}

\theoremstyle{definition}
\newtheorem{definizione}{Definizione}
\newtheorem{esempio}{Esempio}

\theoremstyle{remark}
\newtheorem{osservazione}{Osservazione}


% ]----------------------------------------------------------------[ %


% COMANDI e RINNOVI per la tesis.

\newcommand{\omissis}{[\textellipsis\unkern]}                           % produce "[...]"

\DeclarePairedDelimiter\abs{\lvert}{\rvert}                             % modulo
\DeclarePairedDelimiter\norm{\lVert}{\rVert}                            % norma
% Swap the definition of \abs* and \norm*, so that \abs
% and \norm resizes the size of the brackets, and the 
% starred version does not.
\makeatletter
    \let\oldabs\abs
    \def\abs{\@ifstar{\oldabs}{\oldabs*}}

    \let\oldnorm\norm
    \def\norm{\@ifstar{\oldnorm}{\oldnorm*}}
\makeatother

\newcommand{\opnorm}[1]{\norm{#1}_{\textup{op}}}                        % norma operatore

\newcommand{\N}{\mathbb{N}}                                             % numeri naturali
\newcommand{\Z}{\mathbb{Z}}                                             % interi
\newcommand{\R}{\mathbb{R}}                                             % campo reali
\newcommand{\Rplus}{\R_{\geq 0}}                                        % reali non negativi
\newcommand{\C}{\mathbb{C}}                                             % campo complessi
\newcommand{\K}{\mathbb{K}}                                             % campo generico
\newcommand{\F}{\mathbb{F}}                                             % campo finito; metti a pedice p^q
\renewcommand{\P}[1]{\mathbb{P} \left( #1 \right) }                     % spazio proiettivo (su qualcosa!)

\newcommand{\prob}{\mathds{P}}                                          % probabilit�
\newcommand{\E}{\mathds{E}}                                             % valore atteso
\newcommand{\Borel}{\mathcal{B}}                                        % sigma-algebra di Borel

\newcommand{\U}{\mathds{U}}                                             % standard time evolution (unitaria)

% ]------------------------- SPAZI ----------------------------[ %

\newcommand{\hs}[1]{\mathcal{#1}}                                   % spazi di Hilbert

\newcommand{\duale}[1]{{#1}'}                                           % duale algebrico
\newcommand{\tduale}[1]{{#1}^{\ast}}                                % duale topologico, usa questo per il duale delle mappe

\newcommand{\base}[1]{\mathcal{#1}}                                 % base (ortonormale?)
\newcommand{\puri}[1]{\mathcal{P}\left(#1\right)}                   % stati puri, l'argomento � lo spazio di Hilbert
\newcommand{\stati}[1]{\mathcal{S}\left(#1\right)}                  % stati, c.s.

% CATEGORIE
\newcommand{\vectcat}[1]{\textbf{Vect}_{#1}}                        % categoria degli spazi vettoriali
\newcommand{\hilbcat}{\textbf{Hilb}}                                % sottocategoria degli spazi di hilbert

% (SPAZI DI) OPERATORI (FUNTORI NELLA CATEGORIA)
\newcommand{\hiHom}[2]{\hom\left(\hs{#1},\hs{#2}\right)}            % (handler) per spazi di Hilbert
\DeclareMathOperator{\End}{End}
\newcommand{\hiEnd}[1]{\End\left(\hs{#1}\right)}                    % (handler) per spazi di Hilbert
\DeclareMathOperator{\Bound}{Bou}                                   % omomorfismi limitati
\newcommand{\hiBound}[2]{\Bound\left(\hs{#1},\hs{#2}\right)}        % (handler) per spazi di Hilbert
\newcommand{\hiEndBound}[1]{\Bound\left(\hs{#1}\right)}             % (handler) endomorfismi limitati per sp. Hilbert

\newcommand{\Ident}[1]{\mathds{1}_{#1}}                             % operatore identit� generico (TODO come fo la matrice?)
\newcommand{\hiIdent}[1]{\Ident{\hs{#1}}}                           % (handler) operatore identit� per spazi di Hilbert

\newcommand{\adj}[1]{{#1}^{\dag}}                                   % aggiunzione (era: \ast)

\DeclareMathOperator{\Tr}{Tr}                                       % traccia
\DeclareMathOperator{\Sp}{Sp}                                       % spettro
\DeclareMathOperator{\Imm}{Imm}                                     % immagine di una applicazione
\DeclareMathOperator{\Span}{Span}                                   % chiusura lineare

\DeclareMathOperator{\alghull}{alg}                                 % chiusura algebrica
\newcommand{\alg}[1]{\mathscr{#1}}                                  % algebra associativa
\newcommand{\cstar}{C^{\ast}}                                       % algebre C*
\newcommand{\cstaralgstati}[1]{\tduale{\alg{#1}}_{+,1}}             % (handler) stati di un'algebra C*
\newcommand{\matr}{\mathscr{M}}                                     % matrici, si sottintende quadrate e a entrate in \C

%-----------------------------------------------------------%

\newcommand{\conj}[1]{\overline{#1}}

\newcommand{\eg}{\emph{e.g.~}}                                      % per esempio
\newcommand{\ie}{\emph{i.e.~}}                                      % id est
\newcommand{\st}{\,\mid\;}                                             % tali che, da usare in mathmode all'interno di \Set

% ]-------------- (CLASSI DI) COMPLESSIT� -----------------[ %

\newcommand{\cc}[1]{\mathsf{#1}}                                    % classi di complessit�
\newcommand{\lb}[1]{\mathtt{#1}}                                    % blocco logico O ALGORITMO
\newcommand{\orac}[1]{\mathcal{O}_{#1}}                             % oracolo (black box) come trasformazione: come blocco usa \lb{O}_ <- NO a me questa cosa non mi piace
\newcommand{\vet}[1]{\mathbf{#1}}                                   % vettorizza







\renewcommand{\epsilon}{\varepsilon}
\renewcommand{\theta}{\vartheta}
\renewcommand{\phi}{\varphi}
% \renewcommand{\rho}{\varrho}