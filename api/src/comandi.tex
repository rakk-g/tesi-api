% contributor: from PQC2015
% AMBIENTI PER TEOREMI, DEFINIZIONI ET CETERA (DA AMSTHM)
\theoremstyle{plain}
\newtheorem{teorema}{Teorema}
\newtheorem{problema}{Problema} % usato generalmente nella phorma input - output richiesto
\newtheorem{lemma}{Lemma}
\newtheorem{corollario}{Corollario}
\newmdtheoremenv{postulato}{Postulato}

\theoremstyle{definition}
\newtheorem{definizione}{Definizione}
\newtheorem{esempio}{Esempio}

\theoremstyle{remark}
\newtheorem{osservazione}{Osservazione}


% ]----------------------------------------------------------------[ %


% COMANDI e RINNOVI per la tesis.

\newcommand{\pcite}[1]{(\cite{#1})}

\newcommand{\omissis}{[\textellipsis\unkern]}                           % produce "[...]"

\DeclarePairedDelimiter\abs{\lvert}{\rvert}                             % modulo
\DeclarePairedDelimiter\norm{\lVert}{\rVert}                            % norma
% Swap the definition of \abs* and \norm*, so that \abs
% and \norm resizes the size of the brackets, and the
% starred version does not.
\makeatletter
    \let\oldabs\abs
    \def\abs{\@ifstar{\oldabs}{\oldabs*}}

    \let\oldnorm\norm
    \def\norm{\@ifstar{\oldnorm}{\oldnorm*}}
\makeatother


\newcommand{\N}{\mathbb{N}}                                             % numeri naturali
\newcommand{\Z}{\mathbb{Z}}                                             % interi
\newcommand{\R}{\mathbb{R}}                                             % campo reali
\newcommand{\Rplus}{\R_{\geq 0}}                                        % reali non negativi
\newcommand{\C}{\mathbb{C}}                                             % campo complessi
\newcommand{\K}{\mathbb{K}}                                             % campo generico
\newcommand{\F}{\mathbb{F}}                                             % campo finito; metti a pedice p^q
\renewcommand{\P}[1]{\mathbb{P} \left( #1 \right) }                     % spazio proiettivo (su qualcosa!)

\newcommand{\prob}{\mathds{P}}                                          % probabilità
\newcommand{\E}{\mathds{E}}                                             % valore atteso

\DeclareMathOperator{\Tr}{Tr}                                       % traccia
\DeclareMathOperator{\Sp}{Sp}                                       % spettro
\DeclareMathOperator{\Imm}{Imm}                                     % immagine di una applicazione
\DeclareMathOperator{\Span}{Span}                                   % chiusura lineare


%-----------------------------------------------------------%

\newcommand{\conj}[1]{\overline{#1}}

\newcommand{\eg}{\emph{e.g.~}}                                      % per esempio
\newcommand{\ie}{\emph{i.e.~}}                                      % id est
\newcommand{\st}{\,\mid\;}                                          % tali che, da usare in mathmode all'interno di \Set




% ]-------------- VARIABILI -----------------[ %

\renewcommand{\epsilon}{\varepsilon}
\renewcommand{\theta}{\vartheta}
\renewcommand{\phi}{\varphi}
% \renewcommand{\rho}{\varrho}
