% contributor: from PQC2015
% AMBIENTI PER TEOREMI, DEFINIZIONI ET CETERA (DA AMSTHM)
\theoremstyle{plain}
\newmdtheoremenv{teorema}{Teorema}
\newmdtheoremenv{lemma}{Lemma}
\newmdtheoremenv{corollario}{Corollario}
\newmdtheoremenv{proposizione}{Proposizione}

\theoremstyle{definition}
\newmdtheoremenv{definizione}{Definizione}
\newmdtheoremenv{esempio}{Esempio}

% \theoremstyle{remark}
% \newtheorem{osservazione}{Osservazione}


% ]----------------------------------------------------------------[ %


% COMANDI e RINNOVI per la tesis.

% \newcommand{\pcite}[1]{(\cite{#1})} % UNNECESSARY: \parencite

\newcommand{\omissis}{[\textellipsis\unkern] }                          % produce "[...] "

\DeclarePairedDelimiter\abs{\lvert}{\rvert}                             % modulo
\DeclarePairedDelimiter\norm{\lVert}{\rVert}                            % norma
% Swap the definition of \abs* and \norm*, so that \abs
% and \norm resizes the size of the brackets, and the
% starred version does not.
\makeatletter
    \let\oldabs\abs
    \def\abs{\@ifstar{\oldabs}{\oldabs*}}

    \let\oldnorm\norm
    \def\norm{\@ifstar{\oldnorm}{\oldnorm*}}
\makeatother


\newcommand{\N}{\mathbb{N}}                                             % numeri naturali
\newcommand{\Z}{\mathbb{Z}}                                             % interi
\newcommand{\R}{\mathbb{R}}                                             % campo reali
\newcommand{\Rplus}{\R_{\geq 0}}                                        % reali non negativi

\newcommand{\C}{\mathbb{C}}                                             % campo complessi
\renewcommand{\Re}{\mathfrak{Re}}                                       % parte reale



\DeclareMathOperator{\tr}{tr}                                       % traccia
\DeclareMathOperator{\Sp}{Sp}                                       % spettro
\DeclareMathOperator{\Imm}{Imm}                                     % immagine di una applicazione
\DeclareMathOperator{\Span}{Span}                                   % chiusura lineare
\DeclareMathOperator{\erf}{erf}                                     % funzione degli errori

\newcommand{\de}{\mathop{}\!\mathrm{d}}                             % differenziale (per gli integrali)

%-----------------------------------------------------------%


\newcommand{\ie}{\emph{i.e.~}}                                      % id est
\newcommand{\st}{\,\mid\;}                                          % tali che, da usare in mathmode all'interno di \Set




% ]-------------- VARIABILI -----------------[ %

\renewcommand{\epsilon}{\varepsilon}
\renewcommand{\theta}{\vartheta}
\renewcommand{\phi}{\varphi}
% \renewcommand{\rho}{\varrho}


% ]-------------- AMBIENTI -----------------[ %

\newenvironment{sistema}%
{\left\lbrace\begin{array}{@{}l@{}}}%
{\end{array}\right.}
