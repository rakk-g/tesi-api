\documentclass[]{beamer}
\usepackage[utf8]{inputenc}
\usepackage[T1]{fontenc}
\usepackage[italian]{babel}
% \usepackage{dsfont}
\usepackage{csquotes}
\usepackage{esdiff} % comandi per le derivate: \diff, \diffp, c'è anche la valutazione in un punto!


\usepackage[bibstyle=alphabetic,citestyle=authortitle,backend=biber,backref]{biblatex}
\bibliography{api}

\usetheme{JuanLesPins}
\useinnertheme{rounded}
\useoutertheme{sidebar}
\usecolortheme{crane}

\title{Dinamica delle popolazioni di \emph{Apis mellifera}}
\author[]{Martino De Leo}
\date{25 Marzo 2025}
\logo{\includegraphics[width=15mm]{img/unifiLogo}}

\begin{document}
% Abbiamo block (neutro), alertblock (negativo) e exampleblock (positivo).
% Magari è una buona idea chiamare il primo frame come la sezione, quando la si cambia.

\section{Introduzione}


\begin{frame}
    \maketitle
\end{frame}


\begin{frame}
    \frametitle{Api e civiltà umana}
    \framesubtitle{Una relazione interspecifica millenaria}

    \begin{center}
        \includegraphics[keepaspectratio,width=0.26\textwidth]{img/AranaCaves}
    \end{center}

    \begin{displayquote}[\cite{honeyreligion}]
        ``Humans have eaten, and fixed their wounds with honey and traded with it since history has been recorded.''
    \end{displayquote}
\end{frame}


\begin{frame}
    \frametitle{Api e civiltà umana}
    \framesubtitle{Importanza delle api oggi}

    \begin{center}
        \includegraphics[keepaspectratio,width=0.5\textwidth]{img/forager_bee}
    \end{center}

    Mercato diretto (apicoltura): $8 \times 10^9 \$$ annui.

    \pause
    Mercato indiretto (impollinazione): $570 \times 10^9 \$$ annui.
\end{frame}


\section{Modelli}
\begin{frame}
    \frametitle{Strumenti per i modelli biologici}
    \framesubtitle{Due secoli di progressi}

    Primi modelli (XIX sec.): risposta numerica.
    \begin{itemize}
        \item (Malthus) $\diff{P}{t} = rP$ \pause \quad $\rightarrow P(t) = P_0 \; e^{rt}$.
        \item (Verlhust) $\diff{P}{t} = rP \left( 1 - \frac{P}{K} \right)$
            \pause \quad $\rightarrow P(t) = \frac{K}{1+qe^{-rt}}$.
    \end{itemize}

    \pause
    \begin{center}
        \includegraphics[keepaspectratio,width=0.7\textwidth]{img/logisticF}
    \end{center}
\end{frame}


\begin{frame}
    \frametitle{Strumenti per i modelli biologici}
    \framesubtitle{Modelli matematici moderni}

    Limiti della risposta numerica: max. 2 o 3 specie, interazioni \emph{semplici}.

    \pause
    Risposta \emph{funzionale}: relazioni complesse infra-- ed inter--specifiche.

    \pause
    \begin{exampleblock}{Esempio}
        Funzioni di Holling: $$f(x) = \frac{a x^k}{1 + a h x^k}$$
    \end{exampleblock}
\end{frame}

\begin{frame}
    \frametitle{Strumenti per i modelli biologici}
    \framesubtitle{Esempio: funzioni di Holling}

    \begin{center}
        \includegraphics[keepaspectratio,width=0.8\textwidth]{img/hollo}
    \end{center}
\end{frame}

\begin{frame}
    \frametitle{Strumenti per i modelli biologici}
    \framesubtitle{Esempio: funzioni di Holling (caso delle api)}

    Schiusa delle larve in una colonia: Holling tipo II o III.
    \begin{center}
        \includegraphics[keepaspectratio,width=0.88\textwidth]{img/hollingTypeII-eclosion}
    \end{center}
\end{frame}


\subsection{Epidemiologia}


\begin{frame}
    \frametitle{Modelli a compartimenti}

    Popolazione nel compartimento $x_i$ al tempo $t$: $x_i(t)$.

    \pause
    Le $N$ equazioni differenziali
    $$ \diff{x_i}{t} = G_i (x_1, \dots x_n, t) \qquad i=1, \dots, N$$
    modellano le relazioni tra i compartimenti (omogenei o no).
\end{frame}


\begin{frame}
    \frametitle{Modelli a compartimenti}
    \framesubtitle{Esempio: SIR e derivati}

    \begin{center}
        \includegraphics[keepaspectratio]{img/SIRschema}
    \end{center}

    \pause
    \begin{align}
    \diff{S}{t} &= - \beta I S \\
    \pause
    \diff{I}{t} &= \beta I S - \gamma I \\
    \pause
    \diff{R}{t} &= \gamma I
\end{align}

\end{frame}











\end{document}
