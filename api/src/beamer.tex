\documentclass[]{beamer} % you can't use 'compress' option until you load 'sidebar' as outer theme!
\usepackage[utf8]{inputenc}
\usepackage[T1]{fontenc}
\usepackage[italian]{babel}
% \usepackage{dsfont}
\usepackage{csquotes}
\usepackage{esdiff} % comandi per le derivate: \diff, \diffp, c'è anche la valutazione in un punto!

\usepackage[bibstyle=alphabetic,citestyle=authortitle,backend=biber,backref]{biblatex}
\bibliography{api}

% rubate da comandi per non prenderlo tutto
\newcommand{\st}{\,\mid\;}

% \usetheme{Madrid} % Madrid and similar add the footline
\useinnertheme{rounded}
\useoutertheme{default} % I don't want the footline bar
\usecolortheme{crane}
% \usefonttheme[stillsansseriflarge]{serif}
\usefonttheme[]{serif}

\title[Modelli per \emph{A.~mellifera}]{Dinamica delle popolazioni di \emph{Apis mellifera}}
\author[]{Martino De Leo}
\date{16 Aprile 2025}
% \logo{\includegraphics[width=15mm]{img/unifiLogo}} % così lo mette in tutte le diapo
\titlegraphic{\includegraphics[width=15mm]{img/unifiLogo}} % così soltanto nella diapo del titolo

\begin{document}
% Abbiamo block (neutro), alertblock (negativo) e exampleblock (positivo).
\begin{frame}
    \maketitle
\end{frame}


\begin{frame}
    \frametitle{Perché proprio le api?}
    \framesubtitle{Importanza ecologica, socio--culturale ed economica}

    \begin{columns}
        \begin{column}{0.45\textwidth}
            \includegraphics[keepaspectratio,width=0.6\textwidth]{img/AranaCaves}
        \end{column}
        \pause
        \begin{column}{0.45\textwidth}
            \includegraphics[keepaspectratio,width=0.9\textwidth]{img/forager_bee}

            Mercato diretto: $8 \times 10^9 \$$ annui.

            \pause
            Indiretto: $570 \times 10^9 \$$ annui.
        \end{column}
    \end{columns}
\end{frame}


\begin{frame}
    \frametitle{Modelli a compartimenti}
    \framesubtitle{Esempio: SIR e derivati}

    \begin{center}
        \includegraphics[keepaspectratio,width=0.7\textwidth]{img/SIRschema}
    \end{center}

    \pause
    \begin{align*}
    \diff{S}{t} &= - \beta I S \\
    \diff{I}{t} &= \beta I S - \gamma I \\
    \diff{R}{t} &= \gamma I
    \end{align*}
\end{frame}

\begin{frame}
    \frametitle{Modelli a compartimenti}
    \framesubtitle{Esempio: SIR e derivati}

    SIRVD:
    \pause
    \begin{center}
        \includegraphics[keepaspectratio,width=0.7\textwidth]{img/sirvd}
    \end{center}
\end{frame}


\begin{frame}
    \frametitle{Semplificazioni per le api}
    \framesubtitle{Struttura eusociale}

    $\rightarrow$ considera ogni famiglia come \emph{un unico} super--organismo.

    \pause
    \begin{alertblock}{Nei modelli matematici sulle api}
        \centering
        Il modello di una popolazione \emph{è} il modello di una famiglia.

        \includegraphics[keepaspectratio,width=0.34\textwidth]{img/arnia}
    \end{alertblock}


\end{frame}

\begin{frame}
    \frametitle{Semplificazioni per le api}
    \framesubtitle{Ordini di grandezza tra le caste}

    \begin{center}
        \includegraphics[keepaspectratio,width=\textwidth]{img/qwd1}
    \end{center}
\end{frame}

\begin{frame}
    \frametitle{Semplificazioni per le api}
    \framesubtitle{Ordini di grandezza tra le caste}

    \begin{center}
        \includegraphics[keepaspectratio,width=\textwidth]{img/qwd2}
    \end{center}
\end{frame}

\begin{frame}
    \frametitle{Semplificazioni per le api}
    \framesubtitle{Ordini di grandezza tra le caste}

    \begin{center}
        \includegraphics[keepaspectratio,width=\textwidth]{img/qwd3}
    \end{center}
\end{frame}

\begin{frame}
    \frametitle{Semplificazioni per le api}
    \framesubtitle{Ordini di grandezza tra le caste}

    \begin{center}
        \includegraphics[keepaspectratio,width=\textwidth]{img/qwd4}
    \end{center}
\end{frame}

\begin{frame}
    \frametitle{Semplificazioni per le api}
    \framesubtitle{Sotto--caste di operaie}

    \begin{center}
        \includegraphics[keepaspectratio,width=0.75\textwidth]{img/fanning}
    \end{center}

    \begin{itemize}
        \item operaie \textbf{di nido (giovani)}: \pause nutrici, \pause ceraiole, guardiane, spazzine, \dots
        \item \pause operaie \textbf{di campo (anziane)}: \pause esploratrici, bottinatrici.
    \end{itemize}


\end{frame}

\begin{frame}
    \frametitle{Semplificazioni per le api}
    \framesubtitle{Conseguenze pratiche per tutti (*) i modelli}


    \begin{itemize}
        \item ignora del tutto i fuchi.
        \item \pause della regina considera soltanto:
        \begin{itemize}
            \item \pause il suo tasso di ovideposizione,
            \item \pause le interazioni funzionali della covata coi compartimenti di operaie.
        \end{itemize}
        \item \pause ignora la sciamatura. (*)
        \item \pause effetto \emph{Allee}.
        \item \pause la risorsa cibo si può rappresentare implicitamente nelle relazioni funzionali tra
            bottinatrici e operaie di casa. (*)
    \end{itemize}
\end{frame}




\begin{frame}
    \frametitle{Modellazione matematica e dati sperimentali}
    \framesubtitle{Ricerca accademica e comparto agricolo}

    Ecocidio degli impollinatori (2005--)

    \begin{itemize}
        \item Homing failure da pesticidi
        \item \pause Dinamiche complesse tra le caste di operaie e l'ambiente
        \item \pause Stima dei parametri da misure dirette
    \end{itemize}

    \pause
    Esempio: flightspan $\to$ mortalità \pause
    $$ \mathbb{E} = \sum_{k=1}^{\infty} k m {(1-m)}^k = \pause
    m \underbrace{ \sum_{k=1}^{\infty} k {(1-m)}^k }_{m^{-2}}
    \pause = \frac{1}{m}$$
\end{frame}




\begin{frame}
    \frametitle{Modello a due compartimenti}
    \framesubtitle{Operaie di casa e operaie di campo}

    \cite{khoury2011}

    \pause
    \begin{center}
        \includegraphics[keepaspectratio,width=0.8\textwidth]{img/kh11schema}
    \end{center}
\end{frame}

\begin{frame}
    \frametitle{Modello a due compartimenti}
    \framesubtitle{Equazioni (I)}

    Api di casa:
    $$\diff{H}{t} = \overbrace{ E(H,F) }^{\text{schiusa}} \pause - H \cdot \overbrace{ R(H,F) }^{\text{reclutamento}} $$

    \pause
    Api di campo:
    $$\diff{F}{t} = H \cdot R(H,F)  \pause - m \cdot F $$

    \pause
    Oss.: manca la mortalità delle api di casa.
\end{frame}


\begin{frame}
    \frametitle{Modello a due compartimenti}
    \framesubtitle{Tasso di schiusa}

    Modulazione di $L$: necessità trofiche e sociali della covata (risposta funzionale).

    $$E = L \pause \frac{H+F}{w+H+F}$$

    (Holling II)

\end{frame}

\begin{frame}
    \frametitle{Modello a due compartimenti}
    \framesubtitle{Tasso di schiusa}

    \begin{center}
        \includegraphics[keepaspectratio,width=0.98\textwidth]{img/hollingTypeII-eclosion}
    \end{center}
\end{frame}


\begin{frame}
    \frametitle{Modello a due compartimenti}
    \framesubtitle{Reclutamento sociale}

    In questo modello il termine di trasferimento è
    $$H R = \pause H \left( \alpha - \sigma \frac{F}{H+F} \right)$$
\end{frame}

\begin{frame}
    \frametitle{Modello a due compartimenti}
    \framesubtitle{Equazioni (II)}

    Api di casa:
    $$\diff{H}{t} = \overbrace{L \frac{H+F}{w+H+F}}^{\text{schiusa}} - H
    \overbrace{\left( \alpha - \sigma \frac{F}{H+F} \right)}^{R} $$

    Api di campo:
    $$\diff{F}{t} = H \left( \alpha - \sigma \frac{F}{H+F} \right) - mF $$
\end{frame}

\begin{frame}
    \frametitle{Modello a due compartimenti}
    \framesubtitle{Esempio (simulazione)}

    \begin{center}
        \includegraphics[keepaspectratio,width=\textwidth]{img/alternativeC}
    \end{center}
\end{frame}


\begin{frame}
    \frametitle{Modello a due compartimenti}
    \framesubtitle{Studio della stabilità}

    L'equilibrio banale esiste sempre. Se $\alpha -L/w>0$ e
    $$m> \pause \frac{\alpha L}{w \alpha -L}$$
    è anche asintoticamente stabile.
\end{frame}

\begin{frame}
    \frametitle{Modello a due compartimenti}
    \framesubtitle{Studio della stabilità}

    Se invece la mortalità è contenuta:
    $$m < \frac{L}{2w} \cdot \frac{ \alpha + \sigma + \sqrt{ {(\alpha - \sigma)}^2 + \frac{4 \sigma L}{w} } }{\alpha - L/w}$$
    allora esiste anche un equilibrio (strett.) positivo, univocamente determinato dai parametri.

    \pause
    Quando esiste, è sempre asintoticamente stabile.
\end{frame}

\begin{frame}
    \frametitle{Modello a due compartimenti}
    \framesubtitle{Conclusioni (I)}

    Quanto dimostrato conferma il comportamento ``semplice'' del modello:
    \begin{center}
        \includegraphics[keepaspectratio,width=0.6\textwidth]{img/pone.0018491.g003}
    \end{center}

    \pause
    Limite evidente: parametri \emph{costanti}.
\end{frame}

\begin{frame}
    \frametitle{Modello a due compartimenti}
    \framesubtitle{Conclusioni (II)}

    Alcuni pregi:
    \begin{itemize}
        \item \pause Elementi \emph{sociali} nelle risposte funzionali di $R$ ed $E$.
        \item \pause Effetti secondari nelle variazioni di mortalità $m$ nel comparto $F$.
        \item \pause Effetti di tipo \emph{Allee}.
    \end{itemize}
\end{frame}


\begin{frame}
    \frametitle{Esperimento A}
    \framesubtitle{\emph{FDOD} e mortalità}

    $$\text{FDOD} = \min \left\{ t \geq 0 \st H(t) + F(t) < b_{thres} \right\}$$

    \pause
    \begin{center}
        \includegraphics[keepaspectratio,width=\textwidth]{img/k11EA2-fdodVSm}
    \end{center}
\end{frame}

\begin{frame}
    \frametitle{Esperimento A}
    \framesubtitle{Popolazione iniziale e popolazione finale}

    \begin{center}
        \includegraphics[keepaspectratio,width=\textwidth]{img/k11EA2-fpopVSipop}
    \end{center}
\end{frame}

\begin{frame}
    \frametitle{Esperimento B}
    \framesubtitle{Equilibrio endemico}

    \begin{center}
        \includegraphics[keepaspectratio,width=\textwidth]{img/alternativeD}
    \end{center}
\end{frame}

\begin{frame}
    \frametitle{Esperimento B}
    \framesubtitle{Equilibrio pandemico}

    \begin{center}
        \includegraphics[keepaspectratio,width=\textwidth]{img/alternativeE}
    \end{center}
\end{frame}


\begin{frame}
    \frametitle{Modello a cinque compartimenti}
    \framesubtitle{Operaie di casa e di campo, acari e \emph{ABPV}}

    \cite{ratti2017}

    \vspace{1em}
    Elementi--chiave:
    \begin{itemize}
        \item \pause relazioni complesse tra \emph{A.~mellifera}, \emph{V.~destructor} ed \emph{ABPV}.
        \item \pause \emph{homing failure}, trattamenti anti--acaro.
%             \\ % TODO aggiustare
%             \includegraphics[keepaspectratio,width=0.2\textwidth]{img/imidacloprid}
        \item \pause analisi del sistema con parametri \emph{variabili} (Teoria di Floquet).
    \end{itemize}
\end{frame}

\begin{frame}
    \frametitle{Modello a cinque compartimenti}
    \framesubtitle{Fase riproduttiva delle varroe}

    \begin{center}
        \includegraphics[keepaspectratio,width=\textwidth]{img/varroa-in-brood}
    \end{center}
\end{frame}

\begin{frame}
    \frametitle{Modello a cinque compartimenti}
    \framesubtitle{Fase foretica delle varroe}

    \begin{center}
        \includegraphics[keepaspectratio,width=0.8\textwidth]{img/varroa-on-bee}
    \end{center}
\end{frame}

\begin{frame}
    \frametitle{Modello a cinque compartimenti}
    \framesubtitle{ABPV}

    Le varroe trasmettono anche contagi batterici e virali.

    \begin{center}
        \includegraphics[keepaspectratio,width=0.8\textwidth]{img/iabpv}
    \end{center}
\end{frame}

\begin{frame}
    \frametitle{Modello a cinque compartimenti}
    \framesubtitle{Derivazione del modello}

    \begin{itemize}
        \item le operaie di casa $x_h$,
        \item \pause le operaie di campo $x_f$,
        \item \pause le operaie infette $y$,
        \item \pause le varroe portatrici di ABPV $m$,
        \item \pause le varroe senza virus $n$.
    \end{itemize}
\end{frame}

\begin{frame}
    \frametitle{Modello a cinque compartimenti}
    \framesubtitle{Formulazioni ereditate dal modello del 2011 (I)}

    Assunzioni standard per la risposta funzionlae della schiusa:
    $$g(0)=0 \; , \; \diff{g}{x_h} \geq 0 \; , \; \diff{g}{x_f} \geq 0 \; , %
        \; \lim_{x_h+x_f \to \infty} g(x_h+x_f)=1 \; .$$

    \pause
    È una sigmoide di Hill (Holling tipo III):
    $$g(x_h + x_f) = \frac{ (x_h+x_f)^i }{ K^i + (x_h+x_f)^i } \; , \quad i>1$$

    Nel modello precedente era $\frac{H+F}{w+H+F}$ (Holling tipo II).
\end{frame}

\begin{frame}
    \frametitle{Modello a cinque compartimenti}
    \framesubtitle{Formulazioni ereditate dal modello del 2011 (I)}

    \begin{center}
            \includegraphics[keepaspectratio,width=0.9\textwidth]{img/hillSigmoid}
    \end{center}
\end{frame}

\begin{frame}
    \frametitle{Modello a cinque compartimenti}
    \framesubtitle{Formulazioni ereditate dal modello del 2011 (II)}

    La risposta funzionale del \textbf{reclutamento sociale} è esattamente la stessa:
    $$R = \sigma_1 - \sigma_2 \frac{x_f}{x_h+x_f} \quad (2017)$$
    \pause
    $$R = \alpha - \sigma \frac{F}{H+F} \quad (2011)$$
\end{frame}
%

%
% \begin{frame}
%     \frametitle{Modello a cinque compartimenti}
%     \framesubtitle{Assunzioni (II)}
%
%     \begin{itemize}
%         \item Le pupe che arrivano allo stadio adulto sono libere da virus.
%         \item \pause Il virus non ha preferenze tra api di casa e di campo.
%         \item \pause Le varroe in fase foretica non hanno preferenze tra api di casa e di campo.
%         \item \pause I trattamenti anti--acaro danneggiano anche le api,
%             ma \textbf{molto meno} rispetto alle varroe. \pause (``per costruzione'')
%     \end{itemize}
% \end{frame}


\begin{frame}
    \frametitle{Modello a cinque compartimenti}
    \framesubtitle{Equazioni (I)}

    % operaie di casa sane
    $$ \diff{x_h}{t} = \mu \pause \overbrace{ g( x_h + x_f ) }^{\text{schiusa}} %
        \pause \overbrace{ h(m) }^{\text{virus}} %
        \pause - \overbrace{ \beta_1 m \frac{x_h}{x_h + x_f + y} }^{\text{infezioni} \pause \to y} %
        \pause - \overbrace{ \left( d_1 + \delta_1 \right) x_h }^{\text{mortalità}} $$
    \pause
    $$ - \underbrace{ \gamma_1 (m+n) x_h }_{\text{varroe}} %
        - \underbrace{ x_h \cdot R(x_h, x_f) }_{\text{reclutamento} \pause \to x_f}  $$

    % operaie di campo sane
    $$ \diff{x_f}{t} = \pause  x_h \cdot R(x_h, x_f)  %
    \pause - \beta_1 m \frac{x_f}{x_h + x_f + y} $$
    \pause
    $$ - \left( p + d_2 + \delta_2 \right) x_f %
    - \pause \gamma_2 (m+n) x_f $$
\end{frame}

\begin{frame}
    \frametitle{Modello a cinque compartimenti}
    \framesubtitle{Assunzioni (I)}

    \begin{itemize}
        \item \pause Le api infette $y$ sono \textbf{di casa},
            e non vengono più reclutate per bottinare.
        \item \pause Crescita logistica per le varroe.
        \item \pause Il tasso massimale di crescita per le varroe \textbf{è lo stesso},
            che siano portatrici o no.
        \item \pause La regina non subisce la \emph{Varroa} né il virus.
            \pause La trasmissione di ABPV è esclusivamente \textbf{orizzontale}.
        \item \pause La \emph{Varroa} funge da vettore esclusivamente \textbf{meccanico} per il virus.
    \end{itemize}
\end{frame}

\begin{frame}
    \frametitle{Modello a cinque compartimenti}
    \framesubtitle{Equazioni (II)}

    % operaje infette
    $$\diff{y}{t} = \pause \overbrace{ \beta_1 m \frac{x_h + x_f}{x_h + x_f + y} }^{\text{nuove infezioni}} %
    \pause - \left( d_3 + \delta_3 \right) y  %
    \pause - \gamma_3 (m+n) y $$

    % varroe portatrici
    $$\diff{m}{t} = \pause \overbrace{ rm \left( 1 - \frac{m+n}{ \alpha (x_h + x_f + y) } \right) %
        }^{\text{riproduzione logistica}}%
        \pause + \overbrace{ \beta_2 n \frac{y}{x_h + x_f + y} }^{\text{acquisizione}}$$
    \pause
    $$- \underbrace{ \beta_3 m \frac{x_h + x_f}{x_h + x_f + y} }_{\text{perdita}} %
        \pause -\underbrace{ \delta_4 m }_{\text{trattamenti}} $$
\end{frame}


\begin{frame}
    \frametitle{Modello a cinque compartimenti}
    \framesubtitle{Equazioni (III)}

    % varroe senza virus
    $$\diff{n}{t} = rn \left( 1 - \frac{m+n}{ \alpha (x_h + x_f + y) } \right) - \beta_2 n \frac{y}{x_h + x_f + y}$$
    $$+ \beta_3 m \frac{x_h + x_f}{x_h + x_f + y} - \delta_5 n$$
\end{frame}



\begin{frame}
    \frametitle{Modello a cinque compartimenti}
    \framesubtitle{Studio della stabilità (I)}

    Gli stessi di prima (esistenza e stabilità equilibri (non) banali), \pause
    ed anche
    \begin{itemize}
        \item Equilibri endemici
        \item \pause Equilibri senza patogeni
    \end{itemize}
\end{frame}

\begin{frame}
    \frametitle{Modello a cinque compartimenti}
    \framesubtitle{Studio della stabilità (II)}

    Parametri \textbf{costanti}:
    \begin{enumerate}
        \item \pause Modello ridotto alle sole api $(x_h, x_f)$.
        \item \pause Modello ridotto senza virus $(x_h, x_f, n)$: + equilbrio \emph{senza varroe}.
        \item \pause Modello completo $(x_h, x_f, y, m, n)$: + equilibrio \emph{senza patogeni}.
    \end{enumerate}
\end{frame}

\begin{frame}
    \frametitle{Modello a cinque compartimenti}
    \framesubtitle{Studio della stabilità (III)}

    Riassunto: l'equilibrio banale esiste sempre ed è stabile.


    \begin{block}{Modello ridotto senza patogeni}
        Due equilibri $(x_h^*, x_f^*)$ non banali: esistono se
        $$\frac{\sigma_2 F}{1+F} - d_1 - \sigma_1 <0 \; , \quad %
            a > \frac{Ki}{1+F} (i-1)^{\frac{1}{i} -1}$$

        \pause
        Sono stabili se
        $$\mu \frac{ i K^i (1+F)^{i-1}{x_h^*}^{i-1} }{ {\left[ K^i +(1+F)^i {x_h^*}^i \right]}^2 } %
            < \frac{d_1 + (p+d_2)F}{1+F}$$
    \end{block}
\end{frame}

\begin{frame}
    \frametitle{Modello a cinque compartimenti}
    \framesubtitle{Studio della stabilità (IV)}

    \begin{block}{Modello ridotto senza virus}
        Due equilibri della forma $(x_h^*, F x_h^*, 0)$ esistono se
        $$\frac{\sigma_2 F}{1+F} - d_1 -\delta_1 -\sigma_1 < 0 \; , \quad %
            a > \frac{Ki}{1+F} {(i-1)}^{\frac{1}{i} -1}$$

        \pause
        Sono stabili se $r < \delta_5$ e
        $$\mu \frac{ i K^i (1+F)^{i-1}{x_h^*}^{i-1} }{ {\left[ K^i +(1+F)^i {x_h^*}^i \right]}^2 }
            < \frac{d_1 + \delta_1 + (p+d_2+\delta_2)F}{1+F}$$
    \end{block}
\end{frame}

\begin{frame}
    \frametitle{Modello a cinque compartimenti}
    \framesubtitle{Studio della stabilità (V)}

    \begin{block}{Modello completo}
        Due equilibri della forma $(x_h^*, F x_h^*, 0, 0, 0)$ esistono se
        $$\frac{\sigma_2 F}{1+F} - d_1 -\delta_1 -\sigma_1 < 0 \; , \quad %
        a > \frac{Ki}{1+F} {(i-1)}^{\frac{1}{i} -1}$$

        \pause Sono stabili se $r - \beta_3 - \delta_4 < 0 , \; r - \delta_5 < 0$ e
        $$\mu \frac{ i K^i (1+F)^{i-1}{x_h^*}^{i-1} }{ {\left[ K^i +(1+F)^i {x_h^*}^i \right]}^2 }
            < \frac{d_1 + \delta_1 + (p+d_2+\delta_2)F}{1+F}$$
    \end{block}
\end{frame}

\begin{frame}
    \frametitle{Modello a cinque compartimenti}
    \framesubtitle{Teoria di Floquet (I)}

    Parametri \emph{periodici}: Teoria di Floquet.

    \vspace{1em}
    Sistema omogeneo, lineare a coefficienti periodici
    $$\diff{}{t} \mathbf{x} = A(t) \cdot \mathbf{x}$$

    \pause
    Sia $\Psi (t)$ la sua matrice fondamentale
    \pause (s.p.g. $\Psi (0) = I_n$).

    \pause
    \vspace{1em}
    Matrice di monodromia $C= \Psi^{-1} (0) \Psi (T)$:
    \pause
    i suoi autovalori sono i \emph{moltiplicatori di Floquet}.

    \pause
    \vspace{1em}
    Forma normale di Floquet: $\Psi (t) = Q(t) e^{tB}$
\end{frame}

\begin{frame}
    \frametitle{Modello a cinque compartimenti}
    \framesubtitle{Teoria di Floquet (II)}

    Conseguenza:

    L'equilibrio $\mathbf{0}$ del sistema è
    \begin{itemize}
        \item \pause asintoticamente stabile sse tutti i moltiplicatori di Floquet stanno nella
        parte interna di $D$;
        \item \pause stabile se tutti i $\mu_j \in D$, e quelli sul bordo sono semisemplici;
        \item \pause instabile altrimenti.
    \end{itemize}
\end{frame}

\begin{frame}
    \frametitle{Modello a cinque compartimenti}
    \framesubtitle{Studio della stabilità (VI)}

    Linearizzando nel punto di equilibrio + Teoria di Floquet

    \begin{exampleblock}{Modello completo, parametri periodici}
        La soluzione periodica $\left( x_h^*(t), x_f^*(t), 0, 0, 0 \right)$
        \pause per il modello completo è stabile se
        \begin{enumerate}
            \item \pause $\left( x_h^*(t), x_f^*(t) \right)$ è stabile per il modello ridotto alle sole api;
            \item \pause $\int_0^T (r - \delta_5) dt \leq 0$;
            \item \pause $\int_0^T (r -\beta_3 - \delta_4) dt \leq 0$.
        \end{enumerate}
    \end{exampleblock}
\end{frame}




\begin{frame}
    \frametitle{Conclusioni}
    \framesubtitle{Modelli esaminati}

    \cite{khoury2011} introduce dinamiche \emph{sociali} nel reclutamento e nell'allevamento della covata.

    \pause
    \vspace{1em}
    Riutilizzate in \cite{ratti2017}:
    \begin{itemize}
        \item \pause meccanismi complessi di contagio da \emph{Varroa} ed ABPV
        \item \pause intervento umano ($\delta_i, p$)
        \item \pause parametri periodici!
    \end{itemize}

    Aderenza alla stagionalità reale, \pause dinamica \emph{funzionale} della regina.
\end{frame}

\begin{frame}
    \frametitle{Conclusioni}
    \framesubtitle{Altri modelli}

    Altri modelli per \emph{A.~mellifera} (2005--oggi):
    \begin{itemize}
        \item \pause termini di ritardo nelle equazioni
        \item \pause compartimenti per cibo, cera, propoli, feromoni, \dots
        \item \pause ``stomaco comune''
        \item \pause immunità sociale
        \item \pause spazialità
        \item \pause sciamatura
        \item \pause stagionalità, meteo, \dots
    \end{itemize}

    \pause Osservazioni:
    \begin{enumerate}
        \item Quasi tutti incorporano dinamiche sociali (2 caste di operaie) introdotte nel 2011.
        \item \pause Se considerano \emph{più di un patogeno}, uno di essi è sempre \pause la \emph{Varroa}.
    \end{enumerate}
\end{frame}

\begin{frame}
    \frametitle{Conclusioni}
    \framesubtitle{Collaborazioni extra--matematiche}

    La modellistica matematica è molto utile anche in questo ambito.

    \pause
    \vspace{1em}
    Aziende, associazioni, istituzioni, \dots \emph{collaborano} nella ricerca accademica.

    \pause
    \vspace{1em}
    O.P.
\end{frame}




\begin{frame}

    \vspace{4em}
    \centering
    Grazie per l'attenzione.

    \vspace{4em}
    Domande, interventi?
\end{frame}

\end{document}
