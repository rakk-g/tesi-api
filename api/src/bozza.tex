\documentclass[11pt,a4paper]{book} % opz. draft per vedere dove sfora i margini
\usepackage[utf8]{inputenc}
\usepackage[T1]{fontenc}
\usepackage[english,italian]{babel}
\usepackage[style=italian/guillemets]{csquotes}
\usepackage{amsmath}
\usepackage{amsfonts}
\usepackage{amssymb}
\usepackage{amsthm}
\usepackage{mathrsfs}
\usepackage{mdframed}
\usepackage{dsfont} % per l'operatore identità (unità) (v. \ident)
\usepackage{makeidx}
\usepackage{graphicx} % figure
\usepackage{booktabs} % filetti per le tabelle: [top|mid|bottom]rule

\usepackage{caption}
\captionsetup{tableposition=top,figureposition=bottom,font=small}

\usepackage{subcaption} % spezzare figure in sottofigure
\usepackage{esdiff} % comandi per le derivate: \diff, \diffp, c'è anche la valutazione in un punto!
\usepackage{mathtools}  % norma, modulo
% \usepackage{tikz-cd} % diagrammi commutativi
% \usepackage{qcircuit} % circuiti
\usepackage[left=3cm,right=3cm,top=4cm,bottom=4cm,heightrounded]{geometry}
\usepackage{microtype}

\usepackage[bibstyle=alphabetic,citestyle=authoryear-comp,backend=bibtex,useprefix]{biblatex}
\bibliography{api}

\usepackage[bookmarks,hidelinks]{hyperref} % va caricato per ultimo

% TODO rivedere!
\author{De Leo, Martino}
\title{Dinamica delle popolazioni di \emph{Apis Mellifera}}

% contributor: from PQC2015
% AMBIENTI PER TEOREMI, DEFINIZIONI ET CETERA (DA AMSTHM)
\theoremstyle{plain}
\newtheorem{teorema}{Teorema}
\newtheorem{lemma}{Lemma}
\newtheorem{corollario}{Corollario}
\newtheorem*{proposizione}{Proposizione}
\newmdtheoremenv{postulato}{Postulato}

\theoremstyle{definition}
\newtheorem{definizione}{Definizione}
\newtheorem{esempio}{Esempio}

\theoremstyle{remark}
\newtheorem{osservazione}{Osservazione}


% ]----------------------------------------------------------------[ %


% COMANDI e RINNOVI per la tesis.

\newcommand{\pcite}[1]{(\cite{#1})}

\newcommand{\omissis}{[\textellipsis\unkern] }                          % produce "[...] "

\DeclarePairedDelimiter\abs{\lvert}{\rvert}                             % modulo
\DeclarePairedDelimiter\norm{\lVert}{\rVert}                            % norma
% Swap the definition of \abs* and \norm*, so that \abs
% and \norm resizes the size of the brackets, and the
% starred version does not.
\makeatletter
    \let\oldabs\abs
    \def\abs{\@ifstar{\oldabs}{\oldabs*}}

    \let\oldnorm\norm
    \def\norm{\@ifstar{\oldnorm}{\oldnorm*}}
\makeatother


\newcommand{\N}{\mathbb{N}}                                             % numeri naturali
\newcommand{\Z}{\mathbb{Z}}                                             % interi
\newcommand{\R}{\mathbb{R}}                                             % campo reali
\newcommand{\Rplus}{\R_{\geq 0}}                                        % reali non negativi
\newcommand{\C}{\mathbb{C}}                                             % campo complessi
\newcommand{\K}{\mathbb{K}}                                             % campo generico
\newcommand{\F}{\mathbb{F}}                                             % campo finito; metti a pedice p^q
\renewcommand{\P}[1]{\mathbb{P} \left( #1 \right) }                     % spazio proiettivo (su qualcosa!)

\newcommand{\prob}{\mathds{P}}                                          % probabilità
\newcommand{\E}{\mathds{E}}                                             % valore atteso

\DeclareMathOperator{\Tr}{Tr}                                       % traccia
\DeclareMathOperator{\Sp}{Sp}                                       % spettro
\DeclareMathOperator{\Imm}{Imm}                                     % immagine di una applicazione
\DeclareMathOperator{\Span}{Span}                                   % chiusura lineare


%-----------------------------------------------------------%

\newcommand{\conj}[1]{\overline{#1}}

\newcommand{\eg}{\emph{e.g.~}}                                      % per esempio
\newcommand{\ie}{\emph{i.e.~}}                                      % id est
\newcommand{\st}{\,\mid\;}                                          % tali che, da usare in mathmode all'interno di \Set




% ]-------------- VARIABILI -----------------[ %

\renewcommand{\epsilon}{\varepsilon}
\renewcommand{\theta}{\vartheta}
\renewcommand{\phi}{\varphi}
% \renewcommand{\rho}{\varrho}


% ]-------------- AMBIENTI -----------------[ %

\newenvironment{sistema}%
{\left\lbrace\begin{array}{@{}l@{}}}%
{\end{array}\right.}
 % COMANDI E AMBIENTI

\begin{document}

\maketitle % TODO togliere p. via del frontespizio

\frontmatter

\section*{Sommario}
\input{lorem}

\section*{Notazione}
Da inserire certamente: tutti i nuovi comandi definiti


\tableofcontents

\listoffigures

\chapter{Biologia dell'ape europea}
Questo studio si concentra sulla specie \emph{Apis Mellifera} (Linneo, 1758) del genere \emph{Apis};
comunemente nota come ``ape europea'' o ``ape occidentale''.
Questa specie ha avuto origine in Medio Oriente e si è dispersa su ogni continente (esclusa l'Antartide) durante gli scorsi 2 milioni di anni.

Il rapporto tra \emph{H. Sapiens} e \emph{A. Mellifera} ha origine nel Pleistocene con le prime rudimentali interazioni;
tale relazione perdura ancora oggi articolandosi in forme complesse che comprendono:
\begin{itemize}
    \item La selezione genetica;
    \item La dispersione antropocora;
    \item La dimensione economica: il valore di mercato ``diretto'' dei prodotti dell'apicoltura è stimato intorno agli 8 miliardi di \$ annui \pcite{honeymarket1}. Il valore ``indiretto'' dovuto all'impollinazione delle colture nel settore agricolo è di gran lunga superiore, stimato tra i 235 e i 570 miliardi di \$ all'anno \pcite{honeymarket2,honeymarket3}.
    \item La dimensione ecologica: la presenza del genere \emph{Apis} contribuisce -- assieme ad altri impollinatori -- al mantenimento della biodiversità negli ecosistemi non disturbati dall'attività umana, alla stabilizzazione degli ambienti antropizzati; il declino degli impollinatori aggrava gli effetti negativi del cambiamento climatico e dell'industrializzazione dell'agricoltura \pcite{decline}.
\end{itemize}

L'ape occidentale presenta una organizzazione \emph{eusociale} e per questo conviene concettualizzare l'intero alveare come un macro--organismo, le cui capacità trofiche, riproduttive e di interazione con l'ambiente sono il risultato emergente dalla collettività dei comportamenti degli individui che costituiscono l'alveare.

La suddivisione del lavoro all'interno di un alveare riflette la partizione in caste fertili (regina, fuchi) e non fertili (operaie) della popolazione.

\begin{figure}
    \centering
    \includegraphics[keepaspectratio,width=0.86\textwidth]{img/uovaLarve}

    \caption[Covata giovane.]{Porzione di favo con covata: le celle in alto a sinistra contengono larve di 5--8 giorni; nelle altre celle è presente un uovo (< 3 giorni) \\ (Waugsberg -- 22 Aprile 2007 -- CC BY-SA 3.0)}
    \label{img:uovaLarve}
\end{figure}

L'allevamento della covata (figure~\ref{img:uovaLarve},~\ref{img:lifecycle}) avviene in modo collettivo: la regina depone le uova fecondate, mentre le operaie nutrici si occupano della pulizia, nutrimento e cura della covata.

\begin{figure}
    \centering
    \includegraphics[keepaspectratio,width=0.86\textwidth]{img/honeybeeLifecycle}

    \caption[Ciclo vitale dell'ape.]{ 1) La regina depone un uovo in una cella (da 1 a 3 giorni).
            2) Larva (da 3 a 8/9 giorni).
            3) Le nutrici chiudono la cella con l'opercolo (9° giorno).
            4) La pupa dentro la cella (dal 10° giorno).
            5) L'ape adulta emerge dall cella (21° giorno)
        \\ (illustrazione di Jennifer Sartell -- 22 Aprile 2017 -- ©)}
    \label{img:lifecycle}
\end{figure}

La popolazione di un alveare varia notevolmente col ciclo delle stagioni e mediamente comprende un'ape regina (unica femmina fertile), da 20000 a 100000 operaie (femmine sterili) e qualche centinaia di fuchi (maschi fertili) in primavera/estate.

Le tre caste presentano notevoli differenze morfologiche, che riflettono le mansioni di ognuna all'interno dell'alveare.
Ad esempio nei fuchi e nella regina le ghiandole ceripare e faringee sono decisamente meno sviluppate che nelle operaie; anche il loro apparato boccale è ridotto e per questo motivo sia la regina che i maschi vengono alimentati dalle operaie nutrici.


\section{Organismi patogeni per l'ape}
L'ape europea presenta una varietà di antagonisti naturali, appartenenti a diversi regni.

Nell'ambito dei virus troviamo il ``virus della paralisi'' (Acute Bee Paralysis Virus -- ABPV), il ``virus della regina nera'' (Black Queen Cell Virus) ed altri (Israeli Acute Paralysis Virus, Kashmir Bee Virus, Cloudy Wing Virus, \dots) nella famiglia dei \emph{Dicistroviridae}.

Il nosema (\emph{Nosema ceranae)} è un microsporidio (regno \emph{Fungi}) che nasce come parassita di \emph{Apis cerana} (la specie asiatica di ape mellifera) ma che attacca anche \emph{A. mellifera}.

Le api sono predate abitualmente da alcune specie di uccelli (ad es. il gruccione, \emph{Merops apiaster}) e da alcuni insetti (ad es. il calabrone, v.~figura~\ref{img:calabron}).

Alcune specie non parassitano direttamente le api ma altre componenti dell'alveare, come la falena \emph{Galleria mellonella} che si nutre degli esoscheletri delle pupe di ape sfarfallate, di cera e di tracce di polline che trova nelle celle abbandonate dalle api adulte.

\paragraph{}
Il patogeno di gran lunga più studiato è la varroa, acaro appartenente alle due specie \emph{Varroa destructor} e \emph{V.~jacobsoni}. La varroa parassita le larve, le pupe e le api adulte, nutrendosi delle componenti grasse e dell'emolinfa.

Il ciclo vitale della varroa è suddiviso nella fase riproduttiva e nella fase foretica:
\begin{itemize}
    \item durante la fase riproduttiva, una femmina feconda di varroa si introduce in una cella contenente una larva, prima dell'opercolazione. Dopo l'opercolazione depone le uova da cui fuoriescono gli individui fertili, che si accoppiano immediatamente e si nutrono sulla larva. Con lo sfarfallamento dell'ape, la varroa madre e le giovani femmine fecondate fuoriescono dalla cella, eventualmente attaccandosi al corpo dell'ape adulta. I maschi muoiono dentro la cella.
    \item La fase foretica avviene nelle varroe adulte che si attaccano sulle operaie, nascondendosi negli spazi interstiziali tra i segmenti dell'addome dell'ape, oppure parassitano le larve nelle celle non opercolate.
\end{itemize}

La varroa accorcia l'aspettativa di vita e indebolisce le api adulte che attacca; determina inoltre malformazioni (ad es. alle ali) nelle operaie che vengono parassitate allo stadio larvale, compromettendone le capacità da adulte.

È dimostrato inoltre che la varroa funge da vettore per molti patogeni delle api, sia virali che batterici, tra cui l'ABPV.


\subsection{Declino dovuto all'attività umana}
Molte classi di pesticidi, erbicidi ed altri antiparassitari di sintesi impiegati in agricoltura danneggiano le api: le bottinatrici entrano in contatto con queste sostanze chimiche posandosi sulle colture trattate, importano nel nido polline e nettare contaminati, dove vengono manipolati anche dalle operaie di alveare.

Le molecole imidacloprid, clothianidina, e thiamethoxam (nella classe dei neonicotinoidi) sono state correlate alla sindrome di spopolamento (Colony Collapse Disorder -- CCD) in cui si verifica un crollo repentino della popolazione in alveari apparentemente sani, che rende impossibile la sopravvivenza della famiglia.
Queste molecole vengono considerate anche causa o concausa di patologie cognitive delle api adulte, tra cui una riduzione della memoria a breve termine ed il deterioramento delle facoltà di navigazione: le bottinatrici affette perdono l'orientamento e non riescono a tornare al nido, morendo in breve tempo (per il sopraggiungere della notte o di un predatore).


\begin{figure}
    \centering
    \includegraphics[keepaspectratio,width=0.86\textwidth]{img/fanning}

    \caption[Ventilazione forzata per raffrescare l'interno dell'arnia.]{Gruppo di operaie in prossimità dell'ingresso dell'alveare ``ventilano'' per abbassare la temperatura interna dell'arnia. \\ (Ken Thomas -- 26 Maggio 2008 -- Pubblico dominio)}
\end{figure}

asfaljfafsh

\begin{figure}
    \centering
    \includegraphics[keepaspectratio,width=0.86\textwidth]{img/calabron}

    \caption[Ape predata da un calabrone.]{Una calabrone operaia (\emph{Vespa Crabro}) si nutre di una ape operaia (\emph{Apis Mellifera}) appena catturata. \\ (Böhringer Friedrich -- 20 Agosto 2004 -- CC BY-SA 2.5)}
    \label{img:calabron}
\end{figure}

asfaljfafsh





% <MAIN>
\mainmatter

\part{Dinamica delle popolazioni ed epidemiologia}




\chapter[Modelli]{Modelli matematici}
\section{Sistemi dinamici}

\subsection{Ratti2017}
Esaminiamo un modello proposto da \citeauthor{ratti2017}, che estende alcune proposte precedenti con lo scopo di studiare gli effetti diretti e indiretti di \emph{Varroa destructor} sulla popolazione di un alveare.


\subsubsection{Assunzioni}
\begin{enumerate}
    \item La popolazione delle operaie è suddivisa in due comparti: le operaie più giovani che svolgono mansioni all'interno dell'alveare ($x_h$) e le bottinatrici/esploratrici ($x_f$).

    Questa impostazione ricalca lo studio \pcite{khoury2011}, in cui si propone un modello a due dimensioni per esaminare gli effetti del tasso di schiusa delle pupe e del tasso di reclutamento da operaie d'alveare a bottinatrici sulla dinamica di una popolazione sana.
    \item La popolazione degli acari è suddivisa in due comparti: le varroe portatrici di ABPV ($m$) e le varroe libere dal virus ($n$).
    \item Le operaie infette con ABPV ($y$) perdono rapidamente l'abilità al volo, le loro capacità di orientamento degradano e muoiono in breve tempo. Per questo motivo le api infette sono considerate ``di alveare'', e non vengono reclutate per bottinare.
\end{enumerate}

I compartimenti del modello sono dunque cinque:
\begin{itemize}
    \item operaie ``di alveare'' sane,
    \item operaie bottinatrici sane,
    \item operaie infette,
    \item varroe portatrici di ABPV,
    \item varroe senza virus.
\end{itemize}

Abbiamo inoltre
$$\text{Pop. totale api} = x_h + x_f + y,$$
e
$$\text{Pop. totale varroe} = m + n.$$

\begin{enumerate}
    \setcounter{enumi}{3}
    \item Il tasso massimale di riproduzione delle varroe è lo stesso, che siano portatrici di virus o meno.
    \item Sappiamo che la covata necessita di cure da parte delle operaie nutrici, quindi il tasso di schiusa è dipendente dal numero di operaie nella popolazione, che seguendo \cite{khoury2011} assumiamo pari alla somma $x_h + x_f$.
    \item L'ape regina non subisce gli affetti avversi di ABPV e dell'infestazione da varroe; di conseguenza, il tasso di schiusa non dipende dalla preminenza nell'alveare di varroe o di virus.
    \item La trasmissione di ABPV è esclusivamente orizzontale. Le varroe sono un vettore puramente meccanico per il virus.
    \item Poiché le pupe infette con ABPV muoiono rapidamente prima della schiusa, tutte le giovani operaie che giungono all'età adulta si assumono libere dal virus. L'effetto limitante sul tasso di schiusa dovuto alla presenza di varroe infette viene incorporato in un fattore di modulazione $h(m)$.
    \item Per studiare l'effetto dei trattamenti acaricidi, introduciamo un ulteriore termine di pozzo $\delta_i$ in ogni equazione. Assumiamo che l'impatto dei trattamenti sia di gran lunga maggiore sulle varroe che sulle api: $\delta_4, \delta_5 > \delta_1, \delta_2, \delta_3$.
    \item In tutti e tre i compartimenti delle api, si assume un tasso di mortalità naturale $d_i$ in aggiunta alla mortalità dovuta alle varroe $\gamma_i$ (a sua volta dipendente sia dal parassitismo che dalla trasmissione del virus). Nelle api indebolite dall'infezione da ABPV entrambi questi tassi sono maggiori rispetto alle operaie sane: $d_3 > d_1, d_2$ e $\gamma_3 > \gamma_1, \gamma_2$.

    Inoltre per le bottinatrici si tiene conto anche della mortalità da disorientamento (fattore $p$), dovuto principalmente al degrado delle facoltà cognitive e di navigazione per effetto dei pesticidi sintetici sulle colture.
    \item sfaaf
\end{enumerate}


\subsubsection{Equazione del modello}
Il modello a compartimenti proposto in \cite{ratti2017} è il seguente:
\begin{align}
    % hive sane
    \diff{x_h}{t} &= \mu g( x_h + x_f ) h(m) - \beta_1 m \frac{x_h}{x_h + x_f + y} - \left( d_1 + \delta_1 \right) x_h \notag \\ % barbatrucco
    \label{eq:r17xh}
        &{\;} - \gamma_1 (m+n) x_h - x_h \cdot R(x_h, x_f) \\ % barbatrucco
    % foragers sane
    \label{eq:r17xf}
    \diff{x_f}{t} &= x_h \cdot R(x_h, x_f) - \beta_1 m \frac{x_f}{x_h + x_f + y}
    - \left( p + d_2 + \delta_2 \right) x_f - \gamma_2 (m+n) x_f \\
    % (hive) infette
    \label{eq:r17y}
    \diff{y}{t} &= \beta_1 m \frac{x_h + x_f}{x_h + x_f + y} - \left( d_3 + \delta_3 \right) y - \gamma_3 (m+n) y \\
    % varroe con ABPV
    \label{eq:r17m}
    \diff{m}{t} &= rm \left( 1 - \frac{m+n}{ \alpha (x_h + x_f + y) } \right) + \beta_2 n \frac{y}{x_h + x_f + y}
    - \beta_3 m \frac{x_h + x_f}{x_h + x_f + y} - \delta_4 m \\
    % varroe senza ABPV
    \label{eq:r17n}
    \diff{n}{t} &= rn \left( 1 - \frac{m+n}{ \alpha (x_h + x_f + y) } \right) - \beta_2 n \frac{y}{x_h + x_f + y}
    + \beta_3 m \frac{x_h + x_f}{x_h + x_f + y} - \delta_5 n
\end{align}

dove
\begin{itemize}
    \item $x_h$ è il numero di operaie sane che vivono all'interno dell'alveare (\emph{hive})
    \item $x_f$ è il numero di operaie bottinatrici (\emph{foragers})
    \item $y$ è il numero di operaie infette da ABPV
    \item $m$ è il numero di varroe portatrici di ABPV
    \item $n$ è il numero di varroe non portatrici di virus.
\end{itemize}


Il termine ``di reclutamento'' $x_h \cdot R(x_h, x_f)$ nelle equazioni \ref{eq:r17xh} e \ref{eq:r17xf} rappresenta appunto il passaggio di ruolo delle operaie, da impiegate \emph{all'interno} dell'alveare a \emph{bottinatrici} che escono dall'arnia per esplorare e raccogliere il cibo.








\part{Modelli computazionali}




% </MAIN>

\backmatter
\cleardoublepage
\addcontentsline{toc}{chapter}{\bibname}
\nocite{*} % per fargli stampare anche i riferimenti non citati nella tesi
\printbibliography


\end{document}
