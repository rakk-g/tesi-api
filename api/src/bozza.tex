\documentclass[11pt,a4paper]{book} % opz. draft per vedere dove sfora i margini
% \usepackage[latin1]{inputenc}
\usepackage[T1]{fontenc}
\usepackage[english,italian]{babel}
\usepackage[style=italian/guillemets]{csquotes}
\usepackage{amsmath}
\usepackage{amsfonts}
\usepackage{amssymb}
\usepackage{amsthm}
\usepackage{mathrsfs}
\usepackage{mdframed}
\usepackage{dsfont} % per l'operatore identit� (unit�) (v. \ident)
\usepackage{makeidx}
\usepackage{graphicx} % figure
\usepackage{booktabs} % filetti per le tabelle: [top|mid|bottom]rule
\usepackage{caption}
\captionsetup{tableposition=top,figureposition=bottom,font=small}
\usepackage{subcaption} % spezzare figure in sottofigure
\usepackage{lmodern} % TODO: ?
\usepackage{braket}     % \Set, \Bra, \Ket, \Braket
\usepackage{mathtools}  % norma, modulo
\usepackage{tikz-cd} % diagrammi commutativi
\usepackage{qcircuit} % circuiti
\usepackage[left=3cm,right=3cm,top=4cm,bottom=4cm,heightrounded]{geometry}
\usepackage{microtype}
\usepackage[bookmarks]{hyperref} % va caricato per ultimo

\usepackage[bibstyle=alphabetic,citestyle=authoryear-comp,backend=bibtex,useprefix]{biblatex}
\bibliography{pqc}


% TODO rivedere!
\author{Rachele Grasshopper}
\title{Crittografia Post-Quantistica\\\normalsize{nuove prospettive di difesa in un mondo sempre pi� minaccioso}}


% COMANDI E AMBIENTI
% contributor: from PQC2015
% AMBIENTI PER TEOREMI, DEFINIZIONI ET CETERA (DA AMSTHM)
\theoremstyle{plain}
\newtheorem{teorema}{Teorema}
\newtheorem{lemma}{Lemma}
\newtheorem{corollario}{Corollario}
\newtheorem*{proposizione}{Proposizione}
\newmdtheoremenv{postulato}{Postulato}

\theoremstyle{definition}
\newtheorem{definizione}{Definizione}
\newtheorem{esempio}{Esempio}

\theoremstyle{remark}
\newtheorem{osservazione}{Osservazione}


% ]----------------------------------------------------------------[ %


% COMANDI e RINNOVI per la tesis.

\newcommand{\pcite}[1]{(\cite{#1})}

\newcommand{\omissis}{[\textellipsis\unkern] }                          % produce "[...] "

\DeclarePairedDelimiter\abs{\lvert}{\rvert}                             % modulo
\DeclarePairedDelimiter\norm{\lVert}{\rVert}                            % norma
% Swap the definition of \abs* and \norm*, so that \abs
% and \norm resizes the size of the brackets, and the
% starred version does not.
\makeatletter
    \let\oldabs\abs
    \def\abs{\@ifstar{\oldabs}{\oldabs*}}

    \let\oldnorm\norm
    \def\norm{\@ifstar{\oldnorm}{\oldnorm*}}
\makeatother


\newcommand{\N}{\mathbb{N}}                                             % numeri naturali
\newcommand{\Z}{\mathbb{Z}}                                             % interi
\newcommand{\R}{\mathbb{R}}                                             % campo reali
\newcommand{\Rplus}{\R_{\geq 0}}                                        % reali non negativi
\newcommand{\C}{\mathbb{C}}                                             % campo complessi
\newcommand{\K}{\mathbb{K}}                                             % campo generico
\newcommand{\F}{\mathbb{F}}                                             % campo finito; metti a pedice p^q
\renewcommand{\P}[1]{\mathbb{P} \left( #1 \right) }                     % spazio proiettivo (su qualcosa!)

\newcommand{\prob}{\mathds{P}}                                          % probabilità
\newcommand{\E}{\mathds{E}}                                             % valore atteso

\DeclareMathOperator{\Tr}{Tr}                                       % traccia
\DeclareMathOperator{\Sp}{Sp}                                       % spettro
\DeclareMathOperator{\Imm}{Imm}                                     % immagine di una applicazione
\DeclareMathOperator{\Span}{Span}                                   % chiusura lineare


%-----------------------------------------------------------%

\newcommand{\conj}[1]{\overline{#1}}

\newcommand{\eg}{\emph{e.g.~}}                                      % per esempio
\newcommand{\ie}{\emph{i.e.~}}                                      % id est
\newcommand{\st}{\,\mid\;}                                          % tali che, da usare in mathmode all'interno di \Set




% ]-------------- VARIABILI -----------------[ %

\renewcommand{\epsilon}{\varepsilon}
\renewcommand{\theta}{\vartheta}
\renewcommand{\phi}{\varphi}
% \renewcommand{\rho}{\varrho}


% ]-------------- AMBIENTI -----------------[ %

\newenvironment{sistema}%
{\left\lbrace\begin{array}{@{}l@{}}}%
{\end{array}\right.}


\begin{document}

%\maketitle % TODO togliere

\frontmatter

\chapter{Intro}
\input{lorem}
\section*{Notazione}
Da inserire certamente: tutti i nuovi comandi definiti, probabilit� condizionale, valore atteso, nomenclatura per gli elementi degli Hilbert spaces, notazione bra-ket di Dirac, et cetera...

Notazione $O, \Theta, \Omega$. $\C^\ast = \C \setminus \Set 0$.
\input{lorem}

\tableofcontents


% MAIN
\mainmatter

\part{Computazione Quantistica}
\chapter{Meccanica Quantistica}
Introduzione agli strumenti algebrici (spazi di Hilbert, strutture di $R$-modulo, prodotti tensori, $*$-algebre di operatori, ...) e analitici (per quanto riguarda le POVM) e ai metodi per ``maneggiare'' sistemi quantistici: stati (puri e misti), sistemi composti, misurazioni, osservabili, trasformazioni di stato.
Questo ``impianto'' dovrebbe essere la cornice di riferimento per tutto il seguito; perci� seppure non eccessivamente approfondito (trascurerei le questioni pratiche legate alla realizzazione fisica di esperimenti quantistici) vorrei farlo abbastanza ``solido'' e autocontenuto.

Passerei poi a presentare le principali novit� e opportunit� offerte dalla meccanica quantistica, soprattutto in relazione alla computazione e alla crittografia (entanglement, teletrasporto di stati, no-cloning), ma vorrei anche accennare a delle differenze di fondo rispetto alla Fisica Classica (esperimento di Aspect) che ci impongono di riconsiderare sotto una nuova luce le teorie della Probabilit� e dell'Informazione (da approfondire nel secondo capitolo).
\input{pezzi/postulati_qm}
\input{pezzi/nuove}

\chapter{Teoria dell'Informazione Quantistica}
\label{chap:tiq}
Le teorie della Probabilit� e della Informazione classiche non sono sufficienti a descrivere il mondo su scala quantistica; per questo vorrei presentare le loro ``rivisitazioni'' in chiave quantistica: entropia e informazione, i canali quantistici, una caratterizzazione degli errori su tali canali, e alcuni risultati teorici che hanno un impatto diretto sull'approccio quantistico alla trasmissione ed elaborazione delle informazioni (Holevo bound, capacit� dei canali).
\section{Entropie di Shannon e di Von Neumann}
\section{Canali quantistici}
\subsection{Errori}
\subsection{Informazione accessibile: il limite di Holevo}
\subsection{Capacit� di un canale quantistico}
    
\chapter{Computazione Quantistica}
Vorrei utilizzare il modello dei Quantum Gate Array e presentare alcune peculiarit� importanti della computazione in ambito quantistico: reversibilit�, parallelismo, ... con dei cenni anche alla calcolabilit�/complessit� (universalit�, classi di complessit�). Non vorrei entrare esageratamente nel dettaglio sulle classi di complessit� (relazioni con le classi di complessit� classiche, parallelismo, ...) perch� credo ci sia il rischio di allontanarsi dall'obiettivo (crittografia).

Per quanto riguarda gli algoritmi vorrei presentare quelli pi� emblematici, che rivelano un vantaggio rispetto alle controparti classiche, e analizzare i limiti superiori/inferiori imposti alla complessit� di un problema su cui basare una primitiva crittografica quantistica/post-quantistica; con questo spirito vorrei parlare del problema del sottogruppo nascosto (HSP), delle sue riduzioni, dei problemi aperti nel caso non simmetrico.
\input{pezzi/modello}
\input{pezzi/algos}


\part{Crittografia Quantistica e Post-Quantistica}
    \chapter{Crittografia Quantistica}
    Credo che un ``assaggio'' di crittografia quantistica potrebbe aiutare a marcare le differenze con quella post-quantistica e a delineare il contesto di riferimento (p.e. caratterizzando le classi di attacchi portabili con una macchina quantistica, disturbanza e guadagno di informazione da parte dell'attaccante...)
        \section{Gli schemi BB84, B92, Ekert}
            \subsection{Sicurezza incondizionale di BB84}
            \subsection{Sfruttare l'entanglement}
        \section{Attacchi ai cifrari quantistici}
            \subsection{Disturbanza e guadagno di informazione}

    \chapter{Crittografia Post-Quantistica}
    Per gli schemi basati su hash parlerei del MSS (Merkle) ma senza approfondire troppo questioni che risulterebbero veramente moooolto lunghe (nello specifico: visita dell'albero di autenticazione).
    Qualche riguardo in pi�, anche dal punto di vista teorico e algebrico, lo userei nell'affrontare la crittografia basata sui codici (McEliece, Niederreiter, il problema dell'equivalenza di codici) ma l'argomento su cui investirei di pi� � quello dei reticoli.
    
    Per i reticoli i prerequisiti algebrici sono piuttosto pesanti ma una volta fatto ci� potremmo derivare molte applicazioni crittografiche e mostrarne la robustezza spesso tramite riduzioni dal caso medio al caso pessimo. Di questo parlerei ampiamente e dal nostro ultimo colloquio (e dai materiali che mi ha dato) mi sembra che si possa definire l'argomento centrale della tesi; se nonostante ci� quest'ultimo capitolo sembra poco delineato, in parte � dovuto al fatto che sto ancora cercando di organizzare le idee e i molti risultati per coglierne gli aspetti essenziali (p.e. ho trovato molti articoli che presentano uno stesso cifrario/algoritmo ma a volte la riduzione dell'uno all'altro non � palese, oppure ancora ci sono differenze nelle complessit� dell'ordine dello zerovirgola, ... ).
    Le ultime due sezioni (Hidden Field Equations e Multivariate Quadratic Equations, nell'ordine) hanno il punto interrogativo perch� riguardano primitive crittografiche che ho incontrato solo di sfuggita e non ho ancora approfondito; per questo non so se � il caso di includerle o meno.
        \section{Schemi di firma basati su funzioni hash}
        \section{Crittografia basata su codici autocorrettori}
        \section{Crittografia basata su reticoli}
                \subsection{Funzioni hash resistenti alle collisioni}
                \subsection{Schemi pubblici semanticamente sicuri}
        \section{$HFE^{-v}$ (?)}
        \section{MQE (?)}

        
        
        
% /MAIN


\backmatter
\cleardoublepage
\addcontentsline{toc}{chapter}{\bibname}
\nocite{*} % per fargli stampare anche i riferimenti non citati nella tesi
\printbibliography


\end{document}