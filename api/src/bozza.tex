\documentclass[11pt,a4paper]{book} % opz. draft per vedere dove sfora i margini
\usepackage[utf8]{inputenc}
\usepackage[T1]{fontenc}
\usepackage[english,italian]{babel}
\usepackage[style=italian/guillemets]{csquotes}
\usepackage{amsmath}
\usepackage{amsfonts}
\usepackage{amssymb}
\usepackage{amsthm}
\usepackage{mathrsfs}
\usepackage{mdframed}
\usepackage{dsfont} % per l'operatore identità (unità) (v. \ident)
\usepackage{makeidx}
\usepackage{graphicx} % figure
\usepackage{booktabs} % filetti per le tabelle: [top|mid|bottom]rule

\usepackage{caption}
\captionsetup{tableposition=top,figureposition=bottom,font=small}

\usepackage{subcaption} % spezzare figure in sottofigure
\usepackage{esdiff} % comandi per le derivate: \diff, \diffp, c'è anche la valutazione in un punto!
\usepackage{mathtools}  % norma, modulo
% \usepackage{tikz-cd} % diagrammi commutativi
% \usepackage{qcircuit} % circuiti
\usepackage[left=3cm,right=3cm,top=4cm,bottom=4cm,heightrounded]{geometry}
\usepackage{microtype}
\usepackage[bookmarks]{hyperref} % va caricato per ultimo

\usepackage[bibstyle=alphabetic,citestyle=authoryear-comp,backend=bibtex,useprefix]{biblatex}
\bibliography{api}


% TODO rivedere!
\author{De Leo, Martino}
\title{Modelli di dinamica delle popolazioni di \emph{Apis Mellifera}}


% COMANDI E AMBIENTI
% AMBIENTI PER TEOREMI, DEFINIZIONI ET CETERA (DA AMSTHM)
\theoremstyle{plain}
\newtheorem{teorema}{Teorema}
\newtheorem{problema}{Problema} % usato generalmente nella phorma input - output richiesto
\newtheorem{lemma}{Lemma}
\newtheorem{corollario}{Corollario}
\newmdtheoremenv{postulato}{Postulato}

\theoremstyle{definition}
\newtheorem{definizione}{Definizione}
\newtheorem{esempio}{Esempio}

\theoremstyle{remark}
\newtheorem{osservazione}{Osservazione}


% ]----------------------------------------------------------------[ %


% COMANDI e RINNOVI per la tesis.

\newcommand{\omissis}{[\textellipsis\unkern]}                           % produce "[...]"

\DeclarePairedDelimiter\abs{\lvert}{\rvert}                             % modulo
\DeclarePairedDelimiter\norm{\lVert}{\rVert}                            % norma
% Swap the definition of \abs* and \norm*, so that \abs
% and \norm resizes the size of the brackets, and the 
% starred version does not.
\makeatletter
    \let\oldabs\abs
    \def\abs{\@ifstar{\oldabs}{\oldabs*}}

    \let\oldnorm\norm
    \def\norm{\@ifstar{\oldnorm}{\oldnorm*}}
\makeatother

\newcommand{\opnorm}[1]{\norm{#1}_{\textup{op}}}                        % norma operatore

\newcommand{\N}{\mathbb{N}}                                             % numeri naturali
\newcommand{\Z}{\mathbb{Z}}                                             % interi
\newcommand{\R}{\mathbb{R}}                                             % campo reali
\newcommand{\Rplus}{\R_{\geq 0}}                                        % reali non negativi
\newcommand{\C}{\mathbb{C}}                                             % campo complessi
\newcommand{\K}{\mathbb{K}}                                             % campo generico
\newcommand{\F}{\mathbb{F}}                                             % campo finito; metti a pedice p^q
\renewcommand{\P}[1]{\mathbb{P} \left( #1 \right) }                     % spazio proiettivo (su qualcosa!)

\newcommand{\prob}{\mathds{P}}                                          % probabilit�
\newcommand{\E}{\mathds{E}}                                             % valore atteso
\newcommand{\Borel}{\mathcal{B}}                                        % sigma-algebra di Borel

\newcommand{\U}{\mathds{U}}                                             % standard time evolution (unitaria)

% ]------------------------- SPAZI ----------------------------[ %

\newcommand{\hs}[1]{\mathcal{#1}}                                   % spazi di Hilbert

\newcommand{\duale}[1]{{#1}'}                                           % duale algebrico
\newcommand{\tduale}[1]{{#1}^{\ast}}                                % duale topologico, usa questo per il duale delle mappe

\newcommand{\base}[1]{\mathcal{#1}}                                 % base (ortonormale?)
\newcommand{\puri}[1]{\mathcal{P}\left(#1\right)}                   % stati puri, l'argomento � lo spazio di Hilbert
\newcommand{\stati}[1]{\mathcal{S}\left(#1\right)}                  % stati, c.s.

% CATEGORIE
\newcommand{\vectcat}[1]{\textbf{Vect}_{#1}}                        % categoria degli spazi vettoriali
\newcommand{\hilbcat}{\textbf{Hilb}}                                % sottocategoria degli spazi di hilbert

% (SPAZI DI) OPERATORI (FUNTORI NELLA CATEGORIA)
\newcommand{\hiHom}[2]{\hom\left(\hs{#1},\hs{#2}\right)}            % (handler) per spazi di Hilbert
\DeclareMathOperator{\End}{End}
\newcommand{\hiEnd}[1]{\End\left(\hs{#1}\right)}                    % (handler) per spazi di Hilbert
\DeclareMathOperator{\Bound}{Bou}                                   % omomorfismi limitati
\newcommand{\hiBound}[2]{\Bound\left(\hs{#1},\hs{#2}\right)}        % (handler) per spazi di Hilbert
\newcommand{\hiEndBound}[1]{\Bound\left(\hs{#1}\right)}             % (handler) endomorfismi limitati per sp. Hilbert

\newcommand{\Ident}[1]{\mathds{1}_{#1}}                             % operatore identit� generico (TODO come fo la matrice?)
\newcommand{\hiIdent}[1]{\Ident{\hs{#1}}}                           % (handler) operatore identit� per spazi di Hilbert

\newcommand{\adj}[1]{{#1}^{\dag}}                                   % aggiunzione (era: \ast)

\DeclareMathOperator{\Tr}{Tr}                                       % traccia
\DeclareMathOperator{\Sp}{Sp}                                       % spettro
\DeclareMathOperator{\Imm}{Imm}                                     % immagine di una applicazione
\DeclareMathOperator{\Span}{Span}                                   % chiusura lineare

\DeclareMathOperator{\alghull}{alg}                                 % chiusura algebrica
\newcommand{\alg}[1]{\mathscr{#1}}                                  % algebra associativa
\newcommand{\cstar}{C^{\ast}}                                       % algebre C*
\newcommand{\cstaralgstati}[1]{\tduale{\alg{#1}}_{+,1}}             % (handler) stati di un'algebra C*
\newcommand{\matr}{\mathscr{M}}                                     % matrici, si sottintende quadrate e a entrate in \C

%-----------------------------------------------------------%

\newcommand{\conj}[1]{\overline{#1}}

\newcommand{\eg}{\emph{e.g.~}}                                      % per esempio
\newcommand{\ie}{\emph{i.e.~}}                                      % id est
\newcommand{\st}{\,\mid\;}                                             % tali che, da usare in mathmode all'interno di \Set

% ]-------------- (CLASSI DI) COMPLESSIT� -----------------[ %

\newcommand{\cc}[1]{\mathsf{#1}}                                    % classi di complessit�
\newcommand{\lb}[1]{\mathtt{#1}}                                    % blocco logico O ALGORITMO
\newcommand{\orac}[1]{\mathcal{O}_{#1}}                             % oracolo (black box) come trasformazione: come blocco usa \lb{O}_ <- NO a me questa cosa non mi piace
\newcommand{\vet}[1]{\mathbf{#1}}                                   % vettorizza







\renewcommand{\epsilon}{\varepsilon}
\renewcommand{\theta}{\vartheta}
\renewcommand{\phi}{\varphi}
% \renewcommand{\rho}{\varrho}

\begin{document}

\maketitle % TODO togliere p. via del frontespizio

\frontmatter

\chapter{Intro}
\section*{Sommario}
\input{lorem}

\section*{Notazione}
Da inserire certamente: tutti i nuovi comandi definiti


\tableofcontents

\chapter{Biologia dell'ape europea}
Questo studio si concentra sulla specie \emph{Apis Mellifera} (Linneo, 1758) del genere \emph{Apis};
comunemente nota come ``ape europea'' o ``ape occidentale'', questa specie ha avuto origine in Medio Oriente e si è dispersa su ogni continente (esclusa l'Antartide) durante gli scorsi 2 milioni di anni.

Il rapporto tra \emph{H. Sapiens} e \emph{A. Mellifera} ha origine nel Pleistocene con le prime rudimentali interazioni;
tale rapporto perdura ancora oggi articolandosi in forme complesse che comprendono
\begin{itemize}
    \item La selezione genetica
    \item La dispersione antropocora
    \item La dimensione economica: il valore di mercato ``diretto'' dei prodotti dell'apicoltura è superato di X ordini di grandezza dal valore ``indiretto'' dovuto all'impollinazione delle colture nel settore agricolo;
    \item La presenza di \emph{A. Mellifera} contribuisce al mantenimento della biodiversità negli ecosistemi non disturbati dall'attività antropica;
\end{itemize}

regina operaie fuchi

varroa

pesticidi


% <MAIN>
\mainmatter

\part{Dinamica delle popolazioni di \emph{A. Mellifera}}





\chapter[Modelli]{Modelli matematici}
\section{Sistemi dinamici}

\subsection{Ratti2017}
Esaminiamo un modello proposto da \citeauthor{ratti2017} che tenta di integrare alcune proposte precedenti con lo scopo di studiare gli effetti diretti e indiretti di \emph{Varroa destructor}.

\subsubsection{Assunzioni}
\begin{enumerate}
    \item La popolazione delle operaie è suddivisa in due comparti: le operaie più giovani che svolgono mansioni all'interno dell'alveare e le bottinatrici/esploratrici.
    \item La popolazione degli acari è suddivisa in due comparti: le varroe portatrici di ABPV e le varroe libere dal virus.
    \item Le operaie infette con ABPV perdono rapidamente l'abilità al volo, le capacità di orientamento degradano rapidamente e muoiono in breve tempo. Per questo motivo le api infette sono considerate ``di alveare'', e non vengono reclutate per bottinare.
\end{enumerate}

I compartimenti del modello sono dunque cinque: operaie ``di alveare'' sane, operaie bottinatrici sane, operaie infette, varroe portatrici di ABPV e varroe senza virus.

\begin{enumerate}
    \setcounter{enumi}{3}
    \item Il tasso massimale di riproduzione delle varroe è lo stesso, che siano portatrici di virus o meno.
    \item
\end{enumerate}


\subsubsection{Equazione del modello}
Il modello a compartimenti proposto in \cite{ratti2017} è il seguente:
\begin{align}
    % hive sane
    \diff{x_h}{t} &= \mu g( x_h + x_f ) h(m) - \beta_1 m \frac{x_h}{x_h + x_f + y} - \left( d_1 + \delta_1 \right) x_h \notag \\ % barbatrucco
    \label{eq:r17xh}
        &{\;} - \gamma_1 (m+n) x_h - x_h \cdot R(x_h, x_f) \\ % barbatrucco
    % foragers sane
    \label{eq:r17xf}
    \diff{x_f}{t} &= x_h \cdot R(x_h, x_f) - \beta_1 m \frac{x_f}{x_h + x_f + y}
    - \left( p + d_2 + \delta_2 \right) x_f - \gamma_2 (m+n) x_f \\
    % (hive) infette
    \label{eq:r17y}
    \diff{y}{t} &= \beta_1 m \frac{x_h + x_f}{x_h + x_f + y} - \left( d_3 + \delta_3 \right) y - \gamma_3 (m+n) y \\
    % varroe con ABPV
    \label{eq:r17m}
    \diff{m}{t} &= rm \left( 1 - \frac{m+n}{ \alpha (x_h + x_f + y) } \right) + \beta_2 n \frac{y}{x_h + x_f + y}
    - \beta_3 m \frac{x_h + x_f}{x_h + x_f + y} - \delta_4 m \\
    % varroe senza ABPV
    \label{eq:r17n}
    \diff{n}{t} &= rn \left( 1 - \frac{m+n}{ \alpha (x_h + x_f + y) } \right) - \beta_2 n \frac{y}{x_h + x_f + y}
    + \beta_3 m \frac{x_h + x_f}{x_h + x_f + y} - \delta_5 n
\end{align}

dove
\begin{itemize}
    \item $x_h$ è il numero di operaie sane che vivono all'interno dell'alveare (\emph{hive})
    \item $x_f$ è il numero di operaie bottinatrici (\emph{foragers})
    \item $y$ è il numero di operaie infette da ABPV
    \item $m$ è il numero di varroe portatrici di ABPV
    \item $n$ è il numero di varroe non portatrici di virus.
\end{itemize}


Il termine ``di reclutamento'' $x_h \cdot R(x_h, x_f)$ nelle equazioni \ref{eq:r17xh} e \ref{eq:r17xf} rappresenta appunto il passaggio di ruolo delle operaie, da impiegate \emph{all'interno} dell'alveare a \emph{bottinatrici} che escono dall'arnia per esplorare e raccogliere il cibo.








\section{Modelli computazionali}






\chapter{Computazione Quantistica}
Vorrei utilizzare il modello dei Quantum Gate Array e presentare alcune peculiarità importanti della computazione in ambito quantistico: reversibilità, parallelismo, ... con dei cenni anche alla calcolabilità/complessità (universalità, classi di complessità). Non vorrei entrare esageratamente nel dettaglio sulle classi di complessità (relazioni con le classi di complessità classiche, parallelismo, ...) perché credo ci sia il rischio di allontanarsi dall'obiettivo (crittografia).


% </MAIN>


\backmatter
\cleardoublepage
\addcontentsline{toc}{chapter}{\bibname}
\nocite{*} % per fargli stampare anche i riferimenti non citati nella tesi
\printbibliography


\end{document}
