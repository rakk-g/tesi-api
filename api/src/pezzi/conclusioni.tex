\chapter{Considerazioni finali}
A partire dal secondo lustro del nuovo millennio la ricerca matematica ha prodotto nuovi modelli per
descrivere la dinamica delle popolazioni di imenotteri, in particolare di \emph{Apis mellifera} ed in risposta
ad una crisi globale degli insetti impollinatori.\footcite{chen_review,decline}

\paragraph{}
I due modelli esaminati nel capitolo~\ref{chap:modelz} introducono alcuni elementi importanti per rappresentare
la dinamica di un alveare: nel primo si formulano i termini di schiusa delle uova e di
trasferimento tra i compartimenti di operaie, in modo da tenere conto della struttura \emph{sociale} di tali meccanismi
all'interno della colonia.

Tali formulazioni catturano in modo efficace il comportamento degli alveari (soprattutto in regime di allevamento)
e vengono riutilizzati anche in molti modelli successivi.

I risultati stabiliti qui e nell'articolo originale\footcite{khoury2011} caratterizzano
completamente i punti di equilibrio del sistema.

\paragraph{}
Il secondo modello\footcite{ratti2017} non è storicamente il primo ad introdurre un patogeno, ma esprime con successo
il ruolo fondamentale dell'acaro \emph{Varroa destructor} come ``apripista'' per altre patologie che portano
al rapido collasso della colonia.

La \emph{Varroa} danneggia le api e le larve, innescando le risposte fisiologiche della famiglia. Il patogeno
trasportato dall'acaro (un virus, in questo caso) incontra dunque una colonia \emph{già stressata},
quindi più vulnerabile e prona al collasso.

Tale modello -- esaminato nella sezione~\ref{sec:ratti17} -- prende in considerazione anche la mortalità da
\emph{homing failure}\footnote{Fenomeno dovuto principalmente all'uso di
pesticidi in agricoltura. Cfr. sez.~\ref{ssec:declinoUmano}.}
e gli effetti dei trattamenti anti--acaro somministrati in apicoltura.

\paragraph{}
Questo secondo modello incorpora molti elementi del precedente\footcite{khoury2011}, e stabilisce un nuovo
importante risultato: per ``salvare'' una colonia durante una infestazione da \emph{Varroa} e virus della
paralisi acuta è necessario:
\begin{itemize}
    \item effettuare trattamenti anti--acaro \emph{sufficienti} ed eradicare gli acari \emph{non portatori} di virus;
    \item che le varroe perdano il virus ad un tasso più rapido della loro natalità.
\end{itemize}

Poiché la seconda condizione è generalmente valida per la biologia varroa/ABPV, e praticamente indipendente
dalle condizioni dell'alveare, l'apicoltura di produzione e sperimentale si concentra sul primo fattore,
concentrando la ricerca sui trattamenti anti--varroa.\footcite{privFDL}

\paragraph{}
Altri modelli matematici per \emph{A.~mellifera} introducono termini di ritardo per la transizione da larva
a giovane ape operaia\footcite{messan2021}, fattori come il cibo\footcite{khoury2013,perry2015rapid},
varie forme di ``stomaco comune'' (per la conservazione/trasmissione di cibo e segnali feromonici tra operaie)
e di immunità sociale\footcite{schmickl2016,schmickl2017,laomettachit2021},
o ancora incorporano gli effetti spaziali (ad es. nella distanza tra nido e cibo)\footcite{spatial}
o la sciamatura\footcite{messan2017}.

\paragraph{}
Buona parte dei modelli matematici -- perlomeno quelli più recenti -- per le api tengono conto degli effetti
stagionali (\ie variabilità di alcuni parametri) e del clima.\footcite{chen_review}

Spesso si considera l'effetto di almeno un patogeno; quando nel modello è presente più di un patogeno,
uno di essi è quasi sempre la \emph{varroa}, che funge da vettore per gli altri (virus).\footcite{chen_review}
Altri articoli considerano infine le infestazioni da \emph{Nosema}, o virus diversi da ABPV.

\paragraph{}
Si osservi incidentalmente che alcuni modelli computazionali come \texttt{VARROAPOP} e \texttt{BEEHAVE} precedono
di parecchi anni i modelli matematici sopra citati, e già all'epoca consideravano fattori come infezioni,
temperature, precipitazioni, esposizione luminosa, distanza e qualità del cibo, e i loro effetti sulla colonia. %
\footcite{compMod1,compMod2,compMod3,compMod4}

\paragraph{}
Chi scrive non ha una opinione particolare circa la futura ricerca matematica in quest'area.
Più precisamente, non si ritiene eccesivamente importante \emph{quale} caratteristica aggiungere ad un nuovo
modello più complesso o quali tecniche di analisi esplorare.

Domandando ad un apicoltore o ad un'apicoltrice professionista, ci si sentirà rispondere che \emph{tutto} è
importante ai fini della sopravvivenza della colonia, soprattutto quando
le cose ``si mettono male''.\footcite{privFDL,privFPan}

Le \emph{performance} di una regina, la postazione esposta diversamente sullo stesso campo, \dots
ogni fattore interno ed esterno sembra importante (ed in fondo, lo è) per un organismo così complesso come un alveare.

\paragraph{}
Ben lungi dall'essere ignoranti delle questioni più all'avanguardia nell'ambito della ricerca scientifica,
le aziende di apicoltura sono tra i rari settori che ``collaborano spesso e volentieri'' con la ricerca accademica,
e tra di loro, per sviluppare ed applicare nuove conoscenze.

\paragraph{}
In conclusione, l'augurio di chi scrive per il futuro è un aumento nel \emph{dialogo} tra chi sviluppa modelli matematici
e chi fa ricerca in laboratorio; che si intensifichi la comunicazione tra chi le api le maneggia ogni giorno e chi
-- nelle discipline più varie -- ne fa oggetto di ricerca.



