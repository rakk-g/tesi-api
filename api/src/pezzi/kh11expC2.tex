\section{Experiment C, iteration 2}
Two iterations were made of this simulation with constant parameters $\alpha = 0.25$, $\sigma=0.75$, $L=2500$:
data \texttt{A} was obtained with $w=27000$, while the value $w=9800$ was used for data \texttt{B}.

In both experiments a random value for the parameter $m$ was obtained from a uniform distribution on $[0.006, \, 0.6]$
with a resolution of 30. For each $m$ we selected at random initial values $1 \leq H_0 \leq 15000$ and $1 \leq F_0 \leq 6000$ and ran 30 such simulations for a total of 900 days each.

\subsection{Data A}
\label{sec:kh11expC2A}
Note that with $w=27000$ the condition~\eqref{eq:cond6b} from Theorem~\ref{teo:esistenzPosF} is true: $\alpha - \frac{L}{w} > 0$, which is case 1 of the Theorem,
when a positive equilibrium exists for mortalities $m$ bound from above and from below:
$$ 0.0804 \simeq m_1^* < m < m_2^* \simeq 0.5078 \; .$$

In the remainder of this section, we will call ``weak'' those colonies which have final total
population $N(900) < N_- = 100$ bees. We will say that a colony is ``strong'' if
its final population is more than $N_+ = 40000$ workers.

Results are plotted in Figure~\ref{img:kh11expC2A}.

All colonies which eventually die off have a high rate of forager mortality: $m > m_2^*$.
Some of them don't collapse in 2.5 years and we don't call them ``weak'' in the sense above.

\paragraph{}
For mortalities above a reasonable value and below a fatal threshold, \ie for $m_1^* < m < m_2^*$,
we get steadily close to a positive equilibrium $N^*$: final workers number can get even into
the ``strong'' classification for hives with lower mortality rates, and become lower
(in the hundreds) near the $m_2^*$ end of the spectrum.

\paragraph{}
Outside the shown plot range were final populations in the unreasonable order of
magnitude of $10^6$ and beyond, clearly indicating a diverging behaviour, even with low values of $N(0)$.
All these points lie on the left of $m_1^* \simeq 0.0804$.\footnote{Cfr.~Interpretation at page~\pageref{par:interpretationCond6b}.}

\begin{figure}[pbh]
    \centering
    \includegraphics[keepaspectratio,width=0.98\textwidth]{img/kh11expC2A}

    \caption[Experiment C2A]{Experiment C2A: total number of worker bees $N(t)= H(t)+F(t)$ is shown at $t=0$ and $t=900$ for different levels of forager mortality $m$.

    With the terminology estabilished in page \pageref{sec:kh11expC2A}, we see that
    initial conditions of colonies that become ``weak'' (blue squares) lie all in
    the half--plane $m>0.5078$ (dotted vertical line).
    Similarly, initial conditions of colonies that eventually become ``strong''
    (red x's) all lie around the line $m=0.0804$ (dashed).

    In black we see the initial conditions (+) and final populations (triangles) for colonies that
    don't get to be ``strong'' nor ``weak'' at day $t=900$.

    Final population numbers for ``strong'' hives (purple circles) all happen in the left--hand side
    of the spectrum $0.0804 < m < 0.5078$.
    }

    \label{img:kh11expC2A}
\end{figure}

\subsection{Data B}
In this experiment we ran 900 simulations for 900 days each, within the same setting as above,
where we changed the value of the parameter $w$ only.

Note that in this case inequality~\eqref{eq:cond6b} is reversed, since
$$ \alpha - \frac{L}{w} = \frac{1}{4} - \frac{2500}{9800} < 0 \; .$$

Results are plotted in Figure~\ref{img:kh11expC2B}: for each simulation the fraction $\frac{N(900)}{N(0)}$ is plotted against its level of forager mortality.
This is a rough measure of performance of the hive, for it depends on the colony ability to secure
trophic and reproductive capabilities, within the limits of its given initial conditions.

\paragraph{}
We can see that initial population levels definitely increased in a span of 2.5 years for
almost all initial conditions. A slight decreasing trend in final population levels is
noted for higher mortality rates; still, these higher values for $m$ cannot keep the colony
population at reasonably stable levels and only lower the speed at which population is diverging.

\begin{figure}[pbh]
    \centering
    \includegraphics[keepaspectratio,width=0.98\textwidth]{img/kh11expC2B}

    \caption[Experiment C2B]{Experiment C2B: the ratio $\frac{N(900)}{N(0)}$ of final and initial
    total populations is plotted against $m$ in abscissa.}

    \label{img:kh11expC2B}
\end{figure}

