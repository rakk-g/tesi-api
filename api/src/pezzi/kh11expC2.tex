\section{Esperimento C, iterazione 2}
Il modello di riferimento è~\cite{khoury2011}, di cui sono state eseguite
due iterazioni di simulazione, svolte mantenendo costanti i parametri $\alpha = 0.25$, $\sigma=0.75$,
$L=2500$.
Impostando $w=27000$ si è ottenuto il dataset \texttt{A}, mentre col valore
$w=9800$ il risultato delle simulazioni è denominato \texttt{B}.

\paragraph{}
In entrambi gli esperimenti il valore della mortalità $m$ è estratto in maniera pseudocasuale dalla distribuzione
uniforme su $[0.006, \, 0.6]$
con una risoluzione di 30 passi.

Per ogni $m$ sono state selezionate condizioni iniziali pseudocasuali $1 \leq H_0 \leq 15000$ e
$1 \leq F_0 \leq 6000$ con distribuzione uniforme, quindi eseguito la simulazione corrispondente per $T=900$ giorni.

Per ogni tupla di parametri sono state selezionate 30 condizioni iniziali.

\subsection{Dataset A}
\label{sec:kh11expC2A}
Si noti che con $w=27000$ la condizione~\eqref{eq:cond6b} dalla Proposizione~\ref{teo:esistenzPosF} è vera:
$\alpha - \frac{L}{w} > 0$, che corrisponde al Caso 1 della Proposizione,
in cui l'equilibrio positivo esiste per valori di mortalità $m$ limitati da sopra e da sotto:
$$ 0.0804 \simeq m_1^* < m < m_2^* \simeq 0.5078 \; .$$

Nel resto di questa sezione, sarà utile definire ``deboli'' quelle colonie nella simulazione che hanno
una popolazione finale $N(900) < N_- = 100$.
Diremo che una colonia simulata è ``forte'' se la sua popolazione finale $N(900)$
è maggiore di $N_+ = 40000$ operaie.

I risultati della simulazione sono illustrati in figura~\ref{img:kh11expC2A}.
\begin{figure}[hbt]
    \centering
    \includegraphics[keepaspectratio,width=0.98\textwidth]{img/kh11expC2A}

    \caption[Esperimento C2A]{Viene mostrato il numero totale di operaie $N(t)= H(t)+F(t)$
    al tempo $t=0$ e $t=900$, per valori differenti di mortalità nel comparto bottinatrici $m$.

    Con la terminologia stabilita a p.~\pageref{sec:kh11expC2A}, vediamo che le condizioni iniziali
    per le colonie che diventano ``deboli'' (quadrati blu) giacciono tutte nel semipiano
    $m>0.5078$ (linea a punti).

    In maniera simile, condizioni iniziali per le colonie che alla fine della simulazione
    risultano ``forti'' (x rosse) giacciono tutte attorno alla soglia $m=0.0804$ (linea a tratti).
    Gli alveari con mortalità inferiore alla soglia divergono rapidamente entro i 900 giorni della
    simulazione, mentre i valori di $m$ alla destra $m_1^*$ la popolazione totale converge verso
    valori assai grandi, ma finiti.

    In nero vediamo le condizioni iniziali (+) e popolazioni finali (triangoli) per le colonie che
    non diventano né ``forti'' né ``deboli'' entro il giorno $t=900$.

    Le popolazioni finali per le colonie ``forti'' (punti viola) occorrono tutte nel lato sinistro
    dello spettro $0.0804 < m < 0.5078$.
    }

    \label{img:kh11expC2A}
\end{figure}

Tutte le colonie che prima o poi collassano presentano alti tassi di mortalità tra le bottinatrici: $m > m_2^*$.
Alcune di quelle che non muoiono nell'arco di 2 anni e mezzo, e che non chiamiamo ``deboli'' nel senso sopra,
potrebbero comunque finire per diventarlo avanzando la simulazione per altri mesi.

\paragraph{}
Per valori di mortalità\footnote{Cfr.~Interpretazione a p.~\pageref{par:interpretationCond6b}.}
entro una soglia fatale, cioè per $m_1^* < m < m_2^*$,
la popolazione delle colonie si avvicina rapidamente all'equilibrio positivo $P^*$.

Si può osservare che il numero finale di operaie rientra nella classificazione di ``forti'' per valori
bassi nello spettro delle mortalità, e che si abbassa (fino all'ordine delle centinaia) per $m$ vicino a
$m_2^*$.

\paragraph{}
Al di fuori della regione inserita nel grafico sono presenti popolazioni finali di irragionevoli
magnitudini intorno a $10^7$ ed oltre, chiaramente indicanti un comportamento divergente, perfino
con valori molto bassi di $N(0)$.

Tutti questi punti giacciono nel semipiano alla sinistra di
$m_1^* \simeq 0.0804$.


\subsection{Dataset B}
In questo esperimento sono state eseguite 900 simulazioni di 900 giorni ognuna, entro la configurazione di
cui sopra, in cui cambiamo esclusivamente il valore di $w$.

Si noti che in questo caso $w=9800$ la disequazione~\eqref{eq:cond6b} si inverte, dal momento che
$$ \alpha - \frac{L}{w} = \frac{1}{4} - \frac{2500}{9800} < 0 \; .$$

I risultati sono mostrati nel grafico in figura~\ref{img:kh11expC2B}.
\begin{figure}[hbt]
    \centering
    \includegraphics[keepaspectratio,width=0.98\textwidth]{img/kh11expC2B}

    \caption[Esperimento C2B]{Il rapporto $\frac{N(900)}{N(0)}$ tra le popolazioni totali iniziale e
    finale è mostrato nel grafico, contro $m$ in ascissa.}

    \label{img:kh11expC2B}
\end{figure}

\paragraph{}
Per ogni simulazione la frazione $\frac{N(900)}{N(0)}$ viene graficata contro il livello $m$ di mortalità
nel comparto foraggiatrici.

Ciò corrisponde a una misura piuttosto rudimentale delle prestazioni della colonia, perché dipende
dalla abilità dell'alveare di ``conquistare'' livelli trofici e di capacità riproduttiva che ne garantiscano
la sopravvivenza; entro i limiti dei parametri impostati e delle condizioni iniziali estratte.

\paragraph{}
Possiamo vedere che i livelli di popolazione aumentano decisamente nell'arco di 2 anni e mezzo, per
quasi ogni condizione iniziale. Una leggera tendenza alla decrescita nei livelli finali di popolazione
si può notare per tassi più elevati di mortalità.

Ad ogni modo, anche i valori alti di $m$ non contengono
la popolazione della colonia entro livelli ragionevolmente stabili, e rallentano soltanto la rapidità
con cui la popolazione diverge.



