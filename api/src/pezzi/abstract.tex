\section*{Sommario}
Il ruolo fondamentale di \emph{A.~mellifera} per la specie umana è minacciato dagli antagonisti naturali della
specie, a cui si aggiungono le cause antropiche quali inquinamento, cambiamento climatico, pesticidi.

\paragraph{}
La modellazione matematica di popolazioni di \emph{A.~mellifera}, soprattutto quando incorpora i patogeni ed
altre cause di stress per le api, può rappresentare efficacemente la risposta delle colonie nelle condizioni
reali di campo.
Alcuni modelli più recenti incorporano relazioni complesse come quelle tra api e virus, questi
ultimi trasportati dal parassita \emph{Varroa}, e l'effetto delle tecniche di disinfestazione
impiegate in apicoltura.

In questo lavoro si studiano approfonditamente due modelli (uno con sole api, l'altro con virus e varroe)
e si raffinano le condizioni di stabilità fornite dagli autori. Viene offerta inoltre una interpretazione
pratica di un termine parametrico che compare nelle equazioni, la quale permette di scartare alcuni scenari
irrealistici, superando così alcune limitazioni intrinseche del modello.

Per corroborare i commenti ai risultati teorici ottenuti, sono state svolte nuove simulazioni numeriche al calcolatore.

\vspace{1cm}
\section*{Abstract}
The fundamental role of \emph{A.~mellifera} for the human species is threatened by the natural antagonists of
the species, in addition to anthropogenic causes such as pollution, climate change, and pesticides.

\paragraph{}
The mathematical modeling of \emph{A.~mellifera} populations, especially when incorporating pathogens and
other stress factors affecting bees, can effectively represent colony responses under real field conditions.
Some of the more recent models include complex relationships, such as those between bees and viruses—
the latter being carried by the parasite \emph{Varroa}—and the effects of pest control techniques used in beekeeping.

In this work, we thoroughly study two models (one considering only bees, the other including viruses and Varroa mites)
and refine the stability conditions provided by the authors. Additionally, we offer a practical interpretation
of a parametric term appearing in the equations, which allows us to discard some unrealistic scenarios,
thus overcoming certain intrinsic limitations of the model.

To support the discussion of the theoretical results obtained, new numerical simulations have been carried out using a computer.
