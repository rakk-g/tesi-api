\section*{Sommario}
Il ruolo fondamentale di \emph{A.~mellifera} per la specie umana è minacciato dagli antagonisti naturali della
specie, a cui si aggiungono le cause antropiche quali inquinamento, cambiamento climatico, pesticidi.

Dinamiche simili a quelle delle api hanno portato, negli scorsi anni, ad una crisi generalizzata di
moltissime altre specie di impollinatori, con gravi conseguenze sull'agricoltura e perciò sulla disponibilità di cibo
e sulle economie globali.

\paragraph{}
La modellazione matematica di popolazioni di \emph{A.~mellifera}, soprattutto quando incorpora i patogeni ed
altre cause di stress per le api, può rappresentare efficacemente la risposta delle colonie nelle condizioni
reali di campo.
Alcuni modelli più recenti incorporano relazioni complesse come quelle tra api e virus, questi
ultimi trasportati dal famigerato parassita \emph{Varroa}, e l'effetto delle tecniche di disinfestazione
impiegate in apicoltura.

% Le api si trovano all'intersezione di un vasto gruppo di interessi e di collaborazioni: da parte delle aziende,
% delle associazioni di categoria, della ricerca nei campi della biologia, dell'epidemiologia,
% delle scienze cognitive, dell'informatica, della matematica e molte altre.
%
% Il già fertile dialogo tra i molti campi può migliorare ed essere più efficace, ad es. nell'informare il
% soggetto legislatore.

\vspace{1cm}
\section*{Abstract}
The fundamental role of \emph{A.~mellifera} for the human species is threatened by the natural antagonists of
the species, in addition to anthropogenic causes such as pollution, climate change, and pesticides.

Similar dynamics to those affecting honeybees have led, in recent years, to a widespread crisis
among many other pollinator species,
with severe consequences for agriculture and -- consequently -- for food availability and global economies.

\paragraph{}
The mathematical modeling of \emph{A.~mellifera} populations, especially when incorporating pathogens and
other stress factors affecting bees, can effectively represent colony responses under real field conditions.
Some of the more recent models include complex relationships, such as those between bees and viruses -
the latter being carried by the notorious parasite \emph{Varroa} - and the effects of pest control techniques
used in beekeeping.

% Bees are at the intersection of a vast network of interests and collaborations: from companies, professional associations,
% and research in fields such as biology, epidemiology, cognitive sciences, computer science, mathematics, and many others.
%
% The already fertile dialogue among these fields can be improved and made more effective, for instance, in
% informing policymakers.
