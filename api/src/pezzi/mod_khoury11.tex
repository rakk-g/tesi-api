\section[Modello a due compartimenti con reclutamento]{Modello a due compartimenti con reclutamento sociale}

Nel \citeyear{khoury2011} è stato proposto\footcite{khoury2011} un modello per indagare le dinamiche sociali
tra le caste della popolazione operaia in \emph{Apis~mellifera}.

Nello studio si ottengono due equazioni, in cui le variabili reali e non negative $H$ ed $F$ rappresentano
le operaie di nido e le foraggiatrici, rispettivamente:

\begin{align}
    \diff{H}{t} &= \overbrace{E(H,F)}^{\text{eclosion}}- H \cdot R(H,F) \label{eq:kh11h} \\
    \diff{F}{t} &= H \cdot R(H,F)  - m \cdot F \label{eq:kh11f}
\end{align}
dove $m$ è il tasso di mortalità, mentre $R$ rappresenta il fattore di \emph{reclutamento} nel termine di
trasferimento $H \to F$.

Alcuni oggetti proposti in questo modello si sono dimostrati molto interessanti e sono stati incorporati
in diversi modelli successivi, con variazioni minime.\footcite{ratti2017}

\paragraph{Reclutamento con inibizione sociale.}
Nella funzione di reclutamento
\begin{equation}
    \label{eq:Recr}
    R(H,F) = \alpha - \sigma \frac{F}{H+F} \; ,
\end{equation}
il massimo tasso di reclutamento $\alpha$ si ha quando non c'è nessuna foraggiatrice nella colonia, la quale
dunque risponde forzando il reclutamento precoce delle operaie di nido.

Il secondo termine rappresenta l'inibizione sociale del processo di reclutamento; gli autori assumono
\textquote[{\footcite[3]{khoury2011}}]{\omissis that social inhibition is directly proportional to the fraction of the total number of adult bees that are foragers, such that a high fraction of foragers in the hive results in low recruitment.}

\paragraph{}
La ricerca -- che interseca matematica, biologia e molte altre discipline -- suggerisce che
l'ethyl~oleate\footnote{CAS 111-62-6.} % SURE? TODO
scambiato tra le api operaie sia il principale modulatore nel processo di reclutamento sociale.

Ad esempio una colonia risponderà ad una elevata mortalità di bottinatrici (più anziane) attraverso
la riduzione dell'inibizione del reclutamento (dati i livelli pù bassi di feromone) che dunque
\textquote[\footcite{khoury2011}]{resulting in a precocious onset of foraging behaviour in young bees}.

Nella situazione opposta -- \ie quando vi sono molte bottinatrici e la loro mortalità è bassa -- la colonia
risponde \emph{aumentando} l'inibizione sociale: un maggior numero di bottinatrici si astiene dal volo e
dalle attività all'esterno, ripiegando sui lavori di cura all'interno del nido. Questa risposta inibitoria
è probabilmente dovuta alla maggiore concentrazione di ethyl~oleate:
\textquote[{\footcite[1]{khoury2011}}]{Old forager bees transfer ethyl oleate to young hive bees via trophallaxis, which delays the age at which they begin foraging.}

\paragraph{}
Tramite questo meccanismo mediato dai segnali feromonici, l'alveare tenta di mantenere un equilibrio
demografico tra le due classi sufficiente a soddisfare le necessità della covata e dell'importazione
di polline, nettare, acqua.

In questo scenario la risposta collettiva -- che riduce il tasso di reclutamento -- aumenta con
l'intensità degli scambi di ethyl oleate, che a loro volta sono correlati al rapporto $\frac{F}{H+F}$:
queste assunzioni motivano la presenza di tale fattore nell'Equazione~\eqref{eq:Recr}.


\paragraph{Tassi di mortalità.}
As noted before, foragers are subjected to more diverse and intense stress factors than hive bees, whence the average lifespan of a female honeybee is longer as a hive worker, and is significantly shorter as a forager: \textquote[{\footcite[1]{khoury2011}}]{Survival of bees in the protected hive environment is high, but the survival of forager bees is much lower. \omissis The average foraging life of a bee has been estimated as less than seven days, because of the many risks and severe metabolic costs associated with foraging.}

The sink term $-mF$ in Equation~\eqref{eq:kh11f} accounts for foragers mortality;
the parameter $m$ sets the fraction of forager population that dies every day.

\paragraph{}
The absence of a mortality factor in the hive compartment (Equation~\eqref{eq:kh11h}) is a bold simplifying assumption,
justified by the focus of this paper in relating CCD to higher than normal forager mortality rates.\footcite[1,2,3,5]{khoury2011}


\paragraph{Eclosion.}
The term $E(H,F)$ accounts for eclosion of adult young workers in the hive class.
\textquote[{\footcite[2]{khoury2011}}]{\omissis the number of eggs reared in a colony (and hence the eclosion rate) is related to the number of bees in the hive. Big colonies raise more brood.}

The authors have chosen the form
\begin{equation}
    \label{eq:eclos}
    E(H,F) = L \frac{H+F}{w + H + F} \; ,
\end{equation}
where $L$ is the queen's laying rate and $w$ determines the rate at which $E$ increases to $L$ as $E+F$ gets very large.
\footnote{In this section we will often use the shorthand $N=H+F$, the total number of workers present in the colony at a given time.}

This is a Holling type II function where the attack rate is $a=\frac{1}{w}$ and $w$ is the reverse image of the half--maximal output; in \figurename~\ref{img:eclos} we can see that increasing values of $w$ ``flatten'' the curve
as saturation is attained for greater values of $N$, and the graph of $E(N)$ ``leans'' to the right.

\begin{figure}[pbh]
    \centering
    \includegraphics[keepaspectratio,width=0.86\textwidth]{img/hollingTypeII-eclosion}

    \caption[Holling type II Eclosion]{$E$ models how many of the laid eggs eventually reach adult stage as a functional response of the total number of adult workers $N=H+F$.
    \\
    Drawn for three different values of $w$, with constant laying rate $L=2000$.}
    \label{img:eclos}
\end{figure}

Later models\footcite{ratti2017} incorporated this basic proposal with type III Holling functional response and added modulating factors, but retained the core formulation very similar to what~\citeauthor{khoury2011} proposed in Equation~\refeq{eq:eclos}.


\subsection{Stability analysis}
We now carefully review the results given in the paper about the existence and stability of equilibria in the model \eqref{eq:kh11h}--\eqref{eq:kh11f}.

The authors used \textquote[{\cite[3]{khoury2011}}]{standard linear stability analysis and phase plane analysis} to examine the model equations.
Re--tracing their steps, we will look for criteria for the existence of trivial and non--trivial equilibria, \ie zeroes of the right--hand side of our model equations.

When existence of an equilibrium point is estabilished, the jacobian matrix of the dynamical system is
evaluated at that point; then a simple sufficient criterion for asymptotic
stability\footnote{See Theorem~\ref{teo:pozzoNonLineare}.}
is that $\Re (\lambda) < 0$ for each eigenvalue $\lambda$.
Another simple technique due to Lyapounov is to exhibit a function that has a strong local minimum
over the equilibrium point.\footnote{See Theorem~\ref{teo:lyapounovFunc}.}

\paragraph{}
Note that $H, E$ have non--negative range, \ie $H(t), E(t) \geq 0$ for all $t$.
Also, all parameters in the model have to be strictly positive.

The next result states a necessary condition for all equilibria.

\begin{lemma}
    \label{lem:necessJ}
    All equilibrium points $(H^*, F^*)$, if they exist, have to verify $F^* = J H^*$, where
    \begin{equation}
        J \coloneq \frac{1}{2} \left[
            \frac{\alpha}{m} - \frac{\sigma}{m} - 1 +
            \sqrt{ {\left( \frac{\alpha}{m} - \frac{\sigma}{m} - 1 \right)}^2
                + 4 \frac{\alpha}{m}
            }
        \right].
        \label{eq:kh11posEqJ}
    \end{equation}
\end{lemma}

That is to say, equilibria lie on the ray $F= JH$ in the phase plane, when they exist.

\paragraph{}
We calculated the jacobian matrix of this system as:
\begin{equation}
    \begin{pmatrix}
        L \frac{w}{{(w + H + F)}^2} - \alpha + \sigma \frac{F^2}{{(H+F)}^2} &
        L \frac{w}{{(w + H + F)}^2} + \sigma \frac{H^2}{{(H+F)}^2}
        \\
        \alpha - \sigma \frac{F^2}{{(H+F)}^2} &
        - \sigma \frac{H^2}{{(H+F)}^2} - m
    \end{pmatrix}
    \label{eq:jacoGeneral}
\end{equation}

It is interesting to note that the last row of the jacobian is constant under
condition~\eqref{eq:kh11posEqJ},
\ie substituting $H$ for $\frac{1}{J} F$ in Equation~\eqref{eq:jacoGeneral},
we get:
\begin{gather}
    \diffp*{ \dot{H} }{H}{H= \frac{1}{J} F} =
    L \frac{w}{ {\left[ w + \left( \frac{1}{J} +1 \right) F \right]}^2 } - \alpha
    + \frac{ \sigma }{ {\left( \frac{1}{J} + 1 \right)}^2 } \; ,
    \\
    \diffp*{ \dot{H} }{F}{H= \frac{1}{J} F} =
    L \frac{w}{ {\left[ w + \left( \frac{1}{J} +1 \right) F \right]}^2 }
    + \frac{ \sigma }{ {\left( J + 1 \right)}^2 } \; ,
    \\
    \diffp*{ \dot{F} }{H}{H= \frac{1}{J} F} =
    \alpha - \frac{ \sigma }{ {\left( \frac{1}{J} + 1 \right)}^2 } \; ,
    \\
    \diffp*{ \dot{F} }{F}{H= \frac{1}{J} F} =
    - \frac{ \sigma }{ {\left( J + 1 \right)}^2 } - m \; ,
    \label{eq:jacoOnJFH}
\end{gather}
the last two equations depend only on the parameters and not on the variables H and F.

\subsubsection{Trivial equilibrium}
The trivial $(0,0)$ equilibrium is clearly a solution\footnote{Hence, the trivial equilibrium always exists.}
of $\dot{H}=\dot{F}=0$, and its stability is an important matter of study. Parameter regions where this equilibrium is stable and attractive usually mean the inevitable doom of the entire colony.

The system jacobian evaluated at the origin is:
\begin{equation}
    D_{00} \coloneq
    \begin{pmatrix}
        \frac{L}{w} -\alpha &
        \frac{L}{w}
        \\
        \alpha & -m
    \end{pmatrix}
    \label{eq:jacoEquZero}
\end{equation}

By Theorem~\ref{teo:pozzoNonLineare}, we can see why
condition~\ref{eq:cond6b}
$$\alpha -\frac{L}{w}>0$$
is useful for characterization of stability in the origin.\footcite[3]{khoury2011}

Under that hypothesis,
$$\det  D_{00} > 0 \quad \iff \quad
m > \frac{\alpha L}{w} {\left( \alpha - \frac{L}{w} \right)}^{-1}$$
implies
$$\tr D_{00} = - \left( \alpha - \frac{L}{w} \right) - m < 0$$
then, positive determinant and negative trace imply stability of this equilibrium.


\subsubsection{Positive equilibrium}
We now investigate in detail the existence and stability of positive equilibria; such points should represent a positive outcome for a surviving colony.

\paragraph{}
A couple of results is presented in order to build a criterion for the existence of a positive equilibrium point.

\begin{teorema}[Uniqueness of the positive equilibrium]
    If the model \refeq{eq:kh11h}--\refeq{eq:kh11f} admits a positive equilibrium $(H^*, F^*)$, then it is unique.

    It is determined by both Eq.~\refeq{eq:kh11posEqJ} and the following:
    \begin{equation}
        F^* = \frac{L}{m} - w \frac{J}{1+J} \; .
        \label{eq:kh11posEqF}
    \end{equation}

    \label{teo:exUniqFstarPos}
\end{teorema}

This model always admits one or two distinct equilibrium points, based on the next result:

\begin{teorema}[Existence criterion]
    A necessary and sufficient condition for the model \refeq{eq:kh11h}--\refeq{eq:kh11f} to have a strictly positive equilibrium $P^* \coloneq (H^*, F^*)$ is given by cases:
    \begin{itemize}
        \item[1.] If $\alpha - \frac{L}{w} >0$, then $P^*$ exists $\iff m_1^* < m < m_2^*$;
        \item[2.] If $\alpha - \frac{L}{w} <0$, then $P^*$ exists $\iff m > m_2^*$;
        \item[3.] If $\alpha - \frac{L}{w} =0$, then $P^*$ exists $\iff m > m_3^*$;
    \end{itemize}
    where
    \begin{equation}
        m_3^* = \frac{ \sigma L}{ w (\alpha + \sigma) } \; ,
        \label{eq:FstarPosM3}
    \end{equation}
    and $m_1^*, m_2^*$ are taken always in the order $m_1 < m_2$ from the following
    \begin{equation}
        m_{1,2} = \frac{L}{2w} \frac{ \alpha + \sigma \pm \sqrt{ {(\alpha - \sigma)}^2 +\frac{4 \sigma L}{w}  } }
        {\alpha -\frac{L}{w}} \; .
        \label{eq:FstarPosM12}
    \end{equation}

    \label{teo:esistenzPosF}
\end{teorema}

\paragraph{Interpretation.}
\label{par:interpretationCond6b}
We note that all situations where $m$ is given a lower bound can be related to intrinsic simplifications in the model, and cannot describe accurately a hive in a real apiary on a real field; the non--existence of a positive equilibrium when $m< \hat{m}$ for some $\hat{m}$ is simply because the total population diverges to infinity in such cases.

Conversely, the condition
\begin{equation}
\alpha - \frac{L}{w} > 0
    \label{eq:cond6b}
\end{equation}
can be read easily as $w \alpha >L$, where both sides have unit of bees/day; note also that the right--hand one captures the ``efficiency'' of the social recruitment process in the hive, while the left--hand side is the ``egg--laying efficiency'' of the queen.

That is to say, Equation~\eqref{eq:cond6b} means that the recruitment process $R= R(H, F)$ is capable of sustaining the level $L$ of egg--laying in the hive, by balancing the need for bees in the $H$ compartment (for brood care) and the need for bees in the $F$ compartment (for food intake).

\paragraph{}
Conditions in the form $m< \hat{m}$ typically ensure the survival of the colony;
in such parameter regions, this model can also exhibit the non--existence of a
positive equilibrium, simply because of the divergence of total population numbers.

As noted before, this model doesn't have a mortality factor for the $H$ compartment: in light of this fact, we
can explain condition~\eqref{eq:cond6b} as it as a lower bound for a steady transfer of hive bees to the
forager compartment, in which they have a finite life expectancy.

In the asymptotic limit, $\frac{L}{w} H$ bees enter the hive compartment through eclosion and $\alpha H$ bees
are withdrawn to the $F$ compartment: condition~\eqref{eq:cond6b} then says that the maximal recruitment rate
must be sufficiently large to unload the excess hive bees towards the foragers compartment.
%  moar interpr? TODO
% cfr. condiz per 0

\paragraph{}
Branching of equilibria in this model is illustrated in \figurename~\ref{img:kh11phasePlane}.

\begin{figure}[pbh]
    \centering
    \includegraphics[keepaspectratio,width=0.86\textwidth]{img/pone.0018491.g003}

    \caption[\figurename~3 from \parencite{khoury2011}]{
%         \figurename~3 from \parencite[3]{khoury2011}
        \textquote[{\figurename~3 from \cite[3]{khoury2011}}]{
            Phase plane diagrams of solutions to the model for different values of m.
            Each line on the diagrams represents a solution trajectory, giving the number of foragers $F$ and the number of hive bees $H$.
            As time $t$ increases the solutions change along the trajectory in the direction of the arrows.
            In (a) $m = 0.24$ and the populations tend to a stable equilibrium population, marked by a dot.
            In (b), $m = 0.40$ there is no nonzero equilibrium and the hive populations collapses to zero.

            Parameter values are $L = 2000$,  $\alpha= 0.25$,  $\sigma= 0.75$ and $w = 27 000$.}
    }
    \label{img:kh11phasePlane}
\end{figure}

