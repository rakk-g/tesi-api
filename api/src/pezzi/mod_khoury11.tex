\section[Modello a due compartimenti con reclutamento]{Modello a due compartimenti con reclutamento sociale}

Nel \citeyear{khoury2011} è stato proposto\footcite{khoury2011} un modello per indagare le dinamiche sociali
tra le caste della popolazione operaia in \emph{Apis~mellifera}.

Nello studio si ottengono due equazioni, in cui le variabili reali e non negative $H$ ed $F$ rappresentano
le operaie di nido e le foraggiatrici, rispettivamente:

\begin{align}
    \diff{H}{t} &= \overbrace{E(H,F)}^{\text{eclosion}}- H \cdot R(H,F) \label{eq:kh11h} \\
    \diff{F}{t} &= H \cdot R(H,F)  - m \cdot F \label{eq:kh11f}
\end{align}
dove $m$ è il tasso di mortalità, mentre $R$ rappresenta il fattore di \emph{reclutamento} nel termine di
trasferimento $H \to F$.

Alcuni oggetti proposti in questo modello si sono dimostrati molto interessanti e sono stati incorporati
in diversi modelli successivi, con variazioni minime.\footcite{ratti2017}

\subsection{Reclutamento con inibizione sociale.}
Nella funzione di reclutamento
\begin{equation}
    \label{eq:Recr}
    R(H,F) = \alpha - \sigma \frac{F}{H+F} \; ,
\end{equation}
il massimo tasso di reclutamento $\alpha$ si ha quando non c'è nessuna foraggiatrice nella colonia, che
dunque risponde forzando il reclutamento precoce delle operaie di nido.

Il secondo termine rappresenta l'inibizione sociale del processo di reclutamento; gli autori assumono
\textquote[{\footcite[3]{khoury2011}}]{\omissis that social inhibition is directly proportional to the fraction of the total number of adult bees that are foragers, such that a high fraction of foragers in the hive results in low recruitment.}

\paragraph{}
La ricerca -- che interseca matematica, biologia e molte altre discipline -- suggerisce che
l'ethyl~oleate\footnote{CAS 111-62-6.} % SURE? TODO
scambiato tra le api operaie sia il principale modulatore nel processo di reclutamento sociale.

Ad esempio una colonia risponderà ad una elevata mortalità di bottinatrici (più anziane) attraverso
la riduzione dell'inibizione del reclutamento (dati i livelli più bassi di feromone) che dunque risulta
\textquote[\footcite{khoury2011}]{\omissis in a precocious onset of foraging behaviour in young bees}.

Nella situazione opposta -- \ie quando vi sono molte bottinatrici e la loro mortalità è bassa -- la colonia
risponde \emph{aumentando} l'inibizione sociale: un maggior numero di bottinatrici si astiene dunque dal volo e
dalle attività all'esterno, ripiegando sui lavori di cura all'interno del nido. Questa risposta inibitoria
è probabilmente dovuta alla maggiore concentrazione di ethyl~oleate:
\textquote[{\footcite[1]{khoury2011}}]{Old forager bees transfer ethyl oleate to young hive bees via trophallaxis, which delays the age at which they begin foraging.}

\paragraph{}
Tramite questo meccanismo mediato dai segnali feromonici, l'alveare tenta di mantenere un equilibrio
demografico tra le due classi, sufficiente a soddisfare le necessità della covata e dell'importazione
di polline, nettare, acqua.

In questo scenario la risposta collettiva -- che riduce il tasso di reclutamento -- aumenta con
l'intensità degli scambi di ethyl oleate, che a loro volta sono correlati al rapporto $\frac{F}{H+F}$:
queste assunzioni motivano la presenza di tale fattore nell'equazione~\eqref{eq:Recr}.


\paragraph{Tassi di mortalità.}
Come si è notato in precedenza,\footnote{}
le operaie adulte -- esploratrici e bottinatrici -- sono soggette ad una gamma più ampia di fattori di stress
nell'ambiente esterno, rispetto alle operaie di nido. Di conseguenza l'aspettativa di vita per le operaie di
nido è più lunga rispetto quella delle foraggiatrici:
\textquote[{\footcite[1]{khoury2011}}]{Survival of bees in the protected hive environment is high, but the
survival of forager bees is much lower. \omissis The average foraging life of a bee has been estimated as less
than seven days, because of the many risks and severe metabolic costs associated with foraging.}

Il termine di pozzo $-mF$ nell'equazione~\eqref{eq:kh11f} rappresenta la mortalità delle foraggiatrici;
il tasso parametrico $m$ imposta la frazione del comparto $F$ che muore ogni giorno.

\paragraph{}
L'assenza di un fattore di mortalità nel comparto delle operaie di nido nell'equazione~\eqref{eq:kh11h} è
una assunzione piuttosto severa, e comporta una netta semplificazione del modello.

Tuttavia, essa è giustificata dall'obiettivo principale dell'articolo, ossia modellare la relazione tra le varie
forme di collasso nella popolazione di colonie di \emph{A.~mellifera} e gli eccezionali tassi di mortalità
tra le foraggiatrici.
\footcite[1,2,3,5]{khoury2011}

\subsection{Tasso di schiusa.}
Il termine $E(H,F)$ tiene conto della schiusa della covata che emerge come giovani operaie adulte nel
comparto $H$ delle api di nido.
\blockquote[{\footcite[2]{khoury2011}}]{\omissis the number of eggs reared in a colony (and hence the eclosion rate) is related to the number of bees in the hive. Big colonies raise more brood.}

Come ricordato nel capitolo~\ref{chap:bio}, il sostentamento di una popolazione numerosa dipende non soltanto
dal tasso di deposizione della regina, ma anche dal numero di operaie giovani che rimangono nel nido per
accudire uova e larve, al fine di massimizzare il numero di esse che giunge sana al termine di tutti gli stadi
della metamorfosi.

\paragraph{}
Gli autori scelgono la formulazione
\begin{equation}
    \label{eq:eclos}
    E(H,F) = L \frac{H+F}{w + H + F} \; ,
\end{equation}
ove $L$ rappresenta il tasso di deposizione della regina e $w$ è la rapidità con cui $E$ cresce verso $L$ quando
la popolazione totale $H+F$ aumenta.\footnote{In questa sezione useremo spesso l'abbreviazione
$N=H+F$ per il numero totale di operaie, di nido e di campo, presenti nella colonia ad un dato istante $t$.}

Questa è una funzione Holling di tipo II, in cui il tasso di attacco è $a=\frac{1}{w}$ e $w$
è la retroimmagine del semimassimo nella risposta. In figura~\ref{img:eclos} possiamo vedere come valori
crescenti di $w$ ``schiacciano'' la curva mentre la saturazione è raggiunta per valori sempre maggiori di $N$,
e il grafico di $E(N)$ si ``inclina'' verso la destra.

\begin{figure}[pbh]
    \centering
    \includegraphics[keepaspectratio,width=0.86\textwidth]{img/hollingTypeII-eclosion}

    \caption[Schiusa, Holling tipo II]{$E$ rappresenta la quantità di uova deposte dalla regina che raggiunge lo
    stadio adulto, come risposta funzionale al numero totale di operaie $N=H+F$.
    \\
    Sono illustrati tre diversi valori di $w$, mantenendo costante il tasso di deposizione $L=2000$.}
    \label{img:eclos}
\end{figure}

Modelli successivi\footcite{ratti2017} hanno incorporato questa proposta, modificandola in una funzione Holling
di tipo III e aggiungendo certi fattori di modulazione ulteriori, ma tutto sommato hanno mantenuto il nucleo
della formulazione molto simile a quanto~\citeauthor{khoury2011} propongono nell'equazione~\refeq{eq:eclos}.


\subsection{Analisi della stabilità}
Rivediamo adesso con cautela i risultati forniti nell'articolo, a proposito dell'esistenza e della stabilità
dei punti di equilibrio nel modello~\eqref{eq:kh11h}--\eqref{eq:kh11f}.

Gli autori hanno utilizzato \textquote[{\footcite[3]{khoury2011}}]{standard linear stability analysis and phase
plane analysis} per esaminare le equazioni del modello.

Ripercorrendo i loro passi, forniremo criteri per l'esistenza di equilibri banali e non, \ie zeri del membro
destro delle equazioni del modello.

Una volta stabilita l'esistenza di un punto di equilibrio, calcoliamo la matrice jacobiana del sistema e
la valutiamo sull'equilibrio stesso; un semplice criterio sufficiente per la stabilità asintotica
\footnote{Vedasi il Teorema~\ref{teo:pozzoNonLineare}.}
è che $\Re (\lambda) < 0$ per ogni autovalore $\lambda$ della matrice.

Un'altra semplice tecnica, dovuta a Lyapounov, consiste nell'esibire una funzione che abbia un minimo
locale forte sul punto di equilibrio e che sia localmente debolmente decrescente lungo le
soluzioni.\footnote{Vedasi il Teorema~\ref{teo:lyapounovFunc}.}

\paragraph{}
Si noti che $H, E$ hanno immagine non--negativa, \ie $H(t), E(t) \geq 0$ per ogni $t$.
Inoltre, tutti i parametri del modello debbono essere strettamente positivi.

Il prossimo risultato stabilisce una condizione necessaria per tutti i punti d'equilibrio.

\begin{lemma}
    \label{lem:necessJ}
    Tutti i punti di equilibrio $(H^*, F^*)$
    del modello~\eqref{eq:kh11h}--\eqref{eq:kh11f} verificano $F^* = J H^*$, dove
    \begin{equation}
        J \coloneq \frac{1}{2} \left[
            \frac{\alpha}{m} - \frac{\sigma}{m} - 1 +
            \sqrt{ {\left( \frac{\alpha}{m} - \frac{\sigma}{m} - 1 \right)}^2
                + 4 \frac{\alpha}{m}
            }
        \right].
        \label{eq:kh11posEqJ}
    \end{equation}
\end{lemma}

Vale a dire, i punti di equilibrio giacciono sulla semiretta $F= JH$ nello spazio delle fasi, quando esistono.

\paragraph{}
Calcoliamo la matrice jacobiana del sistema:
\begin{equation}
    \begin{pmatrix}
        L \frac{w}{{(w + H + F)}^2} - \alpha + \sigma \frac{F^2}{{(H+F)}^2} &
        L \frac{w}{{(w + H + F)}^2} + \sigma \frac{H^2}{{(H+F)}^2}
        \\
        \alpha - \sigma \frac{F^2}{{(H+F)}^2} &
        - \sigma \frac{H^2}{{(H+F)}^2} - m
    \end{pmatrix}
    \label{eq:jacoGeneral}
\end{equation}

È interessante notare che la seconda riga della jacobiana diventa costante sotto la
condizione~\eqref{eq:kh11posEqJ}, infatti sostituendo $H$ con $\frac{1}{J} F$
nell'equazione~\eqref{eq:jacoGeneral}, otteniamo
\begin{gather}
    \diffp*{ \dot{H} }{H}{H= \frac{1}{J} F} =
    L \frac{w}{ {\left[ w + \left( \frac{1}{J} +1 \right) F \right]}^2 } - \alpha
    + \frac{ \sigma }{ {\left( \frac{1}{J} + 1 \right)}^2 } \; ,
    \\
    \diffp*{ \dot{H} }{F}{H= \frac{1}{J} F} =
    L \frac{w}{ {\left[ w + \left( \frac{1}{J} +1 \right) F \right]}^2 }
    + \frac{ \sigma }{ {\left( J + 1 \right)}^2 } \; ,
    \\
    \diffp*{ \dot{F} }{H}{H= \frac{1}{J} F} =
    \alpha - \frac{ \sigma }{ {\left( \frac{1}{J} + 1 \right)}^2 } \; ,
    \\
    \diffp*{ \dot{F} }{F}{H= \frac{1}{J} F} =
    - \frac{ \sigma }{ {\left( J + 1 \right)}^2 } - m \; ,
    \label{eq:jacoOnJFH}
\end{gather}
e le ultime due equazioni dipendono soltanto dai parametri, non dalle variabili $H$ ed $F$.

\subsubsection{Equilibrio banale}
L'equilibrio banale $(0,0)$ è chiaramente sempre soluzione\footnote{Dunque il sistema ammette
\emph{sempre} l'equilibrio banale.}
di $\dot{H}=\dot{F}=0$, e lo studio della sua stabilità è cruciale.
Nelle regioni di parametri per cui questo equilibrio è stabile e attrattivo generalmente si ha il declino
inevitabile dell'intera colonia.

La matrice jacobiana del modello valutata nell'origine è:
\begin{equation}
    D_{00} \coloneq
    \begin{pmatrix}
        \frac{L}{w} -\alpha &
        \frac{L}{w}
        \\
        \alpha & -m
    \end{pmatrix}
    \label{eq:jacoEquZero}
\end{equation}

Con il Teorema del pozzo non lineare\footnote{Cfr. p.~\pageref{teo:pozzoNonLineare}}
possiamo vedere il motivo per cui la condizione~\ref{eq:cond6b}
$\alpha -\frac{L}{w}>0$
sia utile per caratterizzare la stabilità dell'origine.\footcite[3]{khoury2011}

Sotto tale ipotesi,
$$\det  D_{00} > 0 \quad \iff \quad
m > \frac{\alpha L}{w} {\left( \alpha - \frac{L}{w} \right)}^{-1}$$
implica che
$$\tr D_{00} = - \left( \alpha - \frac{L}{w} \right) - m < 0$$
dunque abbiamo determinante positivo e traccia negativa, da cui segue che lo spettro di $D_{00}$ è
strettamente negativo. Riassumiamo quanto appena trovato nel

\begin{lemma}[Condizione sufficiente per la stabilità dell'equilibrio banale]
    Se $\alpha - \frac{L}{w} > 0$ e
    $$m > \frac{\alpha L}{w \alpha -L} \; ,$$
    allora l'equilibrio banale del modello \eqref{eq:kh11h}--\eqref{eq:kh11f} è asintoticamente stabile.
\end{lemma}


\subsubsection{Equilibrio positivo}
Indaghiamo adesso nel dettaglio le condizioni per l'esistenza e la stabilità di un equilibrio non banale, ossia
con popolazione strettamente positiva. Tali punti dovrebbero rappresentare l'esito favorevole di una colonia che
sopravvive alle avversità.

\paragraph{}
Presentiamo una successione di risultati più semplici, al fine di costituire un criterio per l'esistenza di
un punto d'equilibrio positivo.

\begin{proposizione}[Unicità dell'equilibrio positivo]
    Se il modello~\refeq{eq:kh11h}--\refeq{eq:kh11f} ammette un equilibrio positivo $(H^*, F^*)$, allora
    questi è unico.

    Esso è determinato dall'equazione~\refeq{eq:kh11posEqJ}, insieme con la seguente:
    \begin{equation}
        F^* = \frac{L}{m} - w \frac{J}{1+J} \; .
        \label{eq:kh11posEqF}
    \end{equation}

    \label{teo:exUniqFstarPos} % WARNING prima era teorema, ho cambiato idea dopo.
\end{proposizione}

Questo modello dunque ammette sempre un solo equilibrio o due equilibri distinti, in base alla seguente

\begin{proposizione}[Criterio di esistenza]
    Una condizione necessaria e sufficiente affinché il modello~\refeq{eq:kh11h}--\refeq{eq:kh11f} abbia
    un equilibrio strettamente positivo $P^* \coloneq (H^*, F^*)$ è data per casi:
    \begin{itemize}
        \item[1.] se $\alpha - \frac{L}{w} >0$, allora $P^*$ esiste $\iff m_1^* < m < m_2^*$;
        \item[2.] se $\alpha - \frac{L}{w} <0$, allora $P^*$ esiste $\iff m > m_2^*$;
        \item[3.] se $\alpha - \frac{L}{w} =0$, allora $P^*$ esiste $\iff m > m_3^*$;
    \end{itemize}
    dove
    \begin{equation}
        m_3^* = \frac{ \sigma L}{ w (\alpha + \sigma) } \; ,
        \label{eq:FstarPosM3}
    \end{equation}
    e $m_1^*, m_2^*$ sono sempre prese nell'ordine $m_1 < m_2$ dalla seguente
    \begin{equation}
        m_{1,2} = \frac{L}{2w} \frac{ \alpha + \sigma \pm \sqrt{ {(\alpha - \sigma)}^2 +\frac{4 \sigma L}{w}  } }
        {\alpha -\frac{L}{w}} \; .
        \label{eq:FstarPosM12}
    \end{equation}

    \label{teo:esistenzPosF} % WARNING prima era teorema, ho cambiato idea dopo.
\end{proposizione}

\paragraph{Interpretazione.}
\label{par:interpretationCond6b}
Notiamo che tutte le situazioni che forniscono un limite inferiore ad $m$ potrebbero essere attribuite alle
semplificazioni intrinseche al modello, dal momento che potrebbero non descrivere accuratamente un alveare
in apiario sul campo. La non esistenza di un equilibrio positivo nelle situazioni in cui
$m< \hat{m}$ per qualche $\hat{m}$ potrebbe semplicemente essere dovuta alla divergenza verso l'infinito
della popolazione in tali casi.

Al contrario, la condizione
\begin{equation}
\alpha - \frac{L}{w} > 0
    \label{eq:cond6b}
\end{equation}
si può leggere altrettanto semplicemente come $w \alpha >L$, dove entrambi i membri hanno unità di api al giorno.
Si noti anche che il membro destro cattura l'``efficienza'' dei processi sociali di reclutamento e cura della
covata, mentre il membro sinistro misura l'``efficienza'' della regina nel deporre uova fresche.

Vale a dire, l'equazione~\eqref{eq:cond6b} significa che il processo di reclutamento $R= R(H, F)$ è capace
di sostenere un livello di deposizione $L$, bilanciando le necessità dell'alveare che richiedono api sia nel
comparto $H$ (per la cura della covata), che nel comparto $F$ (per l'importazione di cibo).

\paragraph{}
Condizioni della forma $m< \hat{m}$ tipicamente servono ad assicurare la sopravvivenza della colonia;
in questa regione dei parametri, il modello può manifestare la non esistenza di un equilibrio positivo,
per via della crescita incontrastata dei numeri nella popolazione totale.

Come notato in precedenza, questo modello non ha un fattore di mortalità nel compartimento $H$: alla luce
di questo fatto, possiamo spiegare la condizione~\eqref{eq:cond6b} come limite inferiore per il tasso
di trasferimento dal comparto delle operaie di casa alle foraggiatrici, dove le api hanno una aspettativa
di vita finita.
Questa equazione esprime la reale necessità di coordinare efficientemente i passaggi tra le suddivisioni
del lavoro per la buona riuscita della colonia; per contro, le limitazioni puramente matematiche che
prevengono la divergenza della popolazione sono conseguenze puramente teoriche e non trovano applicazione
reale sul campo.

% Nel limite asintotico, $\frac{L}{w} H$ bees enter the hive compartment through eclosion and $\alpha H$ bees
% are withdrawn to the $F$ compartment: condition~\eqref{eq:cond6b} then says that the maximal recruitment rate
% must be sufficiently large to unload the excess hive bees towards the foragers compartment.
%  moar interpr? TODO
% cfr. condiz per 0

\paragraph{}
La ramificazione degli equilibri in questo modello è illustrata in figura~\ref{img:kh11phasePlane}.

\begin{figure}[pbh]
    \centering
    \includegraphics[keepaspectratio,width=0.86\textwidth]{img/pone.0018491.g003}

    \caption[\figurename~3 da \parencite{khoury2011}]{Diagramma delle fasi per le
        soluzioni del modello con differenti valori di mortalità $m$.
        Ogni curva del diagramma rappresenta la traiettoria di una soluzione, fornendo il numero di
        operaie di nido $H$ e di foraggiatrici $F$.
        All'aumentare del tempo $t$ le soluzioni variano nel verso delle frecce.

        In (a) abbiamo $m = 0.24$ e la popolazione tende ad un equilibrio positivo stabile,
        segnalato da un punto.

        Nella (b) abbiamo $m = 0.40$ e l'unico equilibrio del sistema è quello banale: la popolazione
        collassa verso lo zero.

        I valori degli altri parametri sono $L = 2000$,  $\alpha= 0.25$,  $\sigma= 0.75$ and $w = 27 000$.}
    \label{img:kh11phasePlane}
\end{figure}

