\section[Modello a due compartimenti con reclutamento]{Modello a due compartimenti con reclutamento sociale}
\label{sec:kh11}

Nel \citeyear{khoury2011} è stato proposto un modello\footcite{khoury2011} per indagare le dinamiche sociali
tra le caste della popolazione operaia in \emph{Apis~mellifera}.

Nello studio si ottengono due equazioni, in cui le variabili reali e non negative $H$ ed $F$ rappresentano
il numero delle operaie di nido e delle foraggiatrici, rispettivamente:

\begin{align}
    \diff{H}{t} &= \overbrace{E(H,F)}^{\text{eclosion}}- H \cdot R(H,F) \label{eq:kh11h} \; , \\
    \diff{F}{t} &= H \cdot R(H,F)  - m \cdot F \label{eq:kh11f} \; ,
\end{align}
dove $m$ è il tasso di mortalità, ed $E$ il tasso di schiusa.
$R$ rappresenta la funzione di \emph{reclutamento} nel termine di
trasferimento $H \to F$.

Un diagramma sintetico di questo modello è illustrato in figura~\ref{img:kh11diagram}.
\begin{figure}[!h]
    \centering
    \includegraphics[keepaspectratio,width=0.76\textwidth]{img/kh11schema}

    \caption[Diagramma del modello a due compartimenti.]{Diagramma del modello che illustra le mutue interazioni
        tra i due compartimenti $H$ (api di casa) ed $F$ (api di campo).
        In corsivo sono evidenziati i termini che compaiono nelle equazioni~\eqref{eq:kh11h}--\eqref{eq:kh11f}.}
    \label{img:kh11diagram}
\end{figure}


\paragraph{}
Alcuni oggetti proposti in questo modello si sono dimostrati molto interessanti e sono stati incorporati
in seguito da altri modelli, con variazioni minime;
uno di questi modelli\footcite{ratti2017} sarà esaminato nella prossima sezione a p.~\pageref{sec:ratti17}.


\subsection{Reclutamento con inibizione sociale.}
\label{ssec:recr}
Nello studio\footcite{khoury2011} la funzione di reclutamento è rappresentata dalla funzione
\begin{equation}
    R(H,F) = \alpha - \sigma \frac{F}{H+F} \; ,
    \label{eq:Recr}
\end{equation}
in cui $\alpha$ rappresenta il tasso massimo di reclutamento.\footnote{Raggiunto quando nella colonia non è
presente nessuna bottinatrice.}

Il secondo termine rappresenta l'inibizione sociale del processo di reclutamento.
Gli autori assumono
\textquote[{\footcite[3]{khoury2011}}]{\omissis that social inhibition is directly proportional to the fraction of the total number of adult bees that are foragers, such that a high fraction of foragers in the hive results in low recruitment.}

\paragraph{}
La ricerca biologica suggerisce che l'ethyl~oleate scambiato tra le api operaie sia
il principale modulatore nel processo di reclutamento sociale.
\footcite{ethyloleate,ethyloleate2,meccanica}

Ad esempio una colonia risponderà ad una elevata mortalità di bottinatrici (più anziane) attraverso
la riduzione dell'inibizione del reclutamento (dati i livelli più bassi di feromone) che dunque risulta
\textquote[\footcite{khoury2011}]{\omissis in a precocious onset of foraging behaviour in young bees}.

Nella situazione opposta -- \ie quando vi sono molte bottinatrici e la loro mortalità è bassa -- la colonia
risponde \emph{aumentando} l'inibizione sociale: un maggior numero di bottinatrici si astiene dunque dal volo e
dalle attività all'esterno, ripiegando sui lavori di cura all'interno del nido. Questa risposta inibitoria
è probabilmente dovuta alla maggiore concentrazione di ethyl~oleate:
\textquote[{\footcite[1]{khoury2011}}]{Old forager bees transfer ethyl oleate to young hive bees via trophallaxis, which delays the age at which they begin foraging.}

\paragraph{}
Tramite questo meccanismo mediato dai segnali feromonici, l'alveare tenta di mantenere un equilibrio
demografico tra le due classi, sufficiente a soddisfare le necessità della covata e dell'importazione
di polline, nettare, acqua.

In questo scenario la risposta collettiva -- che riduce il tasso di reclutamento -- aumenta con
l'intensità degli scambi di ethyl oleate, che a loro volta sono correlati al rapporto $\frac{F}{H+F}$:
queste assunzioni motivano la presenza di tale fattore nell'equazione~\eqref{eq:Recr}.


\paragraph{Tassi di mortalità.}
Come si è notato in precedenza,\footnote{Nel capitolo~\ref{chap:bio}.}
le operaie adulte -- esploratrici e bottinatrici -- sono soggette ad una gamma più ampia di fattori di stress
nell'ambiente esterno, rispetto alle operaie di nido. Di conseguenza l'aspettativa di vita per le operaie di
nido è più lunga rispetto quella delle foraggiatrici:
\textquote[{\footcite[1]{khoury2011}}]{Survival of bees in the protected hive environment is high, but the
survival of forager bees is much lower. \omissis The average foraging life of a bee has been estimated as less
than seven days, because of the many risks and severe metabolic costs associated with foraging.}

Il termine di pozzo $-mF$ nell'equazione~\eqref{eq:kh11f} rappresenta la mortalità delle foraggiatrici;
il tasso parametrico $m$ imposta la frazione del comparto $F$ che muore ogni giorno.\footnote{Cfr. %
p.~\pageref{par:mInverseFlightspan} e tab.~\ref{tab:FDLruoliEta}.}


\paragraph{}
L'assenza di un fattore di mortalità nel comparto delle operaie di nido nell'equazione~\eqref{eq:kh11h} è
una assunzione piuttosto severa, e comporta una netta semplificazione del modello.

Tuttavia, essa è giustificata dall'obiettivo principale dell'articolo, ossia modellare la relazione tra le varie
forme di collasso nella popolazione di colonie di \emph{A.~mellifera} e gli eccezionali tassi di mortalità
tra le foraggiatrici.
\footcite[1,2,3,5]{khoury2011}

\subsection{Tasso di schiusa.}
\label{sec:eclos}
Il termine $E(H,F)$ tiene conto della schiusa della covata che emerge come giovani operaie adulte nel
comparto $H$ delle api di nido.
\blockquote[{\footcite[2]{khoury2011}}]{\omissis the number of eggs reared in a colony (and hence the eclosion rate) is related to the number of bees in the hive. Big colonies raise more brood.}

Come ricordato nel capitolo~\ref{chap:bio}, il sostentamento di una popolazione numerosa dipende non soltanto
dal tasso di deposizione della regina, ma anche dal numero di operaie giovani che rimangono nel nido per
accudire uova e larve, al fine di massimizzare il numero di esse che giunge sana al termine di tutti gli stadi
della metamorfosi.

\paragraph{}
Gli autori scelgono la formulazione
\begin{equation}
    \label{eq:eclos}
    E(H,F) = L \frac{H+F}{w + H + F} \; ,
\end{equation}
ove $L$ rappresenta il tasso di deposizione della regina e $w$ è la rapidità con cui $E$ cresce verso $L$ quando
la popolazione totale $H+F$ aumenta.\footnote{In questa sezione useremo spesso l'abbreviazione
$N=H+F$ per il numero totale di operaie, di nido e di campo, presenti nella colonia ad un dato istante $t$.}

Infatti, sommando le equazioni \eqref{eq:kh11h}--\eqref{eq:kh11f} e trascurando il termine di mortalità otteniamo
il tasso di crescita asintotico della popolazione totale
$$\diff{N}{t} \simeq \frac{L}{ \frac{w}{N} + 1} \approx L \; , \quad \text{quando } \frac{w}{N} \ll 1 \; ,$$
ovvero per popolazioni $N$ molto grandi.

La $E$ è una funzione Holling di tipo II, in cui il tasso di attacco è $a=\frac{1}{w}$ e $w$
è la retroimmagine del semimassimo nella risposta. In figura~\ref{img:eclos} è riportata $E(N)$ in funzione di $N$.
Possiamo vedere come valori
crescenti di $w$ ``schiacciano'' la curva mentre la saturazione è raggiunta per valori sempre maggiori di $N$,
e il grafico di $E(N)$ si ``inclina'' verso la destra.
\begin{figure}[!hb]
    \centering
    \includegraphics[keepaspectratio,width=0.86\textwidth]{img/hollingTypeII-eclosion}

    \caption[Schiusa, Holling tipo II]{$E$ rappresenta la quantità di uova deposte dalla regina che raggiunge lo
    stadio adulto ogni giorno, come risposta funzionale al numero totale di operaie $N=H+F$.
    \\
    Sono illustrati tre diversi valori di $w$, mantenendo costante il tasso di deposizione $L=2000$.}
    \label{img:eclos}
\end{figure}

\paragraph{}
Modelli successivi\footcite{ratti2017} hanno incorporato questa proposta, modificandola in una funzione Holling
di tipo III e aggiungendo certi fattori di modulazione ulteriori, ma tutto sommato hanno mantenuto il nucleo
della formulazione molto simile a quanto~\citeauthor{khoury2011} propongono nell'equazione~\eqref{eq:eclos}.


\subsection[Configurazioni di equilibrio]{Configurazioni di equilibrio e analisi della stabilità}
Rivediamo adesso i risultati forniti nell'articolo, a proposito dell'esistenza e della stabilità
dei punti di equilibrio nel modello~\eqref{eq:kh11h}--\eqref{eq:kh11f}.

Gli autori hanno usato \textquote[{\footcite[3]{khoury2011}}]{standard linear stability analysis and phase
plane analysis} e ripercorrendo i loro passi forniremo criteri per l'esistenza
di equilibri banali e non, \ie zeri del membro destro delle equazioni del modello.

\paragraph{}
Una volta stabilita l'esistenza di un punto di equilibrio, calcoliamo la matrice jacobiana del sistema e
la valutiamo sull'equilibrio stesso; un semplice criterio sufficiente per la stabilità asintotica
\footnote{Vedasi il Teorema~\ref{teo:pozzoNonLineare} a p.~\pageref{teo:pozzoNonLineare}.}
è che $\Re (\lambda) < 0$ per ogni autovalore $\lambda$ della matrice.

Un'altra tecnica, dovuta a Lyapounov, consiste nell'esibire una funzione che abbia un minimo
locale forte sul punto di equilibrio e che sia localmente decrescente lungo le
soluzioni.\footnote{Vedasi il Teorema~\ref{teo:lyapounovFunc} a p.~\pageref{teo:lyapounovFunc}.}

\paragraph{}
Si noti da subito che $H, F$ sono non--negative, \ie $H(t), F(t) \geq 0$ per ogni $t$.
Inoltre, tutti i parametri del modello debbono essere strettamente positivi.

Il prossimo risultato stabilisce una condizione necessaria per tutti i punti d'equilibrio non banali.

\begin{lemma}
    \label{lem:necessJ}
    Tutti i punti di equilibrio $(H^*, F^*) \neq (0,0)$
    del modello~\eqref{eq:kh11h}--\eqref{eq:kh11f} verificano
    \begin{equation}
    F^* = J H^* \; ,
    \label{eq:FeqJH}
    \end{equation}
    dove la costante adimensionale $J$ vale
    \begin{equation}
        J \coloneq \frac{1}{2} \left[
            \frac{\alpha}{m} - \frac{\sigma}{m} - 1 +
            \sqrt{ {\left( \frac{\alpha}{m} - \frac{\sigma}{m} - 1 \right)}^2
                + 4 \frac{\alpha}{m}
            }
        \right].
        \label{eq:kh11posEqJ}
    \end{equation}
\end{lemma}

Vale a dire, i punti di equilibrio giacciono sulla semiretta $F= JH$ nello spazio delle fasi, quando esistono.

\begin{proof}
    Annullando il membro destro dell'equazione~\eqref{eq:kh11f} otteniamo
    $$F^2 - \left( \frac{\alpha}{m} - \frac{\sigma}{m} -1
        \right) H F - \frac{\alpha}{m} H^2 = 0 \; .$$
    Risolvendo rispetto ad $F$ otteniamo il discriminante
    $$\Delta = \left[ {\left( \frac{\alpha}{m} - \frac{\sigma}{m} -1
        \right)}^2 + 4 \frac{\alpha}{m}
        \right] H^2 \; , $$
    che è sempre non negativo.

    \paragraph{}
    Adottando la definizione di $J$ dell'equazione~\eqref{eq:kh11posEqJ}, possiamo vedere
    che ogni punto d'equilibrio $(H^*, F^*)$ deve necessariamente verificare la~\eqref{eq:FeqJH}.
\end{proof}


Calcoliamo adesso la matrice jacobiana del sistema nella configurazione generale $(H, F)$:
\begin{equation}
    D(H, F) =
    \begin{pmatrix}
        L \frac{w}{{(w + H + F)}^2} - \alpha + \sigma \frac{F^2}{{(H+F)}^2} &
        L \frac{w}{{(w + H + F)}^2} + \sigma \frac{H^2}{{(H+F)}^2}
        \\
        \alpha - \sigma \frac{F^2}{{(H+F)}^2} &
        - \sigma \frac{H^2}{{(H+F)}^2} - m
    \end{pmatrix} \; .
    \label{eq:jacoGeneral}
\end{equation}

Se assumiamo che $(H, F) = (H^*, F^*)$ sia un punto d'equilibrio, sotto la condizione ~\eqref{eq:kh11posEqJ}
la matrice jacobiana~\eqref{eq:jacoGeneral} si riduce all'espressione
\begin{equation}
    D \left( F/J, F \right) =
    \begin{pmatrix}
        L \frac{w}{ {\left[ w + \left( \frac{1}{J} +1 \right) F \right]}^2 } - \alpha
        + \frac{ \sigma }{ {\left( \frac{1}{J} + 1 \right)}^2 } &
        %
        L \frac{w}{ {\left[ w + \left( \frac{1}{J} +1 \right) F \right]}^2 }
        + \frac{ \sigma }{ {\left( J + 1 \right)}^2 } \\
        %
        \alpha - \frac{ \sigma }{ {\left( \frac{1}{J} + 1 \right)}^2 } &
        %
        - \frac{ \sigma }{ {\left( J + 1 \right)}^2 } - m
    \end{pmatrix} \; ,
    \label{eq:jacoOnJFH}
\end{equation}
in cui la seconda riga dipende soltanto dai parametri, e non dalle variabili $H$ ed $F$.

% \begin{gather}
%     \diffp*{ \dot{H} }{H}{H= \frac{1}{J} F} =
%     L \frac{w}{ {\left[ w + \left( \frac{1}{J} +1 \right) F \right]}^2 } - \alpha
%     + \frac{ \sigma }{ {\left( \frac{1}{J} + 1 \right)}^2 } \; ,
%     \\
%     \diffp*{ \dot{H} }{F}{H= \frac{1}{J} F} =
%     L \frac{w}{ {\left[ w + \left( \frac{1}{J} +1 \right) F \right]}^2 }
%     + \frac{ \sigma }{ {\left( J + 1 \right)}^2 } \; ,
%     \\
%     \diffp*{ \dot{F} }{H}{H= \frac{1}{J} F} =
%     \alpha - \frac{ \sigma }{ {\left( \frac{1}{J} + 1 \right)}^2 } \; ,
%     \\
%     \diffp*{ \dot{F} }{F}{H= \frac{1}{J} F} =
%     - \frac{ \sigma }{ {\left( J + 1 \right)}^2 } - m \; ,
%     \label{eq:jacoOnJFH}
% \end{gather}
% e le ultime due equazioni dipendono soltanto dai parametri, non dalle variabili $H$ ed $F$.

\subsubsection{Equilibrio banale}
L'equilibrio banale $(0,0)$ è chiaramente sempre soluzione\footnote{Dunque il sistema ammette
\emph{sempre} l'equilibrio banale.}
di $$\dot{H}=\dot{F}=0 \; ,$$ e lo studio della sua stabilità è cruciale.
Infatti, nelle regioni di parametri per cui questo equilibrio è stabile e attrattivo generalmente si ha il declino
inevitabile dell'intera colonia.

\paragraph{}
La matrice jacobiana~\eqref{eq:jacoGeneral} del modello valutata nell'origine è:
\begin{equation}
    D_{00} \coloneq
    \begin{pmatrix}
        \frac{L}{w} -\alpha &
        \frac{L}{w}
        \\
        \alpha & -m
    \end{pmatrix} \; .
    \label{eq:jacoEquZero}
\end{equation}

% Con il Teorema del pozzo non lineare\footnote{Cfr. p.~\pageref{teo:pozzoNonLineare}}
% possiamo vedere il motivo per cui la condizione~\ref{eq:cond6b}
% $\alpha -\frac{L}{w}>0$
% sia utile per caratterizzare la stabilità dell'origine.\footcite[3]{khoury2011}

Abbiamo
$$\det  D_{00} > 0 \quad \iff \quad
m > \frac{\alpha L}{w} {\left( \alpha - \frac{L}{w} \right)}^{-1} \; ,$$
e
$$ \alpha - \frac{L}{w} > 0  \quad \implies \quad
\tr D_{00} = - \left( \alpha - \frac{L}{w} \right) - m < 0 \; ,$$

da cui ricaviamo che se $\alpha -\frac{L}{w}>0$ (condizione~\eqref{eq:cond6b}), allora $D_{00}$ ha
determinante positivo e traccia negativa, cioè gli autovalori sono concordi e negativi;
dunque\footnote{Cfr. p.~\pageref{teo:pozzoNonLineare}} l'origine è stabile.

\paragraph{}
Riassumiamo quanto appena discusso nel seguente
\begin{lemma}[Condizione sufficiente per la stabilità dell'equilibrio banale]
    Se $\alpha - \frac{L}{w} > 0$ e
    $$m > \frac{\alpha L}{w \alpha -L} \; ,$$
    allora l'equilibrio banale del modello \eqref{eq:kh11h}--\eqref{eq:kh11f} è asintoticamente stabile.
    \label{lem:eqBanKh}
\end{lemma}


\subsubsection{Equilibrio positivo}
Indaghiamo adesso nel dettaglio le condizioni per l'esistenza e la stabilità di un equilibrio non banale, ossia
con popolazione strettamente positiva. Tali punti dovrebbero rappresentare l'esito favorevole di una colonia che
sopravvive alle avversità.

\paragraph{}
Presentiamo una successione di risultati più semplici, al fine di costituire un criterio per l'esistenza di
un punto d'equilibrio positivo.

\begin{proposizione}[Unicità dell'equilibrio positivo]
    Se il modello~\eqref{eq:kh11h}--\eqref{eq:kh11f} ammette un equilibrio positivo $(H^*, F^*)$, allora
    questi è unico.

    Esso è determinato dall'equazione~\eqref{eq:kh11posEqJ}, insieme con la seguente:
    \begin{equation}
        F^* = \frac{L}{m} - w \frac{J}{1+J} \; .
        \label{eq:kh11posEqF}
    \end{equation}

    \label{teo:exUniqFstarPos} % WARNING prima era teorema, ho cambiato idea dopo.
\end{proposizione}

La possibilità della non--esistenza dell'equilibrio positivo è implicitamente
catturata nell'equazione~\eqref{eq:kh11posEqF}, laddove si abbia $F^* \leq 0$.

\begin{proof}
    Otteniamo la dimostrazione con alcuni calcoli elementari.

    Dal momento che stiamo cercando un equilibrio non banale $(H^*, F^*)$, per il
    Lemma~\ref{lem:necessJ} possiamo supporre ambo le componenti strettamente positive.

    Ancora per il Lemma~\ref{lem:necessJ}, possiamo sostituire $H=\frac{1}{J} F$ nel membro destro della
    prima equazione~\eqref{eq:kh11h} del modello, ed usare la condizione di equilibrio per uguagliare
    a zero quanto ottenuto. Da
    $$L \frac{ \left( \frac{1}{J} +1 \right) F }{ w +\left( \frac{1}{J} +1 \right) F }
    - \frac{1}{J} F \left( \alpha - \sigma \frac{\cancel{F}}{ \left( \frac{1}{J} +1 \right) \cancel{F} } \right) = 0$$
    possiamo semplificare $F$ perché non nulla, così da ottenere
    $$L \left( \frac{1}{J} +1 \right) \cancel{F} - \frac{1}{J} \left( \alpha - \frac{\sigma}{ \frac{1}{J} +1 } \right)
    \left[ w + \left( \frac{1}{J} +1 \right) F \right] \cancel{F} = 0 \; ,$$
    dove la semplificazione di $F$ procede con lo stesso argomento di sopra; si noti che ciò ammonta a scartare la
    soluzione nulla, che non ci interessa per ipotesi.

    L'unicità dell'equilibrio positivo si ottiene manipolando ulteriormente l'equazione sopra, ottenendo infine
    $$F^* = \frac{LJ}{ \alpha - \frac{\sigma}{ 1 + \frac{1}{J} } } - w \frac{J}{1+J} \; ,$$
    in cui possiamo osservare che il membro destro dipende esclusivamente dai parametri.

    \paragraph{}
    Avendo a disposizione l'ultima equazione, per concludere la dimostrazione -- \ie provare la~\eqref{eq:kh11posEqF}
    -- resta soltanto da dimostrare che
    $$J m = \alpha - \sigma \frac{J}{1+J} \; ,$$
    dove abbiamo semplificato il fattore $L$ ad ambo i membri, essendo non nullo.

    Per stabilire quest'ultima equazione può essere utile razionalizzare l'espressione $\frac{J}{1+J}$ come
    $$\frac{J}{1+J} = \frac{ \alpha + \sigma + m - \sqrt{ {(\alpha - \sigma -m)}^2 - 4 \alpha m } }{2 \sigma}$$
    e raggiungere l'obiettivo tramite semplici manipolazioni algebriche.
\end{proof}

Il modello~\eqref{eq:kh11h}--\eqref{eq:kh11f} ammette quindi sempre l'equilibrio banale, ed un ulteriore
configurazione di equilibrio positivo in alcune circostanze, riassunte nella seguente
\begin{proposizione}[Criterio di esistenza]
    Una condizione necessaria e sufficiente affinché il modello~\eqref{eq:kh11h}--\eqref{eq:kh11f} abbia
    un equilibrio strettamente positivo $P^* \coloneq (H^*, F^*)$ è data per casi:
    \begin{itemize}
        \item[1.] se $\alpha - \frac{L}{w} >0$, allora $P^*$ esiste $\iff m_1^* < m < m_2^*$;
        \item[2.] se $\alpha - \frac{L}{w} <0$, allora $P^*$ esiste $\iff m > m_2^*$;
        \item[3.] se $\alpha - \frac{L}{w} =0$, allora $P^*$ esiste $\iff m > m_3^*$;
    \end{itemize}
    dove
    \begin{equation}
        m_3^* = \frac{ \sigma L}{ w (\alpha + \sigma) } \; ,
        \label{eq:FstarPosM3}
    \end{equation}
    e $m_1^*, m_2^*$ sono sempre prese nell'ordine $m_1^* < m_2^*$ dalla seguente
    \begin{equation}
        m_{1,2}^* = \frac{L}{2w} \cdot \frac{ \alpha + \sigma \pm \sqrt{ {(\alpha - \sigma)}^2 +\frac{4 \sigma L}{w}  } }
        {\alpha -\frac{L}{w}} \; .
        \label{eq:FstarPosM12}
    \end{equation}

    \label{teo:esistenzPosF} % WARNING prima era teorema, ho cambiato idea dopo.
\end{proposizione}

\begin{proof}
Possiamo usare l'equazione~\eqref{eq:kh11posEqF} dalla Proposizione~\ref{teo:exUniqFstarPos} e studiare
la disequazione $F^* >0$ per dimostrare il criterio di esistenza:
$$F^* > 0 \; \iff \;
\left( \frac{2 \alpha w}{L} -1 \right) m - \alpha - \sigma < \sqrt{ {(\alpha - \sigma -m)}^2 + 4 \alpha m } \; .$$

Elevando al quadrato ambo i membri otteniamo la seguente disuguaglianza in $m$:
\begin{equation}
    \left( \frac{\alpha w}{L} -1 \right) m^2 -( \alpha + \sigma) m + \frac{\sigma L}{w} < 0
    \label{eq:mDiseq}
\end{equation}
la quale ha discriminante sempre positivo
$$\Delta = {(\alpha -\sigma)}^2 +4 \frac{\sigma L}{w} \; .$$

L'equazione~\eqref{eq:FstarPosM12} si ottiene applicando la formula delle radici quadratiche.
Nel seguito sceglieremo la notazione per avere sempre $m_1^* < m_2^*$ fino al termine della dimostrazione.

\paragraph{Caso 1.}
Se $\alpha - \frac{L}{w} >0$, allora $F^*>0$ esiste se e soltanto se
$$m_1^* < m < m_2^* \; ,$$
ed anche il limite inferiore ha senso, dal momento che
$m_1^* > 0$ si riduce a $\alpha - \frac{L}{w}>0$, vera per ipotesi del caso.

\paragraph{Caso 2.}
Se $\alpha -\frac{L}{w} <0$, allora dobbiamo prendere $m < m_1^*$ oppure $m > m_2^*$ affinché $F^* >0$ esista.
La prima disequazione si può scartare dal momento che $m_1^* <0$ per ipotesi del caso.

La seconda va mantenuta dal momento che $m_2^* >0$ si riduce alla ipotesi del caso $\alpha - \frac{L}{w} < 0$.

\paragraph{Caso 3.}
Quando $\alpha - \frac{L}{w} =0$, la nostra disequazione~\eqref{eq:mDiseq} diviene lineare e la sua soluzione è
$$m > \frac{\sigma L}{w (\alpha +\sigma)} \; ,$$
ciò che conclude l'ultimo caso restante della dimostrazione.
\end{proof}

\paragraph{}
Un criterio sufficiente per la stabilità dell'equilibrio positivo è dato nella seguente Proposizione
\footcite[3]{khoury2011}; la enunciamo senza dimostrazione poiché richiede calcoli lunghi e tediosi,
seppure non sia particolarmente complicata sotto il profilo teorico.

\begin{proposizione}[Criterio di stabilità dell'equilibrio positivo]
L'equilibrio positivo $(H^*, F^*)$ del modello \eqref{eq:kh11h}--\eqref{eq:kh11f}, determinato
dalla Proposizione~\ref{teo:exUniqFstarPos},
è asintoticamente stabile se $\alpha -\frac{L}{w} >0$ e
$$m < \frac{L}{2w} \cdot \frac{ \alpha + \sigma + \sqrt{ {(\alpha - \sigma)}^2 +\frac{4 \sigma L}{w}  } }
    {\alpha -\frac{L}{w}} \; .$$
    \label{prop:stabPosKh}
\end{proposizione}

\paragraph{Interpretazione.}
\label{par:interpretationCond6b}
Notiamo che tutte le situazioni che forniscono un limite inferiore ad $m$ potrebbero essere attribuite alle
semplificazioni intrinseche al modello, dal momento che potrebbero non descrivere accuratamente un alveare
in apiario sul campo. La non esistenza di un equilibrio positivo nelle situazioni in cui
$m< \hat{m}$ per qualche $\hat{m}$ potrebbe semplicemente essere dovuta alla divergenza verso l'infinito
della popolazione in tali casi.

Al contrario, la condizione che ritroviamo nel Lemma~\ref{lem:eqBanKh} e nelle Proposizioni~\ref{teo:esistenzPosF}
e \ref{prop:stabPosKh} data dalla disuguaglianza
\begin{equation}
\alpha - \frac{L}{w} > 0
    \label{eq:cond6b}
\end{equation}
si può leggere altrettanto semplicemente come $w \alpha >L$, dove entrambi i membri hanno unità di api al giorno.
Si noti anche che il membro destro cattura l'``efficienza'' dei processi sociali di reclutamento e cura della
covata, mentre il membro sinistro misura l'``efficienza'' della regina nel deporre uova fresche.

Vale a dire, la~\eqref{eq:cond6b} significa che il processo di reclutamento $R$ è capace
di sostenere un livello di deposizione $L$, bilanciando le necessità dell'alveare che richiedono api sia nel
comparto $H$ (per la cura della covata), che nel comparto $F$ (per l'importazione di cibo).

\paragraph{}
Condizioni della forma $m< \hat{m}$ tipicamente servono ad assicurare la sopravvivenza della colonia;
in questa regione dei parametri, il modello può manifestare la non esistenza di un equilibrio positivo,
per via della crescita incontrastata dei numeri nella popolazione totale.

Come notato in precedenza, questo modello non ha un fattore di mortalità nel compartimento $H$: alla luce
di questo fatto, possiamo spiegare la condizione~\eqref{eq:cond6b} come limite inferiore per il tasso
di trasferimento dal comparto delle operaie di casa al comparto delle operaie di campo, dove
le api hanno una aspettativa di vita finita.
Questa equazione esprime la reale necessità di coordinare efficientemente i passaggi tra le suddivisioni
del lavoro per la buona riuscita della colonia; per contro, le limitazioni puramente matematiche che
prevengono la divergenza della popolazione sono conseguenze puramente teoriche e non trovano applicazione
reale sul campo.

% Nel limite asintotico, $\frac{L}{w} H$ bees enter the hive compartment through eclosion and $\alpha H$ bees
% are withdrawn to the $F$ compartment: condition~\eqref{eq:cond6b} then says that the maximal recruitment rate
% must be sufficiently large to unload the excess hive bees towards the foragers compartment.
%  moar interpr? TODO
% cfr. condiz per 0

\paragraph{}
La ramificazione degli equilibri in questo modello è illustrata in figura~\ref{img:kh11phasePlane}.
\begin{figure}[hb]
    \centering
    \includegraphics[keepaspectratio,width=0.76\textwidth]{img/pone.0018491.g003}

    \caption[\figurename~3 da \parencite{khoury2011}]{Diagramma delle fasi per le
        soluzioni del modello con differenti valori di mortalità $m$.
        Ogni curva del diagramma rappresenta la traiettoria di una soluzione, fornendo il numero di
        operaie di nido $H$ e di foraggiatrici $F$.
        All'aumentare del tempo $t$ le soluzioni variano nel verso delle frecce.

        In (a) abbiamo $m = 0.24$ e la popolazione tende ad un equilibrio positivo stabile,
        segnalato da un punto.

        Nella (b) abbiamo $m = 0.40$ e l'unico equilibrio del sistema è quello banale: la popolazione
        collassa verso lo zero.

        I valori degli altri parametri sono $L = 2000$,  $\alpha= 0.25$,  $\sigma= 0.75$ e $w = 27 000$.
        \\
        { \footnotesize \ttfamily doi: 10.1371/journal.pone.0018491.g003 }
        }

    \label{img:kh11phasePlane}
\end{figure}

\paragraph{}
\label{par:mInverseFlightspan}
Tra i contributi più utili di questo semplice modello, oltre al ruolo sociale delle funzioni $E$ ed
$R$, rileviamo una concentrazione particolare degli autori sul parametro $m$:
ciò riflette in parte l'attenzione dedicata dalla ricerca alle varie sindromi da
collasso di popolazione in \emph{Apis} rilevate in tutto il mondo.
D'altro canto, è importante per la validazione di ogni modello matematico
un confronto coi dati sperimentali rilevati sul campo:\footcite{rueppell2009honey}
in questo senso $m$ è facilmente derivabile come l'inverso
di una misura diretta, l'aspettativa di vita media per le bottinatrici (\emph{flightspan}).

Se ogni bottinatrice ha mediamente probabilità $1-m$ di sopravvivere al primo giorno in tale ruolo, $(1-m)^2$
probabilità di sopravvivere al secondo, e così di seguito, la sua aspettativa di vita media è
$$ \sum_{k=1}^{\infty} k m {(1-m)}^k = m \underbrace{ \sum_{k=1}^{\infty} k {(1-m)}^k }_{m^{-2}}
= \frac{1}{m} \; .$$


% TODO stabilità dell'equilibrio non banale?
