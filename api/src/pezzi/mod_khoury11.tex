\section[Two--compartments model with social recruitment]{Two--compartments model of workers with social recruitment}

In \citeyear{khoury2011} a simple model was proposed by \citeauthor{khoury2011} to explore the social dynamics in
population interplay between castes of female workers.

Two compartment equations are given, where the non--negative real variables $H$ and $F$ represent the number of hive workers and the number of forager workers, respectively:

\begin{align}
    \diff{H}{t} &= \overbrace{E(H,F)}^{\text{eclosion}}- H \cdot R(H,F) \label{eq:kh11h} \\
    \diff{F}{t} &= H \cdot R(H,F)  - m \cdot F \label{eq:kh11f}
\end{align}

where $m$ is the forager mortality rate, and $R$ represents the \emph{recruitment} factor in the $H \to F$ transfer term.
This proposal has proven insightful, and later incorporated in more sophisticated models with slight variations.\footcite{ratti2017}

\paragraph{Social recruitment.}
In the recruitment function
\begin{equation}
    \label{eq:Recr}
    R(H,F) = \alpha - \sigma \frac{F}{H+F} \; ,
\end{equation}
the maximum rate of recruitment $\alpha$ is attained when there are no foragers present in the colony.
The second term represents social inhibition in the recruitment process, and the authors
\textquote[{\cite[3]{khoury2011}}]{\omissis assumed that social inhibition is directly proportional to the fraction of the total number of adult bees that are foragers, such that a high fraction of foragers in the hive results in low recruitment.}

\paragraph{}
Biological research hints at ethyl oleate exchange between honeybee workers as the main modulating process in social recruitment: for example, a colony will respond to a high mortality of foragers by \emph{reducing} social inhibition and thereby \textquote[\cite{khoury2011}]{resulting in a precocious onset of foraging behaviour in young bees}.

In the opposite situation -- \emph{i.e.} when forager mortality is low -- the colony responds by \emph{increasing} social inhibition: more foragers feel the urge to refrain from flying outside and revert to in--hive duties.
This inhibition response in the recruitment process is allegedly driven by higher concentrations of ethyl oleate:
\textquote[{\cite[1]{khoury2011}}]{Old forager bees transfer ethyl oleate to young hive bees via trophallaxis, which delays the age at which they begin foraging.}

\paragraph{}
Via this pheromonal mechanism the hive can mantain an equilibrium where its basic necessities (brood care and food flow) are satisfied.
In this scenario, the collective pheromonal response inhibiting recruitment increases with the intensity of ethyl oleate exchanges, which in turn is related to the ratio $\frac{F}{H+F}$: these assumptions motivate the presence of this factor in Equation~\eqref{eq:Recr}.

\paragraph{}
As we noted before, forager workers are subjected to more diverse and intense stress factors than hive workers, whence the average lifespan of a female honeybee is longer as a hive worker, and is significantly shorter as a forager: \textquote[{\cite[1]{khoury2011}}]{Survival of bees in the protected hive environment is high, but the survival of forager bees is much lower. \omissis The average foraging life of a bee has been estimated as less than seven days, because of the many risks and severe metabolic costs associated with foraging.}

\paragraph{Eclosion.}
The term $E(H,F)$ accounts for eclosion of adult young workers in the hive class.
\textquote[{\cite[2]{khoury2011}}]{\omissis the number of eggs reared in a colony (and hence the eclosion rate) is related to the number of bees in the hive. Big colonies raise more brood.}

The authors have chosen the form
\begin{equation}
    \label{eq:eclos}
    E(H,F) = L \frac{H+F}{w + H + F} \; ,
\end{equation}
where $L$ is the queen's laying rate and $w$ determines the rate at which $E$ increases to $L$ as $E+F$ gets very large.
\footnote{In this section we will often use the shorthand $N=H+F$, the total number of workers present in the colony at a given time.}

This is a Holling type II function where the attack rate is $a=\frac{1}{w}$ and $w$ is the reverse image of the half--maximal output; in \figurename~\ref{img:eclos} we can see that increasing values of $w$ ``flatten'' the curve
as saturation is attained for greater values of $N$, and the graph of $E(N)$ ``leans'' to the right.

\begin{figure}[pbh]
    \centering
    \includegraphics[keepaspectratio,width=0.86\textwidth]{img/hollingTypeII-eclosion}

    \caption[Holling type II Eclosion]{$E$ models how many of the laid eggs eventually reach adult stage as a functional response of the total number of adult workers $N=H+F$.
    \\
    Drawn for three different values of $w$, with constant laying rate $L=2000$.}
    \label{img:eclos}
\end{figure}

Later models\footcite{ratti2017} incorporated this basic proposal with type III Holling functional response and added modulating factors, but retained the core formulation very similar to what     \citeauthor{khoury2011} proposed in Equation~\refeq{eq:eclos}.

\paragraph{}
The absence of a mortality factor in the hive compartment \refeq{eq:kh11h} is a strong simplifying assumption,
justified by the focus of this paper in relating CCD to higher than normal forager mortality rates.\footcite[1,2,3,5]{khoury2011}

\subsection{Stability analysis}
The authors used \textquote[{\cite[3]{khoury2011}}]{standard linear stability analysis and phase plane analysis} to examine Equations \refeq{eq:kh11h}--\refeq{eq:kh11f}.

They give the two inequalities
\begin{align}
    m &< \frac{L}{2w} \cdot \frac{\alpha + \sigma + \sqrt{{(\alpha - \sigma)}^2
            + 4 \frac{L \sigma}{w} } }{\alpha - \frac{L}{w}}
    \label{eq:posEq1} \\
    \alpha - \frac{L}{w} &> 0 \label{eq:posEq2}
\end{align}
as sufficient conditions for a positive equilibrium $(H_0, F_0)$ to exist in the system; it is globally stable.

If any of these inequalities is reversed, then the trivial equilibrium $(0,0)$ is the only attractor, meaning that the population goes to zero.

This branching in the model is illustrated in \figurename~\ref{img:kh11phasePlane}.

\begin{figure}[pbh]
    \centering
    \includegraphics[keepaspectratio,width=0.86\textwidth]{img/pone.0018491.g003}

    \caption[\figurename~3 from \parencite{khoury2011}]{
%         \figurename~3 from \parencite[3]{khoury2011}
        \textquote[{\figurename~3 from \cite[3]{khoury2011}}]{
            Phase plane diagrams of solutions to the model for different values of m.
            Each line on the diagrams represents a solution trajectory, giving the number of foragers $F$ and the number of hive bees $H$.
            As time $t$ increases the solutions change along the trajectory in the direction of the arrows.
            In (a) $m = 0.24$ and the populations tend to a stable equilibrium population, marked by a dot.
            In (b), $m = 0.40$ there is no nonzero equilibrium and the hive populations collapses to zero.

            Parameter values are $L = 2000$,  $\alpha= 0.25$,  $\sigma= 0.75$ and $w = 27 000$.
        }
    }

    \label{img:kh11phasePlane}
\end{figure}

\subsubsection{Trivial equilibrium}
The trivial $(0,0)$ equilibrium is clearly a solution of $\dot{H}=\dot{F}=0$, and its stability is an important matter of study. Parameter regions where this equilibrium is stable and attractive usually mean the inevitable doom of the entire colony.

We calculate its jacobian matrix as:

BBBB

\subsubsection{Positive equilibrium}
When both equations \eqref{eq:posEq1} and \eqref{eq:posEq2} hold, we have the positive steady state $(H_0, F_0)$:
\begin{align}
    F_0 &= \frac{L}{m} -w \frac{J}{J+1}, \; H_0 = \frac{1}{J} F_0 \quad \text{where}
    \label{eq:kh11posEqJSol} \\
    J &= \frac{1}{2} \left[
    \frac{\alpha}{m} - \frac{\sigma}{m} - 1 +
    \sqrt{ {\left( \frac{\alpha}{m} - \frac{\sigma}{m} - 1 \right)}^2
        + 4 \frac{\alpha}{m}
    }
    \right]
    \label{eq:kh11posEqJ}
\end{align}

Observe that by requesting a strictly positive equilibrium, we get from Equation~\eqref{eq:kh11f} that
$$F_0 = J H_0,$$
as stated in the second half of Equation~\eqref{eq:kh11posEqJSol}.

By substitution, now the \eqref{eq:kh11h} becomes
$$L \frac{(1+J)H}{w +(1+J)H} - H \left( \alpha - \sigma \frac{F}{J}{1+J} \right) = 0, $$
which gives us the first part of \eqref{eq:kh11posEqJSol}.






