\chapter{Biologia dell'ape europea}
\label{chap:bio}
Questo studio si concentra sulla specie \emph{Apis mellifera} (Linneo, 1758) del genere \emph{Apis},
comunemente nota come ``ape europea'' o ``ape occidentale''.
Questa specie ha avuto origine in Africa,\footcite{origins}
diffondendosi poi in Europa, Medio Oriente ed Asia -- dove già esistevano altre specie di \emph{Apis} --
durante l'ultimo milione di anni.

In epoche più recenti la specie umana ha contribuito alla dispersione di \emph{A.~mellifera} su tutti i continenti
(esclusa l'Antartide).\footcite{origins,honeyreligion}

Il rapporto tra \emph{H. sapiens} e \emph{A. mellifera} ha origine nel Pleistocene con le prime rudimentali interazioni
di tipo puramente predatorio (figura~\ref{img:arana}), evolvendosi nell'arco dei millenni in forme complesse e reciprocanti.

\begin{displayquote}[\cite{honeyreligion}]
``Humans have eaten, and fixed their wounds with honey and traded with it since history has been recorded.''
\end{displayquote}

L'interazione simbiotica uomo--ape è oggi articolata in forme complesse, che comprendono:
\begin{itemize}
    \item La selezione genetica: l'uomo seleziona tratti comportamentali vantaggiosi, come la docilità e una ridotta
        tendenza alla sciamatura, per aumentare l’efficienza nella produzione di miele, cera, pappa reale
        e migliorare la resistenza alle malattie.\footcite{behavGenetics,algerianHoney}

        Il genoma di \emph{A. mellifera} è stato interamente sequenziato nel 2006.\footcite{genomeSeq}
    \item La dispersione antropocora: l'espansione di \emph{A.~mellifera} nell'Eurasia continentale è avvenuta in
        almeno due momenti distinti, ed è molto probabile che gli eventi di diffusione più recenti
        siano stati catalizzati dall'uomo.\footcite{antropocora}\footnote{Vedasi la figura~\ref{img:europ}.}

        L'introduzione di svariate sottospecie di \emph{A. mellifera} in America Latina (XVI sec.), in America
        del Nord (XVII sec.) ed in Australia (XIX sec.) è interamente dovuta all'opera umana.
    \item La dimensione economica: il valore di mercato ``diretto'' dei prodotti dell'apicoltura è stimato intorno agli 8 miliardi di \$ annui. \footcite{honeymarket1}

        Il valore ``indiretto'' dovuto all'impollinazione delle colture nel settore agricolo è di gran lunga superiore, stimato tra i 235 e i 570 miliardi di \$ all'anno. \footcite{honeymarket2,honeymarket3}
    \item La dimensione ecologica: la presenza del genere \emph{Apis} contribuisce -- assieme ad altri impollinatori -- al mantenimento della biodiversità negli ecosistemi non disturbati dall'attività umana, alla stabilizzazione degli ambienti antropizzati; il declino degli impollinatori aggrava gli effetti negativi del cambiamento climatico e dell'industrializzazione dell'agricoltura. \footcite{decline}
\end{itemize}

\begin{figure}
    \centering
    \includegraphics[keepaspectratio,width=0.98\textwidth]{img/tesiFDL-distribuzioneRazzeEuropaClean}

    \caption[Distribuzione delle razze d'api in Europa]{``Distribuzione
    delle razze d'api in Europa'', da \cite[24]{tesiFDL}.}
    \label{img:europ}
\end{figure}

L'ape occidentale presenta una organizzazione \emph{eusociale}, considerata una delle forme più
avanzate di organizzazione sociale negli animali.

L'eusocialità è definita dalle seguenti caratteristiche:
\begin{itemize}
    \item La cura della covata è una attività \emph{cooperativa}; gli elementi della colonia dediti a tale attività si occupano anche della progenie degli altri individui.
    \item Sovrapposizione tra le generazioni di individui adulti in una colonia completamente sviluppata.
    \item Suddivisione del lavoro che include la separazione tra i compiti riproduttivi e non riproduttivi.
\end{itemize}

L'organizzazione eusociale di \emph{A.~mellifera} ci permette di concettualizzare l'intero alveare come un macro--organismo, le cui capacità trofiche, riproduttive e di interazione con l'ambiente sono il risultato emergente dalla collettività dei comportamenti degli individui che costituiscono la colonia.

La suddivisione del lavoro in un alveare riflette la partizione in caste fertili (regina, fuchi) e non fertili (operaie) della popolazione.

\begin{figure}[hbp]
    \centering
    \includegraphics[keepaspectratio,width=0.76\textwidth]{img/uovaLarve}

    \caption[Covata giovane.]{Porzione di favo con covata: le celle in alto a sinistra contengono larve di 5--8 giorni; nelle altre celle è presente un uovo (< 3 giorni). \\ (Waugsberg -- 22 Aprile 2007 -- CC BY-SA 3.0)}
    \label{img:uovaLarve}
\end{figure}

L'allevamento della covata (figure~\ref{img:uovaLarve},~\ref{img:lifecycle}) avviene in modo collettivo: la regina depone le uova fecondate, mentre le operaie nutrici si occupano della pulizia, nutrimento e cura delle uova e delle larve.

Al termine dello stadio larvale le operaie nutrici chiudono la cella con un opercolo di cera; all'interno della cella la pupa svolge una metamorfosi da cui emerge dopo 10--11 giorni un'ape adulta.

\begin{figure}[hbp]
    \centering
    \includegraphics[keepaspectratio,width=0.76\textwidth]{img/honeybeeLifecycle}

    \caption[Ciclo vitale dell'ape.]{Schema della riproduzione di \emph{A. mellifera}.
        \\ (illustrazione di Jennifer Sartell (dett.) -- 22 Aprile 2017 -- ©)}
    \label{img:lifecycle}
\end{figure}

\paragraph{}
La popolazione di un alveare varia notevolmente col ciclo delle stagioni e mediamente comprende un'ape regina (unica femmina fertile), da 20000 a 100000 operaie (femmine sterili) e qualche centinaia di fuchi (maschi fertili) in primavera/estate.

Le tre caste presentano notevoli differenze morfologiche, che riflettono le mansioni di ognuna all'interno dell'alveare.
Ad esempio nei fuchi e nella regina le ghiandole ceripare e faringee sono decisamente meno sviluppate che nelle operaie; anche il loro apparato boccale è ridotto e per questo motivo sia la regina che i maschi vengono alimentati dalle operaie nutrici.

\paragraph{}
È importante precisare che un'ape operaia svolge ruoli differenti nella colonia durante l'arco della sua vita.

Una giovane adulta rimane perlopiù all'interno del nido:
\begin{itemize}
    \item come \emph{nutrice} curando la covata, la regina e i fuchi,
    \item come \emph{ceraiola} modellando la cera secreta da speciali ghiandole per costruire il favo,
    \item come \emph{spazzina} che pulisce il nido e cura l'igiene della matrice cerea,
    \item come \emph{guardiana} difendendo l'ingresso da animali estranei.
\end{itemize}

Un'altra funzione importante per il mantenimento della covata è la termoregolazione dell'alveare (figura~\ref{img:fanning}):
le operaie di nido contrastano il freddo formando il \emph{glomere}, ed il caldo eccessivo sfruttando il calore latente di evaporazione dell'acqua.

\begin{figure}[hbp]
    \centering
    \includegraphics[keepaspectratio,width=0.76\textwidth]{img/fanning}

    \caption[Ventilazione forzata per raffrescare l'interno dell'arnia.]{Gruppo di operaie in
        prossimità dell'ingresso dell'alveare ``ventilano'' per abbassare la
        temperatura interna dell'arnia.
        \\ (Ken Thomas -- 26 Maggio 2008 -- Pubblico dominio)
    }
    \label{img:fanning}
\end{figure}

\paragraph{}
Soltanto in età avanzata, con un bagaglio di esperienze che informa le abilità cognitive avanzate di cui è capace, l'ape operaia si avventura fuori dal nido per esplorare, individuare e discriminare le fonti di cibo, raccogliere polline e nettare (``bottinare'') dalle fioriture, importare acqua nell'arnia e scambiare informazioni con le compagne.

Le capacità cognitive per svolgere tali compiti richiedono la coordinazione di svariate abilità, tra cui la navigazione, il conteggio, la misura di distanze ed angoli.
È dimostrato che le bottinatrici più anziane sono più efficienti nelle attività all'esterno del nido, rispetto alle operaie reclutate in età precoce che sono più lente e imprecise nel volo.

\paragraph{}
Dunque le operaie giovani svolgono ruoli fondamentali all'interno dell'alveare come la cura della covata e dell'igiene, e attraverso il reclutamento diventano abili bottinatrici in volo in età più avanzata.
È importante notare la crucialità di entrambe le dinamiche, perché riguardano due aspetti fondamentali di ogni organismo vivente: la \emph{riproduzione} (la cura della covata) e l'approvvigionamento di cibo.

La colonia deve costantemente perseguire un delicato equilibrio nell'allocazione di risorse per soddisfare questi due bisogni fondamentali.
Ne segue che ogni modello matematico che voglia risultare sufficientemente accurato nel riassumere la dinamica di una colonia di \emph{A. mellifera} debba comprendere un effetto di tipo Allee perlomeno sulla popolazione di operaie giovani
\emph{ed anche} sul compartimento delle operaie bottinatrici più anziane.

L'\emph{effetto Allee} in biologia è un fenomeno che correla positivamente la densità di popolazione alla
\emph{fitness} genetica media individuale, per livelli bassi di popolazione.
In generale è vero che un individuo ha migliori probabilità di sopravvivenza se incontra minore competizione per le
limitate risorse disponibili, quindi il successo riproduttivo medio è correlato positivamente a densità di popolazione
più basse (cfr.~crescita \emph{logistica} a p.~\pageref{sec:logistic}).
Però, quando la popolazione è ``drasticamente'' bassa, si può osservare la correlazione inversa: la \emph{fitness}
individuale media migliora con le densità di popolazione relativamente più alte.
Le cause e le modalità di espressione dell'effetto Allee varia da una specie all'altra, e dipende anche dall'ambiente;
si tende a spiegare il fenomeno con l'innesco di relazioni intra--specie di tipo collaborativo per livelli bassi di
popolazione, e di relazioni competitive per grandi popolazioni.

\paragraph{}
Nella tabella~\ref{tab:FDLruoliEta} sono illustrati alcuni ruoli svolti dalle api operaie ed il loro periodo di insorgenza.
Per una disamina più dettagliata dei meccanismi di transizione tra le caste di operaie in \emph{A.~mellifera},
si veda \cite{meccanica}.

Una ape ``estiva'' vive fino a 30--40~giorni, mentre le operaie ``invernali'' nate in autunno
riescono in buona parte a sopravvivere fino alla ripresa della deposizione.

\begin{table}[pbh]
    \centering
    \begin{tabular}{rcc}
        \toprule
        funzione & durata (gg) & età (gg) \\
        \midrule
        riposo e inizio lavori interni & 3 & 0--3 \\
        pulizia dell'arnia e dei favi  & 3 & 4--6 \\
        nutrice                        & 7 & 7--13 \\
        secrezione cera, guardiania, ventilazione & 6 & 14--19 \\
        bottinatrice                            & 20--25 & 20--45 \\
        \bottomrule
    \end{tabular}
    \caption{Demografia media per un alveare sano. Dati~\cite{balio}.}
    \label{tab:FDLruoliEta}
\end{table}




\section{Organismi patogeni per l'ape}
L'ape europea presenta una varietà di antagonisti naturali, appartenenti a diversi regni.

Nell'ambito dei virus troviamo il ``virus della paralisi acuta''\footnote{Acute Bee Paralysis Virus -- ABPV.}, il ``virus della regina nera'' (Black Queen Cell Virus) ed altri (Israeli Acute Paralysis Virus, Kashmir Bee Virus, Cloudy Wing Virus, \dots) nella famiglia dei \emph{Dicistroviridae}.

\paragraph{}
Il nosema (\emph{Nosema ceranae)} è un microsporidio (regno \emph{Fungi}) che nasce come parassita di \emph{Apis cerana} (la specie asiatica di ape mellifera) ma che attacca anche \emph{A. mellifera}.

Le api sono predate abitualmente da alcune specie di uccelli (ad es. il gruccione, \emph{Merops apiaster}) e da alcuni insetti (ad es. il calabrone, v.~figura~\ref{img:calabron}).

\begin{figure}[hbt]
    \centering
    \includegraphics[keepaspectratio,width=0.76\textwidth]{img/calabron}

    \caption[Ape predata da un calabrone.]{Una calabrone operaia (\emph{Vespa Crabro}) si
        nutre di una ape operaia (\emph{Apis mellifera}) appena catturata.
        \\ (Böhringer Friedrich -- 20 Agosto 2004 -- CC BY-SA 2.5)
    }
    \label{img:calabron}
\end{figure}

Alcune specie non parassitano direttamente le api ma altre componenti dell'alveare, come la falena \emph{Galleria mellonella} che si nutre degli esoscheletri delle pupe di ape sfarfallate, di cera e di tracce di polline che trova nelle celle abbandonate dalle api adulte.

\subsection{\emph{Varroa}}
Il patogeno di gran lunga più studiato è la varroa, acaro appartenente alle due specie \emph{Varroa destructor} e \emph{V.~jacobsoni}. La varroa parassita le larve, le pupe e le api adulte, nutrendosi delle componenti grasse e dell'emolinfa.

\paragraph{}
Il ciclo vitale della varroa è suddiviso nella fase riproduttiva e nella fase foretica:
\begin{itemize}
    \item Durante la fase riproduttiva, una femmina feconda di varroa si introduce in una cella contenente una larva, prima dell'opercolazione. Dopo l'opercolazione depone le uova da cui fuoriescono gli individui fertili, che si accoppiano immediatamente e si nutrono sulla larva.

    Con lo sfarfallamento dell'ape adulta, la varroa madre e le giovani femmine fecondate fuoriescono dalla cella, eventualmente attaccandosi al corpo dell'ape.
    I maschi, terminata la loro funzione riproduttiva, muoiono dentro la cella.
    \item La fase foretica avviene nelle varroe adulte che si attaccano sulle api operaie, nascondendosi negli spazi interstiziali tra i segmenti dell'addome.
\end{itemize}

La varroa accorcia l'aspettativa di vita e indebolisce le api adulte che attacca; determina inoltre malformazioni (ad es. alle ali) nelle operaie che vengono parassitate allo stadio larvale, compromettendone le abilità al volo.

È dimostrato inoltre che la varroa funge da vettore per molti patogeni delle api, sia batterici che virali, tra cui l'ABPV.

\paragraph{}
Il sistema immunitario di una colonia, che comprende le barriere fisiologiche a livello di individuo ed i comportamenti sociali di tipo igienico, viene compromesso dal parassitaggio di \emph{V.~destructor} e dalle patologie che essa trasmette.

Questo meccanismo di indebolimento di un sistema di difesa della colonia aggiunge un ulteriore livello di complessità ai processi di interazione tra api e varroe.

\paragraph{}
Una assunzione ubiquitaria nei modelli matematici apistici che considerano l'infestazione da \emph{V.~destructor} è
che le attività sostanziali della regina all'interno della colonia non sono disturbate dalla presenza di varroa e dei patogeni ad essa correlati.

Ciò corrisponde all'esperienza sul campo\footcite{privFDL,privFPan} ed alla letteratura disponibile, secondo le quali l'acaro non si trova \emph{praticamente mai} sul corpo della regina. Non è chiaro  se ciò sia dovuto alla particolare attenzione con cui le operaie attorno alla regina svolgono attività di \emph{grooming}.

\subsection{Declino dovuto all'attività umana}
Molte classi di pesticidi, erbicidi ed altri antiparassitari di sintesi impiegati in agricoltura danneggiano le api: le bottinatrici entrano in contatto con queste sostanze chimiche posandosi sulle colture trattate, importano nel nido polline e nettare contaminati, dove vengono manipolati anche dalle operaie di alveare.

Le molecole \emph{imidacloprid}, \emph{clothianidin}, e \emph{thiamethoxam} (nella classe dei neonicotinoidi) sono state correlate alla sindrome di spopolamento (Colony Collapse Disorder -- CCD) in cui si verifica un crollo repentino della popolazione in alveari apparentemente sani, che rende impossibile la sopravvivenza della famiglia.
Queste molecole vengono considerate anche causa o concausa di patologie cognitive delle api adulte, tra cui una riduzione della memoria a breve termine ed il deterioramento delle facoltà di navigazione: le bottinatrici affette perdono l'orientamento e non riescono a tornare al nido, morendo in breve tempo (per il sopraggiungere della notte o di un predatore).

\paragraph{}
Studi recenti suggeriscono che il CCD sia causato da una \emph{combinazione di fattori}, tra cui l’uso di pesticidi,
la perdita di biodiversità, il cambiamento climatico, l’alterazione delle fioriture e la diffusione di patogeni,
soprattutto legati a \emph{V.~destructor}.

Questa combinazione di concause produce effetti ramificati sul comportamento degli alveari, che a loro volta influenzano il processo di declino della popolazione di api.

\paragraph{}
Ad esempio, sia \emph{imidacloprid} che il parassitaggio da \emph{V.~destructor} inibiscono la memoria spaziale e le abilità di volo delle bottinatrici esperte, che ``si perdono'' più spesso e non riescono a tornare all'alveare (\emph{homing failure}), circostanza quasi sempre fatale. Ciò determina un deficit di operaie anziane rispetto all'equilibrio della popolazione, a cui la colonia reagisce innescando meccanismi sociali che accelerano il processo di reclutamento, per cui le operaie vengono ``promosse'' all'esterno del nido in età precoce.

Le bottinatrici più giovani sono quindi più inesperte e meno efficienti nel bottinare, l'importazione di cibo nella colonia diminuisce; inoltre si perdono più spesso, aumentando così il ``costo unitario'' che la colonia sostiene per una unità di miele importato.

Il reclutamento precoce causa anche una diminuzione del numero di operaie di alveare e più precisamente di nutrici:
con meno cure, la frazione di covata che raggiunge l'età adulta diminuisce per l'aumento di morti precoci allo stadio larvale.

\paragraph{}
\label{par:socialRecr}
Il meccanismo di regolazione sociale del reclutamento è un comportamento naturale di \emph{A.~mellifera}
di risposta allo stress su due elementi vitali principali della colonia -- la \emph{riproduzione} e l'\emph{alimentazione} -- ed in condizioni tipiche garantisce la sopravvivenza della stessa.

Nello scenario d'esempio qui sopra, le attività dannose dell'uomo e di altre specie si intersecano ai molti piani delle attività biologiche di una colonia o di un apiario, i cui parametri vitali sembrano abbastanza stabili nel tempo, ma che improvvisamente collassano per l'occorrenza contemporanea di danni irreversibili a \emph{molti} processi fondamentali che sostengono la vita eusociale dell'ape.

