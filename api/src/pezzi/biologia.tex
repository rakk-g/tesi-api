\chapter{Biologia dell'ape europea}
\label{chap:bio}
Questo studio si concentra sulla specie \emph{Apis mellifera} (Linneo, 1758) del genere \emph{Apis},
comunemente nota come ``ape europea'' o ``ape occidentale''.
Questa specie ha avuto origine in Africa ed in Medio Oriente, diffondendosi poi su ogni continente (esclusa l'Antartide) durante gli scorsi 2 milioni di anni.

\begin{figure}
    \centering
    \includegraphics[keepaspectratio,width=0.56\textwidth]{img/AranaCaves}

    \caption[Pittura rupestre di Arana.]{Pittura rupestre datata 8000 a.C. circa, Cuevas de la Arana, Valencia, Spagna. \\ (Museo Nacional de Ciencias Naturales, Madrid)}
    \label{img:arana}
\end{figure}

Il rapporto tra \emph{H. sapiens} e \emph{A. mellifera} ha origine nel Pleistocene con le prime rudimentali interazioni
di tipo puramente predatorio (figura~\ref{img:arana}), evolvendosi nell'arco dei millenni in forme complesse e reciprocanti.

\begin{displayquote}[{\footcite{honeyreligion}}]
``Humans have eaten, and fixed their wounds with honey and traded with it since history has been recorded.''
\end{displayquote}

L'interazione simbiotica uomo--ape è oggi articolata in forme complesse, che comprendono:
\begin{itemize}
    \item La selezione genetica: l'uomo tenta di rinforzare i tratti comportamentali vantaggiosi che incidono sugli aspetti sociali delle colonie, \footcite{behavGenetics}
    di aumentarne l'efficienza nella produzione di miele, cera, pappa reale, \footcite{algerianHoney}
    oppure di migliorarne la resistenza alle malattie.

    Il genoma di \emph{A. mellifera} è stato interamente sequenziato nel 2006. \footcite{genomeSeq}
    \item La dispersione antropocora: l'espansione di \emph{A. mellifera} nell'Eurasia continentale è avvenuta in almeno due momenti distinti, ed è molto probabile che gli eventi di diffusione più recenti siano stati catalizzati dall'uomo.
    \footcite{antropocora}
    \footnote{Vedasi la figura~\ref{img:europ}.}

        L'introduzione di svariate sottospecie di \emph{A. mellifera} in America Latina (XVI sec.), in America del Nord (XVII sec.) ed in Australia (XIX sec.) è interamente dovuta all'opera umana.
    \item La dimensione economica: il valore di mercato ``diretto'' dei prodotti dell'apicoltura è stimato intorno agli 8 miliardi di \$ annui. \footcite{honeymarket1}

        Il valore ``indiretto'' dovuto all'impollinazione delle colture nel settore agricolo è di gran lunga superiore, stimato tra i 235 e i 570 miliardi di \$ all'anno. \footcite{honeymarket2,honeymarket3}
    \item La dimensione ecologica: la presenza del genere \emph{Apis} contribuisce -- assieme ad altri impollinatori -- al mantenimento della biodiversità negli ecosistemi non disturbati dall'attività umana, alla stabilizzazione degli ambienti antropizzati; il declino degli impollinatori aggrava gli effetti negativi del cambiamento climatico e dell'industrializzazione dell'agricoltura. \footcite{decline}
\end{itemize}

\begin{figure}
    \centering
    \includegraphics[keepaspectratio,width=0.98\textwidth]{img/tesiFDL-distribuzioneRazzeEuropaClean}

    \caption[Distribuzione delle razze d'api in Europa]{Figura~``Distribuzione
    delle razze d'api in Europa'', da \cite[24]{tesiFDL}.}
    \label{img:europ}
\end{figure}

L'ape occidentale presenta una organizzazione \emph{eusociale}, ossia la più elevata forma di organizzazione sociale realizzata da una popolazione animale. L'eusocialità è definita dalle seguenti caratteristiche:
\begin{itemize}
    \item La cura della covata è una attività \emph{cooperativa}; gli elementi della colonia dediti a tale attività si occupano anche della progenie degli altri individui.
    \item Sovrapposizione tra le generazioni di individui adulti in una colonia completamente sviluppata.
    \item Suddivisione del lavoro che include la separazione tra i compiti riproduttivi e non riproduttivi.
\end{itemize}

L'organizzazione eusociale di \emph{A.~mellifera} ci permette di concettualizzare l'intero alveare come un macro--organismo, le cui capacità trofiche, riproduttive e di interazione con l'ambiente sono il risultato emergente dalla collettività dei comportamenti degli individui che costituiscono la colonia.

La suddivisione del lavoro in un alveare riflette la partizione in caste fertili (regina, fuchi) e non fertili (operaie) della popolazione.

\begin{figure}
    \centering
    \includegraphics[keepaspectratio,width=0.86\textwidth]{img/uovaLarve}

    \caption[Covata giovane.]{Porzione di favo con covata: le celle in alto a sinistra contengono larve di 5--8 giorni; nelle altre celle è presente un uovo (< 3 giorni). \\ (Waugsberg -- 22 Aprile 2007 -- CC BY-SA 3.0)}
    \label{img:uovaLarve}
\end{figure}

L'allevamento della covata (figure~\ref{img:uovaLarve},~\ref{img:lifecycle}) avviene in modo collettivo: la regina depone le uova fecondate, mentre le operaie nutrici si occupano della pulizia, nutrimento e cura delle uova e delle larve.

Al termine dello stadio larvale le operaie nutrici chiudono la cella con un opercolo di cera; all'interno della cella la pupa svolge una metamorfosi da cui emerge dopo 10--11 giorni un'ape adulta.

\begin{figure}
    \centering
    \includegraphics[keepaspectratio,width=0.86\textwidth]{img/honeybeeLifecycle}

    \caption[Ciclo vitale dell'ape.]{Schema della riproduzione di \emph{A. mellifera}.
            %1) La regina depone un uovo in una cella (da 1 a 3 giorni).
            %2) Larva (da 3 a 8/9 giorni).
            %3) Le nutrici chiudono la cella con l'opercolo (9° giorno).
            %4) La pupa dentro la cella (dal 10° giorno).
            %5) L'ape adulta emerge dalla cella (21° giorno).
        \\ (illustrazione di Jennifer Sartell (dett.) -- 22 Aprile 2017 -- ©)}
    \label{img:lifecycle}
\end{figure}

\paragraph{}
La popolazione di un alveare varia notevolmente col ciclo delle stagioni e mediamente comprende un'ape regina (unica femmina fertile), da 20000 a 100000 operaie (femmine sterili) e qualche centinaia di fuchi (maschi fertili) in primavera/estate.

Le tre caste presentano notevoli differenze morfologiche, che riflettono le mansioni di ognuna all'interno dell'alveare.
Ad esempio nei fuchi e nella regina le ghiandole ceripare e faringee sono decisamente meno sviluppate che nelle operaie; anche il loro apparato boccale è ridotto e per questo motivo sia la regina che i maschi vengono alimentati dalle operaie nutrici.

\paragraph{}
È importante precisare che un'ape operaia svolge ruoli differenti nella colonia durante l'arco della sua vita.

Una giovane adulta rimane perlopiù all'interno del nido:
\begin{itemize}
    \item come \emph{nutrice} curando la covata, la regina e i fuchi,
    \item come \emph{ceraiola} modellando la cera secreta da speciali ghiandole per costruire il favo,
    \item come \emph{spazzina} che pulisce il nido e cura l'igiene della matrice cerea,
    \item come \emph{guardiana} difendendo l'ingresso da animali estranei.
\end{itemize}

Un'altra funzione importante per il mantenimento della covata è la termoregolazione dell'alveare (figura~\ref{img:fanning}):
le operaie di nido contrastano il freddo formando il \emph{glomere} ed il caldo eccessivo sfruttando il calore latente di evaporazione dell'acqua.

\paragraph{}
Soltanto in età avanzata, con un bagaglio di esperienze che informa le abilità cognitive avanzate di cui è capace, l'ape operaia si avventura fuori dal nido per esplorare, individuare e discriminare le fonti di cibo, raccogliere polline e nettare (``bottinare'') dalle fioriture, importare acqua nell'arnia e scambiare informazioni con le compagne.

Le capacità cognitive per svolgere tali compiti richiedono la coordinazione di svariate abilità, tra cui la navigazione, il conteggio, la misura di distanze ed angoli.
È dimostrato che le bottinatrici più anziane sono più efficienti nelle attività all'esterno del nido, rispetto alle operaie reclutate in età precoce che sono più lente e imprecise nel volo.

\paragraph{}
Dunque le operaie giovani svolgono ruoli fondamentali all'interno dell'alveare come la cura della covata e dell'igiene, e attraverso il reclutamento diventano abili bottinatrici in volo in età più avanzata.
È importante notare la crucialità di entrambe le dinamiche, perché riguardano due aspetti fondamentali di ogni organismo vivente: la \emph{riproduzione} (la cura della covata) e l'approvvigionamento di cibo.

La colonia deve costantemente perseguire un delicato equilibrio nell'allocazione di risorse per soddisfare questi due bisogni fondamentali.
Ne segue che ogni modello matematico che voglia risultare sufficientemente accurato nel riassumere la dinamica di una colonia di \emph{A. mellifera} debba comprendere un effetto Allee perlomeno sulla popolazione di operaie giovani
\emph{ed anche} sulla popolazione di operaie bottinatrici più anziane.

\section{Organismi patogeni per l'ape}
L'ape europea presenta una varietà di antagonisti naturali, appartenenti a diversi regni.

Nell'ambito dei virus troviamo il ``virus della paralisi'' (Acute Bee Paralysis Virus -- ABPV), il ``virus della regina nera'' (Black Queen Cell Virus) ed altri (Israeli Acute Paralysis Virus, Kashmir Bee Virus, Cloudy Wing Virus, \dots) nella famiglia dei \emph{Dicistroviridae}.

Il nosema (\emph{Nosema ceranae)} è un microsporidio (regno \emph{Fungi}) che nasce come parassita di \emph{Apis cerana} (la specie asiatica di ape mellifera) ma che attacca anche \emph{A. mellifera}.

Le api sono predate abitualmente da alcune specie di uccelli (ad es. il gruccione, \emph{Merops apiaster}) e da alcuni insetti (ad es. il calabrone, v.~figura~\ref{img:calabron}).

Alcune specie non parassitano direttamente le api ma altre componenti dell'alveare, come la falena \emph{Galleria mellonella} che si nutre degli esoscheletri delle pupe di ape sfarfallate, di cera e di tracce di polline che trova nelle celle abbandonate dalle api adulte.

\subsection{\emph{Varroa}}
Il patogeno di gran lunga più studiato è la varroa, acaro appartenente alle due specie \emph{Varroa destructor} e \emph{V.~jacobsoni}. La varroa parassita le larve, le pupe e le api adulte, nutrendosi delle componenti grasse e dell'emolinfa.

\paragraph{}
Il ciclo vitale della varroa è suddiviso nella fase riproduttiva e nella fase foretica:
\begin{itemize}
    \item Durante la fase riproduttiva, una femmina feconda di varroa si introduce in una cella contenente una larva, prima dell'opercolazione. Dopo l'opercolazione depone le uova da cui fuoriescono gli individui fertili, che si accoppiano immediatamente e si nutrono sulla larva.

    Con lo sfarfallamento dell'ape adulta, la varroa madre e le giovani femmine fecondate fuoriescono dalla cella, eventualmente attaccandosi al corpo dell'ape.
    I maschi, terminata la loro funzione riproduttiva, muoiono dentro la cella.
    \item La fase foretica avviene nelle varroe adulte che si attaccano sulle api operaie, nascondendosi negli spazi interstiziali tra i segmenti dell'addome.
\end{itemize}

La varroa accorcia l'aspettativa di vita e indebolisce le api adulte che attacca; determina inoltre malformazioni (ad es. alle ali) nelle operaie che vengono parassitate allo stadio larvale, compromettendone le abilità al volo.

È dimostrato inoltre che la varroa funge da vettore per molti patogeni delle api, sia batterici che virali, tra cui l'ABPV.

\paragraph{}
Il sistema immunitario di una colonia, che comprende le barriere fisiologiche a livello di individuo ed i comportamenti sociali di tipo igienico, viene compromesso dal parassitaggio di \emph{V.~destructor} e dalle patologie che essa trasmette.

Questo meccanismo di indebolimento di un sistema di difesa della colonia aggiunge un ulteriore livello di complessità ai processi di interazione tra api e varroe.

\paragraph{}
Una assunzione ubiquitaria nei modelli matematici apistici che considerano l'infestazione da \emph{V.~destructor} è
che le attività sostanziali della regina all'interno della colonia non sono disturbate dalla presenza di varroa e dei patogeni ad essa correlati.

Ciò corrisponde all'esperienza sul campo\footcite{privFDL,privFPan} ed alla letteratura disponibile, secondo le quali l'acaro non si trova \emph{praticamente mai} sul corpo della regina. Non è chiaro  se ciò sia dovuto alla particolare attenzione con cui le operaie attorno alla regina svolgono attività di \emph{grooming}.

\subsection{Declino dovuto all'attività umana}
Molte classi di pesticidi, erbicidi ed altri antiparassitari di sintesi impiegati in agricoltura danneggiano le api: le bottinatrici entrano in contatto con queste sostanze chimiche posandosi sulle colture trattate, importano nel nido polline e nettare contaminati, dove vengono manipolati anche dalle operaie di alveare.

Le molecole \emph{imidacloprid}, \emph{clothianidin}, e \emph{thiamethoxam} (nella classe dei neonicotinoidi) sono state correlate alla sindrome di spopolamento (Colony Collapse Disorder -- CCD) in cui si verifica un crollo repentino della popolazione in alveari apparentemente sani, che rende impossibile la sopravvivenza della famiglia.
Queste molecole vengono considerate anche causa o concausa di patologie cognitive delle api adulte, tra cui una riduzione della memoria a breve termine ed il deterioramento delle facoltà di navigazione: le bottinatrici affette perdono l'orientamento e non riescono a tornare al nido, morendo in breve tempo (per il sopraggiungere della notte o di un predatore).

\paragraph{}
Gli studi più recenti ridimensionano il ruolo dei pesticidi chimici nel CCD ed indicano piuttosto \emph{la combinazione di una molteplicità di fattori} come causa scatenante del declino delle colonie di ape europea.

Tali fattori includono l'uso di pesticidi ed erbicidi sulle colture, la perdita di biodiversità, il cambiamento climatico, l'alterazione delle fioriture (disponibilità di cibo per la colonia) e la proliferazione di patologie, soprattutto legate a \emph{V.~destructor}.

Questa combinazione di concause produce effetti ramificati sul comportamento degli alveari, che a loro volta influenzano il processo di declino della popolazione di api.

\paragraph{}
Ad esempio, sia \emph{imidacloprid} che il parassitaggio da \emph{V.~destructor} inibiscono la memoria spaziale e le abilità di volo delle bottinatrici esperte, che ``si perdono'' più spesso e non riescono a tornare all'alveare (\emph{homing failure}), circostanza quasi sempre fatale. Ciò determina un deficit di operaie anziane rispetto all'equilibrio della popolazione, a cui la colonia reagisce innescando meccanismi sociali che accelerano il processo di reclutamento, per cui le operaie vengono ``promosse'' all'esterno del nido in età precoce.

Le bottinatrici più giovani sono quindi più inesperte e meno efficienti nel bottinare, l'importazione di cibo nella colonia diminuisce; inoltre si perdono più spesso, aumentando così il ``costo unitario'' che la colonia sostiene per una unità di miele importato.

Il reclutamento precoce causa anche una diminuzione del numero di operaie di alveare e più precisamente di nutrici:
con meno cure, la frazione di covata che raggiunge l'età adulta diminuisce per l'aumento di morti precoci allo stadio larvale.

\paragraph{}
Il meccanismo di regolazione sociale del reclutamento è un comportamento naturale di \emph{A.~mellifera}
di risposta allo stress su due elementi vitali principali della colonia -- la \emph{riproduzione} e l'\emph{alimentazione} -- ed in condizioni tipiche garantisce la sopravvivenza della stessa.

Nello scenario d'esempio qui sopra, le attività dannose dell'uomo e di altre specie si intersecano ai molti piani delle attività biologiche di una colonia o di un apiario, i cui parametri vitali sembrano abbastanza stabili nel tempo, ma che improvvisamente collassano per l'occorrenza contemporanea di danni irreversibili a \emph{molti} processi fondamnetali che sostengono la vita eusociale dell'ape.


\begin{figure}
    \centering
    \includegraphics[keepaspectratio,width=0.86\textwidth]{img/fanning}

    \caption[Ventilazione forzata per raffrescare l'interno dell'arnia.]{Gruppo di operaie in prossimità dell'ingresso dell'alveare ``ventilano'' per abbassare la temperatura interna dell'arnia.
        \\ (Ken Thomas -- 26 Maggio 2008 -- Pubblico dominio)}
    \label{img:fanning}
\end{figure}

\begin{figure}
    \centering
    \includegraphics[keepaspectratio,width=0.86\textwidth]{img/calabron}

    \caption[Ape predata da un calabrone.]{Una calabrone operaia (\emph{Vespa Crabro}) si nutre di una ape operaia (\emph{Apis mellifera}) appena catturata. \\ (Böhringer Friedrich -- 20 Agosto 2004 -- CC BY-SA 2.5)}
    \label{img:calabron}
\end{figure}


% SECTION
\section{Ingredienti utili per lo studio dei modelli}
\label{sec:ingredientiBioMat}
I modelli matematici sono dati generalmente sotto forma di sistemi dinamici, \ie di sistemi di equazioni
differenziali del tipo
$$\diff{}{t} \mathbf{x} = F( \mathbf{x} ) \; ,$$
data $F: W \to \R^n$ con $W \subseteq \R^n$ aperto ed $F$ generalmente supposta abbastanza regolare.
%\footnote{A seconda dei contesti, si può richiedere $F$ di classe $C^1$ o soltanto differenziabile.}

Le equazioni differenziali date sopra, unitamente con le condizioni iniziali $\mathbf{x} (t_0) = \mathbf{x}_0$,
costituiscono un \emph{problema di Cauchy}, ed il Teorema di Esistenza e Unicità garantisce che le sue soluzioni
esistano per ogni condizione iniziale $\mathbf{x}_0$, definite in almeno un intervallo.

\paragraph{}
Risultati generali sui sistemi dinamici sono ampiamente disponibili in letteratura e ne diamo un riassunto
a pagina~\pageref{chap:teoria} nell'appendice~\ref{chap:teoria}.

In questa sezione ci soffermeremo piuttosto su alcuni strumenti matematici che trovano applicazione nel campo
più specifico della biomatematica, ossia nella modellazione di popolazioni delle specie animali, vegetali
o microbiche.

\paragraph{}
Intorno al XIX secolo le scienze matematiche iniziano a rivolgersi al campo biologico con rinnovato interesse:
Pierre--François Verhulst propone di utilizzare la \emph{funzione logistica} nel 1838, dopo aver letto il famoso
modello di Malthus.\footcite{malthus1986essay}
La derivazione di Verhulst è correlata anche al modello preda--predatore, pubblicato da Alfred
Lotka nel 1920, che estende la sua applicazione precedente alle reazioni autocatalitiche.

Lo stesso sistema di equazioni viene pubblicato da Vito Volterra nel 1926
\footcite{vito},
che si avvicina all'applicazione della matematica alle scienze biologiche grazie alle interazioni col
biologo marino Umberto D'Ancona.

Negli ultimi due secoli e mezzo sono stati fatti molti progressi in campo matematico e nelle scienze biologiche;
tuttavia, gli studi più recenti lamentano ancora una compartimentazione troppo netta ed auspicano una maggiore
collaborazione tra le scienze matematiche e la ricerca bio-etologica sul campo.

\paragraph{}
Presentiamo in questa sezione alcuni utili strumenti per i sistemi attinenti alla ricerca biomatematica.
Vedremo alcuni oggetti definiti all'epoca di Verhulst, che continuano ad offrire intuizione ed ispirazione
attraverso i secoli, ed argomentazioni più moderne che attorno al principio del terzo millennio hanno iniziato
a combinare lo sguardo matematico con la simulazione numerica, sfruttando la forza delle macchine di calcolo.

% subsection
\section{Funzione Logistica}
La \emph{funzione logistica} è una famiglia di sigmoidi che
\blockquote[\cite{WENlogistic}]{``\omissis trova applicazione in molti campi,
compresi: biologia (specialmente ecologia), biomatematica, chimica, demografia, economia,
geoscienze, psicomatematica, probabilità, sociologia, scienze politiche, linguistica,
statistica e reti neurali.

Ne esistono varie generalizzazioni, a seconda del campo di applicazione.''}

\paragraph{}
Consideriamo dapprima il modello malthusiano\footcite{malthus1986essay} in cui il tasso di crescita di una popolazione $P$
è proporzionale alla popolazione medesima:
$$\diff{P}{t} = rP \; , $$
ove il tasso di riproduzione $r$ è costante. L'equazione differenziale è facilmente integrabile e fornisce la soluzione
$P(t) = P_0 e^{rt}$, in cui $P_0$ rappresenta la popolazione iniziale all'istante $t=0$.

L'assunzione di Malthus è verosimile soltanto nelle fasi iniziali dello sviluppo, ossia
per popolazioni ``relativamente basse'': ad esempio un embrione umano cresce secondo la sequenza
2, 4, 8, 16 cellule etc. subito dopo la fecondazione, ed anche lo sviluppo delle colture batteriche appena inoculate
segue un'andamento esponenziale.

Il feto umano però rallenta il suo sviluppo quando raggiunge dimensioni comparabili con l'utero che lo contiene, così
come la crescita delle popolazioni batteriche satura in vicinanza dei limiti fisici imposti dalla piastra o dall'esaurimento
delle sostanze nutrienti.

La soluzione esponenziale non cattura questi fenomeni reali, e quindi descrive propriamente l'evoluzione di una popolazione soltanto nelle fasi iniziali.

Infatti, in una crescita di tipo esponenziale il tasso di crescita \emph{pro~capite} è costante, quindi la popolazione
aumenta tanto più velocemente quanto è grande il numero di individui.

\paragraph{}
Per risolvere questo problema, Verhulst\footcite{verhulst} introduce l'equazione logistica,
ottenuta supponendo che il tasso di crescita sia proporzionale alla popolazione\footnote{Come già nel modello malthusiano.}
\emph{ma anche} alle risorse disponibili:

\begin{equation}
    \diff{P}{t} = r P \left( 1 - \frac{P}{K} \right) \; ,
    \label{eq:verhulstLogistic}
\end{equation}
dove $K$ rappresenta la \emph{capacità portante} dell'ambiente, ossia il massimo teorico della popolazione
che è possibile sostenere in base alle risorse disponibili.
Il fattore $\frac{K-P}{K}$ nella crescita logistica rappresenta il fatto che il tasso di crescita \emph{pro~capite}
diminuisce quando numero di individui si avvicina al limite ambientale imposto dalla disponibilità di risorse.

Il parametro adimensionale $\frac{P}{K}$ si chiama \emph{competizione intraspecifica} e costituisce la modifica
cruciale di Verhulst al modello malthusiano: rappresenta infatti la diminuzione del tasso di crescita per popolazioni
vicine alla capacità portante, rispetto a popolazioni ``piccole'' (a parità di tasso di riproduzione).

Tramite semplici manipolazioni algebriche possiamo risolvere l'equazione~\eqref{eq:verhulstLogistic} ed ottenere
l'espressione $P=P(t)$ che descrive l'evoluzione temporale della popolazione:
\begin{multline*}
\int_{P_0}^{P(t)} \frac{1}{\pi} + \frac{1}{K-\pi} \de \pi = \int_0^t r \de \tau
\quad \implies \quad
\log \left( \frac{K-P_0}{P_0} \cdot \frac{P}{K-P} \right) = rt
\quad \implies \\
\implies \quad
\frac{K}{P} - 1 = \frac{K-P_0}{P_0} e^{-rt}
\quad \implies \quad
P = \frac{K P_0 e^{rt}}{K + P_0 (e^{rt} -1)} \; ,
\end{multline*}

la quale può essere riscritta come
\begin{equation}
    P(t) = \frac{K}{1 + q e^{-rt}} \; ,
    \label{eq:verhulstLogistic2}
\end{equation}
avendo posto
$$ q \coloneq \frac{K- P_0}{P_0} \; .$$

Si osservi che dalla~\eqref{eq:verhulstLogistic2} segue
$$\lim_{t \to \infty} P(t) = K \; ,$$
ovvero che la popolazione tende alla capacità portante, anziché divergere all'infinito come nel modello esponenziale.

\paragraph{}
Per uno studio analitico più approfondito della funzione logistica conviene adimensionalizzare
la~\eqref{eq:verhulstLogistic2} in modo da ottenere l'equazione della funzione logistica standard:
\begin{equation}
    f(x) \coloneq \frac{1}{1 + e^{-x}} = \frac{e^x}{1+e^x} \; ,
    \label{eq:logisticF}
\end{equation}
la quale è definita per ogni $x \in \R$, anche se nella pratica $f(x)$ si può considerare satura intorno a $\abs{x} > 6$,
come si può vedere in figura~\ref{img:logisticF}.

\begin{figure}[pbh]
    \centering
    \includegraphics[keepaspectratio,width=0.86\textwidth]{img/logisticF}

    \caption[Funzione logistica]{Funzione logistica standard.}
    \label{img:logisticF}
\end{figure}

La funzione logistica è dispari rispetto a $(0, \frac{1}{2})$, e vale $f(0) = \frac{1}{2}$.
Queste osservazioni si applicano anche alla logistica generalizzata, con parametri $L$, $k$ ed $x_0$:
\begin{equation}
    \hat{f}(x) \coloneq \frac{L}{1+e^{-k(x -x_0)}} \; ,
    \label{eq:logisticFgen}
\end{equation}
che è una versione traslata lungo $x$ e riscalata su entrambi gli assi:
\begin{itemize}
    \item $x_0$ adesso è la retroimmagine del semimassimo $\frac{L}{2}$;
    \item è applicata un'omotetia di fattore $\frac{1}{k}$ lungo $x$ ed $L$ lungo $y$.
\end{itemize}

\paragraph{}
Nelle applicazioni è utile conoscere le derivate della $f$ logistica:
nelle figg.~\ref{img:logisticV}, \ref{img:logisticA}, \ref{img:logisticJ} si osservi che
la $f'$ è sempre positiva, e le derivate sono alternativamente pari e dispari, secondo la
facile generalizzazione di un noto risultato.\footnote{Ossia che $f \text{ pari} \implies
    f' \text{ dispari}$ ed analogamente al contrario,
producendo una catena di alternanze.}

\begin{figure}[pbh]
    \centering
    \includegraphics[keepaspectratio,width=0.86\textwidth]{img/logisticV}

    \caption{Derivata logistica prima.}
    \label{img:logisticV}
\end{figure}
\begin{figure}[pbh]
    \centering
    \includegraphics[keepaspectratio,width=0.86\textwidth]{img/logisticA}

    \caption{Derivata logistica seconda.}
    \label{img:logisticA}
\end{figure}
\begin{figure}[pbh]
    \centering
    \includegraphics[keepaspectratio,width=0.86\textwidth]{img/logisticJ}

    \caption{Derivata logistica terza.}
    \label{img:logisticJ}
\end{figure}

Il calcolo esplicito delle derivate della $f$ logistica è facilitato parecchio osservando che
$$\diff{}{x} f(x) = f(x) \left( {1-f(x)} \right) \; ,$$
da cui tutte le derivate successive possono ricavarsi algebricamente.

\paragraph{}
La proprietà di simmetria della funzione logistica
$$1 -f(x) = f(-x)$$
riflette il fatto che la crescita sopra $0$ intorno a valori piccoli di $x$ è
simmetrica rispetto al decadimento al raggiungere il valore--limite di saturazione,
per grandi valori di $x$.

\paragraph{}
Un esempio di applicazione nella dinamica delle popolazioni sarà esaminato nel modello della sezione~\ref{sec:ratti17}:
nelle equazioni \eqref{eq:r17m} e \eqref{eq:r17n} la crescita della popolazione di varroe è di tipo logistico,
e la capacità portante dell'ambiente è proporzionale alla popolazione di api adulte.

\paragraph{}
Concludiamo questa sezione osservando che altre formulazioni sono state proposte ed utilizzate per ottenere
delle sigmoidi nei modelli, ad esempio $\tanh (x)$, $\arctan (x)$, $\erf (x)$ e la funzione di Hill.\footcite{hill}



