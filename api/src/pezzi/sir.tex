\section{Modelli a compartimenti}
I modelli a compartimenti costituiscono una tecnica generale e molto potente per la modellazione, soprattutto nelle
applicazioni di modellazione matematica delle malattie infettive, ambito in cui furono sviluppati nella prima metà
del '900.\footcite{kmk,kdg}

Sono molto utilizzati anche nella modellazione di più popolazioni diverse: l'interazione biologica tra le diverse specie
prese in considerazione può avere carattere predatorio, parassitario, simbiotico, etc.
Questo ultimo ambito è quello di maggior interesse ai fini dei modelli esaminati in questa Tesi, il cui scopo è
la modellazione della popolazione di \emph{Apis~mellifera} tenendo conto della suddivisione in caste del nido, della
presenza di macropatogeni (come la \emph{Varroa}) e di altre infezioni batteriche o virali.

\paragraph{}
In questo tipo di modelli l'intera popolazione è suddivisa in compartimenti etichettati -- ad esempio ``S'' per gli
individui suscettibili ed ``I'' per gli individui infetti -- e le etichette sono da interpretare come variabili del sistema:
ad esempio $I$ è il numero degli individui infetti presenti nella popolazione in un dato momento.

Per specificare completamente il modello a compartimenti occorre anche fornire i tassi di transizione da e verso ogni
compartimento, i quali dipendono dalle variabili e dai parametri del sistema.

\subsection{Modello SIR e sue varianti}
\begin{figure}[hbt]
    \centering
    \includegraphics[keepaspectratio,width=0.58\textwidth]{img/SIRschema}

    \caption{Schema del modello a compartimenti SIR.}
    \label{img:SIRscheme}
\end{figure}

Nonostante la relativa semplicità di questo modello che adotta soltanto tre compartimenti, esso ha un potere predittivo
piuttosto ragionevole\footcite{chinaVirus} nel contesto delle infezioni trasmesse infra--specie in \emph{H.~sapiens}
per le quali la guarigione comporta una resistenza di lungo periodo.\footnote{Ad esempio il morbillo, la parotite,
la rosolia e -- con le dovute cautele -- anche la sindrome respiratoria COVID-19 data dal virus SARS-CoV-2.}

\paragraph{}
Illustrare la derivazione del modello SIR, esaminandone le ipotesi di lavoro, è utile sia per la formalizzazione
di importanti concetti,
\footnote{Quali ad esempio il tasso base di riproduzione $R_0$, l'immunità di gregge, gli
equilibri endemici ed epidemici, \dots}
sia per le modifiche necessarie per la costruzione di modelli più complessi.

Consideriamo dunque una qualche popolazione (animale, vegetale o batterica) soggetta ad una patologia trasmissibile
infra--specie. Sia $N =N(t)$ il numero totale di individui nella popolazione considerata, all'istante $t$.

Suddividiamo la popolazione in tre compartimenti:
\begin{itemize}
    \item il compartimento $S$ rappresenta gli individui \emph{suscettibili}, ossia che possono contrarre
        l'infezione entrando a contatto con un individuo infetto.
    \item Il compartimento $I$ consta degli individui \emph{infetti} della popolazione, e più propriamente
        ``infettivi'' giacché capaci trasmettere l'infezione ad individui sani nel comparto $S$.
    \item Il compartimento $R$ è il gruppo degli individui \emph{rimossi} dalla popolazione infett(iv)a.
        Gli elementi in $R$ non trasmettono e non ricevono l'infezione. Troviamo dunque in $R$ sia gli individui
        infetti che sono guariti (e che dunque hanno sviluppato una immunità di lungo periodo), sia i decessi (che
        in ogni caso non trasmettono né ricevono l'infezione).
\end{itemize}

Si assume inoltre che:
\begin{enumerate}
    \item il tasso di nuove infezioni è proporzionale sia alla popolazione suscettibile che a quella infetta.
        Questa assunzione è giustificata empiricamente dal fatto che la trasmissione del patogeno richiede il
        \emph{contatto} tra un individuo infetto ed uno suscettibile; senza informazioni specifiche ulteriori,
        è ragionevole assumere che il numero di contatti sia proporzionale al prodotto $SI$.
        \footnote{Infatti la semplicità del modello SIR risiede soprattutto nella \emph{assenza di relazioni
            strutturate} entro la popolazione considerata. Modelli sufficientemente informativi sulla dinamica
            di animali che esibiscono comportamenti sociali (quali \emph{H.~sapiens} e \emph{A.~mellifera})
            introducono ulteriori compartimenti e termini di trasferimento proprio per catturare il carattere
            strutturato delle interazioni.}
    \item il tasso delle guarigioni (e dei decessi) è proporzionale al numero $I$ di individui infetti.
\end{enumerate}

Il risultato di queste assunzioni è il modello illustrato in figura~\ref{img:SIRscheme}, con le equazioni seguenti:
\begin{align}
    \diff{S}{t} &= - \beta I S \label{eq:sirS} \\
    \diff{I}{t} &= \beta I S - \gamma I \\
    \diff{R}{t} &= \gamma I \label{eq:sirR} \; ,
\end{align}

in cui il parametro $\beta$ rappresenta il numero medio di contatti tra $S$ ed $I$ (nell'unità di tempo),
moltiplicato per la probabilità di infezione durante un singolo contatto.

Il parametro $\gamma$ è l'inverso del tempo medio di permanenza nello stato infettivo; ovvero è la frequenza
media delle rimozioni dal comparto $I$.

\paragraph{}
Si osservi che questo semplice modello non prevede ingressi nel comparto $S$, ciò che tipicamente rappresenterebbe
le nuove nascite.
Inoltre, poiché i morti sono conteggiati in $R$ assieme alle guarigioni, abbiamo che $\diff{N}{t} = 0$,
da cui si ricava il tasso \emph{pro~capite} di infezioni
$$\diff{(S/N)}{t} = - \frac{\beta I S}{N} \; .$$

Considerando il rapporto
$$\frac{ \diff{S}{t} }{ \diff{R}{t} } = - \frac{\beta}{\gamma} S$$
possiamo ottenere informazioni sull'andamento asintotico dell'infezione: integrando l'equazione precedente otteniamo
$$R(t) = R(0) + \frac{\gamma}{\beta} \log \frac{S(0)}{S(t)} \; .$$

Un esempio di questo andamento è mostrato in figura~\ref{img:sirmodell}: si osservi che l'asse dei tempi è
adimensionalizzato con $d=\gamma^{-1}$ che rappresenta la durata media di una infezione.
Coi valori dei parametri illustrati in didascalia, non vi sono praticamente più individui suscettibili dopo
40 periodi dall'inizio dell'epidemia, e dopo 100 periodi quasi l'intera popolazione è immune.

\begin{figure}[hb]
    \centering
    \includegraphics[keepaspectratio,width=0.78\textwidth]{img/SIR-Modell}

    \caption[Modello SIR]{Andamento di un modello SIR su una popolazione $N=1000$ con $I(0)=3$ e $R(0)=0$.
        I valori dei parametri sono $\beta=0,0004$ e $\gamma=0,04$.}
    \label{img:sirmodell}
\end{figure}

\subsection{Numero di riproduzione di base ed equilibrio endemico}
Negli studi epidemiologici assume un'importanza particolare il \emph{numero di riproduzione di base}, denominato $R_0$.

Questi è il valore atteso del numero di nuovi contagi generati direttamente da un solo individuo infetto, nel contesto
di una malattia infettiva contro la quale non si adottino misure di contrasto.

\paragraph{}
Nei modelli SIR e derivati il numero $R_0$ è correlato al concetto di ``paziente zero'', definito dalle condizioni
iniziali in cui una popolazione è interamente suscettibile: a parte il caso zero, nessun individuo è infetto né immune,
ciò che si traduce in
$$I(0) = 1 \;, \quad S(0) = N(0) - 1 \; .$$

L'argomentazione seguente, che conclude la presente sezione, è esposta considerando il modello SIR a tre compartimenti
per brevità e semplicità, ma può essere estesa ai modelli più complessi\footnote{link SIRVD}
adottati con successo nella modellazione di epidemie reali in \emph{H.~sapiens} ed in altri animali.

Si consideri il modello \eqref{eq:sirS}--\eqref{eq:sirR} e si ponga
\begin{equation}
    R_0 \coloneq \frac{\beta}{\gamma} \; .
    \label{eq:rzero}
\end{equation}

La proporzione di popolazione che rimane suscettibile all'\emph{equilibrio endemico}, cioè nel contesto della presenza
costante dell'infezione nella popolazione $N$, è $\frac{1}{R_0}$ e ci permette di calcolare la
\emph{soglia minima per l'immunità di gregge}\footnote{HIT - Herd Immunity Threshold.}
$$1 - \frac{1}{R_0} \; .$$

Si osservi che in generale $R_0 \neq R_t$, dove $R_t$ rappresenta il valore atteso delle nuove infezioni generate da
un individuo nel comparto $I(t)$ al tempo \emph{attuale}. $R_t$ è denominato \emph{numero effettivo di riproduzione}

\paragraph{}
All'atto pratico, le politiche sanitarie quali ad esempio le vaccinazioni, rimuovono individui dal comparto $S(t)$
trasportandoli al comparto $R(t)$, rendendole immuni ed abbassando la velocità del contagio.

Se la percentuale di popolazione immune è superiore all'immunità di gregge
$$R(t) \geq 1 - \frac{1}{R_0} \; ,$$
allora l'infezione può essere contenuta (equilibrio \emph{endemico}) o addirittura eliminata.

Si distingue quindi la nozione di \emph{endemia} (presenza costante ma contenuta del patogeno nella popolazione)
da quella di \emph{epidemia} o \emph{pandemia}, che invece è il contagio \emph{rapido} di un grande numero di individui
nella popolazione.

\paragraph{}
Concludiamo osservando che l'immunità di gregge non è un concetto statico, ma varia nel tempo.
Così come alcune malattie infettive per gli umani richiedono la ripetizione periodica delle vaccinazioni,
anche alcuni patogeni di \emph{Apis} si riescono a contenere soltanto tramite misure \emph{ripetute} di
immunizzazione. Nell'esempio della \emph{Varroa}, i vari trattamenti\footnote{Ad es. acido ossalico, trattamenti varroacidi
    con o senza ``blocco della regina'' (\ie dell'ovideposizione).}
vengono ripetuti ogni anno e/o durante la stagione.










