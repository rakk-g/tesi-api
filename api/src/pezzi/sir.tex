\section{Modelli a compartimenti}
I modelli a compartimenti costituiscono una tecnica generale e molto potente per la modellazione, soprattutto nelle
applicazioni di modellazione matematica delle malattie infettive, ambito in cui furono sviluppati nella prima metà
del '900.\footcite{kmk,kdg}

Sono molto utilizzati anche nella modellazione di più popolazioni diverse: l'interazione biologica tra le diverse specie
prese in considerazione può avere carattere predatorio, parassitario, simbiotico, etc.
Questo ultimo ambito è quello di maggior interesse ai fini dei modelli esaminati in questa Tesi, il cui scopo è
la modellazione della popolazione di \emph{Apis~mellifera} tenendo conto della suddivisione in caste del nido, della
presenza di macropatogeni (come la \emph{Varroa}) e di altre infezioni batteriche o virali.

\paragraph{}
In questo tipo di modelli l'intera popolazione è suddivisa in compartimenti etichettati -- ad esempio ``S'' per gli
individui suscettibili ed ``I'' per gli individui infetti -- e le etichette sono da interpretare come variabili del sistema:
ad esempio $I$ è il numero degli individui infetti presenti nella popolazione in un dato momento.

Per specificare completamente il modello a compartimenti occorre anche fornire i tassi di transizione da e verso ogni
compartimento, i quali dipendono dalle variabili e dai parametri del sistema.

\subsection{Modello SIR e sue varianti}
Nonostante la relativa semplicità di questo modello che adotta soltanto tre compartimenti, esso ha un potere predittivo
piuttosto ragionevole\footcite{chinaVirus} nel contesto delle infezioni trasmesse infra--specie in \emph{H.~sapiens}
per le quali la guarigione comporta una resistenza di lungo periodo.\footnote{Ad esempio il morbillo, la parotite,
la rosolia e -- con le dovute cautele -- anche la sindrome respiratoria COVID-19 data dal virus SARS-CoV-2.}




