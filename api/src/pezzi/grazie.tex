\cleardoublepage
\addcontentsline{toc}{chapter}{Ringraziamenti} %trick
\section*{Ringraziamenti}
Grazie a Maria Alessandra Soleti, per ogni cosa fondamentale.

\paragraph{}
Al mio relatore prof.~Angiolo Farina, per il lavoro instancabile e preciso; per la curiosità e gli stimoli incessanti.
Egli è anche un ottimo insegnante, ciò che rende inestimabile il suo talento matematico.

\paragraph{}
A mia sorella, a mamma e babbo, ed ai genitori ``supplenti'', per tutto il supporto.

\paragraph{}
Ad Antonio De Dominicis, che mi guida attraverso i meandri della Matematica.
Come fece Virgilio, anche se io in quest'analogia non sono certo Dante.

\paragraph{}
A Maria Vittoria Chellini, per il cammino di formazione fatto fianco a fianco.

\paragraph{}
La comunità degli apicoltori (professionisti e dilettanti) si contraddistingue per una peculiarità
che altrimenti ho ritrovato solo negli ambiti scientifici della ricerca sperimentale: uno spirito di cooperazione
che anima una vasta rete di aziende, cooperative agricole, ricercatori/trici universitari/e di veterinaria, agraria,
matematica ed altre discipline, associazioni di categoria, singoli individui spinti dalla fascinazione per
questo animale\dots

Nella mia esperienza ``ibrida'' di ricercatore matematico e ad un tempo
apicoltore ``semi--professionista''\footnote{\ie un dilettante a stretto contatto con piccole aziende.}
ho sempre riscontrato una \emph{sorprendente} apertura e volontà di collaborazione, nella implicita ed immediata
consapevolezza che \emph{condividere} la conoscenza sia un atto vantaggioso per chiunque, ora e nel futuro.

\paragraph{}
In questi rispetti, la dinamica della comunità degli apicoltori e apicoltrici non è molto differente da
quella di un grande alveare.

\paragraph{}
Particolarmente importanti per la realizzazione di questo lavoro sono state le discussioni con veterinari, biologi,
apicoltori e molti altri professionisti.
Di sicuro, senza il prezioso confronto con Giovanni Guido, Michele Valleri, Francesco Panella
e Francesco De Leo, questo esame di modelli matematici sulle api mancherebbe in larga misura di una ``presa'' sulla realtà:
è grazie al loro contributo se le equazioni qui tracciate sono ``vive e ronzanti''.
% redazione l'Apis?



