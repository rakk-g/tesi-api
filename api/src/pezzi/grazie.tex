\cleardoublepage
\addcontentsline{toc}{chapter}{Ringraziamenti} %trick
\section*{Ringraziamenti}
Grazie a Maria Alessandra Soleti, per le cose fondamentali.

Al mio relatore prof.~Angiolo Farina, per il lavoro instancabile e preciso; per la curiosità e gli stimoli incessanti.
Egli è un ottimo insegnante, e ciò rende inestimabile il suo talento matematico.

A Spicci, mamma e babbo, ed ai genitori ``supplenti'', per tutto il supporto.

\paragraph{}
La comunità degli apicoltori professionisti e dilettanti si contraddistingue per una peculiarità
che altrimenti ho ritrovato solo negli ambiti scientifici della ricerca sperimentale: uno spirito di cooperazione
che anima una vasta rete di aziende, cooperative agricole, ricercatori/trici universitari/e di veterinaria, agraria,
matematica ed altre discipline, associazioni di categoria, singoli individui spinti dalla fascinazione per
questo animale\dots

Nella mia esperienza ``ibrida'' di ricercatore matematico e ad un tempo
apicoltore ``semi--professionista''\footnote{\ie un dilettante a stretto contatto con piccole aziende.}
ho sempre riscontrato una \emph{sorprendente} apertura e volontà di collaborazione, nella implicita ed immediata
consapevolezza che \emph{condividere} la conoscenza sia un atto vantaggioso per tutti i soggetti coinvolti.

\paragraph{}
In questi rispetti, la dinamica della comunità degli apicoltori/apicoltrici, non è molto differente da
quella di un grande alveare.

Particolarmente importanti per la realizzazione di questo lavoro sono state le discussioni coi veterinari apistici
e gli apicoltori professionisti: senza il prezioso confronto con Giovanni Guido, Michele Valleri, Francesco Panella
e Francesco De Leo, questo esame di modelli matematici mancherebbe in larga misura di una ``presa'' sulla realtà.




