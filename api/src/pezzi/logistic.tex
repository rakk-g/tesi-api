\subsection{Funzione Logistica}
La \emph{funzione logistica} è una famiglia di sigmoidi che
\textquote[\footcite{WENlogistic}]{\omissis trova applicazione in molti campi,
compresi: biologia (specialmente ecologia), biomatematica, chimica, demografia, economia,
geoscienze, psicomatematica, probabilità, sociologia, scienze politiche, linguistica,
statistica e reti neurali.

Ne esistono varie generalizzazioni, a seconda del campo di applicazione.}

\paragraph{}
L'equazione della funzione logistica standard è
\begin{equation}
    f(x) \coloneq \frac{1}{1 + e^{-x}} = \frac{e^x}{1+e^x} \; ,
    \label{eq:logisticF}
\end{equation}
ed è definita per ogni $x \in \R$, anche se in pratica $f(x)$ satura intorno a $\abs{x} > 6$.
Vedasi la figura~\ref{img:logisticF}.

\begin{figure}[pbh]
    \centering
    \includegraphics[keepaspectratio,width=0.86\textwidth]{img/logisticF}

    \caption[Funzione logistica]{Funzione logistica.}
    \label{img:logisticF}
\end{figure}

La funzione logistica è dispari rispetto a %$y=\frac{1}{2}$ e $0 = f^{-1} \left( \frac{1}{2} \right)$.
$(0, \frac{1}{2})$, ed $f(0) = \frac{1}{2}$.
Queste osservazioni si applicano anche alla logistica generalizzata, con parametri $L$, $k$ ed $x_0$:
\begin{equation}
    \hat{f}(x) \coloneq \frac{L}{1+e^{-k(x -x_0)}} \; ,
    \label{eq:logisticFgen}
\end{equation}
che è una versione traslata lungo $x$ e riscalata su entrambi gli assi:
\begin{itemize}
    \item $x_0$ adesso è la retroimmagine del semimassimo $\frac{L}{2}$;
    \item è applicata un'omotetia di fattore $\frac{1}{k}$ lungo $x$ ed $L$ lungo $y$.
\end{itemize}

\paragraph{}
Nelle applicazioni è utile conoscere le derivate della $f$ logistica:
nelle figg.~\ref{img:logisticV}, \ref{img:logisticA}, \ref{img:logisticJ} si osservi che
la $f'$ è sempre positiva, e le derivate sono alternativamente pari e dispari, secondo la
facile generalizzazione di un noto risultato.\footnote{Ossia che $f \text{ pari} \implies
    f' \text{ dispari}$ ed analogamente al contrario,
producendo una catena di alternanze.}

\begin{figure}[pbh]
    \centering
    \includegraphics[keepaspectratio,width=0.86\textwidth]{img/logisticV}

    \caption{Derivata logistica prima.}
    \label{img:logisticV}
\end{figure}
\begin{figure}[pbh]
    \centering
    \includegraphics[keepaspectratio,width=0.86\textwidth]{img/logisticA}

    \caption{Derivata logistica seconda.}
    \label{img:logisticA}
\end{figure}
\begin{figure}[pbh]
    \centering
    \includegraphics[keepaspectratio,width=0.86\textwidth]{img/logisticJ}

    \caption{Derivata logistica terza.}
    \label{img:logisticJ}
\end{figure}

Il calcolo esplicito delle derivate della $f$ logistica è facilitato parecchio osservando che
$$\diff{}{x} f(x) = f(x) \left( {1-f(x)} \right) \; ,$$
da cui tutte le derivate successive possono ricavarsi algebricamente.

\paragraph{}
La proprietà di simmetria della logistica
$$1 -f(x) = f(-x)$$
riflette il fatto che la crescita sopra $0$ intorno a valori piccoli di $x$ è
simmetrica rispetto al decadimento al raggiungere il valore--limite di saturazione,
per grandi valori di $x$.


% e l'integrale?
