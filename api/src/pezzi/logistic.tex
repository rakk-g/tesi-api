\section{Funzione Logistica}
La \emph{funzione logistica} è una famiglia di sigmoidi che
\blockquote[\cite{WENlogistic}]{``\omissis trova applicazione in molti campi,
compresi: biologia (specialmente ecologia), biomatematica, chimica, demografia, economia,
geoscienze, psicomatematica, probabilità, sociologia, scienze politiche, linguistica,
statistica e reti neurali.

Ne esistono varie generalizzazioni, a seconda del campo di applicazione.''}

\paragraph{}
Consideriamo dapprima il modello malthusiano\footcite{malthus1986essay} in cui il tasso di crescita di una popolazione $P$
è proporzionale alla popolazione medesima:
$$\diff{P}{t} = rP \; , $$
ove il tasso di riproduzione $r$ è costante. L'equazione differenziale è facilmente integrabile e fornisce la soluzione
$P(t) = P_0 e^{rt}$, in cui $P_0$ rappresenta la popolazione iniziale all'istante $t=0$.

L'assunzione di Malthus è verosimile soltanto nelle fasi iniziali dello sviluppo, ossia
per popolazioni ``relativamente basse'': ad esempio un embrione umano cresce secondo la sequenza
2, 4, 8, 16 cellule etc. subito dopo la fecondazione, ed anche lo sviluppo delle colture batteriche appena inoculate
segue un'andamento esponenziale.

Il feto umano però rallenta il suo sviluppo quando raggiunge dimensioni comparabili con l'utero che lo contiene, così
come la crescita delle popolazioni batteriche satura in vicinanza dei limiti fisici imposti dalla piastra o dall'esaurimento
delle sostanze nutrienti.

La soluzione esponenziale non cattura questi fenomeni reali, e quindi descrive propriamente l'evoluzione di una popolazione soltanto nelle fasi iniziali.

Infatti, in una crescita di tipo esponenziale il tasso di crescita \emph{pro~capite} è costante, quindi la popolazione
aumenta tanto più velocemente quanto è grande il numero di individui.

\paragraph{}
Per risolvere questo problema, Verhulst\footcite{verhulst} introduce l'equazione logistica,
ottenuta supponendo che il tasso di crescita sia proporzionale alla popolazione\footnote{Come già nel modello malthusiano.}
\emph{ma anche} alle risorse disponibili:

\begin{equation}
    \diff{P}{t} = r P \left( 1 - \frac{P}{K} \right) \; ,
    \label{eq:verhulstLogistic}
\end{equation}
dove $K$ rappresenta la \emph{capacità portante} dell'ambiente, ossia il massimo teorico della popolazione
che è possibile sostenere in base alle risorse disponibili.
Il fattore $\frac{K-P}{K}$ nella crescita logistica rappresenta il fatto che il tasso di crescita \emph{pro~capite}
diminuisce quando numero di individui si avvicina al limite ambientale imposto dalla disponibilità di risorse.

Il parametro adimensionale $\frac{P}{K}$ si chiama \emph{competizione intraspecifica} e costituisce la modifica
cruciale di Verhulst al modello malthusiano: rappresenta infatti la diminuzione del tasso di crescita per popolazioni
vicine alla capacità portante, rispetto a popolazioni ``piccole'' (a parità di tasso di riproduzione).

Tramite semplici manipolazioni algebriche possiamo risolvere l'equazione~\eqref{eq:verhulstLogistic} ed ottenere
l'espressione $P=P(t)$ che descrive l'evoluzione temporale della popolazione:
\begin{multline*}
\int_{P_0}^{P(t)} \frac{1}{\pi} + \frac{1}{K-\pi} \de \pi = \int_0^t r \de \tau
\quad \implies \quad
\log \left( \frac{K-P_0}{P_0} \cdot \frac{P}{K-P} \right) = rt
\quad \implies \\
\implies \quad
\frac{K}{P} - 1 = \frac{K-P_0}{P_0} e^{-rt}
\quad \implies \quad
P = \frac{K P_0 e^{rt}}{K + P_0 (e^{rt} -1)} \; ,
\end{multline*}

la quale può essere riscritta come
\begin{equation}
    P(t) = \frac{K}{1 + q e^{-rt}} \; ,
    \label{eq:verhulstLogistic2}
\end{equation}
avendo posto
$$ q \coloneq \frac{K- P_0}{P_0} \; .$$

Si osservi che dalla~\eqref{eq:verhulstLogistic2} segue
$$\lim_{t \to \infty} P(t) = K \; ,$$
ovvero che la popolazione tende alla capacità portante, anziché divergere all'infinito come nel modello esponenziale.

\paragraph{}
Per uno studio analitico più approfondito della funzione logistica conviene adimensionalizzare
la~\eqref{eq:verhulstLogistic2} in modo da ottenere l'equazione della funzione logistica standard:
\begin{equation}
    f(x) \coloneq \frac{1}{1 + e^{-x}} = \frac{e^x}{1+e^x} \; ,
    \label{eq:logisticF}
\end{equation}
la quale è definita per ogni $x \in \R$, anche se nella pratica $f(x)$ si può considerare satura intorno a $\abs{x} > 6$,
come si può vedere in figura~\ref{img:logisticF}.

\begin{figure}[pbh]
    \centering
    \includegraphics[keepaspectratio,width=0.86\textwidth]{img/logisticF}

    \caption[Funzione logistica]{Funzione logistica standard.}
    \label{img:logisticF}
\end{figure}

La funzione logistica è dispari rispetto a $(0, \frac{1}{2})$, e vale $f(0) = \frac{1}{2}$.
Queste osservazioni si applicano anche alla logistica generalizzata, con parametri $L$, $k$ ed $x_0$:
\begin{equation}
    \hat{f}(x) \coloneq \frac{L}{1+e^{-k(x -x_0)}} \; ,
    \label{eq:logisticFgen}
\end{equation}
che è una versione traslata lungo $x$ e riscalata su entrambi gli assi:
\begin{itemize}
    \item $x_0$ adesso è la retroimmagine del semimassimo $\frac{L}{2}$;
    \item è applicata un'omotetia di fattore $\frac{1}{k}$ lungo $x$ ed $L$ lungo $y$.
\end{itemize}

\paragraph{}
Nelle applicazioni è utile conoscere le derivate della $f$ logistica:
nelle figg.~\ref{img:logisticV}, \ref{img:logisticA}, \ref{img:logisticJ} si osservi che
la $f'$ è sempre positiva, e le derivate sono alternativamente pari e dispari, secondo la
facile generalizzazione di un noto risultato.\footnote{Ossia che $f \text{ pari} \implies
    f' \text{ dispari}$ ed analogamente al contrario,
producendo una catena di alternanze.}

\begin{figure}[pbh]
    \centering
    \includegraphics[keepaspectratio,width=0.86\textwidth]{img/logisticV}

    \caption{Derivata logistica prima.}
    \label{img:logisticV}
\end{figure}
\begin{figure}[pbh]
    \centering
    \includegraphics[keepaspectratio,width=0.86\textwidth]{img/logisticA}

    \caption{Derivata logistica seconda.}
    \label{img:logisticA}
\end{figure}
\begin{figure}[pbh]
    \centering
    \includegraphics[keepaspectratio,width=0.86\textwidth]{img/logisticJ}

    \caption{Derivata logistica terza.}
    \label{img:logisticJ}
\end{figure}

Il calcolo esplicito delle derivate della $f$ logistica è facilitato parecchio osservando che
$$\diff{}{x} f(x) = f(x) \left( {1-f(x)} \right) \; ,$$
da cui tutte le derivate successive possono ricavarsi algebricamente.

\paragraph{}
La proprietà di simmetria della funzione logistica
$$1 -f(x) = f(-x)$$
riflette il fatto che la crescita sopra $0$ intorno a valori piccoli di $x$ è
simmetrica rispetto al decadimento al raggiungere il valore--limite di saturazione,
per grandi valori di $x$.

\paragraph{}
Un esempio di applicazione nella dinamica delle popolazioni sarà esaminato nel modello della sezione~\ref{sec:ratti17}:
nelle equazioni \eqref{eq:r17m} e \eqref{eq:r17n} la crescita della popolazione di varroe è di tipo logistico,
e la capacità portante dell'ambiente è proporzionale alla popolazione di api adulte.

\paragraph{}
Concludiamo questa sezione osservando che altre formulazioni sono state proposte ed utilizzate per ottenere
delle sigmoidi nei modelli, ad esempio $\tanh (x)$, $\arctan (x)$, $\erf (x)$ e la funzione di Hill.\footcite{hill}

