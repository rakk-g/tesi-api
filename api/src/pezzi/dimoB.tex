\chapter{Teoria di Floquet}
\label{chap:floquetTheory}
Gaston Floquet (1847, Épinal -- 1920, Nancy)
appartiene al novero ``glorioso'' dei matematici francesi che a cavallo del XIX~secolo rinnovarono
le fondamenta e le prospettive dell'Analisi e di altri campi scientifici.
\footnote{Senza pretesa di esaustività citiamo tra i connazionali di Floquet, all'incirca suoi coevi: Sophie Germain,
Jacques Charles François Sturm, Henri Poincaré, Henri Padé, Gaspard Monge, Joseph Liouville,
Jules Antoine Lissajous, Adrien-Marie Legendre, Joseph-Louis Lagrange (n.~Giuseppe Luigi Lagrangia),
Camille Jordan, Charles Hermite, Henri Lebesgue, Pierre-Simon Laplace,
Évariste Galois, Jean Frédéric Frenet, Joseph Fourier, Jean Gaston Darboux, Gaspard-Gustave de Coriolis,
Auguste Comte, André-Louis Cholesky, Augustin-Louis Cauchy, Lazare Carnot, Émile Borel.}

La Teoria di Floquet è la branca dello studio delle equazioni differenziali a coefficienti periodici,
fondata all'epoca dello studio seminale\footcite{gFloquet} di Floquet, in cui è enunciato il
risultato cruciale che ci permette di scrivere la matrice fondamentale di sistemi del tipo~\eqref{eq:sdcGenerale}
in cui la $F$ è periodica in $t$.

\paragraph{}
Consideriamo un sistema omogeneo, lineare e periodico della forma
\begin{equation}
    \diff{}{t} \mathbf{x} = A(t) \cdot \mathbf{x} \; ,
    \label{eq:sdcPeriodico}
\end{equation}
in cui la mappa $t \mapsto A(t) \in \R^{n \times n}$ è periodica: esiste ben definito $T >0$ tale che
$$A(t+T) =A(t) \quad \forall t \in \R \; .$$

\begin{definizione}[Matrice fondamentale]
    Una mappa
    $$\Psi (t) : \; \R \rightarrow \C^{n \times n}$$
    è una \emph{matrice fondamentale} per il sistema~\eqref{eq:sdcPeriodico} se le sue colonne sono soluzioni
    linearmente indipendenti per ogni $t \in \R$.
\end{definizione}
Equivalentemente, $\Psi$ è una matrice fondamentale se $\Imm \Psi \subseteq GL(n)$ e
$$ \dot{\Psi} (t) = A(t) \Psi (t) \quad (\forall t \in \R) \; .$$

Osserviamo che la matrice fondamentale non è unica, ma lo diventa se imponiamo $\Psi(0)=I_n$, ciò
che è sempre possibile.
\footnote{A meno di premoltiplicare per $\Psi^{-1} (0)$.}

\begin{definizione}[Moltiplicatori di Floquet]
    Una matrice
    $$C= \Psi^{-1} (0) \Psi (T) $$
    si chiama \emph{matrice di monodromia} del sistema.

    Gli autovalori $\mu_j$ della matrice di monodromia $C$ si chiamano \emph{moltiplicatori di Floquet}.
\end{definizione}

Se assumiamo senza perdita di generalità che $\Psi (0) = I_n$, la matrice di monodromia si riduce a $C= \Psi (T)$.

La non unicità della matrice di monodromia è conseguenza delle ramificazioni del logaritmo complesso
di matrici, la cui esistenza è necessaria per la dimostrazione del Teorema di Floquet, assieme ai
seguenti Lemmi:
\begin{itemize}
    \item Se $\det A \neq 0$ allora $A \cdot \Psi (t)$ è fondamentale $(\forall t)$.
    \item $\Psi (t)$ fondamentale $\implies \Psi (t+T)$ fondamentale $(\forall t)$, ove $T$ è il periodo.
    \item $\exists C$ tale che $\Psi (t+T) = C \Psi (t) \; (\forall t)$.\footnote{Una $C$ siffatta è la matrice di monodromia.}
\end{itemize}

\paragraph{}
Il seguente risultato è noto come Teorema di Bloch nelle applicazioni di Fisica della materia condensata.

\begin{teorema}[di Floquet]
    Consideriamo un sistema omogeneo a coefficienti periodici della forma~\eqref{eq:sdcPeriodico}, e
    sia $\Psi (t)$ la sua matrice fondamentale con $\Psi (0) = I_n$.

    Esiste una decomposizione in \emph{Forma normale di Floquet} della matrice fondamentale data da
    $$\Psi (t) = Q(t) e^{tB} \; ,$$
    dove la mappa $Q : \R \rightarrow GL(n)$ è $T$--periodica, di classe $C^1$ e verifica $Q(0) = I_n$.
    \label{teo:floquet}
\end{teorema}
Si osservi che la matrice di monodromia si ottiene quindi dalla seguente:
$$C= \Psi (T) = e^{TB} \; .$$

\begin{definizione}[Esponenti di Floquet]
    Gli autovalori $\phi_j$ di $B$ si chiamano \emph{esponenti di Floquet} del sistema.
\end{definizione}

Anche la $B$ non è univocamente determinata,\footnote{Per lo stesso motivo per cui non lo è la matrice di monodromia.}
dunque non lo sono neanche gli esponenti di Floquet. Si osservi però che per via della periodicità vale
la seguente relazione tra gli
esponenti ed i moltiplicatori di Floquet:
$$\mu_j = e^{\phi_j T} \; .$$

\paragraph{}
Grazie alla forma normale di Floquet siamo in grado di formulare il seguente risultato,
analogo al Teorema~\ref{teo:pozzoNonLineare} visto nel caso dei sistemi autonomi.

\begin{teorema}
    Siano $\mu_j$ i moltiplicatori di Floquet di un sistema omogeneo a coefficienti periodici.
    Sia $D \subset \C$ il disco unitario $D= { \abs{z} \leq 1 }$.

    L'equilibrio $\mathbf{0}$ del sistema è
    \begin{itemize}
        \item asintoticamente stabile se e soltanto se tutti i moltiplicatori di Floquet stanno nella
        parte interna di $D$;
        \item stabile se tutti i $\mu_j \in D$, e quelli sul bordo sono semisemplici;
        \item instabile altrimenti.
    \end{itemize}
    \label{teo:stabPeriod}
\end{teorema}
