\chapter{Propagazione delle soluzioni}
Il supporto del calcolatore elettronico ha un valore inestimabile nella ricerca Matematica.

Nella applicazione d'interesse -- i modelli matematici per le api -- viene utilizzato
per manipolare le equazioni dei modelli, sia tramite il calcolo simbolico che svolgendo simulazioni numeriche.

\paragraph{}
Il calcolo simbolico è stato usato come supporto per le dimostrazioni.

Nello specifico, per la dimostrazione delle affermazioni che accompagnano i modelli\footcite{khoury2011,ratti2017}
esaminati nel capitolo~\ref{chap:modelz}, ci si è avvalsi del software libero \texttt{Maxima},
versione \texttt{5.47.0} impiegando la GUI \texttt{wxMaxima}, versione \texttt{22.09.0}.

\paragraph{}
L'Analisi Numerica viene invece impiegata per approssimare le soluzioni di una vasta classe di sistemi di equazioni,
che spesso non ammettono una risoluzione analitica.

È possibile inoltre ripetere l'approssimazione di una soluzione di un problema di Cauchy del tipo~\eqref{eq:sistDin}
variandone i parametri e le condizioni iniziali: in questo modo è possibile indagare le relazioni tra i parametri e
la dinamica del sistema, ad esempio le regioni di parametri in cui il sistema è stabile.

\paragraph{}
L'approssimazione numerica di una soluzione problema di Cauchy della forma~\eqref{eq:sistDin}
\begin{equation}
\begin{sistema}
\diff{}{t} \mathbf{x} = F( \mathbf{x}, t )  \\
\mathbf{x}(t_0) = \mathbf{x}_0
\end{sistema}
\; ,
\label{eq:probCauchy} % unused TODO
\end{equation}
generalmente si ottiene con i software dedicati al calcolo numerico\footnote{Ad esempio \texttt{Octave} o
\texttt{Mathematica}.} chiamando una routine--solutore
\begin{equation}
\mathbf{y} = \mathtt{sol} ( F, \mathbf{x}_0, t_0, \Delta t, p) \; .
\label{eq:numSolutor}
\end{equation}
Nella pseudo--sintassi di $\mathtt{sol}$ nella~\eqref{eq:numSolutor} si vuole indicare che la routine prende in ingresso
\begin{itemize}
    \item una specifica della $F$, la rappresentazione del membro destro dell'equazione~\eqref{eq:sistDin}.
        Qui si possono anche variare i parametri del sistema;
    \item le condizioni iniziali $\mathbf{x} (t_0) = \mathbf{x}_0$;
    \item il passo di integrazione $\Delta t$ ;
    \item ulteriori parametri $p$: scelta del metodo di approssimazione numerica, di adattamento dinamico del passo
        $\Delta t$, dei criteri d'arresto, etc.
\end{itemize}
Generalmente una siffatta routine $\mathtt{sol}$ ritorna un array di valori in virgola mobile\footnote{Più
precisamente \emph{una matrice}, in cui ogni colonna rappresenta il vettore--soluzione $\mathbf{x}(t)$ approssimato
ad un certo tempo $t$.}
\begin{equation}
\mathbf{y} = \left( y_0, y_0, \dots y_N \right) \; ,
\label{eq:approxSol}
\end{equation}
i quali verificano $y_0 \approx \mathbf{x}_0$ e
$$\frac{y_i - y_{i-1}}{\Delta t} \approx F \left( y_i, i \cdot \Delta t \right) \, , \quad 0 < i \leq N.$$

Ciò rappresenta l'approssimazione \emph{finita} della soluzione di un sistema dinamico, la cui natura intrinsecamente
infinitesimale non è rappresentabile in un calcolatore con risorse limitate.

\paragraph{}
L'intera branca matematica dell'Analisi Numerica è dedicata allo studio di metodi di approssimazione numerica come quello
appena descritto, progettando nuovi algoritmi più precisi ed efficienti, stabilendo limiti teorici a ciò che è possibile
calcolare efficientemente, fornendo limitazioni superiori per l'errore (la differenza tra la soluzione calcolata e la soluzione esatta), costituendo un vastissimo corpus di risultati, che esula dagli scopi di quanto qui esposto.

Per una introduzione all'Analisi Numerica si rimanda ad alcuni testi classici\footcite{ananum,ananumlin,demmel}.

\paragraph{}
Ai fini della presente, si osservi che una soluzione del tipo~\eqref{eq:approxSol}, ottenuta applicando il solutore
numerico~\eqref{eq:numSolutor} ai modelli di \emph{Apis} esposti nel capitolo~\ref{chap:modelz} fornisce l'andamento
di una soluzione approssimata $\mathbf{y}$ sotto forma di matrice: il vettore--colonna $y_i$ contiene il valore di
tutti i compartimenti al tempo $t = i \Delta t$.

% TODO figura con l'andamento di *una* soluzione.

Questo tipo di simulazione numerica è stato utilizzato dagli autori dei modelli\footcite{khoury2011,ratti2017}
sia per validare empiricamente i risultati teorici dimostrati nei rispettivi articoli, sia per estendere
quanto già stabilito che per indirizzare la ricerca futura.

\paragraph{}
Presentiamo adesso alcune nuove simulazioni svolte per il modello a due compartimenti\footcite{khoury2011}
della Sezione~\ref{sec:kh11}, che confermano ed estendono alcuni risultati degli autori.

Nella vasta classe di metodi per l'approssimazione di soluzioni di un sistema di equazioni
non lineari troviamo i metodi Runge--Kutta, che includono il metodo di Eulero:
nel caso presente è stato usato il software libero \texttt{GNU Octave}, versione \texttt{9.2.0} e nello specifico
la routine \texttt{ode45} che implementa il metodo di Dormand--Prince al quarto ordine, un solutore
di equazioni differenziali ordinarie della famiglia Runge--Kutta.

\paragraph{}
Alcuni sistemi presentano equazioni cosiddette \emph{rigide} poiché alcuni metodi numerici si rivelano
molto instabili numericamente a meno di prendere un passo d'approssimazione esageratamente breve.
Questo fenomeno si è verificato nella nostra sperimentazione in alcune regioni di parametri ragionevolmente piccole.

Si noti inoltre che nella chiamata al solutore~\eqref{eq:numSolutor} si è generalmente utilizzato $\Delta t=1~\text{giorno}$
ed un criterio d'arresto $T=k \cdot 365,25$ per terminare la simulazione dopo $k$ anni.

In alcune chiamate al solutore, l'iterazione si è arrestata per $t < T$, ossia prima della soglia richiesta, per
l'occorrenza del fenomeno di instabilità numerica esposto sopra, oppure per la divergenza della popolazione in almeno
uno dei compartimenti in esame.

In ogni caso, un filtraggio in fase di \emph{post--processing} dei dati ottenuti permette di scartare le iterazioni
invalide, e di calcolare numeri di condizionamento\footnote{Il \emph{numero di condizionamento} $\rho$ di un sistema
descrive l'instabilità numerica di una sua soluzione approssimata: per $\rho \gg 1$ la precisione richiesta può essere
fuori dalla portata delle risorse computazionali a disposizione.}
irragionevolmente alti per le istanze corrispondenti, in linea con i risultati teorici.\footcite{demmel}



% Link al codice? Licens? % TODO
\section{Experiment A, iteration 2}

\subsection{Rapidità di estinzione e mortalità}
\begin{figure}[pbh]
    \centering
    \includegraphics[keepaspectratio,width=0.98\textwidth]{img/k11EA2-fdooVSm}

    \caption[Experiment A2:1]{}

    \label{img:kh11expA21}
\end{figure}

\subsection{Popolazione finale e iniziale}
TODO: colora in base a $m$. % TODO
\begin{figure}[pbh]
    \centering
    \includegraphics[keepaspectratio,width=0.98\textwidth]{img/k11EA2-fpopVSipop}

    \caption[Experiment A2:2]{}

    \label{img:kh11expA22}
\end{figure}

\subsection{Popolazione finale e $w$}
\begin{figure}[pbh]
    \centering
    \includegraphics[keepaspectratio,width=0.98\textwidth]{img/k11EA2-finpopVSw}

    \caption[Experiment A2:3]{}

    \label{img:kh11expA23}
\end{figure}

\subsection{$m_i^*$ e $w$}
\begin{figure}[pbh]
    \centering
    \includegraphics[keepaspectratio,width=0.98\textwidth]{img/k11EA2-mistarVSw}

    \caption[Experiment A2:4]{}

    \label{img:kh11expA24}
\end{figure}


\section{Experiment C, iteration 2, data A and B}
Two iterations were made of this simulation with constant parameters $\alpha = 0.25$, $\sigma=0.75$, $L=2500$:
data \texttt{A} was obtained with $w=27000$, while the value $w=9800$ was used for data \texttt{B}.

In both experiments a random value for the parameter $m$ was obtained from a uniform distribution on $[0.006, \, 0.6]$
with a resolution of 30. For each $m$ we selected at random initial values $1 \leq H_0 \leq 15000$ and $1 \leq F_0 \leq 6000$ and ran 30 such simulations for a total of 900 days.

\subsection{Data A}
\label{sec:kh11expC2A}
Note that with $w=27000$ and the other parameter values, the condition~\eqref{eq:cond6b} from Theorem~\ref{teo:esistenzPosF} is true: $\alpha - \frac{L}{w} > 0$, which is case 1 of the Theorem,
when a positive equilibrium exists for mortalities $m$ bound from above and from below.

In the remainder of this section, we will call ``weak'' colonies those which have final total
population $N(900)$ less than $N_- = 100$ bees. We will say that a colony is ``strong'' if
its final population is more than $N_+ = 40000$ workers.

Results are plotted in Figure~\ref{img:kh11expC2A}.

\paragraph{}
Outside the shown plot range were final populations in the unreasonable order of
magnitude of $10^6$ and beyond, clearly indicating a diverging behaviour.
All these points lie on the left of $m_1^* \simeq 0.0804$.\footnote{Cfr.~interpretation at page~\pageref{par:interpretationCond6b}.}

\begin{figure}[pbh]
    \centering
    \includegraphics[keepaspectratio,width=0.96\textwidth]{img/kh11expC2A}

    \caption[Experiment C2A]{Experiment C2A: total number of worker bees $N(t)= H(t)+F(t)$ is shown at $t=0$ and $t=900$ for different levels of forager mortality $m$.

    With the terminology estabilished in page \pageref{sec:kh11expC2A}, we see that
    initial conditions of colonies that become ``weak'' (blue squares) lie all in
    the half--plane $m>0.5078$ (dotted vertical line).
    Similarly, initial conditions of colonies that eventually become ``strong''
    (red x's) all lie around the line $m=0.0804$ (dashed).

    In black we see the initial conditions (+) and final populations (triangles) for colonies that
    don't get to be ``strong'' nor ``weak'' at day $t=900$.

    Final population numbers for ``strong'' hives (purple circles) all happen in the left--hand side
    of the spectrum $0.0804 < m < 0.5078$.
    }

    \label{img:kh11expC2A}
\end{figure}

