\chapter{Propagazione delle soluzioni}
Il calcolatore elettronico è stato utilizzato nella presente tesi per assistere nelle dimostrazioni (calcolo simbolico)
e per le simulazioni dei modelli (calcolo numerico).

Nello specifico, per la dimostrazione delle affermazioni che accompagnano i modelli\footcite{khoury2011,ratti2017}
esaminati nel capitolo~\ref{chap:modelz}, ci si è avvalsi del software libero \texttt{Maxima},
versione \texttt{5.47.0} impiegando la GUI \texttt{wxMaxima}, versione \texttt{22.09.0}.

\paragraph{}
Le simulazioni impiegano invece alcune tecniche standard dell'Analisi Numerica,
che risolvendo le equazioni di un modello forniscono l'andamento temporale dei compartimenti $x_i (t)$.

Per una introduzione all'Analisi Numerica si rimanda ad alcuni testi classici\footcite{ananum,ananumlin,demmel}.

\paragraph{}
Un esempio di simulazione del modello a 5 compartimenti esaminato nella sezione~\ref{sec:ratti17} è illustrato
in figura~\ref{img:r17sim}: fornendo al calcolatore le equazioni del modello~\eqref{eq:r17xh}--\eqref{eq:r17n},
i parametri e le condizioni iniziali,
si ottiene in output una rappresentazione discreta delle soluzioni corrispondenti.
\begin{figure}[!h]
    \centering
    \includegraphics[keepaspectratio,width=0.5\textwidth]{img/r17fig5b}

    \caption[\figurename~5(b) da \cite{ratti2017}]{Un esempio di illustrazione (fig.~5(b))
        degli autori~\parencite[28]{ratti2017}
        per una soluzione del modello ridotto api--\emph{Varroa} (cfr.~sez.~\ref{sec:r17rid3D}), cioè senza virus.

        In blu il numero $x_h$ di operaie di nido, in nero tratteggiato il numero $x_f$ delle bottinatrici,
        in rosso tratteggiato il numero $n$ di varroe (non portatrici di ABPV).}
    \label{img:r17sim}
\end{figure}

Nel caso in esame il grafico illustra l'andamento delle soluzioni, ossia dei compartimenti di api operaie e del
compartimento della \emph{Varroa}.

\paragraph{}
I modelli sono generalmente forniti sotto forma di problema di Cauchy\footnote{Cfr. sez.~\ref{sec:ingredientiBioMat} e
l'appendice~\ref{chap:teoria}.}
\begin{equation}
\begin{sistema}
\diff{}{t} \mathbf{x} = F( \mathbf{x}, t )  \\
\mathbf{x}(t_0) = \mathbf{x}_0
\end{sistema}
\; ,
\label{eq:probCauchy}
\end{equation}
dove $\mathbf{x}(t)$ è un array di $n$ funzioni incognite.

La soluzione si ottiene con i software dedicati al calcolo numerico\footnote{Ad esempio \texttt{Octave},
\texttt{Mathematica}, \texttt{MATLAB}, \texttt{NumPy/SciPy}, etc.} chiamando una routine--solutore del tipo
\begin{equation}
\mathbf{y} = \mathtt{sol} ( F, \mathbf{x}_0, t_0, \Delta t, p) \; .
\label{eq:numSolutor}
\end{equation}
Nella pseudo--sintassi di $\mathtt{sol}$ nella~\eqref{eq:numSolutor} si vuole indicare che la routine prende in ingresso
\begin{itemize}
    \item la funzione vettoriale $F$, ovvero il membro destro dell'equazione~\eqref{eq:probCauchy}.
        Qui si possono anche variare i parametri del sistema;
    \item le condizioni iniziali $\mathbf{x} (t_0) = \mathbf{x}_0$;
    \item il passo di integrazione $\Delta t$ ;
    \item ulteriori parametri $p$ quali: scelta del metodo di approssimazione numerica, di adattamento dinamico del passo
        $\Delta t$, dei criteri d'arresto, etc.
\end{itemize}
Generalmente una siffatta routine $\mathtt{sol}$ ritorna un array di valori in virgola mobile\footnote{Più
precisamente \emph{una matrice}, in cui ogni colonna rappresenta il vettore--soluzione $\mathbf{x}(t)$ approssimato
ad un certo tempo $t$.}
\begin{equation}
\mathbf{y} = \left( y_0, y_1, \dots y_N \right) \; ,
\label{eq:approxSol}
\end{equation}
i quali verificano $y_0 \approx \mathbf{x}_0$ e
$$\frac{y_i - y_{i-1}}{\Delta t} \approx F \left( y_i, i \cdot \Delta t \right) \, , \quad 0 < i \leq N.$$

\paragraph{}
Ai fini della presente, si osservi che una soluzione del tipo~\eqref{eq:approxSol}, ottenuta applicando il solutore
numerico~\eqref{eq:numSolutor} ai modelli di \emph{Apis} esposti nel capitolo~\ref{chap:modelz} fornisce l'andamento
di una soluzione approssimata $\mathbf{y}$ sotto forma di matrice, in cui l'$i$-esimo
vettore--colonna $y_i$ contiene il valore di
tutti i compartimenti (api ed eventuali patogeni) al tempo $t = i \cdot \Delta t$.

Questo tipo di simulazione numerica è stato utilizzato anche dagli autori dei modelli\footcite{khoury2011,ratti2017},
sia per validare empiricamente i risultati teorici dimostrati nei rispettivi articoli, sia per estendere
quanto già stabilito ed indirizzare la ricerca futura. Vedasi in proposito le figure~\ref{img:kh11phasePlane} e
\ref{img:r17sim}.

\paragraph{}
Presentiamo adesso alcune nuove simulazioni svolte per il modello a due compartimenti\footcite{khoury2011}
studiato nella sezione~\ref{sec:kh11}, che confermano ed estendono alcuni risultati degli autori.

È stato usato il software libero \texttt{GNU Octave}, versione \texttt{9.2.0} e nello specifico
la routine \texttt{ode45} che implementa il metodo di Dormand--Prince al quarto ordine, un solutore
di equazioni differenziali ordinarie della famiglia Runge--Kutta.

Nella chiamata al solutore~\eqref{eq:numSolutor} si è generalmente utilizzato $\Delta t=1~\text{giorno}$
ed un criterio d'arresto $t \geq T=k \cdot 365,25$ per terminare la simulazione dopo $k$ anni.

\paragraph{}
In alcune chiamate al solutore, l'iterazione si è arrestata per $t < T$, ossia prima della soglia richiesta, per
l'occorrenza di fenomeni di instabilità numerica, oppure per la divergenza della popolazione in almeno
uno dei compartimenti in esame.

In ogni caso, un filtraggio in fase di \emph{post--processing} dei dati ottenuti permette di scartare le iterazioni
invalide, e di calcolare numeri di condizionamento in linea con i risultati teorici.\footnote{Cfr.~\cite{demmel}.}


% TODO figura con l'andamento di *una* soluzione. -> ma non qui?

% Link al codice? Licens? % TODO
\section{Esperimento A}
\label{sec:esperimentoA}
La simulazione per gli esperimenti descritti in questa sezione utilizza il modello a due compartimenti\footcite{khoury2011}
esaminato nella sezione~\ref{sec:kh11},
definito dalle equazioni~\eqref{eq:kh11h}--\eqref{eq:kh11f} a p.~\pageref{eq:kh11h}, che per comodità
di lettura riportiamo qui sotto:
$$\begin{sistema}
    \dot{H} = E(H,F)- H \cdot R(H,F)\; , \\
    \dot{F} = H \cdot R(H,F)  - m \cdot F\; ,
\end{sistema}$$
ove $E(H,F) = L \frac{H+F}{w + H + F}$ rappresenta la schiusa di nuove operaie di nido,
mentre il reclutamento sociale è dato dalla $R(H,F) = \alpha - \sigma \frac{F}{H+F}$.
La dinamica dei compartimenti è illustrata nel diagramma  riportato
in figura~\ref{img:kh11diagram} a p.~\pageref{img:kh11diagram}.

\paragraph{}
Tutte le simulazioni sono state svolte mantenendo costanti i seguenti parametri:
\begin{itemize}
    \item il tasso di deposizione della regina $L=2000$;
    \item il tasso massimo di reclutamento $\alpha = \frac{1}{4}$;
    \item il tasso di inibizione sociale $\sigma = \frac{3}{4}$;
\end{itemize}

Per esplorare lo spazio dei restanti parametri, sono state svolte 2304 simulazioni, risolvendo
il problema di Cauchy
con parametri e condizioni iniziali scelti come descritto di seguito.

I parametri $m,w$ e la popolazione iniziale $b=H_0 + F_0$ (condizioni iniziali) vengono scelti prima di ogni simulazione,
in modo pseudorandom uniforme in un intervallo specificato dall'utente. La distribuzione uniforme è stata scelta per esplorare estensivamente
lo spazio dei parametri: si noti che invece i dati sperimentali si distribuiscono secondo una normale gaussiana.

La mortalità $m \in \left[ m_-, m_+ \right]$ è stata scelta tra $m_-=0,016$ ed $m_+ =0,7$.

Il parametro $w \in \left[ w_-, w_+ \right]$, che regola la risposta funzionale nella schiusa $E$, è
compreso nell'intervallo tra $w_- = 3000$ e $w_+ = 30000$.

\paragraph{}
Le condizioni iniziali sono state riassunte in un solo parametro $b = H_0 +F_0$, anch'esso scelto
in modo pseudorandom uniforme nell'intervallo $\left[ b_-, b_+ \right] = \left[50, 9000\right]$.
Si noti che tale raggruppamento è concettualmente sensato solo per i compartimenti omologhi di un modello.

La soglia di popolazione $b_{thres} = 20$ sotto la quale una colonia si considera ``irreversibilmente perduta''
è utilizzata nell'esperimento descritto nella sottosezione~\ref{sec:fdod}.

\paragraph{}
Il tempo di simulazione è $T=2~\text{anni}$ ed il passo d'integrazione $\Delta t = 1~\text{giorno}$.

Nello spazio dei parametri in figura~\ref{img:param3D}, ogni punto di coordinate $(m,w,b)$
rappresenta la simulazione di una colonia con parametri $m,w$ e condizioni iniziali $b=H_0+F_0$; la dimensione di ogni
punto è proporzionale alla popolazione finale per esprimere visivamente l'esito della simulazione.

\paragraph{}
\label{par:colore}
Il colore di ogni punto rappresenta la quantità $\alpha w - L$, che compare nel criterio di esistenza per gli equilibri
del modello (proposizione~\ref{teo:esistenzPosF} a p.~\pageref{teo:esistenzPosF}).

Le regioni dei parametri per cui $\alpha w -L$ è negativo (colore rosso) non rappresentano realisticamente un
alveare di campo, poiché l'equilibrio tra i processi interni di ovideposizione ($L$), reclutamento ($\alpha$) ed
allevamento sociale della covata ($w$) è fortemente sbilanciato e determina un eccesso insensato di api di casa.

Inoltre, per le caratteristiche algebriche del sistema (mortalità $m$ soltanto nel comparto $F$),
per $\alpha w -L<0$ si ha quasi sempre la \emph{divergenza} della popolazione nel comparto $H$: questo è un artefatto
dovuto alla semplicità del modello, che tuttavia si verifica \emph{a priori} soltanto in siutazioni inverosimili.

Le simulazioni con $\alpha w -L >0$ rappresentano più realisticamente le colonie di api (colore verde: valori positivi,
colore blu: valori ``molto grandi''), in cui il tasso di deposizione della regina $L$ viene forzatamente
regolato dalle nutrici per rispondere alle variazioni dei ritmi fisiologici della famiglia. %
\footnote{Cfr. la discussione a p.~\pageref{par:interpretationCond6b} per una interpretazione dettagliata
della condizione~\eqref{eq:cond6b} $\alpha w -L>0$.}

Per tutti i grafici in questa sezione escluso l'ultimo (cioè per le figg.~\ref{img:param3D}--\ref{img:kh11expA23}),
l'interpretazione del colore dei
\emph{datapoint} è la stessa della figura~\ref{img:param3D}.

\begin{figure}[!h]
    \centering
    \includegraphics[keepaspectratio,width=\textwidth]{img/k11EA2-parameterSpace3D}

    \caption[Esperimento A2, spazio dei parametri.]{Ogni punto $(m,w,b)$ corrisponde ad una simulazione effettuata
        coi parametri $m,w$ e popolazione iniziale $b$. La dimensione del punto rappresenta la popolazione finale
        $H(T)+F(T)$ raggiunta al termine della simulazione, compresa tra 100 e $10^5$ operaie.

        Il colore rappresenta la quantità $\alpha w - L$, che è una misura di quanto la condizione~\eqref{eq:cond6b}
        sia soddisfatta o meno nella simulazione corrente (cfr.~p.~\pageref{par:interpretationCond6b} per una interpretazione quantitativa).
    }
    \label{img:param3D}
\end{figure}

\paragraph{}
Nella proiezione dello spazio dei parametri sul piano $(m, \, b)$ (figura~\ref{img:param2D}) possiamo confermare
empiricamente
che in questo semplice modello il parametro $m$ influenza definitivamente le capacità di sopravvivenza della colonia;
gli altri parametri determinano il valore del punto di equilibrio nella popolazione.
\begin{figure}[!h]
    \centering
    \includegraphics[keepaspectratio,width=\textwidth]{img/k11EA2-parameterSpace2D}

    \caption[Esperimento A, proiezione dello spazio dei parametri.]{Il diagramma a dispersione della
        figura~\ref{img:param3D} proiettato sul piano $(m, \, b)$.}
    \label{img:param2D}
\end{figure}

Dalla figura~\ref{img:param2D} emerge un altro interessante aspetto di questo semplice modello:
la popolazione iniziale \emph{non influenza} l'esito finale della colonia;
in effetti il punto di equilibrio (cfr. sezione~\ref{sec:kh11}, Proposizione~\ref{teo:exUniqFstarPos}) è determinato
soltanto dai parametri.

\paragraph{}
Nelle sottosezioni che seguono sono esaminati i dati risultanti dalle simulazioni appena descritte.

\subsection{Rapidità di estinzione e mortalità}
\label{sec:fdod}
Per valutare la rapidità e la gravità con cui una colonia si estingue in presenza di un alto tasso di
mortalità nel comparto bottinatrici, definiamo il \emph{FDOD (First Day Of (colony) Death)} come
$$\text{FDOD} \coloneq \min \left\{ t \geq 0 \st H(t) + F(t) < b_{thres} \right\} \; .$$

Il risultato di questa analisi è illustrato in Figura~\ref{img:kh11expA21}: ogni punto blu di coordinate $(x,y)$
rappresenta una simulazione del modello con mortalità $m=x$: la corrispondente popolazione
della colonia è scesa sotto la soglia $b_{thres}=20$ al giorno $y$ per la prima volta.
\begin{figure}[!h]
    \centering
    \includegraphics[keepaspectratio,width=0.98\textwidth]{img/k11EA2-fdodVSm}

    \caption[Esperimento A, \emph{FDOD} vs. mortalità.]{Il \emph{FDOD} (in giorni) degli alveari che
        muoiono entro i due anni,
        contro il tasso di mortalità. Con fit lineare (retta celeste).

        La dimensione del punto rappresenta qui la popolazione iniziale $b$.
    }

    \label{img:kh11expA21}
\end{figure}

\paragraph{}
Si può osservare che se il tasso di mortalità è contenuto ($m< \hat{m}$ con $\hat{m} \approx 0,4$),
la colonia sopravvive con successo. % pché mancano i datapoint a sinistra
Quando invece il tasso di mortalità è eccessivo ($m \gtrapprox 0,4$) le famiglie di \emph{Apis} muoiono
ad un ritmo proporzionalmente accelerato, ed
il tempo di sopravvivenza dell'alveare si riduce in funzione della mortalità (che nel modello innesca un reclutamento
precoce delle operaie di nido verso il compartimento delle bottinatrici), e neanche le famiglie (inizialmente) più numerose
riescono a superare una o due stagioni di raccolta.

\paragraph{}
Se da un lato la popolazione iniziale non determina affatto il destino della colonia, dall'altro un effetto
catturato con successo da questo semplice modello è che la numerosità iniziale delle api permette alle colonie
di resistere più a lungo, anche negli scenari in cui la morte è inevitabile.

\paragraph{}
Si osservi infine che nella figura~\ref{img:kh11expA21} mancano del tutto punti di colore rosso,
per cui la condizione~\eqref{eq:cond6b} è falsa: in tali colonie la popolazione diverge all'infinito (irrealisticamente)
e perciò non scende mai sotto la soglia $b_{thres} =20$.

Questa è una ulteriore conferma delle considerazioni circa la condizione $\alpha w -L >0$ fatte durante l'analisi
della stabilità del modello (p.~\pageref{par:interpretationCond6b} e segg.),
e già in questo esperimento (p.~\pageref{par:colore} e segg.).


\subsection{Popolazione iniziale e finale}
In figura~\ref{img:kh11expA22} è mostrato su scala logaritmica il risultato delle simulazioni, con la popolazione
iniziale $b=H(0)+F(0)$ in ascissa e la popolazione finale $H(T)+F(T)$ in ordinata.

Ricordiamo che il periodo di ogni simulazione è $T = 2~\text{anni}$.
\begin{figure}[!h]
    \centering
    \includegraphics[keepaspectratio,width=0.98\textwidth]{img/k11EA2-fpopVSipop}

    \caption[Esperimento A, popolazione finale vs. Popolazione iniziale.]{Esperimento A, popolazione finale $H(T)+F(T)$
        vs. Popolazione iniziale $H(0)+F(0)$. I triangoli rappresentano le simulazioni per cui la condizione~\eqref{eq:cond6b} è falsa,
        mentre per i cerchi è vera.

        La dimensione di ogni punto è proporzionale alla mortalità $m$, compresa nell'intervallo
        $\left[ 0,016, \, 0,7 \right]$.
    }

    \label{img:kh11expA22}
\end{figure}

Il modello considerato è abbastanza predittivo ed in linea con le osservazioni sugli apiari in campo, purché si
utilizzino misure accurate o quantomeno stime ragionevoli dei parametri da impiegare.%
\footnote{Cfr. p.~\pageref{par:interpretationCond6b} e segg. a chiusura della sezione~\ref{sec:kh11}.}

I \emph{datapoint} col triangolo indicano le simulazioni per cui la condizione~\eqref{eq:cond6b} è falsa: si
osservi che la popolazione corrispondentemente diverge irragionevolmente, anche per condizioni iniziali
relativamente sfavorevoli (\ie basse numerosità iniziali).%
\footnote{Si veda il paragrafo dedicato all'interpretazione della condizione~\eqref{eq:cond6b}
a p.~\pageref{par:interpretationCond6b}.}

I triangoli compaiono soltanto nella parte alta del grafico (\ie popolazioni finali superiori a $10^4$ individui);
popolazioni finali irrealisticamente alte ($>10^5$ api) si trovano sempre più in alto in corrispondenza di
mortalità sempre più basse (punti più piccoli).

\paragraph{}
Come confermano gli apicoltori\footcite{privFDL,privFPan,meccanica},
la ``forza'' di un alveare (genericamente intesa come la numerosità della popolazione)
sostiene meglio e più a lungo la famiglia attraverso gli stress ambientali;
ma alveare -- seppur molto popoloso -- che incontri continuativamente avversità troppo intense (ad es. attacchi da
molteplici patogeni) necessita di un intervento umano per non soccombere.

\subsection[Popolazione finale e w]{Popolazione finale e $w$}
Per studiare la relazione tra la ``efficienza'' di una colonia nella combinazione delle attività foretiche e di
cura della covata (ingresso di nuovi individui nel comparto $H$ delle operaie di nido),
si consideri il grafico in figura~\ref{img:kh11expA23}: ogni \emph{datapoint} alle coordinate $(x,y)$ rappresenta
una simulazione con parametro $w=x$ e popolazione finale $N(T)=y$; ricordiamo che in questo esperimento il
tempo di ogni simulazione è $T=2~\text{anni}$, mentre la popolazione iniziale $b=H(0)+F(0)$ è scelta
casualmente per ogni simulazione tra $b_-=50$ e $b_+ =9000$ individui.
\begin{figure}[!h]
    \centering
    \includegraphics[keepaspectratio,width=0.98\textwidth]{img/k11EA2-finpopVSw}

    \caption[Esperimento A, popolazione finale vs $w$.]{Esperimento A, popolazione finale contro $w$.
        I triangoli rappresentano le simulazioni per cui la condizione~\eqref{eq:cond6b} è falsa,
        mentre per i cerchi è vera.

        La dimensione di ogni punto è proporzionale alla mortalità $m \in \left[0,016, \, 0,7 \right]$.
    }
    \label{img:kh11expA23}
\end{figure}

Il gradiente di colore orizzontale nella figura~\ref{img:kh11expA23} è dovuto al fatto che in $\alpha w - L$ i
parametri $\alpha, L$ vengono mantenuti costanti attraverso le simulazioni.

\paragraph{}
Il parametro $w$ compare nell'equazione~\eqref{eq:eclos} e nella risposta funzionale
(Holling di tipo~II\footnote{Cfr. sezioni~\ref{sec:rispFunz} e~\ref{sec:eclos}.})
modula la schiusa delle api adulte con il tasso di ovideposizione $L$ della regina; come
affermato dagli stessi autori del modello\footcite[2]{khoury2011}, senza informazioni aggiuntive è ragionevole
proporre che il tasso di schiusa delle uova (deposte 21 giorni prima) approcci $L$ dal basso per $N(t) \to \infty$,
in modo liscio.

Come già discusso nelle sezioni~\ref{sec:rispFunz} e~\ref{sec:eclos}, il parametro $w$ è l'inverso del tasso di
attacco nella risposta funzionale della schiusa al numero di operaie (di casa e di campo) disponibili nell'alveare.

Nella figura~\ref{img:kh11expA23} si può osservare che il parametro $w$ impatta in ogni caso il destino di un
alveare simulato, e che per alti valori di $w$ (\ie famiglie ``inefficienti'' nella cura della prole) gli altri
fattori ambientali (\ie la mortalità $m$) influiscono progressivamente sempre di più sulla
numerosità della popolazione finale.


\subsection{Soglie di mortalità}
\label{ssec:simSoglie}
Il risultato fondamentale che accompagna il modello di~\citeauthor{khoury2011}\footcite{khoury2011},
precisato e dimostrato nella
Proposizione~\ref{teo:esistenzPosF} (p.~\pageref{teo:esistenzPosF}), fornisce le condizioni affinché esista un
equilibrio positivo.

In teoria, mantenere i parametri calcolati a partire dalle misure dirette sul campo
entro i limiti ivi stabiliti, dovrebbe consentire di garantire la sopravvivenza di una colonia in apiario.

\paragraph{}
Nella figura~\ref{img:kh11expA24} è mostrato il grafico delle soglie di mortalità $m_1^*$ (in blu) ed
$m_2^*$ (in verde) stabilite nella proposizione~\ref{teo:esistenzPosF}.
Gli altri punti alle coordinate $(x,y)$ sono simulazioni con parametri $w=x$ ed $m=y$, in cui il
colore rappresenta la popolazione finale $N(T)$ raggiunta dalla colonia.

Ricordiamo che in questo esperimento il
tempo di ogni simulazione è $T=2~\text{anni}$, mentre la popolazione iniziale $b=H(0)+F(0)$ è scelta
casualmente per ogni simulazione tra $b_-=50$ e $b_+ =9000$ individui.

\begin{figure}[!h]
    \centering
    \includegraphics[keepaspectratio,width=0.98\textwidth]{img/k11EA2-mistarVSw}

    \caption[Esperimento A, soglie di mortalità]{Previsioni del modello per $m_1^*$ (in blu) ed $m_2^*$ (in verde).
        Gli altri punti sono le simulazioni con parametri $(w,m)$ ed il colore rappresenta la popolazione
        finale raggiunta dalla colonia.
    }

    \label{img:kh11expA24}
\end{figure}

La linea tratteggiata delimita la regione del parametro $w$ per cui la condizione~\eqref{eq:cond6b} è vera (a destra),
in cui la popolazione di api si stabilizza ad un equilibrio positivo per mortalità $m$ superiori ad $m_1^*$
(altrimenti la popolazione diverge) ed inferiori ad $m_2^*$ (altrimenti la colonia si estingue).
Si noti che quando la~\eqref{eq:cond6b} diventa falsa, la soglia $m_2^*$ funge
da limite \emph{inferiore} alla mortalità, come affermato nella proposizione~\ref{teo:esistenzPosF}.

\paragraph{}
Questa conferma sperimentale ci consente di assimilare definitivamente i casi 2 e 3 della
proposizione~\ref{teo:esistenzPosF}, e di considerare solamente il caso 1 come regione di parametri ammissibile
per questo modello -- e successivi, più raffinati -- per descrivere efficacemente le reali colonie di \emph{A.~mellifera},
sia allo stato selvatico che in allevamento.


\section{Esperimento B}
L'esperimento descritto in questa sezione utilizza ancora il modello a due compartimenti\footcite{khoury2011}
esaminato nella sezione~\ref{sec:kh11}.

Nelle equazioni del modello si introduce il termine di reclutamento $R$ per il passaggio tra il compartimento $H$ delle
operaie di nido (che curano la covata) ed $F$, il compartimento delle operaie di campo (che raccolgono il cibo).

L'obiettivo dell'esperimento è studiare qualitativamente la risposta funzionale $R$ quando la colonia simulata viene
sottoposta a una mortalità \emph{variabile} nel tempo.

\paragraph{}
Gli autori\footcite[2,3]{khoury2011} affermano che la formulazione~\eqref{eq:Recr}, ovvero
$$R = \alpha - \sigma \frac{F}{H+F}$$
rappresenta la risposta funzionale della colonia all'equilibrio/disequilibrio tra il numero di operaie in $H$ ed $F$,
ad esempio accelerando il reclutamento precoce in caso di deficit di api bottinatrici.

In tal proposito, si vedano la sottosezione~\ref{ssec:recr} (p.~\pageref{ssec:recr}) per un esame di questa formulazione
di $R$ nel modello di~\citeauthor{khoury2011} ivi esaminato, e la sottosezione~\ref{ssec:r17eqns}
(p.~\pageref{ssec:r17eqns})
per un impiego di una espressione molto simile in un modello più raffinato\footcite{ratti2017} che analizza
la risposta del sistema a stimoli non costanti, bensì periodici.

\paragraph{}
Sono state svolte diverse simulazioni di una singola colonia con parametri costanti
$$w= 25000, \quad L=2000 \; , \quad \alpha=\frac{1}{4}, \quad \sigma= \frac{3}{4} \; ,$$
fino a $T=750~\text{giorni}$, con passo $\Delta t = 1~\text{giorno}$.
Si noti che per questa scelta dei parametri la condizione~\eqref{eq:cond6b} è verificata.

Le condizioni iniziali della colonia sono sempre $(H_0, F_0) = (10000, 4000)$ in tutte le simulazioni.

\paragraph{}
L'unica variazione tra le diverse simulazioni è la mortalità $m$: tale parametro costante viene sostituito
con una funzione variabile $m(t)$ di diverse forme:
\begin{enumerate}
    \item a scalino (funzione discontinua);
    \item triangolare (funzione continua ma non differenziabile);
    \item normale gaussiana (una ``campana'' di classe $C^\infty$);
    \item sinusoidale;
\end{enumerate}
i dettagli e le motivazioni di tali scelte si trovano nelle rispettive sottosezioni che seguono.

\paragraph{}
Le figure~\ref{img:expB1}--\ref{img:expB5} illustrano il risultato delle simulazioni della colonia, variando
la forma dell'impulso nel parametro della mortalità $m(t)$.

Tali figure riportano nei grafici superiori le numerosità nei comparti $H, F$
(in unità di api) ed il tasso di schiusa $E$ (in unità di api/die).
I grafici inferiori riportano invece grandezze consistenti, giacché la mortalità $m$ ed il tasso di
reclutamento $R$ sono entrambi in unità di 1/die.

Nel seguito è analizzata nel dettaglio la risposta dell'alveare alle diverse forme esaminate di mortalità variabili.


\subsection{Stimolo a scalino}
Una famiglia sottoposta inizialmente a mortalità $m= m_- =0,15$ è improvvisamente soggetta
a mortalità $m= m_+ =0,55$, a partire dal trecentesimo giorno.

Il risultato della simulazione (illustrato in figura~\ref{img:expB1}) conferma quanto affermato
nella proposizione~\ref{teo:esistenzPosF} (sez.~\ref{sec:kh11}) e quanto visto nell'esperimento numerico
della sezione~\ref{ssec:simSoglie}:
per $m<m_2^* =0,4$ la colonia affronta con successo le avversità e la popolazione tende ad un equilibrio positivo.
\begin{figure}[hbp]
    \centering
    \includegraphics[keepaspectratio,width=\textwidth]{img/alternativeA}

    \caption[Esperimento B, stimolo a scalino.]{Sopra: in blu le numerosità nei comparti $H$ (linea continua)
        ed $F$ (linea tratteggiata). In verde il tasso di schiusa $E=E(H,F)$.

        Sotto: in rosso la mortalità $m=m(t)$ ed in blu il tasso di reclutamento $R=R(H,F)$.
    }
    \label{img:expB1}
\end{figure}

\paragraph{}
Quando invece la mortalità è eccessiva ($m>m_2^*$) la colonia soccombe, anche per via del repentino aumento
di reclutamento precoce delle api giovani ($R$ in blu nel grafico inferiore, figura~\ref{img:expB1}),
che inibisce per troppo tempo la schiusa di un numero sufficiente di nuove operaie
($E$ in verde nel grafico superiore), portando l'alveare verso l'inevitabile collasso.


\subsection{Stimolo limitato}
Nelle figure~\ref{img:expB2} e \ref{img:expB3} è illustrata la simulazione dello stesso modello
(equazioni~\eqref{eq:kh11h}--\eqref{eq:kh11f} nella sez.~\ref{sec:kh11})
in risposta ad uno stimolo triangolare e gaussiano nella mortalità.

Stimoli di mortalità di questo tipo (un aumento \emph{limitato nel tempo}) possono essere usati per simulare
l'insorgenza \emph{episodica} di un fattore di stress, incrementando temporaneamente la mortalità in uno
o più compartimenti. Quando l'episodio stressante termina\footnote{Ad es. un patogeno viene sconfitto,
cessa una carestia di polline/nettare o una siccità, etc.}, la mortalità ritorna ai livelli precedenti.

\paragraph{}
Per lo stimolo triangolare (figura~\ref{img:expB2}) si è usata una funzione costante $m(t) \equiv m_- = 0,15$
fino a $t=250~\text{giorni}$, dopodiché $m(t)$ aumenta linearmente fino al picco $m_+ =0,55$ al
tempo $t=300~\text{giorni}$.
In maniera simmetrica, dopo $t=300~\text{giorni}$ il tasso di mortalità $m$ ritorna verso $m_-$ linearmente,
nello stesso tempo di $50~\text{giorni}$.
\begin{figure}[hbp]
    \centering
    \includegraphics[keepaspectratio,width=\textwidth]{img/alternativeB}

    \caption[Esperimento B, stimolo triangolare.]{Sopra: in blu le numerosità nei comparti $H$ (linea continua)
        ed $F$ (linea tratteggiata). In verde il tasso di schiusa $E=E(H,F)$.

        Sotto: in rosso la mortalità $m=m(t)$ ed in blu il tasso di reclutamento $R=R(H,F)$.
    }
    \label{img:expB2}
\end{figure}

\paragraph{}
In figura~\ref{img:expB3} è riportata la simulazione per uno stimolo analogo a quello triangolare
sia per durata che per ampiezza, ma per confrontare la $m(t)$ triangolare con una funzione liscia, è stata utilizzata
una normale gaussiana.
\begin{figure}[hbp]
    \centering
    \includegraphics[keepaspectratio,width=\textwidth]{img/alternativeC}

    \caption[Esperimento B, stimolo gaussiano.]{Sopra: in blu le numerosità nei comparti $H$ (linea continua)
        ed $F$ (linea tratteggiata). In verde il tasso di schiusa $E=E(H,F)$.

        Sotto: in rosso la mortalità $m=m(t)$ ed in blu il tasso di reclutamento $R=R(H,F)$.
    }
    \label{img:expB3}
\end{figure}

In entrambe le figure (\ref{img:expB2}, \ref{img:expB3}) possiamo verificare che la colonia risponde all'innalzamento
(temporaneo)
di mortalità nel comparto bottinatrici $F$ aumentando il reclutamento precoce ($R$ in blu, grafici in basso).

Questa risposta fisiologica\footnote{Cfr. discussione a p.~\pageref{par:socialRecr} nel capitolo~\ref{chap:bio}.}
della colonia
all'aumento di mortalità tra le bottinatrici comporta un deficit di operaie di nido $H$, a cui segue in
breve un declino di nuove api sfarfallate, dal momento che molte larve nella covata non giungono a maturazione
($E$ in verde, grafici in alto).

\paragraph{}
Quando la mortalità si riporta ai livelli precedenti ($m \simeq m_-$ per $t>350~\text{giorni}$), la colonia
ripristina gradualmente la popolazione in $H,F$ verso l'equilibrio, ed anche $E,R$ ritornano ai ritmi
antecedenti il fattore di stress.

Questo meccanismo è noto in apicoltura~\footcite{privFDL,privFPan,meccanica} ed è uno dei motivi per cui
l'impatto dei molteplici fattori di stress (sia naturali che dovuti all'intervento umano) è dettato principalmente
dalla \emph{durata di esposizione} delle colonie.

Infatti i meccanismi fisiologici e sociali delle api permettono agli alveari di rispondere dinamicamente alle criticità
-- anche piuttosto profonde -- e di ritornare dopo gli episodi stressanti al funzionamento ``normale'',
a patto però che i fattori principali della crisi si risolvano entro tempi ragionevoli.

\paragraph{}
Altrimenti, ossia laddove la colonia è esposta a stress eccessivi per un periodo \emph{troppo prolungato},
anche i meccanismi di difesa naturali falliscono e l'intera famiglia collassa, sotto il peso di
deficit nutrizionali o di patologie particolarmente invalidanti.

Questa considerazione -- incorporata in molti
modelli per \emph{A.~mellifera}\footcite{khoury2011,ratti2017,chen_review,betti2014} --
è corroborata anche dall'esperimento con stimolo a scalino (visto sopra), e da quello descritto nella
prossima sottosezione (figure~\ref{img:expB4} e \ref{img:expB5}).


\subsection{Stimolo periodico}
Nelle figure~\ref{img:expB4} e~\ref{img:expB5} vediamo il risultato di due simulazioni in cui la
mortalità $m(t)$ è costante al valore $m_-=0,15$ fino a $t=150~\text{giorni}$: come nei casi precedenti,
in questo periodo iniziale le due colonie si portano verso lo stesso equilibrio di una colonia
sottoposta a mortalità costante $m(t) \equiv m_-$ (grafici sopra, in grigio).

A partire da $t=150~\text{giorni}$ le colonie sono sottoposte ad una mortalità periodica, oscillante tra $m_-$ ed
una soglia superiore $m_+ \simeq 0,186$ (figura~\ref{img:expB4}) oppure $m_+ \simeq 0,674$ per la figura~\ref{img:expB5}.
In entrambi i casi la lunghezza d'onda è $\lambda=50~\text{giorni}$.

\paragraph{}
Stimoli di questo tipo nella mortalità o in parametri/compartimenti simili servono a simulare l'insorgenza di uno stato
\emph{endemico} di un patogeno\footnote{Cfr. cap.~\ref{chap:bio}.} che attacca la popolazione senza sterminarla né
essere eradicato: la popolazione ed il patogeno coesistono in un equilibrio dinamico.
\begin{figure}[hbp]
    \centering
    \includegraphics[keepaspectratio,width=\textwidth]{img/alternativeD}

    \caption[Esperimento B, stimolo sinusoidale debole.]{Sopra: in blu le numerosità nei comparti $H$ (linea continua)
        ed $F$ (linea tratteggiata). In verde il tasso di schiusa $E=E(H,F)$.
        In grigio una colonia di controllo, sottoposta a mortalità costante $m_-=0,15$.

        Sotto: in rosso la mortalità $m=m(t)$ ed in blu il tasso di reclutamento $R=R(H,F)$.
    }
    \label{img:expB4}
\end{figure}

\paragraph{}
Si può osservare che se l'incremento medio sul periodo porta la mortalità ad un livello inferiore ad $m_2^*=0,4$
(cfr. proposizione~\ref{teo:esistenzPosF}), la colonia risponde con successo ai fattori di stress, anche
se la popolazione in entrambi i compartimenti ovviamente diminuisce quando incontra le avversità.

Nel grafico in alto della figura~\ref{img:expB4}, entrambi i comparti simulati (linee blu) scendono sotto
i livelli di una corrispondente colonia con mortalità costante $m_-$ (linee grigie) a partire dall'inizio
dello stimolo ($t=150~\text{giorni}$), ma la famiglia riesce a sopravvivere.

\paragraph{}
Ben diverso è l'esito di uno stimolo di ampiezza maggiore come quello in figura~\ref{img:expB5}, in cui la mortalità media
sul periodo supera $m_2^*$: la risposta dinamica della colonia non è sufficiente a fronteggiare lo stress
prolungato, e la popolazione declina inesorabilmente.
\begin{figure}[hbp]
    \centering
    \includegraphics[keepaspectratio,width=\textwidth]{img/alternativeE}

    \caption[Esperimento B, stimolo sinusoidale forte.]{Sopra: in blu le numerosità nei comparti $H$ (linea continua)
        ed $F$ (linea tratteggiata). In verde il tasso di schiusa $E=E(H,F)$.
        In grigio una colonia di controllo, sottoposta a mortalità costante $m_-=0,15$.

        Sotto: in rosso la mortalità $m=m(t)$ ed in blu il tasso di reclutamento $R=R(H,F)$.
    }
    \label{img:expB5}
\end{figure}

\paragraph{}
In altre simulazioni (non riportate per brevità) si è variata la forma dello stimolo periodico, i massimi e minimi,
l'ampiezza. I risultati confermano essenzialmente quanto già visto sperimentalmente e quanto affermato
nel criterio di stabilità (proposizione~\ref{teo:esistenzPosF}): se la mortalità \emph{media sul periodo}
(\ie analiticamente, la media integrale $\frac{1}{T} \int_0^T m \de t$) è inferiore alla soglia $m_2^*$ la colonia
sopravvive, altrimenti la popolazione declina rapidamente.

Nella simulazione corrente i valori dei parametri forniscono $m_2^*=0,4$.


\paragraph{}
Concludiamo la sezione osservando che la differenza qualitativa tra un equilibrio endemico ed un equilibrio banale
(\ie in cui la popolazione di api tende a zero) è catturata in entrambi i modelli studiati nel capitolo~\ref{chap:modelz}
(cfr. propp.~\ref{teo:esistenzPosF} e \ref{prop:exist5D}, rispettivamente), e motivata dalle considerazioni
nel capitolo~\ref{chap:bio} (cfr. p.~\pageref{par:alleeEffect}).

Anche gli apicoltori professionisti\footcite{privFPan,meccanica} concordano sulla necessità, dopo un decremento di
popolazione, per la colonia
\begin{displayquote}[\cite{privFDL}]
    \omissis di un periodo abbastanza lungo con buone condizioni ambientali, per ristabilire i livelli
    nelle varie caste, e in generale una ``efficienza'' della famiglia, che si riprende gradualmente.
\end{displayquote}

Si osservi infine che tale considerazione si applica indipendentemente dalla causa dello spopolamento: che si tratti
di un patogeno, di una sciamatura (anche ``artificiale'') o dell'effetto di pesticidi agricoli%
\footnote{Cfr. cap.~\ref{chap:bio}, ssez.~\ref{ssec:declinoUmano}.}
o di altri meccanismi, un alveare sembra aver bisogno di un ``periodo di grazia'' per potersi ``riprendere'' dopo un evento
stressante, e tornare ai livelli di funzionamento fisiologici.




% \section{Experiment C, iteration 2}
Two iterations were made of this simulation with constant parameters $\alpha = 0.25$, $\sigma=0.75$, $L=2500$:
data \texttt{A} was obtained with $w=27000$, while the value $w=9800$ was used for data \texttt{B}.

In both experiments a random value for the parameter $m$ was obtained from a uniform distribution on $[0.006, \, 0.6]$
with a resolution of 30. For each $m$ we selected at random initial values $1 \leq H_0 \leq 15000$ and $1 \leq F_0 \leq 6000$ and ran 30 such simulations for a total of 900 days each.

\subsection{Data A}
\label{sec:kh11expC2A}
Note that with $w=27000$ the condition~\eqref{eq:cond6b} from Theorem~\ref{teo:esistenzPosF} is true: $\alpha - \frac{L}{w} > 0$, which is case 1 of the Theorem,
when a positive equilibrium exists for mortalities $m$ bound from above and from below:
$$ 0.0804 \simeq m_1^* < m < m_2^* \simeq 0.5078 \; .$$

In the remainder of this section, we will call ``weak'' those colonies which have final total
population $N(900) < N_- = 100$ bees. We will say that a colony is ``strong'' if
its final population is more than $N_+ = 40000$ workers.

Results are plotted in Figure~\ref{img:kh11expC2A}.

All colonies which eventually die off have a high rate of forager mortality: $m > m_2^*$.
Some of them don't collapse in 2.5 years and we don't call them ``weak'' in the sense above.

\paragraph{}
For mortalities above a reasonable value and below a fatal threshold, \ie for $m_1^* < m < m_2^*$,
we get steadily close to a positive equilibrium $N^*$: final workers number can get even into
the ``strong'' classification for hives with lower mortality rates, and become lower
(in the hundreds) near the $m_2^*$ end of the spectrum.

\paragraph{}
Outside the shown plot range were final populations in the unreasonable order of
magnitude of $10^6$ and beyond, clearly indicating a diverging behaviour, even with low values of $N(0)$.
All these points lie on the left of $m_1^* \simeq 0.0804$.\footnote{Cfr.~Interpretation at page~\pageref{par:interpretationCond6b}.}

\begin{figure}[pbh]
    \centering
    \includegraphics[keepaspectratio,width=0.98\textwidth]{img/kh11expC2A}

    \caption[Experiment C2A]{Experiment C2A: total number of worker bees $N(t)= H(t)+F(t)$ is shown at $t=0$ and $t=900$ for different levels of forager mortality $m$.

    With the terminology estabilished in page \pageref{sec:kh11expC2A}, we see that
    initial conditions of colonies that become ``weak'' (blue squares) lie all in
    the half--plane $m>0.5078$ (dotted vertical line).
    Similarly, initial conditions of colonies that eventually become ``strong''
    (red x's) all lie around the line $m=0.0804$ (dashed).

    In black we see the initial conditions (+) and final populations (triangles) for colonies that
    don't get to be ``strong'' nor ``weak'' at day $t=900$.

    Final population numbers for ``strong'' hives (purple circles) all happen in the left--hand side
    of the spectrum $0.0804 < m < 0.5078$.
    }

    \label{img:kh11expC2A}
\end{figure}

\subsection{Data B}
In this experiment we ran 900 simulations for 900 days each, within the same setting as above,
where we changed the value of the parameter $w$ only.

Note that in this case inequality~\eqref{eq:cond6b} is reversed, since
$$ \alpha - \frac{L}{w} = \frac{1}{4} - \frac{2500}{9800} < 0 \; .$$

Results are plotted in Figure~\ref{img:kh11expC2B}: for each simulation the fraction $\frac{N(900)}{N(0)}$ is plotted against its level of forager mortality.
This is a rough measure of performance of the hive, for it depends on the colony ability to secure
trophic and reproductive capabilities, within the limits of its given initial conditions.

\paragraph{}
We can see that initial population levels definitely increased in a span of 2.5 years for
almost all initial conditions. A slight decreasing trend in final population levels is
noted for higher mortality rates; still, these higher values for $m$ cannot keep the colony
population at reasonably stable levels and only lower the speed at which population is diverging.

\begin{figure}[pbh]
    \centering
    \includegraphics[keepaspectratio,width=0.98\textwidth]{img/kh11expC2B}

    \caption[Experiment C2B]{Experiment C2B: the ratio $\frac{N(900)}{N(0)}$ of final and initial
    total populations is plotted against $m$ in abscissa.}

    \label{img:kh11expC2B}
\end{figure}


