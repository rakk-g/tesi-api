\chapter{Metodi di propagazione delle soluzioni}
Il supporto del calcolatore elettronico ha un valore inestimabile nella ricerca Matematica.

Nella applicazione d'interesse -- i modelli matematici per le api -- viene utilizzato
per manipolare le equazioni dei modelli, sia tramite il calcolo simbolico che svolgendo simulazioni numeriche.

\paragraph{}
L'Analisi Numerica offre una vasta classe di metodi per l'approssimazione di soluzioni di un sistema di equazioni
non lineari, tra cui i metodi Runge--Kutta che includono il metodo di Eulero.

Alcuni sistemi presentano equazioni cosiddette \emph{rigide} poiché alcuni metodi numerici si rivelano
molto instabili numericamente a meno di prendere un passo d'approssimazione esageratamente breve.

Per i due modelli esaminati in questa tesi, una analisi dettagliata delle simulazioni numeriche
si trova nei rispettivi articoli.\footcite{khoury2011,ratti2017}

Presentiamo adesso le simulazioni svolte per il modello a due compartimenti della Sezione~\ref{sec:kh11},
che confermano ed estendono alcuni risultati degli autori.

\paragraph{}
È stato usato il software libero \texttt{GNU Octave}, versione \texttt{9.2.0} e nello specifico
la routine \texttt{ode45} che implementa il metodo di Dormand--Prince al quarto ordine, un solutore
di equazioni differenziali ordinarie della famiglia Runge--Kutta.

% Link al codice? Licens? % TODO
