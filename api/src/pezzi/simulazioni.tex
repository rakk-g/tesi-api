\chapter{Simulazione per propagazione delle soluzioni}
Il supporto del calcolatore elettronico ha un valore inestimabile nella ricerca Matematica.

Nella applicazione d'interesse -- i modelli matematici per le api -- viene utilizzato
per manipolare le equazioni dei modelli, sia tramite il calcolo simbolico che svolgendo simulazioni numeriche.

\paragraph{}
L'Analisi Numerica offre una vasta classe di metodi per l'approssimazione di soluzioni di un sistema di equazioni
non lineari, tra cui i metodi Runge--Kutta che includono il metodo di Eulero.

Alcuni sistemi presentano equazioni cosiddette \emph{rigide} poiché alcuni metodi numerici si rivelano
molto instabili numericamente a meno di prendere un passo d'approssimazione esageratamente breve.

\paragraph{}
Per i due modelli esaminati in questa tesi, una analisi dettagliata delle simulazioni numeriche
si trova nei rispettivi articoli.\footcite{khoury2011,ratti2017}

Presentiamo adesso le simulazioni svolte per il modello a due compartimenti della Sezione~\ref{sec:kh11},
che confermano ed estendono alcuni risultati degli autori.

\paragraph{}
È stato usato il software libero \texttt{GNU Octave}, versione \texttt{9.2.0} e nello specifico
la routine \texttt{ode45} che implementa il metodo di Dormand--Prince al quarto ordine, un solutore
di equazioni differenziali ordinarie della famiglia Runge--Kutta.

\paragraph{}
Per la dimostrazione dei risultati qui esposti circa le affermazioni che accompagnano i modelli
esaminati\footcite{khoury2011,ratti2017} ci si è avvalsi del software libero \texttt{Maxima},
versione \texttt{5.47.0} per assistere nel calcolo simbolico,
impiegando la GUI \texttt{wxMaxima}, versione \texttt{22.09.0}.

% Link al codice? Licens? % TODO
\section{Experiment A, iteration 2}

\subsection{Rapidità di estinzione e mortalità}
\begin{figure}[pbh]
    \centering
    \includegraphics[keepaspectratio,width=0.98\textwidth]{img/k11EA2-fdooVSm}

    \caption[Experiment A2:1]{}

    \label{img:kh11expA21}
\end{figure}

\subsection{Popolazione finale e iniziale}
TODO: colora in base a $m$. % TODO
\begin{figure}[pbh]
    \centering
    \includegraphics[keepaspectratio,width=0.98\textwidth]{img/k11EA2-fpopVSipop}

    \caption[Experiment A2:2]{}

    \label{img:kh11expA22}
\end{figure}

\subsection{Popolazione finale e $w$}
\begin{figure}[pbh]
    \centering
    \includegraphics[keepaspectratio,width=0.98\textwidth]{img/k11EA2-finpopVSw}

    \caption[Experiment A2:3]{}

    \label{img:kh11expA23}
\end{figure}

\subsection{$m_i^*$ e $w$}
\begin{figure}[pbh]
    \centering
    \includegraphics[keepaspectratio,width=0.98\textwidth]{img/k11EA2-mistarVSw}

    \caption[Experiment A2:4]{}

    \label{img:kh11expA24}
\end{figure}


\section{Experiment C, iteration 2, data A and B}
Two iterations were made of this simulation with constant parameters $\alpha = 0.25$, $\sigma=0.75$, $L=2500$:
data \texttt{A} was obtained with $w=27000$, while the value $w=9800$ was used for data \texttt{B}.

In both experiments a random value for the parameter $m$ was obtained from a uniform distribution on $[0.006, \, 0.6]$
with a resolution of 30. For each $m$ we selected at random initial values $1 \leq H_0 \leq 15000$ and $1 \leq F_0 \leq 6000$ and ran 30 such simulations for a total of 900 days.

\subsection{Data A}
\label{sec:kh11expC2A}
Note that with $w=27000$ and the other parameter values, the condition~\eqref{eq:cond6b} from Theorem~\ref{teo:esistenzPosF} is true: $\alpha - \frac{L}{w} > 0$, which is case 1 of the Theorem,
when a positive equilibrium exists for mortalities $m$ bound from above and from below.

In the remainder of this section, we will call ``weak'' colonies those which have final total
population $N(900)$ less than $N_- = 100$ bees. We will say that a colony is ``strong'' if
its final population is more than $N_+ = 40000$ workers.

Results are plotted in Figure~\ref{img:kh11expC2A}.

\paragraph{}
Outside the shown plot range were final populations in the unreasonable order of
magnitude of $10^6$ and beyond, clearly indicating a diverging behaviour.
All these points lie on the left of $m_1^* \simeq 0.0804$.\footnote{Cfr.~interpretation at page~\pageref{par:interpretationCond6b}.}

\begin{figure}[pbh]
    \centering
    \includegraphics[keepaspectratio,width=0.96\textwidth]{img/kh11expC2A}

    \caption[Experiment C2A]{Experiment C2A: total number of worker bees $N(t)= H(t)+F(t)$ is shown at $t=0$ and $t=900$ for different levels of forager mortality $m$.

    With the terminology estabilished in page \pageref{sec:kh11expC2A}, we see that
    initial conditions of colonies that become ``weak'' (blue squares) lie all in
    the half--plane $m>0.5078$ (dotted vertical line).
    Similarly, initial conditions of colonies that eventually become ``strong''
    (red x's) all lie around the line $m=0.0804$ (dashed).

    In black we see the initial conditions (+) and final populations (triangles) for colonies that
    don't get to be ``strong'' nor ``weak'' at day $t=900$.

    Final population numbers for ``strong'' hives (purple circles) all happen in the left--hand side
    of the spectrum $0.0804 < m < 0.5078$.
    }

    \label{img:kh11expC2A}
\end{figure}

