\section{Modello a cinque compartimenti con operaie, \emph{Varroa} e ABPV}
Esaminiamo adesso il modello proposto nel \citeyear{ratti2017}, che estende alcune proposte precedenti con lo scopo di studiare gli effetti diretti e indiretti di \emph{Varroa destructor} sulla popolazione
di un alveare.\footcite{ratti2017}

\subsection{Derivazione del modello}
Per ottenere una formulazione del modello, gli autori propongono le seguenti assunzioni preliminari:
\begin{enumerate}
    \item La popolazione delle operaie sane (non infette da ABPV) è suddivisa in due comparti: le operaie più giovani che svolgono mansioni all'interno dell'alveare ($x_h$) e le bottinatrici/esploratrici ($x_f$).
    \footnote{Questa impostazione ricalca lo studio esaminato nella sezione
    precedente~\parencite{khoury2011}.}
%     in cui si propone un modello a due compartimenti per esaminare gli effetti del tasso di schiusa delle pupe e del tasso di reclutamento da operaie d'alveare a bottinatrici sulla dinamica di una popolazione sana.
    \item La popolazione degli acari è suddivisa in due comparti: le varroe portatrici di ABPV ($m$) e le varroe libere dal virus ($n$).
    \item Le operaie infette con ABPV ($y$) perdono rapidamente l'abilità al volo, le loro capacità di
    orientamento degradano e muoiono in breve tempo. Per questo motivo le api infette sono considerate
    ``di alveare'', e non vengono più reclutate per bottinare.
\end{enumerate}

I compartimenti del modello sono dunque cinque:
\begin{itemize}
    \item il numero di operaie di nido sane, $x_h$
    \item il numero di operaie bottinatrici sane, $x_f$
    \item il numero di operaie infette, $y$
    \item il numero di varroe portatrici di ABPV, $m$
    \item il numero di varroe senza virus, $n$
\end{itemize}

Abbiamo inoltre
$$\text{Pop. totale api operaie} = x_h + x_f + y,$$
e
$$\text{Pop. totale varroe} = m + n.$$

\paragraph{}
Proseguiamo con l'esame delle assunzioni del modello:
\begin{enumerate}
    \setcounter{enumi}{3}
    \item Il tasso massimale di riproduzione delle varroe è lo stesso, che siano portatrici di virus o meno.
    Ciò si riflette nelle equazioni \eqref{eq:r17m} e \eqref{eq:r17n}, in cui il primo termine descrive le nuove nascite: esso è proporzionale alla popolazione del compartimento ed il fattore di proporzionalità è lo stesso in entrambe le equazioni. In esso, $r$ è il tasso massimo di riproduzione del parassita \emph{Varroa}, mentre $\alpha$ è la capacità portante dell'ospite \emph{Apis}.
    \item Sappiamo che la covata necessita di cure da parte delle operaie nutrici, quindi il tasso di schiusa è dipendente dal numero di operaie sane nella popolazione, che assumiamo pari alla somma $x_h + x_f$.
    \footnote{Cfr.~\cite{khoury2011}.}
    \item L'ape regina non subisce gli effetti avversi di ABPV e dell'infestazione da varroe\footcite{privFDL}; di conseguenza, il tasso di schiusa non dipende dalla preminenza nell'alveare di varroe o di virus.
    \item La trasmissione di ABPV è esclusivamente orizzontale:\footcite{privFDL}
    ciò significa che l'infezione di una operaia avviene per trasmissione del virus da un'altra operaia infetta.\footnote{La trasmissione \emph{verticale} di un patogeno riguarda invece l'infezione della prole a causa dei genitori già infetti.}

    Le varroe sono un vettore puramente meccanico per il virus: la carica virale non è incentivata né
    inibita dalle varroe, che per l'ABPV fungono meramente da ``mezzo di trasporto'' tra le api.
    \item Poiché le pupe infette con ABPV muoiono rapidamente prima della schiusa, tutte le giovani operaie che
    giungono all'età adulta si assumono libere dal virus. L'effetto limitante sul tasso di schiusa dovuto alla
    presenza di varroe infette viene incorporato in un fattore di modulazione $h(m)$ nella~\eqref{eq:r17xh}.
    \item Per studiare l'effetto dei trattamenti acaricidi, introduciamo un ulteriore termine di pozzo $\delta_i$ in ogni equazione. Assumiamo che l'impatto dei trattamenti sia di gran lunga maggiore sulle varroe che sulle api: $\delta_4, \delta_5 \gg \delta_1, \delta_2, \delta_3$.
    \item In tutti e tre i compartimenti delle api, si assume un tasso di mortalità naturale $d_i$ in aggiunta alla mortalità dovuta alle varroe $\gamma_i$ (a sua volta dipendente sia dal parassitismo che dalla trasmissione del virus). Nelle api indebolite dall'infezione da ABPV entrambi questi tassi sono maggiori rispetto alle operaie sane: $d_3 > d_1, d_2$ e $\gamma_3 > \gamma_1, \gamma_2$.

    Inoltre per le bottinatrici si tiene conto anche della mortalità da disorientamento (fattore $p$), dovuto
    principalmente al degrado delle facoltà cognitive e di navigazione per effetto dei pesticidi
    sintetici sulle colture visitate.
\end{enumerate}


\subsubsection{Equazione del modello}
Il modello a compartimenti proposto\footcite[8]{ratti2017} assume quindi la seguente formulazione:
\begin{align}
    % hive sane
    \diff{x_h}{t} &= \mu g( x_h + x_f ) h(m) - \beta_1 m \frac{x_h}{x_h + x_f + y} - \left( d_1 + \delta_1 \right) x_h \notag \\ % barbatrucco
    \label{eq:r17xh}
        &{\;} - \gamma_1 (m+n) x_h - x_h \cdot R(x_h, x_f) \\ % barbatrucco
    % foragers sane
    \label{eq:r17xf}
    \diff{x_f}{t} &= x_h \cdot R(x_h, x_f) - \beta_1 m \frac{x_f}{x_h + x_f + y}
    - \left( p + d_2 + \delta_2 \right) x_f - \gamma_2 (m+n) x_f \\
    % (hive) infette
    \label{eq:r17y}
    \diff{y}{t} &= \beta_1 m \frac{x_h + x_f}{x_h + x_f + y} - \left( d_3 + \delta_3 \right) y - \gamma_3 (m+n) y \\
    % varroe con ABPV
    \label{eq:r17m}
    \diff{m}{t} &= rm \left( 1 - \frac{m+n}{ \alpha (x_h + x_f + y) } \right) + \beta_2 n \frac{y}{x_h + x_f + y}
    - \beta_3 m \frac{x_h + x_f}{x_h + x_f + y} - \delta_4 m \\
    % varroe senza ABPV
    \label{eq:r17n}
    \diff{n}{t} &= rn \left( 1 - \frac{m+n}{ \alpha (x_h + x_f + y) } \right) - \beta_2 n \frac{y}{x_h + x_f + y}
    + \beta_3 m \frac{x_h + x_f}{x_h + x_f + y} - \delta_5 n
\end{align}

Il termine ``di reclutamento'' $x_h \cdot R(x_h, x_f)$ nelle equazioni \eqref{eq:r17xh} e \eqref{eq:r17xf} rappresenta appunto il passaggio di ruolo delle operaie, da impiegate \emph{all'interno} dell'alveare a \emph{bottinatrici} che escono dall'arnia per esplorare e raccogliere il cibo.

Come nello studio precedente,\footnote{\cite{khoury2011}, cfr.~p.~\pageref{eq:Recr} e segg.}
il fattore $R$ rappresenta gli effetti \emph{sociali}
della colonia sul tasso di reclutamento, con la seguente formulazione:
\begin{equation}
    R(x_h, x_f) = \sigma_1 - \sigma_2 \frac{x_f}{x_h + x_f}
    \label{eq:reclutam}
\end{equation}
in cui $\sigma_1$ è il tasso massimo di reclutamento delle operaie, quando non ci sono bottinatrici nella colonia.
Il termine $\sigma_2 \frac{x_f}{x_h + x_f}$ descrive l'inibizione sociale del reclutamento, ossia il rallentamento del processo di ``promozione'' all'esterno del nido, proporzionale alla frazione di popolazione impiegata nella bottinatura; quando tale frazione è in surplus, le bottinatrici vengono ``riallocate'' in attività interne all'alveare.

\paragraph{}
Il tasso di nuove nascite è descritto dal primo termine nella equazione \eqref{eq:r17xh}, in cui troviamo il
parametro $\mu$ che è determinato principalmente dalla regina e dalla stagione, e rappresenta il tasso massimo
di schiusa.

La funzione $g$ descrive il bisogno di un numero minimo di operaie sane nella colonia per allevare la covata; se questo numero scende sotto una certa soglia -- che può presentare variazioni stagionali -- la covata non produce più api adulte.
La $g$ rappresenta la risposta funzionale dell'allevamento della covata; nelle parole degli autori:
\begin{displayquote}[{\footcite[9]{ratti2017}}]
``We think of $g(x_h + x_f)$ as a switch function.''
\end{displayquote}

È necessario che $g$ verifichi le condizioni
$$g(0)=0 \; , \quad %
\diff{g}{x_h} \geq 0 \; , \quad %
\diff{g}{x_f} \geq 0 \; ,$$
ed anche $\lim_{x_h+x_f \to \infty} g(x_h+x_f)=1$.

Un'utile formulazione per $g$ è data dalla sigmoide di Hill (figura~\ref{img:hillSigmoid}):
\begin{equation}
    g(x_h + x_f) = \frac{ (x_h+x_f)^i }{ K^i + (x_h+x_f)^i }
    \label{eq:hillSigmoid}
\end{equation}

\begin{figure}
    \centering
    \includegraphics[keepaspectratio,width=0.86\textwidth]{img/hillSigmoid}

    \caption[Sigmoidi di Hill.]{Sigmoide di Hill per diversi valori dell'esponente $i$.
        \\ Il parametro $K$ rappresenta la retroimmagine del semimassimo.}
        % TODO citarla negli strumenti quando si dicono le Holling
    \label{img:hillSigmoid}
\end{figure}

dove $K$ è la dimensione della popolazione di operaie nell'alveare corrispondente al tasso di schiusa semimassimo, e l'esponente $i>1$ si assume intero per semplicità di analisi.

\paragraph{}
Sempre nel termine delle nascite di api adulte, un altro fattore modulante tiene conto degli effetti deleteri per la covata dovuti alla presenza degli acari portanti ABPV nell'alveare: le larve e le pupe infette dal virus muoiono rapidamente prima di raggiungere lo stadio di adulte, quindi la funzione $h(m)$ nell'equazione~\ref{eq:r17xh} verifica le condizioni $h(0)=1$, $\diff{h}{m} <0$ e $\lim_{m \to \infty} h(m) = 0$.

In \cite{sumMar04} si suggerisce $h(m) \approx e^{-km}$ con $k\geq0$, e questa è la forma utilizzata in \cite{ratti2017} per le simulazioni numeriche.

Si noti che assumere $h=h(m)$ porta a una semplificazione del sistema, mentre in realtà anche la
popolazione $n$ di varroe non infette preda abitualmente le pupe di ape.

% MANCHEREBBE DI PARLARE ALMENO UNA VOLTA DEI \beta_i
% The parameter β2 in (4) is the rate at which mites that do not carry the virus acquire
% it. The rate at which infected mites lose their virus to an uninfected host is β3 . The
% rate at which uninfected (hive and forager) bees become infected is β1 , in units of
% bees per virus-carrying mite and time. We assumed that the rate at which hive bees get
% infected (i.e. β1 ) is the same as the rate at which foragers get infected (i.e. β2 ). With
% this assumption, the first term in (3) becomes the rate at which total bees get infected.

\subsection{Analisi del modello con parametri costanti}
Seguiamo l'impostazione dello studio \cite{ratti2017} per l'analisi degli equilibri, iniziando con l'esame di sotto-casi più semplici ed estendendo progressivamente i risultati fino al modello completo.

\paragraph{}
Dapprima assumiamo l'ipotesi di lavoro di coefficienti costanti; utilizzeremo in seguito la teoria di Floquet per analizzare il caso di coefficienti periodici (sottosezione~\ref{sez:paramPeriodici}).

% \paragraph{}
% Per le dimostrazioni dei risultati esposti in questa sottosezione, si rimanda all'articolo originale \cite[11--15]{ratti2017}.
Una volta stabilite le condizioni per l'esistenza dei punti di equilibrio, la metodologia standard per analizzarne la stabilità consiste nel linearizzare il sistema in un intorno di un punto di equilibrio: il carattere attrattivo o repulsivo dell'equilibrio è determinato dagli \emph{esponenti di Lyapounov locali}, ossia dalle parti reali degli autovalori della matrice jacobiana.

Questo approccio metodologico costituisce la base delle dimostrazioni in questa sottosezione e nella successiva.
\footnote{Cfr. p.~\pageref{chap:teoria} e segg. per una sintesi dei risultati teorici qui sfruttati.}

\subsubsection{Modello bidimensionale senza patogeni}
In assenza di \emph{Varroa} e di virus, il modello \eqref{eq:r17xh}--\eqref{eq:r17n} si riduce alle due sole equazioni per le api sane:
\begin{align}
    \diff{x_h}{t} &= \mu g( x_h + x_f ) - d_1 x_h - x_h \left( \sigma_1 - \sigma_2 \frac{x_f}{x_h + x_f} \right)
    \label{eq:rattiRidotto1prima}
    \\
    \diff{x_f}{t} &= x_h \left( \sigma_1 - \sigma_2 \frac{x_f}{x_h + x_f} \right) - (p + d_2) x_f
    \label{eq:rattiRidotto1seconda}
\end{align}
dove $g(x_h + x_f) = \frac{ (x_h+x_f)^i }{ K^i + (x_h+x_f)^i }$ come nella \eqref{eq:hillSigmoid}.

\paragraph{}
L'equilibrio banale $(x_h^*, x_f^*) = (0,0)$ esiste sempre, ed è asintoticamente stabile; questo risultato è contenuto nelle due proposizioni che seguono.

Nelle parole degli autori, la stabilità asintotica dell'equilibrio banale significa che
\begin{displayquote}[\footcite{ratti2017}]
``\omissis in order to establish itself as a properly working colony, a sufficiently large healthy adult bee population is required to take care of the brood.''
\end{displayquote}

E viceversa, se il numero di operaie sane scende sotto un valore critico, la colonia si avvia verso una morte inesorabile.
Si noti che questo effetto di tipo Allee si osserva già nel modello senza patogeni; l'introduzione di \emph{Varroa} e virus non può che esacerbare fenomeni di questo tipo.

\paragraph{}
Proponiamo adesso una analisi dei punti di equilibrio del modello \eqref{eq:rattiRidotto1prima}--\eqref{eq:rattiRidotto1seconda}, introducendo per comodità di notazione le due costanti
\begin{equation}
    F \coloneq \frac{1}{2} \left( \frac{ \sigma_1 - \sigma_2 - p - d_2 }{p+d_2} +
    \sqrt{ {\left( \frac{ \sigma_1 - \sigma_2 - p - d_2 }{p+d_2} \right)}^2 + \frac{4 \sigma_1}{p+d_2} } \, \right),
    \label{eq:rattiFconst}
\end{equation}
ed
$$ a \coloneq - \frac{ \mu }{ \frac{\sigma_2 F}{1+F} - d_1 - \sigma_1 }.$$

\begin{proposizione}[3.1, Existence of equilibria {\footcite[11]{ratti2017}}]
    Il modello ridotto alle sole api \eqref{eq:rattiRidotto1prima}--\eqref{eq:rattiRidotto1seconda} ammette sempre l'equilibrio banale $(0,0)$, e se $\frac{\sigma_2 F}{1+F} - d_1 - \sigma_1 >0$ allora esso è unico.

    Viceversa, se $\frac{\sigma_2 F}{1+F} - d_1 - \sigma_1 <0$ allora il sistema ammette due ulteriori equilibri strettamente positivi, a patto che sia verificata l'ulteriore condizione
    $$a > \frac{Ki}{1+F} (i-1)^{\frac{1}{i} -1}.$$
    \label{prop:rExist}
\end{proposizione}

\begin{proof}
    Le equazioni del modello sono omogenee, dunque l'equilibrio banale $(0,0)$ esiste sempre.
    Dalla \eqref{eq:rattiRidotto1seconda} segue che ogni punto di equilibrio
    $(x_h^*, x_f^*) > (0,0)$ verifica
    $$x_f^2 + \frac{\sigma_2 - \sigma_1 + p + d_2}{p +d_2} x_h x_f - \frac{\sigma_1}{p+d_2} x_h^2 = 0,$$
    da cui ricaviamo la relazione tra le componenti $x_f^* = F x_h^*$.

    Inserendo quest'ultima nell'equazione \eqref{eq:rattiRidotto1prima} otteniamo
    $$ \mu g \left( (1+F) x_h \right) -d_1 x_h -x_h \left( \sigma_1 - \sigma_2 \frac{F}{1+F} \right) = 0 \; ,$$
    da cui
    $$ \underbrace{ x_h^i -a x_h^{i-1} + \left( { \frac{K}{1+F} } \right)^i }_{ f(x_h) \coloneq } = 0 \; .$$

    Per il criterio di Descartes, il polinomio $f(x_h)$ ha due radici positive\footnote{Contate con molteplicità.}
    se $a>0$, e nessuna se $a<0$. Quindi vogliamo $a>0$ ossia $\frac{\sigma_2 F}{1+F} -d_1 - \sigma_1 < 0$, ciò che
    stabilisce la prima parte della proposizione.

    Assumendo $a>0$, abbiamo che $f'$ cambia segno in $\hat{ x_h } = a \frac{i-1}{i}$, da cui
    possiamo imporre
    $$\min f = f( \hat{ x_h } ) =
    - \frac{1}{i} a^i { \left( \frac{i-1}{i} \right) }^{i-1} + {\left( \frac{K}{1+F} \right) }^i < 0
    \; ,$$
    per ricavare infine
    $$a > \frac{Ki}{1+F} {(i-1)}^{ \frac{1}{i} - 1 }.$$
\end{proof}

L'equilibrio banale è incondizionatamente asintoticamente stabile.
Il fatto matematico che $(0,0)$ appartenga alla \emph{parte interna} del suo bacino d'attrazione
ha una applicazione sintetica sul campo: se per qualsiasi motivo una colonia si trova ``pericolosamente''
vicina ad una soglia inferiore di popolazione, non riuscirà a mantenere in funzione gli apparati vitali
dell'alveare, ed è fatalmente condannata.

L'esperienza degli apicoltori professionisti\footcite{privFDL,privFPan} che hanno praticato il nomadismo in Italia
fin dagli anni '80 conferma questo dato:
\begin{displayquote}[{\footcite{privFDL}}]
Per garantire il proprio sviluppo, una famiglia necessita di una popolazione di
almeno 12000 operaie attive; circa due telai pienamente popolati.

La stessa soglia è utilizzata per l'``equilibratura'': idealmente due famiglie nello stesso apiario,
se hanno meno di 12000 api ciascuna, vengono unite.
\end{displayquote}

\begin{proposizione}[3.2, Stability of trivial equilibrium {\footcite[12]{ratti2017}}]
    L'equilibrio banale del modello ridotto alle sole api
    \eqref{eq:rattiRidotto1prima}--\eqref{eq:rattiRidotto1seconda}

    è stabile.
\end{proposizione}

\begin{proof}
Consideriamo $V(x_h, x_f) = x_h + x_f$ e dimostriamo che è di Lyapounov per l'origine.
La $V$ ha un minimo locale forte in $(0,0)$; inoltre
$$\dot{V} = \diff{x_h}{t} + \diff{x_f}{t} = \mu g(x_h + x_f) -d_1 x_h - (p+d_2) x_f
\leq
 \mu g(x_h + x_f) - \delta (x_h + x_f)
\; ,$$
prendendo $0 < \delta \leq \min (d_1, p + d_2)$.
Si conclude osservando che $\mu g(x) - \delta x <0$ per $x>0$ sufficientemente piccolo.
\end{proof}

La matrice jacobiana del sistema in configurazione generica $(x_h, x_f)$ è
\begin{equation}
    D(x_h, x_f) =
    \begin{pmatrix}
        \mu g' -d_1 - \sigma_1 + \sigma_2 \frac{x_f^2}{{(x_h+x_f)}^2}
        & \mu g' + \sigma_2 \frac{x_h^2}{{(x_h+x_f)}^2}
        \\
        \sigma_1 - \sigma_2 \frac{x_f^2}{{(x_h+x_f)}^2}
        & - \sigma_2 \frac{x_h^2}{{(x_h+x_f)}^2} - (p+d_2)
    \end{pmatrix} \; ,
    \label{eq:r17jaco}
\end{equation}
dove è indicato per brevità di notazione
$$
    g' = g'(x_h+x_f) \coloneq \diffp{g}{{x_h}} (x_h + x_f) = \diffp{g}{{x_f}} (x_h + x_f) =
    \frac{ iK^i {(x_h + x_f)}^{i-1} }{ { \left( K^i + {(x_h +x_f)}^i \right)  }^2 }
    \; .
$$

\begin{proposizione}[3.3, Stability of non--trivial equilibria {\footcite[13]{ratti2017}}]
    Sia $(x_h^*, x_f^*)$ un equilibrio non banale, come da proposizione~\ref{prop:rExist}.
    Tale punto è stabile se
    $$ \mu \frac{ i K^i (1+F)^{i-1}{x_h^*}^{i-1} }{ {\left[ K^i +(1+F)^i {x_h^*}^i \right]}^2 }
    < \frac{d_1 + (p+d_2)F}{1+F} \; .$$
\end{proposizione}
\begin{proof}
    Sostituendo $x_f = F x_h$ nella jacobiana~\eqref{eq:r17jaco} otteniamo
    \begin{equation}
    D(x_h, F x_h) =
    \begin{pmatrix}
        \mu g_h' -d_1 - \sigma_1 + \sigma_2 \frac{F^2}{(1+F)^2}
        & \mu g_h' + \frac{\sigma_2}{(1+F)^2}
        \\
        \sigma_1 - \sigma_2 \frac{F^2}{(1+F)^2}
        & - \frac{\sigma_2}{(1+F)^2} - (p+d_2)
    \end{pmatrix} \; ,
    \label{eq:r17jacoFxh}
\end{equation}
dove analogamente a poco sopra usiamo per brevità
$$g_h' \coloneq g'( x_h + F x_h ) = \frac{i K^i {(1+F)}^{i-1} x_h^{i-1} }{ { \left( K^i + {(1+F)}^i x_h^i
        \right)  }^2 } \; .$$

Dalla definizione di $F$~\eqref{eq:rattiFconst} possiamo ricavare
$\sigma_1 = (p+d_2)F + \sigma_2 \frac{F}{1+F}$, che sostituita nella prima colonna della
della jacobiana~\eqref{eq:r17jacoFxh} ci permette di trovare l'espressione per il determinante
% Ricaviamo adesso per il determinante l'espressione
% \begin{multline}
%     \det D(x_h, F x_h) = - \frac{\sigma_2}{ {(1+F)}^2 } \left( \mu g_h' -d_1 + (p+d_2) F^2 \right) - \\
%     - (p+d_2) (\mu g_h' -d_1) -\sigma_1 \left( \mu g_h' -(p+d_2) \right) + \\
%     + \sigma_2 \frac{F^2}{{(1+F)}^2} \mu g_h' % questo termine Ratti e amici non ce l'hanno
%     \; ,
% \end{multline}
$$
\det D(x_h, F x_h) =
\left[ \frac{\sigma_2}{(1+F)^2} +(p+d_2) \right] \cdot
\left[ { - \mu g_h' (1+F) + \left( d_1 +(p+d_2)F \right) } \right]
\; ,
$$
da cui
$$
\det D(x_h, F x_h) > 0 \quad \iff \quad
\mu g_h' < \frac{d_1 + (p+d_2)F}{1+F} \; ,$$
che è la disequazione nella tesi.

\paragraph{}
Si noti che la $\det D(x_h, F x_h) > 0$ implica $\tr D(x_h, F x_h) <0$: infatti se la disequazione
precedente è valida, allora abbiamo
\begin{multline*}
\tr D(x_h, F x_h) = \mu g_h' -d_1 -(p+d_2)F - \frac{\sigma_2 F}{(1+F)^2} - \frac{\sigma_2}{(1+F)^2} - (p + d_2) \overbrace{<}^{\det >0} \\
< - \frac{d_1 F}{1+F} - \frac{p +d_2}{1+F} - \frac{\sigma_2}{1+F} < 0 \; ,
\end{multline*}
dunque la somma degli autovalori è negativa, e la loro parte reale è negativa.

Ciò implica che il punto d'equilibrio sia stabile.
\end{proof}

% \paragraph{LORO DICONO: }
% Se esistono, gli equilibri positivi sono stabili se
% \begin{equation}
% \frac{i \mu K^i {(x_h^*)}^{i-1} {(1+F)}^{i-1}}{ {\left( K^i + {(x_h^*)}^i {(1+F)}^i \right)}^2 }
% <
% \frac{d_1 + (p+d_2) F}{{(1+F)}^2}.
% \label{eq:rattiRidotto1stabilitySECONDOLORO}
% \end{equation}

\subsubsection{Modello tridimensionale api--acaro}
Introduciamo adesso nella colonia le varroe senza virus, ed indaghiamo la conseguente alterazione della stabilità degli equilibri individuati nel caso precedente.

Si eliminano quindi dal modello \eqref{eq:r17xh}--\eqref{eq:r17n} le equazioni per le api infette da ABPV e per le varroe che trasportano il virus; notiamo che da $m \equiv 0$ segue che $h(m) \equiv 1$.

\paragraph{}
Il modello ridotto che si ottiene è il seguente:

\begin{align}
    \diff{x_h}{t} &= \mu g( x_h + x_f ) - (d_1 + \delta_1) x_h - \gamma_1 n x_h
        - x_h \left( \sigma_1 - \sigma_2 \frac{x_f}{x_h + x_f} \right)
        \label{eq:rattiRidotto2prima}
    \\
    \diff{x_f}{t} &= x_h \left( \sigma_1 - \sigma_2 \frac{x_f}{x_h + x_f} \right) - (p + d_2 + \delta_2) x_f
        - \gamma_2 n x_f
        \label{eq:rattiRidotto2seconda}
    \\
    \diff{n}{t} &= rn \left( 1 - \frac{n}{ \alpha (x_h + x_f) } \right) - \delta_5 n
        \label{eq:rattiRidotto2terza}
\end{align}

Anche in questo caso, l'equilibrio banale $(x_h, x_f, n)=(0,0,0)$ esiste ed è sempre asintoticamente stabile, con la medesima interpretazione pratica del caso precedente (effetto Allee forte).

\paragraph{Equilibrio senza varroe. } Consideriamo adesso un equilibrio non banale $(x_h^*, x_f^*)$ per il modello \eqref{eq:rattiRidotto1prima}--\eqref{eq:rattiRidotto1seconda}, secondo le condizioni di esistenza date sopra.
Il punto $(x_h^*, x_f^*, 0)$ è un equilibrio per il modello \eqref{eq:rattiRidotto2prima}--\eqref{eq:rattiRidotto2terza}.

Se $r > \delta_5$ allora è instabile; se viceversa $r< \delta_5$ allora il punto $(x_h^*, x_f^*, 0)$ eredita la stabilità di $(x_h^*, x_f^*)$ sotto le \eqref{eq:rattiRidotto1prima}--\eqref{eq:rattiRidotto1seconda}.

\paragraph{}
Ciò significa che l'equilibrio senza varroe è stabile se la \eqref{eq:rattiRidotto1stability} è soddisfatta
ed il tasso massimo di natalità delle varroe è inferiore alla loro mortalità dovuta ai trattamenti.
Si osservi inoltre che in assenza di trattamenti varroacidi l'alveare non riesce a debellare l'infestazione dell'acaro.

Nel caso instabile, non è chiaro se il sistema converge all'equilibrio banale oppure ad un equilibrio endemico. \cite[15]{ratti2017}


\subsubsection{Modello completo api--\emph{Varroa}--ABPV}
Consideriamo infine il modello completo \eqref{eq:r17xh}--\eqref{eq:r17n} e studiamo la stabilità dell'equilibrio senza patogeni.

Sia $(x_h^*, x_f^*)$ un equilibrio positivo per il modello ridotto \eqref{eq:rattiRidotto1prima}--\eqref{eq:rattiRidotto1seconda}, sempre secondo le condizioni di esistenza di cui sopra; che $(x_h^*, x_f^*, 0,0,0)$ sia un equilibrio per \eqref{eq:r17xh}--\eqref{eq:r17n} è ovvio.

Se valgono entrambe le disuguaglianze
$$
\begin{sistema}
    r - \beta_3 - \delta_4 < 0 \\
    r - \delta_5 < 0,
\end{sistema}
$$
allora $(x_h^*, x_f^*, 0,0,0)$ eredita la stabilità di $(x_h^*, x_f^*)$ sotto le \eqref{eq:rattiRidotto1prima}--\eqref{eq:rattiRidotto1seconda}.
Se invece (almeno) una delle disuguaglianze è invertita, l'equilibrio è instabile.

\paragraph{}
Quest'ultimo risultato si può interpetare con la seguente condizione: affinché l'alveare sconfigga completamente la \emph{Varroa} ed il virus, è necessario e sufficiente che
\begin{itemize}
    \item i trattamenti varroacidi siano abbastanza potenti da eradicare gli acari che non portano ABPV;
    \item le varroe portatrici perdano la loro carica virale ad un tasso più rapido della loro natalità.
\end{itemize}

L'esperienza sul campo dell'apicoltura, sia sperimentale che in produzione, indica come estremamente improbabile l'eradicazione completa della \emph{Varroa}.~\cite{privFPan}

Piuttosto, è più ragionevole che \textquote[\cite{privFDL}]{a livello di alveare o di apiario, si riesca a mantenere per un certo periodo (almeno 3 o 4 stagioni) un equilibrio in cui api e varroe coesistono senza causare il collasso delle colonie.}

Ciò corrisponderebbe alla presenza di un equilibrio endemico nel modello, la cui esistenza tuttavia non è ancora dimostrata.
Notiamo inoltre che abbiamo condizioni sufficienti per la stabilità degli equilibri positivi, ma non per la stabilità asintotica: un punto di equilibrio stabile ma non attrattivo potrebbe presentare dei cicli-limite.


\paragraph{}
Le attività in apiario che promuovono la sopravvivenza dell'alveare richiedono un monitoraggio costante e diversi tipi di interventi, che includono i trattamenti acaricidi, ma che non possono limitarsi al contrasto della \emph{Varroa}: è infatti necessario il mantenimento di \textquote[\cite{privFDL}]{\omissis uno stato di salute complessivamente buono nella colonia, che riguarda anche l'efficienza del sistema immunitario, la ``forza'' della famiglia in termini di operaie sane, le scorte di cibo, la salute della regina, e molti altri fattori.}

Questo elemento di complessità è catturato anche nel modello relativamente semplice di \cite{ratti2017}, in quanto le condizioni di esistenza e stabilità dei punti di equilibrio includono una molteplicità di parametri, che nel modello rappresentano non solo gli effetti ``diretti'' dovuti all'acaro e al virus, ma anche lo stato di salute dell'alveare, compresa la sua capacità di allevare covata e di approvvigionarsi di cibo.


\subsection{Analisi del modello con parametri periodici}
\label{sez:paramPeriodici}
Questo ultimo passaggio è fondamentale per incorporare nel modello le variazioni stagionali nel tasso di deposizione da parte della regina, l'arresto della raccolta di cibo nel periodo invernale e di molte altre attività dell'alveare.

Il periodo dei coefficienti è dunque $T= 1 \text{ anno}$.

suloz non banali, Floquet

\begin{proposizione}[4.1, Stability of mite--free periodic solution {\cite[19]{ratti2017}} ]
    Sia $\left( x_h^*(t), x_f^*(t) \right)$ sia una soluzione periodica per il modello ridotto alle sole api \eqref{eq:rattiRidotto1prima}--\eqref{eq:rattiRidotto1seconda}.

    Allora $\left( x_h^*(t), x_f^*(t), 0 \right)$ è una soluzione periodica per il modello ridotto api--acaro \eqref{eq:rattiRidotto2prima}--\eqref{eq:rattiRidotto2terza}; è stabile se
    \begin{enumerate}
        \item $\left( x_h^*(t), x_f^*(t) \right)$ è stabile per il modello ridotto alle sole api;
        \item $\int_0^T (r - \delta_5) dt \leq 0$.
    \end{enumerate}
\end{proposizione}


\begin{proposizione}[4.2, Stability of the disease--free periodic solution {\cite[21]{ratti2017}} ]
    Sia $\left( x_h^*(t), x_f^*(t) \right)$ sia una soluzione periodica per il modello ridotto alle sole api \eqref{eq:rattiRidotto1prima}--\eqref{eq:rattiRidotto1seconda}.

    Allora $\left( x_h^*(t), x_f^*(t), 0, 0, 0 \right)$ è una soluzione periodica per il modello completo \eqref{eq:r17xh}--\eqref{eq:r17n}; è stabile se
    \begin{enumerate}
        \item $\left( x_h^*(t), x_f^*(t) \right)$ è stabile per il modello ridotto alle sole api;
        \item $\int_0^T (r - \delta_5) dt \leq 0$;
        \item $\int_0^T (r -\beta_3 - \delta_4) dt \leq 0$.
    \end{enumerate}
\end{proposizione}



\subsection{Riduzioni ai modelli precedenti}

Se nell'espressione di $g$ (equazione~\eqref{eq:hillSigmoid}) prendiamo $K=0$ ci riduciamo al modello di \cite{sumMar04}.

Con $i=1$ e togliendo i compartimenti di virus e varroe, e riducendo a zero la mortalità nel
comparto $x_h$, ci si riduce a \cite{khoury2011}. Si veda in proposito la tabella~\ref{tab:kh11VSratti17} per
un confronto tra le notazioni.

\begin{table}[pbh]
    $$\begin{array}{ccc}
        \toprule
        \text{Khoury~2011}
        & \text{Ratti~2017}
        & \text{u.m.} \\
        \midrule
        H & x_h & \text{api} \\
        F & x_f & \text{api} \\
        m & d_2 & \\ % TODO
        L & \mu & \\
        \alpha & \sigma_1 & \\
        \sigma & \sigma_2 & \\
        E & g & \\
        \midrule
        \multicolumn{2}{c}{R} & \\
        \midrule
        J & F & \\
        \bottomrule
    \end{array}$$
    \caption{Notazione in Khoury~2011 vs. Ratti~2017.}
    \label{tab:kh11VSratti17}
\end{table}

