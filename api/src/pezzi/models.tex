\chapter{Modelli matematici}
\label{chap:modelz}
\begin{flushright}
    \small
    \textquote[Andrea~Milani~Comparetti]{Adesso vi darò \emph{la mia} definizione di stabilità asintotica.
        Di questo concetto esistono tante definizioni -- cioè interpretazioni -- quante sono le persone
        che studiano i sistemi dinamici.}
\end{flushright}

In questo capitolo discuteremo alcuni modelli proposti in letteratura per descrivere alveari o apiari di
\emph{A.~mellifera}, e le sue interazioni con patogeni e altri fattori di stress.

\paragraph{}
Per una panoramica sui principali modelli a proposito di api rimandiamo alla review di
\citeauthor{chen_review}\footcite{chen_review}, in cui si comparano molte proposte diverse e si forniscono utili
tabelle di parametri coi valori rilevati sul campo durante gli studi citati.
Risulta efficace anche una tabella in cui si riassumono le caratteristiche dei modelli esaminati, facilitando
la loro comparazione.

\paragraph{}
Il primo modello che verrà illustrato nella sezione~\ref{sec:kh11} è piuttosto
semplice\footcite{khoury2011}, eppure esprime efficacemente alcune
caratteristiche critiche nei procesi demografici di \emph{A. mellifera}, e le proposte--chiave di questo
articolo sono state integrate spesso in modelli successivi e più sofisticati.

\blockquote[{\footcite[4]{khoury2011}}]{The model that we present here is very simple and focuses on the effect
of varying forager death rate on brood and adult bee population dynamics.}

\paragraph{}
Il secondo modello, considerato nella sezione~\ref{sec:ratti17}, divide la popolazione di
operaie sane tra ``api di casa''
e ``api di campo'', caratteristica ereditata dallo studio precedente\footcite{khoury2011}.

Gli altri compartimenti del modello riguardano le api infette dal virus della paralisi (ABPV) e degli acari
\emph{V.~destructor}, che possono trasportare o meno il virus.

Lo studio originale di questo modello\footcite{ratti2017}  mira ad evidenziare il ruolo \emph{preventivo}
dei trattamenti anti--\emph{Varroa} nel contrastare lo spopolamento degli alveari.
In questo contesto, la \emph{Varroa} è uno soltanto dei molteplici fattori di stress, ma sembra giocare
un ruolo centrale nel \emph{catalizzare} altri effetti nefasti, il cui risultato combinato porta al collasso
della colonia se sostenuto nel tempo oltre i livelli--limite.

\section[Modello a due compartimenti con reclutamento]{Modello a due compartimenti con reclutamento sociale}
\label{sec:kh11}

Nel \citeyear{khoury2011} è stato proposto un modello\footcite{khoury2011} per indagare le dinamiche sociali
tra le caste della popolazione operaia in \emph{Apis~mellifera}.

Nello studio si ottengono due equazioni, in cui le variabili reali e non negative $H$ ed $F$ rappresentano
il numero delle operaie di nido e delle foraggiatrici, rispettivamente:

\begin{align}
    \diff{H}{t} &= \overbrace{E(H,F)}^{\text{eclosion}}- H \cdot R(H,F) \label{eq:kh11h} \; , \\
    \diff{F}{t} &= H \cdot R(H,F)  - m \cdot F \label{eq:kh11f} \; ,
\end{align}
dove $m$ è il tasso di mortalità, ed $E$ il tasso di schiusa.
$R$ rappresenta la funzione di \emph{reclutamento} nel termine di
trasferimento $H \to F$.

Un diagramma sintetico di questo modello è illustrato in figura~\ref{img:kh11diagram}.

\begin{figure}[hbp]
    \centering
    \includegraphics[keepaspectratio,width=0.76\textwidth]{img/pone.0018491.g001}

    \caption[\figurename~1 da \parencite{khoury2011}]{Diagramma del modello che illustra le mutue interazioni
        tra i compartimenti. Si noti che la mortalità delle api di nido è trascurabile.
        \\
        { \footnotesize \ttfamily doi: 10.1371/journal.pone.0018491.g001 }
        }
    \label{img:kh11diagram}
\end{figure}


\paragraph{}
Alcuni oggetti proposti in questo modello si sono dimostrati molto interessanti e sono stati incorporati
in seguito da altri modelli, con variazioni minime;
uno di questi modelli\footcite{ratti2017} sarà esaminato nella prossima sezione a p.~\pageref{sec:ratti17}.


\subsection{Reclutamento con inibizione sociale.}
Nello studio\footcite{khoury2011} la funzione di reclutamento è rappresentata dalla funzione
\begin{equation}
    \label{eq:Recr}
    R(H,F) = \alpha - \sigma \frac{F}{H+F} \; ,
\end{equation}
in cui $\alpha$ rappresenta il tasso massimo di reclutamento.\footnote{Raggiunto quando nella colonia non è
presente nessuna bottinatrice.}

Il secondo termine rappresenta l'inibizione sociale del processo di reclutamento.
Gli autori assumono
\textquote[{\footcite[3]{khoury2011}}]{\omissis that social inhibition is directly proportional to the fraction of the total number of adult bees that are foragers, such that a high fraction of foragers in the hive results in low recruitment.}

\paragraph{}
La ricerca biologica suggerisce che l'ethyl~oleate scambiato tra le api operaie sia
il principale modulatore nel processo di reclutamento sociale.
\footcite{ethyloleate,ethyloleate2,meccanica}

Ad esempio una colonia risponderà ad una elevata mortalità di bottinatrici (più anziane) attraverso
la riduzione dell'inibizione del reclutamento (dati i livelli più bassi di feromone) che dunque risulta
\textquote[\footcite{khoury2011}]{\omissis in a precocious onset of foraging behaviour in young bees}.

Nella situazione opposta -- \ie quando vi sono molte bottinatrici e la loro mortalità è bassa -- la colonia
risponde \emph{aumentando} l'inibizione sociale: un maggior numero di bottinatrici si astiene dunque dal volo e
dalle attività all'esterno, ripiegando sui lavori di cura all'interno del nido. Questa risposta inibitoria
è probabilmente dovuta alla maggiore concentrazione di ethyl~oleate:
\textquote[{\footcite[1]{khoury2011}}]{Old forager bees transfer ethyl oleate to young hive bees via trophallaxis, which delays the age at which they begin foraging.}

\paragraph{}
Tramite questo meccanismo mediato dai segnali feromonici, l'alveare tenta di mantenere un equilibrio
demografico tra le due classi, sufficiente a soddisfare le necessità della covata e dell'importazione
di polline, nettare, acqua.

In questo scenario la risposta collettiva -- che riduce il tasso di reclutamento -- aumenta con
l'intensità degli scambi di ethyl oleate, che a loro volta sono correlati al rapporto $\frac{F}{H+F}$:
queste assunzioni motivano la presenza di tale fattore nell'equazione~\eqref{eq:Recr}.


\paragraph{Tassi di mortalità.}
Come si è notato in precedenza,\footnote{Nel capitolo~\ref{chap:bio}.}
le operaie adulte -- esploratrici e bottinatrici -- sono soggette ad una gamma più ampia di fattori di stress
nell'ambiente esterno, rispetto alle operaie di nido. Di conseguenza l'aspettativa di vita per le operaie di
nido è più lunga rispetto quella delle foraggiatrici:
\textquote[{\footcite[1]{khoury2011}}]{Survival of bees in the protected hive environment is high, but the
survival of forager bees is much lower. \omissis The average foraging life of a bee has been estimated as less
than seven days, because of the many risks and severe metabolic costs associated with foraging.}

Il termine di pozzo $-mF$ nell'equazione~\eqref{eq:kh11f} rappresenta la mortalità delle foraggiatrici;
il tasso parametrico $m$ imposta la frazione del comparto $F$ che muore ogni giorno.\footnote{Cfr. %
p.~\pageref{par:mInverseFlightspan} e tab.~\ref{tab:FDLruoliEta}.}


\paragraph{}
L'assenza di un fattore di mortalità nel comparto delle operaie di nido nell'equazione~\eqref{eq:kh11h} è
una assunzione piuttosto severa, e comporta una netta semplificazione del modello.

Tuttavia, essa è giustificata dall'obiettivo principale dell'articolo, ossia modellare la relazione tra le varie
forme di collasso nella popolazione di colonie di \emph{A.~mellifera} e gli eccezionali tassi di mortalità
tra le foraggiatrici.
\footcite[1,2,3,5]{khoury2011}

\subsection{Tasso di schiusa.}
Il termine $E(H,F)$ tiene conto della schiusa della covata che emerge come giovani operaie adulte nel
comparto $H$ delle api di nido.
\blockquote[{\footcite[2]{khoury2011}}]{\omissis the number of eggs reared in a colony (and hence the eclosion rate) is related to the number of bees in the hive. Big colonies raise more brood.}

Come ricordato nel capitolo~\ref{chap:bio}, il sostentamento di una popolazione numerosa dipende non soltanto
dal tasso di deposizione della regina, ma anche dal numero di operaie giovani che rimangono nel nido per
accudire uova e larve, al fine di massimizzare il numero di esse che giunge sana al termine di tutti gli stadi
della metamorfosi.

\paragraph{}
Gli autori scelgono la formulazione
\begin{equation}
    \label{eq:eclos}
    E(H,F) = L \frac{H+F}{w + H + F} \; ,
\end{equation}
ove $L$ rappresenta il tasso di deposizione della regina e $w$ è la rapidità con cui $E$ cresce verso $L$ quando
la popolazione totale $H+F$ aumenta.\footnote{In questa sezione useremo spesso l'abbreviazione
$N=H+F$ per il numero totale di operaie, di nido e di campo, presenti nella colonia ad un dato istante $t$.}

Infatti, sommando le equazioni \eqref{eq:kh11h}--\eqref{eq:kh11f} e trascurando il termine di mortalità otteniamo
il tasso di crescita asintotico della popolazione totale
$$\diff{N}{t} \simeq \frac{L}{ \frac{w}{N} + 1} \approx L \; , \quad \text{quando } \frac{w}{N} \ll 1 \; ,$$
ovvero per popolazioni $N$ molto grandi.

La $E$ è una funzione Holling di tipo II, in cui il tasso di attacco è $a=\frac{1}{w}$ e $w$
è la retroimmagine del semimassimo nella risposta. In figura~\ref{img:eclos} è riportata $E(N)$ in funzione di $N$.
Possiamo vedere come valori
crescenti di $w$ ``schiacciano'' la curva mentre la saturazione è raggiunta per valori sempre maggiori di $N$,
e il grafico di $E(N)$ si ``inclina'' verso la destra.

\begin{figure}[hbp]
    \centering
    \includegraphics[keepaspectratio,width=0.86\textwidth]{img/hollingTypeII-eclosion}

    \caption[Schiusa, Holling tipo II]{$E$ rappresenta la quantità di uova deposte dalla regina che raggiunge lo
    stadio adulto, come risposta funzionale al numero totale di operaie $N=H+F$.
    \\
    Sono illustrati tre diversi valori di $w$, mantenendo costante il tasso di deposizione $L=2000$.}
    \label{img:eclos}
\end{figure}

\paragraph{}
Modelli successivi\footcite{ratti2017} hanno incorporato questa proposta, modificandola in una funzione Holling
di tipo III e aggiungendo certi fattori di modulazione ulteriori, ma tutto sommato hanno mantenuto il nucleo
della formulazione molto simile a quanto~\citeauthor{khoury2011} propongono nell'equazione~\eqref{eq:eclos}.


\subsection[Configurazioni di equilibrio]{Configurazioni di equilibrio e analisi della stabilità}
Rivediamo adesso i risultati forniti nell'articolo, a proposito dell'esistenza e della stabilità
dei punti di equilibrio nel modello~\eqref{eq:kh11h}--\eqref{eq:kh11f}.

Gli autori hanno usato \textquote[{\footcite[3]{khoury2011}}]{standard linear stability analysis and phase
plane analysis} e ripercorrendo i loro passi forniremo criteri per l'esistenza
di equilibri banali e non, \ie zeri del membro destro delle equazioni del modello.

\paragraph{}
Una volta stabilita l'esistenza di un punto di equilibrio, calcoliamo la matrice jacobiana del sistema e
la valutiamo sull'equilibrio stesso; un semplice criterio sufficiente per la stabilità asintotica
\footnote{Vedasi il Teorema~\ref{teo:pozzoNonLineare} a p.~\pageref{teo:pozzoNonLineare}.}
è che $\Re (\lambda) < 0$ per ogni autovalore $\lambda$ della matrice.

Un'altra tecnica, dovuta a Lyapounov, consiste nell'esibire una funzione che abbia un minimo
locale forte sul punto di equilibrio e che sia localmente decrescente lungo le
soluzioni.\footnote{Vedasi il Teorema~\ref{teo:lyapounovFunc} a p.~\pageref{teo:lyapounovFunc}.}

\paragraph{}
Si noti da subito che $H, F$ sono non--negative, \ie $H(t), F(t) \geq 0$ per ogni $t$.
Inoltre, tutti i parametri del modello debbono essere strettamente positivi.

Il prossimo risultato stabilisce una condizione necessaria per tutti i punti d'equilibrio non banali.

\begin{lemma}
    \label{lem:necessJ}
    Tutti i punti di equilibrio $(H^*, F^*) \neq (0,0)$
    del modello~\eqref{eq:kh11h}--\eqref{eq:kh11f} verificano
    \begin{equation}
    F^* = J H^* \; ,
    \label{eq:FeqJH}
    \end{equation}
    dove la costante adimensionale $J$ vale
    \begin{equation}
        J \coloneq \frac{1}{2} \left[
            \frac{\alpha}{m} - \frac{\sigma}{m} - 1 +
            \sqrt{ {\left( \frac{\alpha}{m} - \frac{\sigma}{m} - 1 \right)}^2
                + 4 \frac{\alpha}{m}
            }
        \right].
        \label{eq:kh11posEqJ}
    \end{equation}
\end{lemma}

Vale a dire, i punti di equilibrio giacciono sulla semiretta $F= JH$ nello spazio delle fasi, quando esistono.

\begin{proof}
    Annullando il membro destro dell'equazione~\eqref{eq:kh11f} otteniamo
    $$F^2 - \left( \frac{\alpha}{m} - \frac{\sigma}{m} -1
        \right) H F - \frac{\alpha}{m} H^2 = 0 \; .$$
    Risolvendo rispetto ad $F$ otteniamo il discriminante
    $$\Delta = \left[ {\left( \frac{\alpha}{m} - \frac{\sigma}{m} -1
        \right)}^2 + 4 \frac{\alpha}{m}
        \right] H^2 \; , $$
    che è sempre non negativo.

    \paragraph{}
    Adottando la definizione di $J$ dell'equazione~\eqref{eq:kh11posEqJ}, possiamo vedere
    che ogni punto d'equilibrio $(H^*, F^*)$ deve necessariamente verificare la~\eqref{eq:FeqJH}.
\end{proof}


Calcoliamo adesso la matrice jacobiana del sistema nella configurazione generale $(H, F)$:
\begin{equation}
    D(H, F) =
    \begin{pmatrix}
        L \frac{w}{{(w + H + F)}^2} - \alpha + \sigma \frac{F^2}{{(H+F)}^2} &
        L \frac{w}{{(w + H + F)}^2} + \sigma \frac{H^2}{{(H+F)}^2}
        \\
        \alpha - \sigma \frac{F^2}{{(H+F)}^2} &
        - \sigma \frac{H^2}{{(H+F)}^2} - m
    \end{pmatrix} \; .
    \label{eq:jacoGeneral}
\end{equation}

Se assumiamo che $(H, F) = (H^*, F^*)$ sia un punto d'equilibrio, sotto la condizione ~\eqref{eq:kh11posEqJ}
la matrice jacobiana~\eqref{eq:jacoGeneral} si riduce all'espressione
\begin{equation}
    D \left( F/J, F \right) =
    \begin{pmatrix}
        L \frac{w}{ {\left[ w + \left( \frac{1}{J} +1 \right) F \right]}^2 } - \alpha
        + \frac{ \sigma }{ {\left( \frac{1}{J} + 1 \right)}^2 } &
        %
        L \frac{w}{ {\left[ w + \left( \frac{1}{J} +1 \right) F \right]}^2 }
        + \frac{ \sigma }{ {\left( J + 1 \right)}^2 } \\
        %
        \alpha - \frac{ \sigma }{ {\left( \frac{1}{J} + 1 \right)}^2 } &
        %
        - \frac{ \sigma }{ {\left( J + 1 \right)}^2 } - m
    \end{pmatrix} \; ,
    \label{eq:jacoOnJFH}
\end{equation}
in cui la seconda riga dipende soltanto dai parametri, e non dalle variabili $H$ ed $F$.

% \begin{gather}
%     \diffp*{ \dot{H} }{H}{H= \frac{1}{J} F} =
%     L \frac{w}{ {\left[ w + \left( \frac{1}{J} +1 \right) F \right]}^2 } - \alpha
%     + \frac{ \sigma }{ {\left( \frac{1}{J} + 1 \right)}^2 } \; ,
%     \\
%     \diffp*{ \dot{H} }{F}{H= \frac{1}{J} F} =
%     L \frac{w}{ {\left[ w + \left( \frac{1}{J} +1 \right) F \right]}^2 }
%     + \frac{ \sigma }{ {\left( J + 1 \right)}^2 } \; ,
%     \\
%     \diffp*{ \dot{F} }{H}{H= \frac{1}{J} F} =
%     \alpha - \frac{ \sigma }{ {\left( \frac{1}{J} + 1 \right)}^2 } \; ,
%     \\
%     \diffp*{ \dot{F} }{F}{H= \frac{1}{J} F} =
%     - \frac{ \sigma }{ {\left( J + 1 \right)}^2 } - m \; ,
%     \label{eq:jacoOnJFH}
% \end{gather}
% e le ultime due equazioni dipendono soltanto dai parametri, non dalle variabili $H$ ed $F$.

\subsubsection{Equilibrio banale}
L'equilibrio banale $(0,0)$ è chiaramente sempre soluzione\footnote{Dunque il sistema ammette
\emph{sempre} l'equilibrio banale.}
di $$\dot{H}=\dot{F}=0 \; ,$$ e lo studio della sua stabilità è cruciale.
Infatti, nelle regioni di parametri per cui questo equilibrio è stabile e attrattivo generalmente si ha il declino
inevitabile dell'intera colonia.

\paragraph{}
La matrice jacobiana~\eqref{eq:jacoGeneral} del modello valutata nell'origine è:
\begin{equation}
    D_{00} \coloneq
    \begin{pmatrix}
        \frac{L}{w} -\alpha &
        \frac{L}{w}
        \\
        \alpha & -m
    \end{pmatrix} \; .
    \label{eq:jacoEquZero}
\end{equation}

% Con il Teorema del pozzo non lineare\footnote{Cfr. p.~\pageref{teo:pozzoNonLineare}}
% possiamo vedere il motivo per cui la condizione~\ref{eq:cond6b}
% $\alpha -\frac{L}{w}>0$
% sia utile per caratterizzare la stabilità dell'origine.\footcite[3]{khoury2011}

Abbiamo
$$\det  D_{00} > 0 \quad \iff \quad
m > \frac{\alpha L}{w} {\left( \alpha - \frac{L}{w} \right)}^{-1} \; ,$$
e
$$ \alpha - \frac{L}{w} > 0  \quad \implies \quad
\tr D_{00} = - \left( \alpha - \frac{L}{w} \right) - m < 0 \; ,$$

da cui ricaviamo che se $\alpha -\frac{L}{w}>0$ (condizione~\eqref{eq:cond6b}), allora $D_{00}$ ha
determinante positivo e traccia negativa, cioè gli autovalori sono concordi e negativi;
dunque\footnote{Cfr. p.~\pageref{teo:pozzoNonLineare}} l'origine è stabile.

\paragraph{}
Riassumiamo quanto appena discusso nel seguente
\begin{lemma}[Condizione sufficiente per la stabilità dell'equilibrio banale]
    Se $\alpha - \frac{L}{w} > 0$ e
    $$m > \frac{\alpha L}{w \alpha -L} \; ,$$
    allora l'equilibrio banale del modello \eqref{eq:kh11h}--\eqref{eq:kh11f} è asintoticamente stabile.
    \label{lem:eqBanKh}
\end{lemma}


\subsubsection{Equilibrio positivo}
Indaghiamo adesso nel dettaglio le condizioni per l'esistenza e la stabilità di un equilibrio non banale, ossia
con popolazione strettamente positiva. Tali punti dovrebbero rappresentare l'esito favorevole di una colonia che
sopravvive alle avversità.

\paragraph{}
Presentiamo una successione di risultati più semplici, al fine di costituire un criterio per l'esistenza di
un punto d'equilibrio positivo.

\begin{proposizione}[Unicità dell'equilibrio positivo]
    Se il modello~\eqref{eq:kh11h}--\eqref{eq:kh11f} ammette un equilibrio positivo $(H^*, F^*)$, allora
    questi è unico.

    Esso è determinato dall'equazione~\eqref{eq:kh11posEqJ}, insieme con la seguente:
    \begin{equation}
        F^* = \frac{L}{m} - w \frac{J}{1+J} \; .
        \label{eq:kh11posEqF}
    \end{equation}

    \label{teo:exUniqFstarPos} % WARNING prima era teorema, ho cambiato idea dopo.
\end{proposizione}

La possibilità della non--esistenza dell'equilibrio positivo è implicitamente
catturata nell'equazione~\eqref{eq:kh11posEqF}, laddove si abbia $F^* \leq 0$.

\begin{proof}
    Otteniamo la dimostrazione con alcuni calcoli elementari.

    Dal momento che stiamo cercando un equilibrio non banale $(H^*, F^*)$, per il
    Lemma~\ref{lem:necessJ} possiamo supporre ambo le componenti strettamente positive.

    Ancora per il Lemma~\ref{lem:necessJ}, possiamo sostituire $H=\frac{1}{J} F$ nel membro destro della
    prima equazione~\eqref{eq:kh11h} del modello, ed usare la condizione di equilibrio per uguagliare
    a zero quanto ottenuto. Da
    $$L \frac{ \left( \frac{1}{J} +1 \right) F }{ w +\left( \frac{1}{J} +1 \right) F }
    - \frac{1}{J} F \left( \alpha - \sigma \frac{\cancel{F}}{ \left( \frac{1}{J} +1 \right) \cancel{F} } \right) = 0$$
    possiamo semplificare $F$ perché non nulla, così da ottenere
    $$L \left( \frac{1}{J} +1 \right) \cancel{F} - \frac{1}{J} \left( \alpha - \frac{\sigma}{ \frac{1}{J} +1 } \right)
    \left[ w + \left( \frac{1}{J} +1 \right) F \right] \cancel{F} = 0 \; ,$$
    dove la semplificazione di $F$ procede con lo stesso argomento di sopra; si noti che ciò ammonta a scartare la
    soluzione nulla, che non ci interessa per ipotesi.

    L'unicità dell'equilibrio positivo si ottiene manipolando ulteriormente l'equazione sopra, ottenendo infine
    $$F^* = \frac{LJ}{ \alpha - \frac{\sigma}{ 1 + \frac{1}{J} } } - w \frac{J}{1+J} \; ,$$
    in cui possiamo osservare che il membro destro dipende esclusivamente dai parametri.

    \paragraph{}
    Avendo a disposizione l'ultima equazione, per concludere la dimostrazione -- \ie provare la~\eqref{eq:kh11posEqF}
    -- resta soltanto da dimostrare che
    $$J m = \alpha - \sigma \frac{J}{1+J} \; ,$$
    dove abbiamo semplificato il fattore $L$ ad ambo i membri, essendo non nullo.

    Per stabilire quest'ultima equazione può essere utile razionalizzare l'espressione $\frac{J}{1+J}$ come
    $$\frac{J}{1+J} = \frac{ \alpha + \sigma + m - \sqrt{ {(\alpha - \sigma -m)}^2 - 4 \alpha m } }{2 \sigma}$$
    e raggiungere l'obiettivo tramite semplici manipolazioni algebriche.
\end{proof}

Il modello~\eqref{eq:kh11h}--\eqref{eq:kh11f} ammette quindi sempre l'equilibrio banale, ed un ulteriore
configurazione di equilibrio positivo in alcune circostanze, riassunte nella seguente
\begin{proposizione}[Criterio di esistenza]
    Una condizione necessaria e sufficiente affinché il modello~\eqref{eq:kh11h}--\eqref{eq:kh11f} abbia
    un equilibrio strettamente positivo $P^* \coloneq (H^*, F^*)$ è data per casi:
    \begin{itemize}
        \item[1.] se $\alpha - \frac{L}{w} >0$, allora $P^*$ esiste $\iff m_1^* < m < m_2^*$;
        \item[2.] se $\alpha - \frac{L}{w} <0$, allora $P^*$ esiste $\iff m > m_2^*$;
        \item[3.] se $\alpha - \frac{L}{w} =0$, allora $P^*$ esiste $\iff m > m_3^*$;
    \end{itemize}
    dove
    \begin{equation}
        m_3^* = \frac{ \sigma L}{ w (\alpha + \sigma) } \; ,
        \label{eq:FstarPosM3}
    \end{equation}
    e $m_1^*, m_2^*$ sono sempre prese nell'ordine $m_1^* < m_2^*$ dalla seguente
    \begin{equation}
        m_{1,2}^* = \frac{L}{2w} \cdot \frac{ \alpha + \sigma \pm \sqrt{ {(\alpha - \sigma)}^2 +\frac{4 \sigma L}{w}  } }
        {\alpha -\frac{L}{w}} \; .
        \label{eq:FstarPosM12}
    \end{equation}

    \label{teo:esistenzPosF} % WARNING prima era teorema, ho cambiato idea dopo.
\end{proposizione}

\begin{proof}
Possiamo usare l'equazione~\eqref{eq:kh11posEqF} dalla Proposizione~\ref{teo:exUniqFstarPos} e studiare
la disequazione $F^* >0$ per dimostrare il criterio di esistenza:
$$F^* > 0 \; \iff \;
\left( \frac{2 \alpha w}{L} -1 \right) m - \alpha - \sigma < \sqrt{ {(\alpha - \sigma -m)}^2 + 4 \alpha m } \; .$$

Elevando al quadrato ambo i membri otteniamo la seguente disuguaglianza in $m$:
\begin{equation}
    \left( \frac{\alpha w}{L} -1 \right) m^2 -( \alpha + \sigma) m + \frac{\sigma L}{w} < 0
    \label{eq:mDiseq}
\end{equation}
la quale ha discriminante sempre positivo
$$\Delta = {(\alpha -\sigma)}^2 +4 \frac{\sigma L}{w} \; .$$

L'equazione~\eqref{eq:FstarPosM12} si ottiene applicando la formula delle radici quadratiche.
Nel seguito sceglieremo la notazione per avere sempre $m_1^* < m_2^*$ fino al termine della dimostrazione.

\paragraph{Caso 1.}
Se $\alpha - \frac{L}{w} >0$, allora $F^*>0$ esiste se e soltanto se
$$m_1^* < m < m_2^* \; ,$$
ed anche il limite inferiore ha senso, dal momento che
$m_1^* > 0$ si riduce a $\alpha - \frac{L}{w}>0$, vera per ipotesi del caso.

\paragraph{Caso 2.}
Se $\alpha -\frac{L}{w} <0$, allora dobbiamo prendere $m < m_1^*$ oppure $m > m_2^*$ affinché $F^* >0$ esista.
La prima disequazione si può scartare dal momento che $m_1^* <0$ per ipotesi del caso.

La seconda va mantenuta dal momento che $m_2^* >0$ si riduce alla ipotesi del caso $\alpha - \frac{L}{w} < 0$.

\paragraph{Caso 3.}
Quando $\alpha - \frac{L}{w} =0$, la nostra disequazione~\eqref{eq:mDiseq} diviene lineare e la sua soluzione è
$$m > \frac{\sigma L}{w (\alpha +\sigma)} \; ,$$
ciò che conclude l'ultimo caso restante della dimostrazione.
\end{proof}

\paragraph{}
Un criterio sufficiente per la stabilità dell'equilibrio positivo è dato nella seguente Proposizione
\footcite[3]{khoury2011}; la enunciamo qui senza dimostrazione poiché essa richiede calcoli lunghi e tediosi,
seppure non sia particolarmente complicata sotto il profilo teorico.

\begin{proposizione}[Criterio di stabilità dell'equilibrio positivo]
L'equilibrio positivo $(H^*, F^*)$ del modello \eqref{eq:kh11h}--\eqref{eq:kh11f}, determinato
dalla Proposizione~\ref{teo:exUniqFstarPos},
è asintoticamente stabile se $\alpha -\frac{L}{w} >0$ e
$$m < \frac{L}{2w} \cdot \frac{ \alpha + \sigma + \sqrt{ {(\alpha - \sigma)}^2 +\frac{4 \sigma L}{w}  } }
    {\alpha -\frac{L}{w}} \; .$$
    \label{prop:stabPosKh}
\end{proposizione}

\paragraph{Interpretazione.}
\label{par:interpretationCond6b}
Notiamo che tutte le situazioni che forniscono un limite inferiore ad $m$ potrebbero essere attribuite alle
semplificazioni intrinseche al modello, dal momento che potrebbero non descrivere accuratamente un alveare
in apiario sul campo. La non esistenza di un equilibrio positivo nelle situazioni in cui
$m< \hat{m}$ per qualche $\hat{m}$ potrebbe semplicemente essere dovuta alla divergenza verso l'infinito
della popolazione in tali casi.

Al contrario, la condizione che ritroviamo nel Lemma~\ref{lem:eqBanKh} e nelle Proposizioni~\ref{teo:esistenzPosF}
e \ref{prop:stabPosKh} data dalla disuguaglianza
\begin{equation}
\alpha - \frac{L}{w} > 0
    \label{eq:cond6b}
\end{equation}
si può leggere altrettanto semplicemente come $w \alpha >L$, dove entrambi i membri hanno unità di api al giorno.
Si noti anche che il membro destro cattura l'``efficienza'' dei processi sociali di reclutamento e cura della
covata, mentre il membro sinistro misura l'``efficienza'' della regina nel deporre uova fresche.

Vale a dire, la~\eqref{eq:cond6b} significa che il processo di reclutamento $R$ è capace
di sostenere un livello di deposizione $L$, bilanciando le necessità dell'alveare che richiedono api sia nel
comparto $H$ (per la cura della covata), che nel comparto $F$ (per l'importazione di cibo).

\paragraph{}
Condizioni della forma $m< \hat{m}$ tipicamente servono ad assicurare la sopravvivenza della colonia;
in questa regione dei parametri, il modello può manifestare la non esistenza di un equilibrio positivo,
per via della crescita incontrastata dei numeri nella popolazione totale.

Come notato in precedenza, questo modello non ha un fattore di mortalità nel compartimento $H$: alla luce
di questo fatto, possiamo spiegare la condizione~\eqref{eq:cond6b} come limite inferiore per il tasso
di trasferimento dal comparto delle operaie di casa al comparto delle operaie di campo, dove
le api hanno una aspettativa di vita finita.
Questa equazione esprime la reale necessità di coordinare efficientemente i passaggi tra le suddivisioni
del lavoro per la buona riuscita della colonia; per contro, le limitazioni puramente matematiche che
prevengono la divergenza della popolazione sono conseguenze puramente teoriche e non trovano applicazione
reale sul campo.

% Nel limite asintotico, $\frac{L}{w} H$ bees enter the hive compartment through eclosion and $\alpha H$ bees
% are withdrawn to the $F$ compartment: condition~\eqref{eq:cond6b} then says that the maximal recruitment rate
% must be sufficiently large to unload the excess hive bees towards the foragers compartment.
%  moar interpr? TODO
% cfr. condiz per 0

\paragraph{}
La ramificazione degli equilibri in questo modello è illustrata in figura~\ref{img:kh11phasePlane}.

\begin{figure}[hpb]
    \centering
    \includegraphics[keepaspectratio,width=0.76\textwidth]{img/pone.0018491.g003}

    \caption[\figurename~3 da \parencite{khoury2011}]{Diagramma delle fasi per le
        soluzioni del modello con differenti valori di mortalità $m$.
        Ogni curva del diagramma rappresenta la traiettoria di una soluzione, fornendo il numero di
        operaie di nido $H$ e di foraggiatrici $F$.
        All'aumentare del tempo $t$ le soluzioni variano nel verso delle frecce.

        In (a) abbiamo $m = 0.24$ e la popolazione tende ad un equilibrio positivo stabile,
        segnalato da un punto.

        Nella (b) abbiamo $m = 0.40$ e l'unico equilibrio del sistema è quello banale: la popolazione
        collassa verso lo zero.

        I valori degli altri parametri sono $L = 2000$,  $\alpha= 0.25$,  $\sigma= 0.75$ e $w = 27 000$.
        \\
        { \footnotesize \ttfamily doi: 10.1371/journal.pone.0018491.g003 }
        }

    \label{img:kh11phasePlane}
\end{figure}

\paragraph{}
\label{par:mInverseFlightspan}
Tra i contributi più utili di questo semplice modello, oltre al ruolo sociale delle funzioni $E$ ed
$R$, rileviamo una concentrazione particolare degli autori sul parametro $m$:
ciò riflette in parte l'attenzione dedicata dalla ricerca alle varie sindromi da
collasso di popolazione in \emph{Apis} rilevate in tutto il mondo.
D'altro canto, è importante per la validazione di ogni modello matematico
un confronto coi dati sperimentali rilevati sul campo:\footcite{rueppell2009honey}
in questo senso $m$ è facilmente derivabile come l'inverso
di una misura diretta, l'aspettativa di vita media per le bottinatrici (\emph{flightspan}).

Se ogni bottinatrice ha mediamente probabilità $1-m$ di sopravvivere al primo giorno in tale ruolo, $(1-m)^2$
probabilità di sopravvivere al secondo, e così di seguito, la sua aspettativa di vita media è
$$ \sum_{k=1}^{\infty} k m {(1-m)}^k = m \underbrace{ \sum_{k=1}^{\infty} k {(1-m)}^k }_{m^{-2}}
= \frac{1}{m} \; .$$


% TODO stabilità dell'equilibrio non banale?


\section{Modello a cinque compartimenti con operaie, \emph{Varroa} e ABPV}
Esaminiamo adesso il modello proposto nel \citeyear{ratti2017}, che estende alcune proposte precedenti con lo scopo di studiare gli effetti diretti e indiretti di \emph{Varroa destructor} sulla popolazione
di un alveare.\footcite{ratti2017}

\subsection{Derivazione del modello}
Per ottenere una formulazione del modello, gli autori propongono le seguenti assunzioni preliminari:
\begin{enumerate}
    \item La popolazione delle operaie sane (non infette da ABPV) è suddivisa in due comparti: le operaie più giovani che svolgono mansioni all'interno dell'alveare ($x_h$) e le bottinatrici/esploratrici ($x_f$).
    \footnote{Questa impostazione ricalca lo studio esaminato nella sezione
    precedente~\parencite{khoury2011}.}
%     in cui si propone un modello a due compartimenti per esaminare gli effetti del tasso di schiusa delle pupe e del tasso di reclutamento da operaie d'alveare a bottinatrici sulla dinamica di una popolazione sana.
    \item La popolazione degli acari è suddivisa in due comparti: le varroe portatrici di ABPV ($m$) e le varroe libere dal virus ($n$).
    \item Le operaie infette con ABPV ($y$) perdono rapidamente l'abilità al volo, le loro capacità di
    orientamento degradano e muoiono in breve tempo. Per questo motivo le api infette sono considerate
    ``di alveare'', e non vengono più reclutate per bottinare.
\end{enumerate}

I compartimenti del modello sono dunque cinque:
\begin{itemize}
    \item il numero di operaie di nido sane, $x_h$
    \item il numero di operaie bottinatrici sane, $x_f$
    \item il numero di operaie infette, $y$
    \item il numero di varroe portatrici di ABPV, $m$
    \item il numero di varroe senza virus, $n$
\end{itemize}

Abbiamo inoltre
$$\text{Pop. totale api operaie} = x_h + x_f + y,$$
e
$$\text{Pop. totale varroe} = m + n.$$

\paragraph{}
Proseguiamo con l'esame delle assunzioni del modello:
\begin{enumerate}
    \setcounter{enumi}{3}
    \item Il tasso massimale di riproduzione delle varroe è lo stesso, che siano portatrici di virus o meno.
    Ciò si riflette nelle equazioni \eqref{eq:r17m} e \eqref{eq:r17n}, in cui il primo termine descrive le nuove nascite: esso è proporzionale alla popolazione del compartimento ed il fattore di proporzionalità è lo stesso in entrambe le equazioni. In esso, $r$ è il tasso massimo di riproduzione del parassita \emph{Varroa}, mentre $\alpha$ è la capacità portante dell'ospite \emph{Apis}.
    \item Sappiamo che la covata necessita di cure da parte delle operaie nutrici, quindi il tasso di schiusa è dipendente dal numero di operaie sane nella popolazione, che assumiamo pari alla somma $x_h + x_f$.
    \footnote{Cfr.~\cite{khoury2011}.}
    \item L'ape regina non subisce gli effetti avversi di ABPV e dell'infestazione da varroe\footcite{privFDL}; di conseguenza, il tasso di schiusa non dipende dalla preminenza nell'alveare di varroe o di virus.
    \item La trasmissione di ABPV è esclusivamente orizzontale:\footcite{privFDL}
    ciò significa che l'infezione di una operaia avviene per trasmissione del virus da un'altra operaia infetta.\footnote{La trasmissione \emph{verticale} di un patogeno riguarda invece l'infezione della prole a causa dei genitori già infetti.}

    Le varroe sono un vettore puramente meccanico per il virus: la carica virale non è incentivata né
    inibita dalle varroe, che per l'ABPV fungono meramente da ``mezzo di trasporto'' tra le api.
    \item Poiché le pupe infette con ABPV muoiono rapidamente prima della schiusa, tutte le giovani operaie che
    giungono all'età adulta si assumono libere dal virus. L'effetto limitante sul tasso di schiusa dovuto alla
    presenza di varroe infette viene incorporato in un fattore di modulazione $h(m)$ nella~\eqref{eq:r17xh}.
    \item Per studiare l'effetto dei trattamenti acaricidi, introduciamo un ulteriore termine di pozzo $\delta_i$ in ogni equazione. Assumiamo che l'impatto dei trattamenti sia di gran lunga maggiore sulle varroe che sulle api: $\delta_4, \delta_5 \gg \delta_1, \delta_2, \delta_3$.
    \item In tutti e tre i compartimenti delle api, si assume un tasso di mortalità naturale $d_i$ in aggiunta alla mortalità dovuta alle varroe $\gamma_i$ (a sua volta dipendente sia dal parassitismo che dalla trasmissione del virus). Nelle api indebolite dall'infezione da ABPV entrambi questi tassi sono maggiori rispetto alle operaie sane: $d_3 > d_1, d_2$ e $\gamma_3 > \gamma_1, \gamma_2$.

    Inoltre per le bottinatrici si tiene conto anche della mortalità da disorientamento (fattore $p$), dovuto
    principalmente al degrado delle facoltà cognitive e di navigazione per effetto dei pesticidi
    sintetici sulle colture visitate.
\end{enumerate}


\subsubsection{Equazione del modello}
Il modello a compartimenti proposto\footcite[8]{ratti2017} assume quindi la seguente formulazione:
\begin{align}
    % hive sane
    \diff{x_h}{t} &= \mu g( x_h + x_f ) h(m) - \beta_1 m \frac{x_h}{x_h + x_f + y} - \left( d_1 + \delta_1 \right) x_h \notag \\ % barbatrucco
    \label{eq:r17xh}
        &{\;} - \gamma_1 (m+n) x_h - x_h \cdot R(x_h, x_f) \\ % barbatrucco
    % foragers sane
    \label{eq:r17xf}
    \diff{x_f}{t} &= x_h \cdot R(x_h, x_f) - \beta_1 m \frac{x_f}{x_h + x_f + y}
    - \left( p + d_2 + \delta_2 \right) x_f - \gamma_2 (m+n) x_f \\
    % (hive) infette
    \label{eq:r17y}
    \diff{y}{t} &= \beta_1 m \frac{x_h + x_f}{x_h + x_f + y} - \left( d_3 + \delta_3 \right) y - \gamma_3 (m+n) y \\
    % varroe con ABPV
    \label{eq:r17m}
    \diff{m}{t} &= rm \left( 1 - \frac{m+n}{ \alpha (x_h + x_f + y) } \right) + \beta_2 n \frac{y}{x_h + x_f + y}
    - \beta_3 m \frac{x_h + x_f}{x_h + x_f + y} - \delta_4 m \\
    % varroe senza ABPV
    \label{eq:r17n}
    \diff{n}{t} &= rn \left( 1 - \frac{m+n}{ \alpha (x_h + x_f + y) } \right) - \beta_2 n \frac{y}{x_h + x_f + y}
    + \beta_3 m \frac{x_h + x_f}{x_h + x_f + y} - \delta_5 n
\end{align}

Il termine ``di reclutamento'' $x_h \cdot R(x_h, x_f)$ nelle equazioni \eqref{eq:r17xh} e \eqref{eq:r17xf} rappresenta appunto il passaggio di ruolo delle operaie, da impiegate \emph{all'interno} dell'alveare a \emph{bottinatrici} che escono dall'arnia per esplorare e raccogliere il cibo.

Come nello studio precedente,\footnote{\cite{khoury2011}, cfr.~p.~\pageref{eq:Recr} e segg.}
il fattore $R$ rappresenta gli effetti \emph{sociali}
della colonia sul tasso di reclutamento, con la seguente formulazione:
\begin{equation}
    R(x_h, x_f) = \sigma_1 - \sigma_2 \frac{x_f}{x_h + x_f}
    \label{eq:reclutam}
\end{equation}
in cui $\sigma_1$ è il tasso massimo di reclutamento delle operaie, quando non ci sono bottinatrici nella colonia.
Il termine $\sigma_2 \frac{x_f}{x_h + x_f}$ descrive l'inibizione sociale del reclutamento, ossia il rallentamento del processo di ``promozione'' all'esterno del nido, proporzionale alla frazione di popolazione impiegata nella bottinatura; quando tale frazione è in surplus, le bottinatrici vengono ``riallocate'' in attività interne all'alveare.

\paragraph{}
Il tasso di nuove nascite è descritto dal primo termine nella equazione \eqref{eq:r17xh}, in cui troviamo il
parametro $\mu$ che è determinato principalmente dalla regina e dalla stagione, e rappresenta il tasso massimo
di schiusa.

La funzione $g$ descrive il bisogno di un numero minimo di operaie sane nella colonia per allevare la covata; se questo numero scende sotto una certa soglia -- che può presentare variazioni stagionali -- la covata non produce più api adulte.
La $g$ rappresenta la risposta funzionale dell'allevamento della covata; nelle parole degli autori:
\begin{displayquote}[{\footcite[9]{ratti2017}}]
``We think of $g(x_h + x_f)$ as a switch function.''
\end{displayquote}

È necessario che $g$ verifichi le condizioni
$$g(0)=0 \; , \quad %
\diff{g}{x_h} \geq 0 \; , \quad %
\diff{g}{x_f} \geq 0 \; ,$$
ed anche $\lim_{x_h+x_f \to \infty} g(x_h+x_f)=1$.

Un'utile formulazione per $g$ è data dalla sigmoide di Hill (figura~\ref{img:hillSigmoid}):
\begin{equation}
    g(x_h + x_f) = \frac{ (x_h+x_f)^i }{ K^i + (x_h+x_f)^i }
    \label{eq:hillSigmoid}
\end{equation}

\begin{figure}
    \centering
    \includegraphics[keepaspectratio,width=0.86\textwidth]{img/hillSigmoid}

    \caption[Sigmoidi di Hill.]{Sigmoide di Hill per diversi valori dell'esponente $i$.
        \\ Il parametro $K$ rappresenta la retroimmagine del semimassimo.}
        % TODO citarla negli strumenti quando si dicono le Holling
    \label{img:hillSigmoid}
\end{figure}

dove $K$ è la dimensione della popolazione di operaie nell'alveare corrispondente al tasso di schiusa semimassimo, e l'esponente $i>1$ si assume intero per semplicità di analisi.

\paragraph{}
Sempre nel termine delle nascite di api adulte, un altro fattore modulante tiene conto degli effetti deleteri per la covata dovuti alla presenza degli acari portanti ABPV nell'alveare: le larve e le pupe infette dal virus muoiono rapidamente prima di raggiungere lo stadio di adulte, quindi la funzione $h(m)$ nell'equazione~\ref{eq:r17xh} verifica le condizioni $h(0)=1$, $\diff{h}{m} <0$ e $\lim_{m \to \infty} h(m) = 0$.

In \cite{sumMar04} si suggerisce $h(m) \approx e^{-km}$ con $k\geq0$, e questa è la forma utilizzata in \cite{ratti2017} per le simulazioni numeriche.

Si noti che assumere $h=h(m)$ porta a una semplificazione del sistema, mentre in realtà anche la
popolazione $n$ di varroe non infette preda abitualmente le pupe di ape.

% MANCHEREBBE DI PARLARE ALMENO UNA VOLTA DEI \beta_i
% The parameter β2 in (4) is the rate at which mites that do not carry the virus acquire
% it. The rate at which infected mites lose their virus to an uninfected host is β3 . The
% rate at which uninfected (hive and forager) bees become infected is β1 , in units of
% bees per virus-carrying mite and time. We assumed that the rate at which hive bees get
% infected (i.e. β1 ) is the same as the rate at which foragers get infected (i.e. β2 ). With
% this assumption, the first term in (3) becomes the rate at which total bees get infected.

\subsection{Analisi del modello con parametri costanti}
Seguiamo l'impostazione dello studio \cite{ratti2017} per l'analisi degli equilibri, iniziando con l'esame di sotto-casi più semplici ed estendendo progressivamente i risultati fino al modello completo.

\paragraph{}
Dapprima assumiamo l'ipotesi di lavoro di coefficienti costanti; utilizzeremo in seguito la teoria di Floquet per analizzare il caso di coefficienti periodici (sottosezione~\ref{sez:paramPeriodici}).

% \paragraph{}
% Per le dimostrazioni dei risultati esposti in questa sottosezione, si rimanda all'articolo originale \cite[11--15]{ratti2017}.
Una volta stabilite le condizioni per l'esistenza dei punti di equilibrio, la metodologia standard per analizzarne la stabilità consiste nel linearizzare il sistema in un intorno di un punto di equilibrio: il carattere attrattivo o repulsivo dell'equilibrio è determinato dagli \emph{esponenti di Lyapounov locali}, ossia dalle parti reali degli autovalori della matrice jacobiana.

Questo approccio metodologico costituisce la base delle dimostrazioni in questa sottosezione e nella successiva.
\footnote{Cfr. p.~\pageref{chap:teoria} e segg. per una sintesi dei risultati teorici qui sfruttati.}

\subsubsection{Modello bidimensionale senza patogeni}
In assenza di \emph{Varroa} e di virus, il modello \eqref{eq:r17xh}--\eqref{eq:r17n} si riduce alle due sole equazioni per le api sane:
\begin{align}
    \diff{x_h}{t} &= \mu g( x_h + x_f ) - d_1 x_h - x_h \left( \sigma_1 - \sigma_2 \frac{x_f}{x_h + x_f} \right)
    \label{eq:rattiRidotto1prima}
    \\
    \diff{x_f}{t} &= x_h \left( \sigma_1 - \sigma_2 \frac{x_f}{x_h + x_f} \right) - (p + d_2) x_f
    \label{eq:rattiRidotto1seconda}
\end{align}
dove $g(x_h + x_f) = \frac{ (x_h+x_f)^i }{ K^i + (x_h+x_f)^i }$ come nella \eqref{eq:hillSigmoid}.

\paragraph{}
L'equilibrio banale $(x_h^*, x_f^*) = (0,0)$ esiste sempre, ed è asintoticamente stabile; questo risultato è contenuto nelle due proposizioni che seguono.

Nelle parole degli autori, la stabilità asintotica dell'equilibrio banale significa che
\begin{displayquote}[\footcite{ratti2017}]
``\omissis in order to establish itself as a properly working colony, a sufficiently large healthy adult bee population is required to take care of the brood.''
\end{displayquote}

E viceversa, se il numero di operaie sane scende sotto un valore critico, la colonia si avvia verso una morte inesorabile.
Si noti che questo effetto di tipo Allee si osserva già nel modello senza patogeni; l'introduzione di \emph{Varroa} e virus non può che esacerbare fenomeni di questo tipo.

\paragraph{}
Proponiamo adesso una analisi dei punti di equilibrio del modello \eqref{eq:rattiRidotto1prima}--\eqref{eq:rattiRidotto1seconda}, introducendo per comodità di notazione le due costanti
$$
    F \coloneq \frac{1}{2} \left( \frac{ \sigma_1 - \sigma_2 - p - d_2 }{p+d_2} +
    \sqrt{ {\left( \frac{ \sigma_1 - \sigma_2 - p - d_2 }{p+d_2} \right)}^2 + \frac{4 \sigma_1}{p+d_2} } \, \right),
$$
ed
$$ a \coloneq - \frac{ \mu }{ \frac{\sigma_2 F}{1+F} - d_1 - \sigma_1 }.$$

\begin{proposizione}[3.1, Existence of equilibria {\footcite[11]{ratti2017}}]
    Il modello ridotto alle sole api \eqref{eq:rattiRidotto1prima}--\eqref{eq:rattiRidotto1seconda} ammette sempre l'equilibrio banale $(0,0)$, e se $\frac{\sigma_2 F}{1+F} - d_1 - \sigma_1 >0$ allora esso è unico.

    Viceversa, se $\frac{\sigma_2 F}{1+F} - d_1 - \sigma_1 <0$ allora il sistema ammette due ulteriori equilibri strettamente positivi, a patto che sia verificata l'ulteriore condizione
    $$a > \frac{Ki}{1+F} (i-1)^{\frac{1}{i} -1}.$$
    \label{prop:rExist}
\end{proposizione}

\begin{proof}
    Le equazioni del modello sono omogenee, dunque l'equilibrio banale $(0,0)$ esiste sempre.
    Dalla \eqref{eq:rattiRidotto1seconda} segue che ogni punto di equilibrio
    $(x_h^*, x_f^*) > (0,0)$ verifica
    $$x_f^2 + \frac{\sigma_2 - \sigma_1 + p + d_2}{p +d_2} x_h x_f - \frac{\sigma_1}{p+d_2} x_h^2 = 0,$$
    da cui ricaviamo la relazione tra le componenti $x_f^* = F x_h^*$.

    Inserendo quest'ultima nell'equazione \eqref{eq:rattiRidotto1prima} otteniamo
    $$ \mu g \left( (1+F) x_h \right) -d_1 x_h -x_h \left( \sigma_1 - \sigma_2 \frac{F}{1+F} \right) = 0 \; ,$$
    da cui
    $$ \underbrace{ x_h^i -a x_h^{i-1} + \left( { \frac{K}{1+F} } \right)^i }_{ f(x_h) \coloneq } = 0 \; .$$

    Per il criterio di Descartes, il polinomio $f(x_h)$ ha due radici positive\footnote{Contate con molteplicità.}
    se $a>0$, e nessuna se $a<0$. Quindi vogliamo $a>0$ ossia $\frac{\sigma_2 F}{1+F} -d_1 - \sigma_1 < 0$, ciò che
    stabilisce la prima parte della proposizione.

    Assumendo $a>0$, abbiamo che $f'$ cambia segno in $\hat{ x_h } = a \frac{i-1}{i}$, da cui
    possiamo imporre
    $$\min f = f( \hat{ x_h } ) =
    - \frac{1}{i} a^i { \left( \frac{i-1}{i} \right) }^{i-1} + {\left( \frac{K}{1+F} \right) }^i < 0
    \; ,$$
    per ricavare infine
    $$a > \frac{Ki}{1+F} {(i-1)}^{ \frac{1}{i} - 1 }.$$
\end{proof}




\paragraph{}
Se esistono, gli equilibri positivi sono stabili se
\begin{equation}
\frac{i \mu K^i {(x_h^*)}^{i-1} {(1+F)}^{i-1}}{ {\left( K^i + {(x_h^*)}^i {(1+F)}^i \right)}^2 }
<
\frac{d_1 + (p+d_2) F}{{(1+F)}^2}.
\label{eq:rattiRidotto1stability}
\end{equation}

\subsubsection{Modello tridimensionale api--acaro}
Introduciamo adesso nella colonia le varroe senza virus, ed indaghiamo la conseguente alterazione della stabilità degli equilibri individuati nel caso precedente.

Si eliminano quindi dal modello \eqref{eq:r17xh}--\eqref{eq:r17n} le equazioni per le api infette da ABPV e per le varroe che trasportano il virus; notiamo che da $m \equiv 0$ segue che $h(m) \equiv 1$.

\paragraph{}
Il modello ridotto che si ottiene è il seguente:

\begin{align}
    \diff{x_h}{t} &= \mu g( x_h + x_f ) - (d_1 + \delta_1) x_h - \gamma_1 n x_h
        - x_h \left( \sigma_1 - \sigma_2 \frac{x_f}{x_h + x_f} \right)
        \label{eq:rattiRidotto2prima}
    \\
    \diff{x_f}{t} &= x_h \left( \sigma_1 - \sigma_2 \frac{x_f}{x_h + x_f} \right) - (p + d_2 + \delta_2) x_f
        - \gamma_2 n x_f
        \label{eq:rattiRidotto2seconda}
    \\
    \diff{n}{t} &= rn \left( 1 - \frac{n}{ \alpha (x_h + x_f) } \right) - \delta_5 n
        \label{eq:rattiRidotto2terza}
\end{align}

Anche in questo caso, l'equilibrio banale $(x_h, x_f, n)=(0,0,0)$ esiste ed è sempre asintoticamente stabile, con la medesima interpretazione pratica del caso precedente (effetto Allee forte).

\paragraph{Equilibrio senza varroe. } Consideriamo adesso un equilibrio non banale $(x_h^*, x_f^*)$ per il modello \eqref{eq:rattiRidotto1prima}--\eqref{eq:rattiRidotto1seconda}, secondo le condizioni di esistenza date sopra.
Il punto $(x_h^*, x_f^*, 0)$ è un equilibrio per il modello \eqref{eq:rattiRidotto2prima}--\eqref{eq:rattiRidotto2terza}.

Se $r > \delta_5$ allora è instabile; se viceversa $r< \delta_5$ allora il punto $(x_h^*, x_f^*, 0)$ eredita la stabilità di $(x_h^*, x_f^*)$ sotto le \eqref{eq:rattiRidotto1prima}--\eqref{eq:rattiRidotto1seconda}.

\paragraph{}
Ciò significa che l'equilibrio senza varroe è stabile se la \eqref{eq:rattiRidotto1stability} è soddisfatta
ed il tasso massimo di natalità delle varroe è inferiore alla loro mortalità dovuta ai trattamenti.
Si osservi inoltre che in assenza di trattamenti varroacidi l'alveare non riesce a debellare l'infestazione dell'acaro.

Nel caso instabile, non è chiaro se il sistema converge all'equilibrio banale oppure ad un equilibrio endemico. \cite[15]{ratti2017}


\subsubsection{Modello completo api--\emph{Varroa}--ABPV}
Consideriamo infine il modello completo \eqref{eq:r17xh}--\eqref{eq:r17n} e studiamo la stabilità dell'equilibrio senza patogeni.

Sia $(x_h^*, x_f^*)$ un equilibrio positivo per il modello ridotto \eqref{eq:rattiRidotto1prima}--\eqref{eq:rattiRidotto1seconda}, sempre secondo le condizioni di esistenza di cui sopra; che $(x_h^*, x_f^*, 0,0,0)$ sia un equilibrio per \eqref{eq:r17xh}--\eqref{eq:r17n} è ovvio.

Se valgono entrambe le disuguaglianze
$$
\begin{sistema}
    r - \beta_3 - \delta_4 < 0 \\
    r - \delta_5 < 0,
\end{sistema}
$$
allora $(x_h^*, x_f^*, 0,0,0)$ eredita la stabilità di $(x_h^*, x_f^*)$ sotto le \eqref{eq:rattiRidotto1prima}--\eqref{eq:rattiRidotto1seconda}.
Se invece (almeno) una delle disuguaglianze è invertita, l'equilibrio è instabile.

\paragraph{}
Quest'ultimo risultato si può interpetare con la seguente condizione: affinché l'alveare sconfigga completamente la \emph{Varroa} ed il virus, è necessario e sufficiente che
\begin{itemize}
    \item i trattamenti varroacidi siano abbastanza potenti da eradicare gli acari che non portano ABPV;
    \item le varroe portatrici perdano la loro carica virale ad un tasso più rapido della loro natalità.
\end{itemize}

L'esperienza sul campo dell'apicoltura, sia sperimentale che in produzione, indica come estremamente improbabile l'eradicazione completa della \emph{Varroa}.~\cite{privFPan}

Piuttosto, è più ragionevole che \textquote[\cite{privFDL}]{a livello di alveare o di apiario, si riesca a mantenere per un certo periodo (almeno 3 o 4 stagioni) un equilibrio in cui api e varroe coesistono senza causare il collasso delle colonie.}

Ciò corrisponderebbe alla presenza di un equilibrio endemico nel modello, la cui esistenza tuttavia non è ancora dimostrata.
Notiamo inoltre che abbiamo condizioni sufficienti per la stabilità degli equilibri positivi, ma non per la stabilità asintotica: un punto di equilibrio stabile ma non attrattivo potrebbe presentare dei cicli-limite.


\paragraph{}
Le attività in apiario che promuovono la sopravvivenza dell'alveare richiedono un monitoraggio costante e diversi tipi di interventi, che includono i trattamenti acaricidi, ma che non possono limitarsi al contrasto della \emph{Varroa}: è infatti necessario il mantenimento di \textquote[\cite{privFDL}]{\omissis uno stato di salute complessivamente buono nella colonia, che riguarda anche l'efficienza del sistema immunitario, la ``forza'' della famiglia in termini di operaie sane, le scorte di cibo, la salute della regina, e molti altri fattori.}

Questo elemento di complessità è catturato anche nel modello relativamente semplice di \cite{ratti2017}, in quanto le condizioni di esistenza e stabilità dei punti di equilibrio includono una molteplicità di parametri, che nel modello rappresentano non solo gli effetti ``diretti'' dovuti all'acaro e al virus, ma anche lo stato di salute dell'alveare, compresa la sua capacità di allevare covata e di approvvigionarsi di cibo.


\subsection{Analisi del modello con parametri periodici}
\label{sez:paramPeriodici}
Questo ultimo passaggio è fondamentale per incorporare nel modello le variazioni stagionali nel tasso di deposizione da parte della regina, l'arresto della raccolta di cibo nel periodo invernale e di molte altre attività dell'alveare.

Il periodo dei coefficienti è dunque $T= 1 \text{ anno}$.

suloz non banali, Floquet

\begin{proposizione}[4.1, Stability of mite--free periodic solution {\cite[19]{ratti2017}} ]
    Sia $\left( x_h^*(t), x_f^*(t) \right)$ sia una soluzione periodica per il modello ridotto alle sole api \eqref{eq:rattiRidotto1prima}--\eqref{eq:rattiRidotto1seconda}.

    Allora $\left( x_h^*(t), x_f^*(t), 0 \right)$ è una soluzione periodica per il modello ridotto api--acaro \eqref{eq:rattiRidotto2prima}--\eqref{eq:rattiRidotto2terza}; è stabile se
    \begin{enumerate}
        \item $\left( x_h^*(t), x_f^*(t) \right)$ è stabile per il modello ridotto alle sole api;
        \item $\int_0^T (r - \delta_5) dt \leq 0$.
    \end{enumerate}
\end{proposizione}


\begin{proposizione}[4.2, Stability of the disease--free periodic solution {\cite[21]{ratti2017}} ]
    Sia $\left( x_h^*(t), x_f^*(t) \right)$ sia una soluzione periodica per il modello ridotto alle sole api \eqref{eq:rattiRidotto1prima}--\eqref{eq:rattiRidotto1seconda}.

    Allora $\left( x_h^*(t), x_f^*(t), 0, 0, 0 \right)$ è una soluzione periodica per il modello completo \eqref{eq:r17xh}--\eqref{eq:r17n}; è stabile se
    \begin{enumerate}
        \item $\left( x_h^*(t), x_f^*(t) \right)$ è stabile per il modello ridotto alle sole api;
        \item $\int_0^T (r - \delta_5) dt \leq 0$;
        \item $\int_0^T (r -\beta_3 - \delta_4) dt \leq 0$.
    \end{enumerate}
\end{proposizione}



\subsection{Riduzioni ai modelli precedenti}

Se nell'espressione di $g$ (equazione~\eqref{eq:hillSigmoid}) prendiamo $K=0$ ci riduciamo al modello di \cite{sumMar04}.

Con $i=1$ e togliendo i compartimenti di virus e varroe, e riducendo a zero la mortalità nel
comparto $x_h$, ci si riduce a \cite{khoury2011}. Si veda in proposito la tabella~\ref{tab:kh11VSratti17} per
un confronto tra le notazioni.

\begin{table}[pbh]
    $$\begin{array}{ccc}
        \toprule
        \text{Khoury~2011}
        & \text{Ratti~2017}
        & \text{u.m.} \\
        \midrule
        H & x_h & \text{api} \\
        F & x_f & \text{api} \\
        m & d_2 & \\ % TODO
        L & \mu & \\
        \alpha & \sigma_1 & \\
        \sigma & \sigma_2 & \\
        E & g & \\
        \midrule
        \multicolumn{2}{c}{R} & \\
        \midrule
        J & F & \\
        \bottomrule
    \end{array}$$
    \caption{Notazione in Khoury~2011 vs. Ratti~2017.}
    \label{tab:kh11VSratti17}
\end{table}


