\section{Risposta funzionale}
\label{sec:rispFunz}
Nello studio della dinamica delle popolazioni l'interesse è spesso concentrato sulla \emph{risposta numerica}, ossia
sul tasso riproduttivo dei vari compartimenti in funzione delle risorse disponibili nell'ambiente.

Nella modellazione di relazioni strutturate tra i compartimenti (infra-- ed inter-- specie) è spesso necessario
considerare anche le \emph{risposte funzionali} per descrivere i termini di trasferimento nelle equazioni~\eqref{eq:sistDin}.

\paragraph{}
La risposta funzionale modella il tasso di consumo di un singolo individuo, in funzione della densità di risorse.

Si noti che adottare la densità di risorse piuttosto che la quantità effettiva di risorse permette --
nei modelli più avanzati -- di modellare la \emph{spazialità} nelle azioni dei viventi, che operano, si muovono
ed interagiscono con l'ambiente cercando il cibo e di evitare minacce, in un certo raggio di prossimità.

\paragraph{}
Tra i primi tentativi di modellazione matematica della risposta funzionale troviamo quella definita da
\citeauthor{holling59}\footcite{holling59}:
\begin{equation}
    f(x) = \frac{a x^k}{1+ a h x^k} \; ,
    \label{eq:holling}
\end{equation}
dove $f$ è il tasso di consumo individuale, $x$ è la densità di risorse, ed i parametri $a$ ed $h$ sono rispettivamente
il tasso di attacco ed il tempo medio di manipolazione.

\paragraph{}
Il tasso di attacco $a$ rappresenta la frequenza con cui il consumatore incontra il cibo (in unità di densità).
Il tempo di manipolazione $h$ è il tempo medio che il consumatore impiega per processare una unità di cibo/risorsa.

Il parametro $k$ distingue tre classi di funzioni di Holling:
\begin{enumerate}
    \item f. di Holling di \textbf{tipo I}: risposta lineare. Si ottiene con $k=1$ e $h=0$, assumendo cioè che il tempo
        necessario alla manipolazione/assunzione di cibo sia trascurabile rispetto al tempo trascorso nel cercare cibo.
        Una ulteriore semplificazione qui implicata, seppur meno evidente, è che l'atto di procacciarsi il cibo non sia
        in competizione con il suo consumo, ciò che in generale è falso nelle specie animali.

        Questa risposta funzionale lineare è quella adottata nel modello di Lotka--Volterra.\footcite{vito}
    \item f. di Holling di \textbf{tipo II}: si ha per $k=1$ e $a,h \neq 0$. Assume che il tasso di consumo sia limitato
        dalla capacità di ogni consumatore di processare la risorsa.

        Le funzioni di tipo II e III hanno un plateau a quota $f=\frac{1}{h}$, cfr.~fig.~\ref{img:hollo}.

    \item f. di Holling di \textbf{tipo III}: con $k>1$ si modella l'apprendimento del consumatore nell'approccio alla
        risorsa~cibo. Un elemento viene accettato e processato al $k$-esimo incontro e rifiutato/ignorato nei
        primi $k-1$ incontri.
\end{enumerate}

\begin{figure}[hbt]
    \centering
    \includegraphics[keepaspectratio,width=0.68\textwidth]{img/hollo}

    \caption{Risposta funzionale di Holling, tipi I--III.}
    \label{img:hollo}
\end{figure}

\paragraph{}
Nei modelli a proposito di \emph{Apis~mellifera} esaminati nel capitolo~\ref{chap:modelz} vedremo un esempio di
funzione di Holling (tipo II) per descrivere la schiusa delle api dopo la metamorfosi delle larve, ciò che corrisponde alla
introduzione di nuovi individui giovani nel compartimento delle operaie di nido.
Si veda in proposito l'equazione~\eqref{eq:eclos} a p.~\pageref{eq:eclos} e la sua illustrazione in figura~\ref{img:eclos}.

Nel modello di~\citeauthor{ratti2017} (Sezione~\ref{sec:ratti17}) si impiega una sigmoide di Hill\footnote{Matematicamente, le sigmoidi di Hill fanno parte della famiglia delle funzioni di Holling II e III.}
con lo stesso scopo del precedente: modulare ``nuove nascite'' con la risposta funzionale alla ``risorsa'' api operaie,
``consumata'' dalle larve in termini di tempo dedicato alla cura della covata.

\paragraph{}
Pensando alle popolazioni di mammiferi è generalmente vera una serie di assunzioni che ci permettono di modellare la
risposta numerica in termini di relazioni con l'ambiente (densità di cibo, minacce da predatori o patogeni, \dots) e di
relazioni infra--specifiche. Ad esempio l'\emph{omogeneità} delle popolazioni considerate (su scala geografica
sufficientemente ristretta) è un'assunzione utilizzata dapprima nella epidemiologia, e successivamente è stata
adattata con profitto alla modellazione matematica delle popolazioni viventi in prospettive più ampie.
% omog: la uso per esempio per dire che le gravidanze sono proporzionali agli incontri

\paragraph{}
Nelle specie animali altamente socializzate l'esistenza di relazioni strutturate comporta che molte delle assunzioni
di cui sopra divengano false: la popolazione è disomogenea anche su scala di branco o famiglia, e le relazioni
complesse tra i compartimenti generano nuovi termini nelle equazioni del modello.
Tali relazioni complesse si esprimono efficacemente con lo strumento della risposta funzionale; le misure sul campo
relative alle interazioni sociali\footnote{Ad esempio il tempo medio di ``manipolazione'' che una ape nutrice impiega per
ogni larva visitata, il numero medio di larve visitate al giorno, etc.}
permettono talvolta di stimare i parametri delle funzioni espresse nel modello.



