\chapter{Richiami di teoria dei sistemi dinamici}
\label{chap:teoria}
In questo capitolo si richiamano alcuni risultati nell'ambito dei sistemi dinamici, utili strumenti ai fini
di questa tesi.
Un'ampia letteratura è disponibile in materia, anche in lingua italiana.
\footnote{Ad es. \cite{ricciSistDin} Cap.~2, \cite{introSD} Capp.~1--4.}

Presentiamo qui soltanto i teoremi e le definizioni che si applicano all'ambito generale; alcuni oggetti
e risultati riguardanti più nello specifico i sistemi biomatematici, di modellazione ecologica delle popolazioni
e delle patologie sono esaminati nel sezione~\ref{sec:ingredientiBioMat}.

\paragraph{}
È utile definire un po' di notazione preliminare, prima di procedere oltre.
Sia
\begin{equation}
\begin{sistema}
\dot{\mathbf{x}} = F ( \mathbf{x}, t ) \\
\mathbf{x} (t_0) = \mathbf{x}_0
\end{sistema}
\label{eq:sdcGenerale}
\end{equation}
un sistema dinamico con condizione iniziale $\mathbf{x}_0$. Supporremo nel seguito del capitolo che $F$ sia
definita su qualche aperto $W \subseteq \R^n$ e che sia di classe $C^1$ almeno.

Per una classe abbastanza ampia\footnote{Tra le molte varianti del Teorema, scegliamo come ipotesi $F$
localmente lipschitziana rispetto a $\mathbf{x}$, uniformemente rispetto a $t$.}
di sistemi nella forma~\eqref{eq:sdcGenerale} vale il Teorema di Esistenza e Unicità locale delle soluzioni;
le soluzioni dipendono con continuità dal dato iniziale $\mathbf{x}_0$.

\paragraph{}
Se
$$\diffp{F}{t} = 0 \; ,$$
\ie il sistema non dipende esplicitamente dal tempo, esso si dice \emph{autonomo}.
Per tali sistemi non è restrittivo supporre la condizione iniziale data all'istante zero: $t_0=0$.

Lo spazio delle soluzioni è sempre uno spazio vettoriale, ma nel caso dei sistemi autonomi abbiamo
la possibilità di esprimere -- almeno localmente -- in forma canonica la soluzione generale come
combinazione lineare delle colonne di una matrice fondamentale che si ottiene direttamente
dalla $F$ del sistema.

Vedremo nella prossima Appendice~\ref{chap:floquetTheory} una forma canonica che è possibile
ottenere nei sistemi non autonomi a coefficienti periodici.

\paragraph{}
Possiamo descrivere sinteticamente le soluzioni di un sistema dinamico tramite la
\emph{soluzione generale}\footnote{Talvolta anche \emph{``flusso integrale''}.}
$$\Phi \, : \, D \to W, \quad (t, \mathbf{x}_0 ) \mapsto \Phi_t ( \mathbf{x}_0 ) \; ,$$
che mappa $(t, \mathbf{x}_0 )$ nella soluzione $\mathbf{x} (t)$ con condizioni iniziali $\mathbf{x}_0$,
valutata al tempo $t$. Questa è una applicazione ben definita grazie al Teorema di esistenza e unicità delle
soluzioni; purtroppo non abbiamo garanzie a priori sulla forma del dominio di ogni soluzione, che dipende
comunque dalle condizioni iniziali.
Dunque la forma di $D \subseteq \R \times W$ non ci è nota a priori.

Se indichiamo con $p$ la proiezione $D \to W$, sappiamo che ogni fibra
$p^{-1} ( \mathbf{x}_0 )$ è un intervallo aperto contenente $t_0$.

Per definizione, la mappa $t \mapsto \Phi_t ( \mathbf{x}_0 )$ è la soluzione del problema di Cauchy
$$\begin{sistema}
\diff{ }{t} \Phi_t ( \mathbf{x}_0 ) = F \left( \Phi_t ( \mathbf{x}_0 ) \right) \\
\Phi_{t_0} ( \mathbf{x}_0 ) = \mathbf{x}_0
\end{sistema}$$

Richiamiamo la \emph{proprietà di semigruppo} del flusso integrale:
$$\Phi_h \circ \Phi_t = \Phi_{h+t}$$
per ogni $h, t \geq t_0$.

Inoltre, la dipendenza continua delle soluzioni di~\eqref{eq:sdcGenerale} dal dato iniziale si riassume nel
Teorema di Continuità del Flusso (\ie $\Phi_t$ è continua in $\mathbf{x}_0$).

\begin{definizione}[Punto di Equilibrio]
    Diciamo che $\mathbf{x}_S \in W$ è un \emph{punto di equilibrio} per il sistema dinamico,
    se $F( \mathbf{x}_s ) = \mathbf{0}$.
\end{definizione}

Sopra ogni punto di equilibrio abbiamo la soluzione costante $\mathbf{x} (t) \equiv \mathbf{x}_S$, che è un'orbita,
ma ogni altra soluzione -- comunque vicina a $\mathbf{x}_S$ -- non può raggiungere $\mathbf{x}_S$
per nessun $t$ finito.

\begin{definizione}[Stabilità]
    Un punto $\mathbf{x}_S \in W$ si dice \emph{stabile} se per ogni intorno $U$ di $\mathbf{x}_S$
    esiste un intorno $V$ di $\mathbf{x}_S$
    tale che
    \footnote{Senza perdita di generalità $U, V, \subseteq W$.}
    $$(\forall \mathbf{x}_0 \in V) \; \Phi_t ( \mathbf{x}_0 ) \; ,$$
    per ogni $t \geq 0$.
\end{definizione}

Per definizione, le soluzioni possono essere ``costrette'' a restare arbitrariamente vicine ad un punto
stabile, purché si scelgano condizioni iniziali sufficiente vicine ad esso.

Si osservi che ogni punto stabile è anche di equilibrio.

\begin{definizione}[Attrazione e repulsione]
    Diciamo che un punto $\mathbf{x}_S \in W$ è \emph{attrattivo} se esiste un intorno
    $U$ di $\mathbf{x}_S$ tale che
    $$(\forall \mathbf{x}_0 \in U) \; \exists \lim_{t \to +\infty} \Phi_t (\mathbf{x}_0) = \mathbf{x}_S \; .$$

    Quando otteniamo la stessa cosa con $t \to -\infty$, diciamo che $\mathbf{x}_S$ è \emph{repulsivo}.
\end{definizione}

Si noti che essendo un punto limite, ogni punto attrattivo o repulsivo è anche un punto di equilibrio.

\begin{definizione}[Stabilità asintotica]
    Un punto che è sia attrattivo che stabile in avanti si dice \emph{asintoticamente stabile}.
\end{definizione}

\begin{definizione}{Esponenti di Lyapounov}
    Sia $J = DF (\mathbf{x}_S)$ la matrice jacobiana di $F$ valutata sul punto di equilibrio $\mathbf{x}_S$.

    Gli \emph{esponenti di Lyapounov} di questo equilibrio sono le parti reali degli autovalori di $J$.

    Diciamo che $\mathbf{x}_S$ è un \emph{pozzo} se tutti gli esponenti di Lyapounov sono strettamente negativi.
\end{definizione}

\begin{teorema}[del pozzo non lineare]
    Sia $\mathbf{x}_S \in W$ un pozzo per il sistema dinamico continuo $\dot{ \mathbf{x} } = F(\mathbf{x})$,
    sia $c>0$ tale che $\Re ( \lambda ) < -c$ per ogni autovalore $\lambda$ di $DF (\mathbf{x}_S)$.

    Allora esiste un intorno $U$ di $\mathbf{x}_S$ tale che:
    \begin{enumerate}
        \item la mappa $\Phi_t (\mathbf{x}_0)$ è definita per ogni $t \geq 0$, per ogni $\mathbf{x}_0 \in U$;
        \item $(\exists B>0) \, (\forall \mathbf{x}_0 \in U) \;
        \norm{ \Phi_t (\mathbf{x}_0) - \mathbf{x}_S } \leq
        B e^{-ct} \norm{ \mathbf{x}_0 - \mathbf{x}_S }.$
    \end{enumerate}
    \label{teo:pozzoNonLineare}
\end{teorema}


\begin{definizione}[Funzione di Lyapounov]
    Sia $\dot{\mathbf{x}} = F(\mathbf{x})$ un sistema dinamico continuo con $F \in C^1(W,\R^n)$, dove
    $W \subseteq \R^n$ è aperto.

    Diciamo che una funzione $V \in C^1(B, \R)$, per qualche aperto $B \subseteq W$, è una
    \emph{funzione di Lyapounov} per l'equilibrio $\mathbf{x}_S \in B$ se:
    \begin{itemize}
        \item $\dot{V} ( \mathbf{x} ) \leq 0 \; (\forall \mathbf{x} \in B),$
        \item $V( \mathbf{x}_S ) \lneq V( \mathbf{x} ) \; (\forall \mathbf{x} \in B, \mathbf{x} \neq \mathbf{x}_S).$
    \end{itemize}

    Diciamo inoltre che una funzione di Lyapounov $V$ è \emph{stretta} se
    $$\dot{V}(\mathbf{x}) \lneq 0 \; (\forall \mathbf{x} \in B, \mathbf{x} \neq \mathbf{x}_S) \; .$$
\end{definizione}

In pratica, $V$ ha un minimo locale forte sul punto di equilibrio, ed è non crescente
lungo le soluzioni vicine.

Possiamo vedere come $\dot{V} ( \mathbf{x} )$ sia una misura della crescita
di $V$ \emph{lungo le soluzioni} del sistema
dinamico definito dal campo vettoriale $F$, dal momento che
$$\dot{V} ( \mathbf{x} ) = \diff{}{t} \left( \dot{V} \left( \mathbf{x} (t) \right) \right) =
\sum_{i=1}^n \diffp{V}{{\mathbf{x}_i}} \left( \mathbf{x} (t) \right) \cdot \diff{ \mathbf{x}_i }{t} (t) =
\langle \nabla V ( \mathbf{x} ) , \dot{ \mathbf{x} } \rangle \; .$$

\begin{teorema}[della funzione di Lyapounov]
    Se un punto di equilibrio ammette una funzione di Lyapounov, allora esso è stabile.
    \label{teo:lyapounovFunc}
\end{teorema}

\begin{corollario}
    Se un punto di equilibrio ammette una funzione di Lyapounov stretta, allora esso è asintoticamente stabile.
\end{corollario}
