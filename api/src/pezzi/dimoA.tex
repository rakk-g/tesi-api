\chapter{Proofs}

\section{About KH11}
\subsection{ Lemma~\ref{lem:necessJ} }
From page~\pageref{lem:necessJ}.
\begin{proof}
    By equating to zero the right--hand side of Equation~\eqref{eq:kh11f} we get
    $$F^2 - \left( \frac{\alpha}{m} - \frac{\sigma}{m} -1
        \right) H F - \frac{\alpha}{m} H^2 = 0 \; ,$$
    which has discriminant
    $$\Delta = \left[ {\left( \frac{\alpha}{m} - \frac{\sigma}{m} -1
        \right)}^2 + 4 \frac{\alpha}{m}
        \right] H^2 \; ; $$
    we incidentally note that the parametric factor is always strictly positive.

    \paragraph{}
    Adopting the definition of $J$ from Equation~\eqref{eq:kh11posEqJ} we can see that any equilibrium point $(H^*, F^*)$ has to verify $F^* = J H^*$.
\end{proof}

\subsection{ Theorem~\ref{teo:exUniqFstarPos} }
From page~\pageref{teo:exUniqFstarPos}

\begin{proof}
    We get the proof done by somewhat tedious, but quite elementary calculations.

    Since we are looking for a non--trival equilibrium point $(H^*, F^*)$, by
    Lemma~\ref{lem:necessJ} we can suppose both components strictly positive in the following.

    By Lemma~\ref{lem:necessJ} we can substitute for $H=\frac{1}{J} F$ in the right--hand side of the first equation~\eqref{eq:kh11h} of the model, which is equal to zero by the equilibrium condition. From
    $$L \frac{ \left( \frac{1}{J} +1 \right) F }{ w +\left( \frac{1}{J} +1 \right) F }
    - \frac{1}{J} F \left( \alpha - \sigma \frac{\cancel{F}}{ \left( \frac{1}{J} +1 \right) \cancel{F} } \right) = 0$$
    we can simplify $F$, for it is non--zero, in order to get
    $$L \left( \frac{1}{J} +1 \right) \cancel{F} - \frac{1}{J} \left( \alpha - \frac{\sigma}{ \frac{1}{J} +1 } \right)
    \left[ w + \left( \frac{1}{J} +1 \right) F \right] \cancel{F} = 0 \; ,$$
    where the simplification of $F$ goes by the same argument as before; note that this amounts to discarding the zero solution, in which we are not interested by hypothesis.

    We get the uniqueness of the positive equilibrium by further algebraic manipulation of the above equations to finally get
    $$F^* = \frac{LJ}{ \alpha - \frac{\sigma}{ 1 + \frac{1}{J} } } - w \frac{J}{1+J} \; ,$$
    where we can see that the right--hand side is entirely dependent on the parameters.

    \paragraph{}
    Since we got last equation holding, in order to conclude our proof --
    \ie to estabilish equation \eqref{eq:kh11posEqF} -- it remains to prove that
    $$J m = \alpha - \sigma \frac{J}{1+J} \; ,$$
    where again we simplified the factor $L$ at both sides, for it is non--zero.

    To estabilish this last equation could be useful to rationalize the expression $\frac{J}{1+J}$ as
    $$\frac{J}{1+J} = \frac{ \alpha + \sigma + m - \sqrt{ {(\alpha - \sigma -m)}^2 - 4 \alpha m } }{2 \sigma}$$
    and get to it via algebraic manipulation.

    \paragraph{}
    We have estabilished Equation~\eqref{eq:kh11posEqF} and this concludes the proof.
\end{proof}


\subsection{ Theorem~\ref{teo:esistenzPosF} }
From page \pageref{teo:esistenzPosF}.

\begin{proof}
We can use Equation~\eqref{eq:kh11posEqF} from Theorem~\ref{teo:exUniqFstarPos} and then study the
inequality $F^* >0$ to prove the existence criterion:
$$F^* > 0 \; \iff \;
\left( \frac{2 \alpha w}{L} -1 \right) m - \alpha - \sigma < \sqrt{ {(\alpha - \sigma -m)}^2 + 4 \alpha m }$$

we can square both sides in order to get the following inequality in $m$:
\begin{equation}
    \left( \frac{\alpha w}{L} -1 \right) m^2 -( \alpha + \sigma) m + \frac{\sigma L}{w} < 0
    \label{eq:mDiseq}
\end{equation}
which has always positive discriminant
$$\Delta = {(\alpha -\sigma)}^2 +4 \frac{\sigma L}{w} \; .$$

Via the quadratic roots formula we get Equation~\eqref{eq:FstarPosM12}.
We will select the nomenclature as to always have $m_1^* < m_2^*$ in the rest of this proof.

\paragraph{Case 1.}
If $\alpha - \frac{L}{w} >0$ then $F^*>0$ exists for $m_1^* < m < m_2^*$, and the lower bound makes sense since
$m_1^* > 0$ reduces to $\alpha - \frac{L}{w}>0$, which is true by case hypothesis.

\paragraph{Case 2.}
If $\alpha -\frac{L}{w} <0$ then we must take $m < m_1^*$ or $m > m_2^*$ for $F^* >0$ to exist.
The first inequality can be discarded as is $m_1^* <0$ by case hypothesis.

The second one must be kept since $m_2^* >0$ reduces to our case hypothesis $\alpha - \frac{L}{w} < 0$.

\paragraph{Case 3.}
When $\alpha - \frac{L}{w} =0$ our inequality~\eqref{eq:mDiseq} becomes linear and its solution is
$$m > \frac{\sigma L}{w (\alpha +\sigma)} \; ,$$
which concludes our last case and the proof.
\end{proof}


\section{About Ratti17}
\subsection{ Proposition~\ref{prop:rExist} }
From page~\pageref{prop:rExist}.

\begin{proof}
    Le equazioni del modello sono omogenee, dunque l'equilibrio banale $(0,0)$ esiste sempre.
    Dalla \eqref{eq:rattiRidotto1seconda} segue che ogni punto di equilibrio strettamente positivo $(x_h^*, x_f^*)$ verifica
    $$x_f^2 + \frac{\sigma_2 - \sigma_1 + p + d_2}{p +d_2} x_h x_f - \frac{\sigma_1}{p+d_2} x_h^2 = 0,$$
    da cui ricaviamo la relazione tra le componenti $x_f^* = F x_h^*$.

    [DA CONCLUDERE QUANDO SI SPIEGA L'ESPONENTE $i$] % TODO
\end{proof}








\chapter{Theory}

\begin{teorema}[Linear Well]
    If $\Re{\lambda} < 0$ for all eigenvalues \omissis % TODO

    \label{teo:pozzoLineare}
\end{teorema}


\begin{definizione}[Lyapounov function]
    Let $\dot{x} = F(x)$ be a continuous dynamical system with $F \in C^1(W,\R^n)$, where $W \subseteq \R^n$ is open.

    We say that a function $V \in C^1(B, \R)$, for some open $B \subseteq W$, is a \emph{Lyapounov function} for the equilibrium point $x_S \in B$ if:
    \begin{itemize}
        \item $\dot{V} (x) \leq 0 \; (\forall x \in B),$
        \item $V(x_S) \lneq V(x) \; (\forall x \in B, x \neq x_S).$
    \end{itemize}

    We also say that a Lyapounov function $V$ is \emph{strict} if $\dot{V}(x) \lneq 0 \; (\forall x \in B, x \neq x_S)$.
\end{definizione}

% comment. % TODO

\begin{teorema}[Stability via Lyapounov function]

    \label{teo:lyapounovFunc}
\end{teorema}

\begin{corollario}[Asymptotic stability via Lyapounov function]

\end{corollario}
