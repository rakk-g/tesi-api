\chapter{Stated results}

\section{About KH11}
\subsection{ Lemma~\ref{lem:necessJ} }
From page~\pageref{lem:necessJ}.
\begin{proof}
    By equating to zero the right--hand side of Equation~\eqref{eq:kh11f} we get
    $$F^2 - \left( \frac{\alpha}{m} - \frac{\sigma}{m} -1
        \right) H F - \frac{\alpha}{m} H^2 = 0 \; ,$$
    which has discriminant
    $$\Delta = \left[ {\left( \frac{\alpha}{m} - \frac{\sigma}{m} -1
        \right)}^2 + 4 \frac{\alpha}{m}
        \right] H^2 \; ; $$
    we incidentally note that the parametric factor is always strictly positive.

    \paragraph{}
    Adopting the definition of $J$ from Equation~\eqref{eq:kh11posEqJ} we can see that any equilibrium point $(H^*, F^*)$ has to verify $F^* = J H^*$.
\end{proof}

\subsection{ Theorem~\ref{teo:exUniqFstarPos} }
From page~\pageref{teo:exUniqFstarPos}

\begin{proof}
    We get the proof done by somewhat tedious, but quite elementary calculations.

    Since we are looking for a non--trival equilibrium point $(H^*, F^*)$, by
    Lemma~\ref{lem:necessJ} we can suppose both components strictly positive in the following.

    By Lemma~\ref{lem:necessJ} we can substitute for $H=\frac{1}{J} F$ in the right--hand side of the first equation~\eqref{eq:kh11h} of the model, which is equal to zero by the equilibrium condition. From
    $$L \frac{ \left( \frac{1}{J} +1 \right) F }{ w +\left( \frac{1}{J} +1 \right) F }
    - \frac{1}{J} F \left( \alpha - \sigma \frac{\cancel{F}}{ \left( \frac{1}{J} +1 \right) \cancel{F} } \right) = 0$$
    we can simplify $F$, for it is non--zero, in order to get
    $$L \left( \frac{1}{J} +1 \right) \cancel{F} - \frac{1}{J} \left( \alpha - \frac{\sigma}{ \frac{1}{J} +1 } \right)
    \left[ w + \left( \frac{1}{J} +1 \right) F \right] \cancel{F} = 0 \; ,$$
    where the simplification of $F$ goes by the same argument as before; note that this amounts to discarding the zero solution, in which we are not interested by hypothesis.

    We get the uniqueness of the positive equilibrium by further algebraic manipulation of the above equations to finally get
    $$F^* = \frac{LJ}{ \alpha - \frac{\sigma}{ 1 + \frac{1}{J} } } - w \frac{J}{1+J} \; ,$$
    where we can see that the right--hand side is entirely dependent on the parameters.

    \paragraph{}
    Since we got last equation holding, in order to conclude our proof --
    \ie to estabilish equation \eqref{eq:kh11posEqF} -- it remains to prove that
    $$J m = \alpha - \sigma \frac{J}{1+J} \; ,$$
    where again we simplified the factor $L$ at both sides, for it is non--zero.

    To estabilish this last equation could be useful to rationalize the expression $\frac{J}{1+J}$ as
    $$\frac{J}{1+J} = \frac{ \alpha + \sigma + m - \sqrt{ {(\alpha - \sigma -m)}^2 - 4 \alpha m } }{2 \sigma}$$
    and get to it via algebraic manipulation.

    \paragraph{}
    We have estabilished Equation~\eqref{eq:kh11posEqF} and this concludes the proof.
\end{proof}


\subsection{ Theorem~\ref{teo:esistenzPosF} }
From page \pageref{teo:esistenzPosF}.

\begin{proof}
We can use Equation~\eqref{eq:kh11posEqF} from Theorem~\ref{teo:exUniqFstarPos} and then study the
inequality $F^* >0$ to prove the existence criterion:
$$F^* > 0 \; \iff \;
\left( \frac{2 \alpha w}{L} -1 \right) m - \alpha - \sigma < \sqrt{ {(\alpha - \sigma -m)}^2 + 4 \alpha m }$$

we can square both sides in order to get the following inequality in $m$:
\begin{equation}
    \left( \frac{\alpha w}{L} -1 \right) m^2 -( \alpha + \sigma) m + \frac{\sigma L}{w} < 0
    \label{eq:mDiseq}
\end{equation}
which has always positive discriminant
$$\Delta = {(\alpha -\sigma)}^2 +4 \frac{\sigma L}{w} \; .$$

Via the quadratic roots formula we get Equation~\eqref{eq:FstarPosM12}.
We will select the nomenclature as to always have $m_1^* < m_2^*$ in the rest of this proof.

\paragraph{Case 1.}
If $\alpha - \frac{L}{w} >0$ then $F^*>0$ exists for $m_1^* < m < m_2^*$, and the lower bound makes sense since
$m_1^* > 0$ reduces to $\alpha - \frac{L}{w}>0$, which is true by case hypothesis.

\paragraph{Case 2.}
If $\alpha -\frac{L}{w} <0$ then we must take $m < m_1^*$ or $m > m_2^*$ for $F^* >0$ to exist.
The first inequality can be discarded as is $m_1^* <0$ by case hypothesis.

The second one must be kept since $m_2^* >0$ reduces to our case hypothesis $\alpha - \frac{L}{w} < 0$.

\paragraph{Case 3.}
When $\alpha - \frac{L}{w} =0$ our inequality~\eqref{eq:mDiseq} becomes linear and its solution is
$$m > \frac{\sigma L}{w (\alpha +\sigma)} \; ,$$
which concludes our last case and the proof.
\end{proof}


\section{About Ratti17}
\subsection{ Proposition~\ref{prop:rExist} }
From page~\pageref{prop:rExist}.

\begin{proof}
    Le equazioni del modello sono omogenee, dunque l'equilibrio banale $(0,0)$ esiste sempre.
    Dalla \eqref{eq:rattiRidotto1seconda} segue che ogni punto di equilibrio strettamente positivo $(x_h^*, x_f^*)$ verifica
    $$x_f^2 + \frac{\sigma_2 - \sigma_1 + p + d_2}{p +d_2} x_h x_f - \frac{\sigma_1}{p+d_2} x_h^2 = 0,$$
    da cui ricaviamo la relazione tra le componenti $x_f^* = F x_h^*$.

    [DA CONCLUDERE QUANDO SI SPIEGA L'ESPONENTE $i$] % TODO
\end{proof}








\chapter{Theory}

In this chapter we recall some useful results from the theory of dynamical systems.

It is useful to define some preliminary notation, to be used in the rest of this chapter.
Let
$$\begin{sistema}
\dot{\mathbf{x}} = F ( \mathbf{x} ) \\
\mathbf{x} (t_0) = \mathbf{x}_0
\end{sistema}$$
be a dynamical system with initial condition $\mathbf{x}_0$. We will suppose throughout the rest of this
chapter that $F$ is defined on some open $W \subseteq \R^n$ and that it is of class $C^1$ at least.

If
$$\diffp{F}{t} = 0 \; ,$$
\ie the system equation does not depend explicitly on time, we say that the system is \emph{autonomous}.
For such system is unrestrictive to suppose the initial condition is given at the instant $t_0=0$.

\paragraph{}
We can describe synthetically the solutions of a dynamical system via the \emph{general solution}\footnote{Sometimes
\emph{``integral flow''}.} map
$$\Phi \, : \, D \to W, \quad (t, \mathbf{x}_0 ) \mapsto \Phi_t ( \mathbf{x}_0 ) \; ,$$
that sends $(t, \mathbf{x}_0 )$ to the solution $\mathbf{x} (t)$ with initial condition $\mathbf{x}_0$, evaluated at time $t$. This is a well--defined application via the Existence/Uniqueness Theorem; it does not, however,
state any condition on the form the domain of each solution, which still depend on the initial condition.
Thus $D \subseteq \R times W$ is not known in advance.

If we denote the projection $D \to W$ by $p$, we know that each fiber $p^{-1} ( \mathbf{x}_0 )$ is an open interval
containing $t_0$.

By definition, the map $t \mapsto \Phi_t ( \mathbf{x}_0 )$ is a solution to the dynamical system
$$\begin{sistema}
\diff{ \Phi_t ( \mathbf{x}_0 ) }{t} = F \left( \Phi_t ( \mathbf{x}_0 ) \right) \\
\Phi_{t_0} ( \mathbf{x}_0 ) = \mathbf{x}_0
\end{sistema}$$

We recall the \emph{semigroup property} of the general solution
$$\Phi_h \circ \Phi_t = \Phi_{h+t}$$
for all $h, t \geq t_0$.

\begin{definizione}[Equilibrium point]
    We say that $\mathbf{x}_S \in W$ is an \emph{equilibrium point} for our dynamical system,
    if $F( \mathbf{x}_s ) = \mathbf{0}$.
\end{definizione}

On any equilibrium point we have a constant orbit $\mathbf{x} (t) \equiv \mathbf{x}_S$, but any different solution
-- however near $\mathbf{x}_S$ -- cannot reach this point for any finite $t$ value.

\begin{definizione}[Stable point]
    A point $\mathbf{x}_S \in W$ is said to be \emph{stable} if for every neighbourhood $U$ of $\mathbf{x}_S$
    there exists a neighbourhood $V$ of $\mathbf{x}_S$\footnote{Without loss of generality $U, V, \subseteq W$.}
    such that
    $$(\forall \mathbf{x}_0 \in V) \; \Phi_t ( \mathbf{x}_0 ) \; ,$$
    for every $t \geq 0$.
\end{definizione}

By definition, solutions can be constrained to remain arbitrarily near a stable point, provided that we choose
initial conditions sufficiently close to it.

Observe that any stable point is an equilibrium point also.

\begin{definizione}[Attractive point]
    We say that a point $\mathbf{x}_S \in W$ is \emph{attractive} if there exists a
    neighbourhood $U$ of $\mathbf{x}_S$ such that
    $$(\forall \mathbf{x}_0 \in U) \; \exists \lim_{t \to +\infty} \Phi_t (\mathbf{x}_0) = \mathbf{x}_S \; .$$

    When we get the same with $t \to -\infty$, we say that $\mathbf{x}_S$ is \emph{repulsive}.
\end{definizione}

Note that, being a limit point, any attractive or repulsive point is an equilibrium point also.

\begin{definizione}[Asymptotic stability]
    A point which is both attractive and forward stable, is said \emph{asymptotically stable}.
\end{definizione}

\begin{definizione}{Lyapounov Exponents}
    Let $J = DF (\mathbf{x}_S)$ be the jacobian matrix evaluated on the equilibrium point $\mathbf{x}_S$.

    The \emph{Lyapounov exponents} of this equilibrium are the real parts o the eigenvalues of $J$.
    We say that $\mathbf{x}_S$ is a \emph{sink} if every Lyapounov exponent is strictly negative.
\end{definizione}

\begin{teorema}[Non--linear sink]
    Let $\mathbf{x}_S \in W$ be a sink for the continuous system $\dot{ \mathbf{x} } = F(\mathbf{x})$,
    let $c>0$ be such that $\Re ( \lambda ) < -c$ for each eigenvalue $\lambda$ of $DF (\mathbf{x}_S)$.

    There exists a neighbourhood $U$ of $\mathbf{x}_S$ such that:
    \begin{enumerate}
        \item $\Phi_t (\mathbf{x}_0)$ is defined for $t \geq 0$, for all $\mathbf{x}_0 \in U$;
        \item $(\exists B>0) \, (\forall \mathbf{x}_0 \in U) \;
        \norm{ \Phi_t (\mathbf{x}_0) - \mathbf{x}_S } \leq
        B e^{-ct} \norm{ \mathbf{x}_0 - \mathbf{x}_S }.$
    \end{enumerate}
    \label{teo:pozzoNonLineare}
\end{teorema}


\begin{definizione}[Lyapounov function]
    Let $\dot{\mathbf{x}} = F(\mathbf{x})$ be a continuous dynamical system with $F \in C^1(W,\R^n)$, where $W \subseteq \R^n$ is open.

    We say that a function $V \in C^1(B, \R)$, for some open $B \subseteq W$, is a \emph{Lyapounov function} for the equilibrium point $\mathbf{x}_S \in B$ if:
    \begin{itemize}
        \item $\dot{V} ( \mathbf{x} ) \leq 0 \; (\forall \mathbf{x} \in B),$
        \item $V( \mathbf{x}_S ) \lneq V( \mathbf{x} ) \; (\forall \mathbf{x} \in B, \mathbf{x} \neq \mathbf{x}_S).$
    \end{itemize}

    We also say that a Lyapounov function $V$ is \emph{strict} if $\dot{V}(\mathbf{x}) \lneq 0 \; (\forall \mathbf{x} \in B, \mathbf{x} \neq \mathbf{x}_S)$.
\end{definizione}

We can see that $\dot{V} ( \mathbf{x} )$ is a measure of $V$ growth along solutions of the dynamical system
defined by the vector field $F$, since
$$\dot{V} ( \mathbf{x} ) = \diff{}{t} \left( \dot{V} \left( \mathbf{x} (t) \right) \right) =
\sum_{i=1}^n \diffp{V}{{\mathbf{x}_i}} \left( \mathbf{x} (t) \right) \cdot \diff{ \mathbf{x}_i }{t} (t) =
\langle \nabla V ( \mathbf{x} ) , \dot{ \mathbf{x} } \rangle.$$


\begin{teorema}[Stability via Lyapounov function]
    If an equilibrium point admits a Lyapounov function, then it is stable.
    \label{teo:lyapounovFunc}
\end{teorema}

\begin{corollario}[Asymptotic stability via Lyapounov function]
    If an equilibrium point admits a strict Lyapounov function, then it is asymptotically stable.
\end{corollario}
