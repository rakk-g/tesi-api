% WARNING questo capitolo l'ho scorporato mettendo le dimostrazioni dopo gli enunciati.
% \chapter{Risultati}
%
% \section{A proposito di KH11}
% \subsection{ Lemma~\ref{lem:necessJ} }
% L'enunciato si trova a pagina~\pageref{lem:necessJ}.
% \begin{proof}
%     Annullando il membro destro dell'equazione~\eqref{eq:kh11f} otteniamo
%     $$F^2 - \left( \frac{\alpha}{m} - \frac{\sigma}{m} -1
%         \right) H F - \frac{\alpha}{m} H^2 = 0 \; ,$$
%     che ha determinante
%     $$\Delta = \left[ {\left( \frac{\alpha}{m} - \frac{\sigma}{m} -1
%         \right)}^2 + 4 \frac{\alpha}{m}
%         \right] H^2 \; ; $$
%     notiamo incidentalmente che il fattore parametrico è sempre strettamente positivo.
%
%     \paragraph{}
%     Adottando la definizione di $J$ dell'equazione~\eqref{eq:kh11posEqJ}, possiamo vedere
%     che ogni punto d'equilibrio $(H^*, F^*)$ deve necessariamente verificare $F^* = J H^*$.
% \end{proof}


% \subsection{ Proposizione~\ref{teo:exUniqFstarPos} }
% L'enunciato si trova a pagina~\pageref{teo:exUniqFstarPos}
%
% \begin{proof}
%     Otteniamo la dimostrazione con alcuni calcoli elementari.
%
%     Dal momento che stiamo cercando un equilibrio non banale $(H^*, F^*)$, per il
%     Lemma~\ref{lem:necessJ} possiamo supporre ambo le componenti strettamente positive nel seguito.
%
%     Ancora per il Lemma~\ref{lem:necessJ}, possiamo sostituire $H=\frac{1}{J} F$ nel membro destro della
%     prima equazione~\eqref{eq:kh11h} del modello, ed usare la condizione di equilibrio per uguagliare
%     a zero quanto ottenuto. Da
%     $$L \frac{ \left( \frac{1}{J} +1 \right) F }{ w +\left( \frac{1}{J} +1 \right) F }
%     - \frac{1}{J} F \left( \alpha - \sigma \frac{\cancel{F}}{ \left( \frac{1}{J} +1 \right) \cancel{F} } \right) = 0$$
%     possiamo semplificare $F$ perché non nulla, così da ottenere
%     $$L \left( \frac{1}{J} +1 \right) \cancel{F} - \frac{1}{J} \left( \alpha - \frac{\sigma}{ \frac{1}{J} +1 } \right)
%     \left[ w + \left( \frac{1}{J} +1 \right) F \right] \cancel{F} = 0 \; ,$$
%     dove la semplificazione di $F$ procede con lo stesso argomento di sopra; si noti che ciò ammonta a scartare la
%     soluzione nulla, che non ci interessa per ipotesi.
%
%     L'unicità dell'equilibrio positivo si ottiene manipolando ulteriormente l'equazione sopra, ottenendo infine
%     $$F^* = \frac{LJ}{ \alpha - \frac{\sigma}{ 1 + \frac{1}{J} } } - w \frac{J}{1+J} \; ,$$
%     in cui possiamo osservare che il membro destro dipende esclusivamente dai parametri.
%
%     \paragraph{}
%     Avendo a disposizione l'ultima equazione, per concludere la dimostrazione -- \ie stabilire
%     l'equazione~\eqref{eq:kh11posEqF} -- resta soltanto da provare che
%     $$J m = \alpha - \sigma \frac{J}{1+J} \; ,$$
%     dove abbiamo semplificato il fattore $L$ ad ambo i membri, essendo non nullo.
%
%     Per stabilire quest'ultima equazione può essere utile razionalizzare l'espressione $\frac{J}{1+J}$ come
%     $$\frac{J}{1+J} = \frac{ \alpha + \sigma + m - \sqrt{ {(\alpha - \sigma -m)}^2 - 4 \alpha m } }{2 \sigma}$$
%     e raggiungere l'obiettivo tramite manipolazioni algebriche.
%
%     \paragraph{}
%     Abbiamo stabilito l'equazione~\eqref{eq:kh11posEqF} e ciò conclude la prova.
% \end{proof}


% \subsection{ Proposizione~\ref{teo:esistenzPosF} }
% Dalla pagina~\pageref{teo:esistenzPosF}.
%
% \begin{proof}
% Possiamo usare la equazione~\eqref{eq:kh11posEqF} dalla Proposizione~\ref{teo:exUniqFstarPos} e studiare
% la disequazione $F^* >0$ per dimostrare il criterio di esistenza:
% $$F^* > 0 \; \iff \;
% \left( \frac{2 \alpha w}{L} -1 \right) m - \alpha - \sigma < \sqrt{ {(\alpha - \sigma -m)}^2 + 4 \alpha m }$$
%
% possiamo elevare al quadrato ambo i membri per ottenere la seguente disuguaglianza in $m$:
% \begin{equation}
%     \left( \frac{\alpha w}{L} -1 \right) m^2 -( \alpha + \sigma) m + \frac{\sigma L}{w} < 0
%     \label{eq:mDiseq}
% \end{equation}
% la quale ha determinante sempre positivo
% $$\Delta = {(\alpha -\sigma)}^2 +4 \frac{\sigma L}{w} \; .$$
%
% L'equazione~\eqref{eq:FstarPosM12} si ottiene applicando la formula delle radici quadratiche.
% Nel seguito selezioneremo la nomenclatura per avere sempre $m_1^* < m_2^*$ fino al termine della dimostrazione.
%
% \paragraph{Caso 1.}
% Se $\alpha - \frac{L}{w} >0$ allora $F^*>0$ esiste $m_1^* < m < m_2^*$, ed il limite inferiore ha senso,
% dal momento che
% $m_1^* > 0$ si riduce a $\alpha - \frac{L}{w}>0$, vera per ipotesi del caso.
%
% \paragraph{Caso 2.}
% Se $\alpha -\frac{L}{w} <0$ allora dobbiamo prendere $m < m_1^*$ oppure $m > m_2^*$ affinché $F^* >0$ esista.
% La prima disequazione si può scartare dal momento che $m_1^* <0$ per ipotesi del caso.
%
% La seconda va mantenuta dal momento che $m_2^* >0$ si riduce alla ipotesi del caso $\alpha - \frac{L}{w} < 0$.
%
% \paragraph{Caso 3.}
% Quando $\alpha - \frac{L}{w} =0$ la nostra disequazione~\eqref{eq:mDiseq} diviene lineare e la sua soluzione è
% $$m > \frac{\sigma L}{w (\alpha +\sigma)} \; ,$$
% ciò che conclude l'ultimo caso restante della dimostrazione.
% \end{proof}


% \section{A proposito di Ratti17}
% \subsection{ Proposizione~\ref{prop:rExist} }
% Da pagina~\pageref{prop:rExist}.
%
% \begin{proof}
%     Le equazioni del modello sono omogenee, dunque l'equilibrio banale $(0,0)$ esiste sempre.
%     Dalla \eqref{eq:rattiRidotto1seconda} segue che ogni punto di equilibrio strettamente positivo $(x_h^*, x_f^*)$ verifica
%     $$x_f^2 + \frac{\sigma_2 - \sigma_1 + p + d_2}{p +d_2} x_h x_f - \frac{\sigma_1}{p+d_2} x_h^2 = 0,$$
%     da cui ricaviamo la relazione tra le componenti $x_f^* = F x_h^*$.
%
%     [DA CONCLUDERE QUANDO SI SPIEGA L'ESPONENTE $i$] % TODO
% \end{proof}








\chapter{Cenni di teoria}
\label{chap:teoria}
In questo capitolo si richiamano alcuni risultati nell'ambito dei sistemi dinamici, utili strumenti ai fini
di questa tesi.

Presentiamo qui soltanto i teoremi e le definizioni che si applicano all'ambito generale; alcuni oggetti
e risultati riguardanti più nello specifico i sistemi biomatematici, di modellazione ecologica delle popolazioni
e delle patologie sono esaminati nel sezione~\ref{sec:ingredientiBioMat}.

\paragraph{}
È utile definire un po' di notazione preliminare, prima di procedere oltre.
Sia
$$\begin{sistema}
\dot{\mathbf{x}} = F ( \mathbf{x} ) \\
\mathbf{x} (t_0) = \mathbf{x}_0
\end{sistema}$$
un sistema dinamico con condizione iniziale $\mathbf{x}_0$. Supporremo nel seguito del capitolo che $F$ sia
definita su qualche aperto $W \subseteq \R^n$ e che sia di classe $C^1$ almeno.

Se
$$\diffp{F}{t} = 0 \; ,$$
\ie il sistema non dipende esplicitamente dal tempo, esso si dice \emph{autonomo}.
Per tali sistemi non è restrittivo supporre la condizione iniziale data all'istante zero: $t_0=0$.

\paragraph{}
Possiamo descrivere sinteticamente le soluzioni di un sistema dinamico tramite la
\emph{soluzione generale}\footnote{Talvolta anche \emph{``flusso integrale''}.}
$$\Phi \, : \, D \to W, \quad (t, \mathbf{x}_0 ) \mapsto \Phi_t ( \mathbf{x}_0 ) \; ,$$
che mappa $(t, \mathbf{x}_0 )$ nella soluzione $\mathbf{x} (t)$ con condizioni iniziali $\mathbf{x}_0$,
valutata al tempo $t$. Questa è una applicazione ben definita grazie al Teorema di esistenza e unicità delle
soluzioni; purtroppo non abbiamo garanzie a priori sulla forma del dominio di ogni soluzione, che dipende
comunque dalle condizioni iniziali.
Dunque la forma di $D \subseteq \R \times W$ non ci è nota a priori.

Se indichiamo con $p$ la proiezione $D \to W$, sappiamo che ogni fibra
$p^{-1} ( \mathbf{x}_0 )$ è un intervallo aperto contenente $t_0$.

Per definizione, la mappa $t \mapsto \Phi_t ( \mathbf{x}_0 )$ è la soluzione del problema di Cauchy
$$\begin{sistema}
\diff{ \Phi_t ( \mathbf{x}_0 ) }{t} = F \left( \Phi_t ( \mathbf{x}_0 ) \right) \\
\Phi_{t_0} ( \mathbf{x}_0 ) = \mathbf{x}_0
\end{sistema}$$

Richiamiamo la \emph{proprietà di semigruppo} del flusso integrale:
$$\Phi_h \circ \Phi_t = \Phi_{h+t}$$
per ogni $h, t \geq t_0$.

\begin{definizione}[Punto di Equilibrio]
    Diciamo che $\mathbf{x}_S \in W$ è un \emph{punto di equilibrio} per il sistema dinamico,
    se $F( \mathbf{x}_s ) = \mathbf{0}$.
\end{definizione}

Sopra ogni punto di equilibrio abbiamo la soluzione costante $\mathbf{x} (t) \equiv \mathbf{x}_S$, che è un'orbita,
ma ogni altra soluzione -- comunque vicina a $\mathbf{x}_S$ -- non può raggiungere $\mathbf{x}_S$
per nessun $t$ finito.

\begin{definizione}[Stabilità]
    Un punto $\mathbf{x}_S \in W$ si dice \emph{stabile} se per ogni intorno $U$ di $\mathbf{x}_S$
    esiste un intorno $V$ di $\mathbf{x}_S$
    tale che
    \footnote{Senza perdita di generalità $U, V, \subseteq W$.}
    $$(\forall \mathbf{x}_0 \in V) \; \Phi_t ( \mathbf{x}_0 ) \; ,$$
    per ogni $t \geq 0$.
\end{definizione}

Per definizione, le soluzioni possono essere ``costrette'' a restare arbitrariamente vicine ad un punto
stabile, purché si scelgano condizioni iniziali sufficiente vicine ad esso.

Si osservi che ogni punto stabile è anche di equilibrio.

\begin{definizione}[Attrazione e repulsione]
    Diciamo che un punto $\mathbf{x}_S \in W$ è \emph{attrattivo} se esiste un intorno
    $U$ di $\mathbf{x}_S$ tale che
    $$(\forall \mathbf{x}_0 \in U) \; \exists \lim_{t \to +\infty} \Phi_t (\mathbf{x}_0) = \mathbf{x}_S \; .$$

    Quando otteniamo la stessa cosa con $t \to -\infty$, diciamo che $\mathbf{x}_S$ è \emph{repulsivo}.
\end{definizione}

Si noti che essendo un punto limite, ogni punto attrattivo o repulsivo è anche un punto di equilibrio.

\begin{definizione}[Stabilità asintotica]
    Un punto che è sia attrattivo che stabile in avanti si dice \emph{asintoticamente stabile}.
\end{definizione}

\begin{definizione}{Esponenti di Lyapounov}
    Sia $J = DF (\mathbf{x}_S)$ la matrice jacobiana di $F$ valutata sul punto di equilibrio $\mathbf{x}_S$.

    Gli \emph{esponenti di Lyapounov} di questo equilibrio sono le parti reali degli autovalori di $J$.

    Diciamo che $\mathbf{x}_S$ è un \emph{pozzo} se tutti gli esponenti di Lyapounov sono strettamente negativi.
\end{definizione}

\begin{teorema}[del pozzo non lineare]
    Sia $\mathbf{x}_S \in W$ un pozzo per il sistema dinamico continuo $\dot{ \mathbf{x} } = F(\mathbf{x})$,
    sia $c>0$ tale che $\Re ( \lambda ) < -c$ per ogni autovalore $\lambda$ di $DF (\mathbf{x}_S)$.

    Allora esiste un intorno $U$ di $\mathbf{x}_S$ tale che:
    \begin{enumerate}
        \item la mappa $\Phi_t (\mathbf{x}_0)$ è definita per ogni $t \geq 0$, per ogni $\mathbf{x}_0 \in U$;
        \item $(\exists B>0) \, (\forall \mathbf{x}_0 \in U) \;
        \norm{ \Phi_t (\mathbf{x}_0) - \mathbf{x}_S } \leq
        B e^{-ct} \norm{ \mathbf{x}_0 - \mathbf{x}_S }.$
    \end{enumerate}
    \label{teo:pozzoNonLineare}
\end{teorema}


\begin{definizione}[Funzione di Lyapounov]
    Sia $\dot{\mathbf{x}} = F(\mathbf{x})$ un sistema dinamico continuo con $F \in C^1(W,\R^n)$, dove
    $W \subseteq \R^n$ è aperto.

    Diciamo che una funzione $V \in C^1(B, \R)$, per qualche aperto $B \subseteq W$, è una
    \emph{funzione di Lyapounov} per l'equilibrio $\mathbf{x}_S \in B$ se:
    \begin{itemize}
        \item $\dot{V} ( \mathbf{x} ) \leq 0 \; (\forall \mathbf{x} \in B),$
        \item $V( \mathbf{x}_S ) \lneq V( \mathbf{x} ) \; (\forall \mathbf{x} \in B, \mathbf{x} \neq \mathbf{x}_S).$
    \end{itemize}

    Diciamo inoltre che una funzione di Lyapounov $V$ è \emph{stretta} se
    $$\dot{V}(\mathbf{x}) \lneq 0 \; (\forall \mathbf{x} \in B, \mathbf{x} \neq \mathbf{x}_S) \; .$$
\end{definizione}

In pratica, $V$ ha un minimo locale forte sul punto di equilibrio, ed è non crescente lungo le soluzioni vicine.

Possiamo vedere come $\dot{V} ( \mathbf{x} )$ sia una misura della crescita di $V$ lungo le soluzioni del sistema
dinamico definito dal campo vettoriale $F$, dal momento che
$$\dot{V} ( \mathbf{x} ) = \diff{}{t} \left( \dot{V} \left( \mathbf{x} (t) \right) \right) =
\sum_{i=1}^n \diffp{V}{{\mathbf{x}_i}} \left( \mathbf{x} (t) \right) \cdot \diff{ \mathbf{x}_i }{t} (t) =
\langle \nabla V ( \mathbf{x} ) , \dot{ \mathbf{x} } \rangle \; .$$

\begin{teorema}[della funzione di Lyapounov]
    Se un punto di equilibrio ammette una funzione di Lyapounov, allora esso è stabile.
    \label{teo:lyapounovFunc}
\end{teorema}

\begin{corollario}
    Se un punto di equilibrio ammette una funzione di Lyapounov stretta, allora esso è asintoticamente stabile.
\end{corollario}
