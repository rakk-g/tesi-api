\section{Esperimento A, iterazione 2}
Il modello di riferimento a due compartimenti\footcite{khoury2011} esaminato nella sezione~\ref{sec:kh11} è
definito dalle equazioni~\eqref{eq:kh11h}--\eqref{eq:kh11f} a p.~\pageref{eq:kh11h}, che per comodità
di lettura riportiamo qui sotto:
\begin{align*}
    \dot{H} &= E(H,F)- H \cdot R(H,F)\; , \\
    \dot{F} &= H \cdot R(H,F)  - m \cdot F\; ,
\end{align*}
ove $E(H,F) = L \frac{H+F}{w + H + F}$ rappresenta la schiusa di nuove operaie di nido,
mentre il reclutamento sociale è dato dalla $R(H,F) = \alpha - \sigma \frac{F}{H+F}$.

\paragraph{}
Tutte le simulazioni sono state svolte mantenendo costanti i seguenti parametri:
\begin{itemize}
    \item il tasso di deposizione della regina $L=2000$;
    \item il tasso massimo di reclutamento $\alpha = \frac{1}{4}$;
    \item il tasso di inibizione sociale $\sigma = \frac{3}{4}$;
\end{itemize}

Per esplorare lo spazio dei restanti parametri, sono state svolte 27648 simulazioni, risolvendo istanze del problema
con parametri e condizioni iniziali scelti come descritto di seguito.
\begin{figure}[hb]
    \centering
    \includegraphics[keepaspectratio,width=\textwidth]{img/k11EA2-parameterSpace3D}

    \caption[Esperimento A2, spazio dei parametri.]{Ogni punto $(m,w,b)$ corrisponde ad una simulazione effettuata
        coi parametri $m,w$ e popolazione iniziale $b$. La dimensione del punto rappresenta la popolazione finale
        $H(T)+F(T)$ raggiunta al termine della simulazione, compresa tra 100 e $10^5$ operaie.

        Il colore rappresenta la quantità $\alpha w - L$, che è una misura di quanto la condizione~\eqref{eq:cond6b}
        sia soddisfatta o meno nella simulazione corrente. (Cfr.~p.~\pageref{par:interpretationCond6b} per una interpretazione quantitativa.)
    }
    \label{img:param3D}
\end{figure}

\paragraph{}
Il parametro di mortalità $m \in \left[ m_-, m_+ \right]$ è stato esplorato nell'intervallo tra $m_- = 0,016$ ed
$m_+ = 0,7$, scegliendo 16 punti equispaziati. Tale scelta produce i piani verticali nella figura~\ref{img:param3D}.

Il parametro $w \in \left[ w_-, w_+ \right]$ regola la risposta funzionale nella schiusa $E$, ed è
stato scelto in modo pseudorandom uniforme nell'intervallo tra $w_- = 3000$ e $w_+ = 30000$.

\paragraph{}
Le condizioni iniziali sono state riassunte in un solo parametro $b = H_0 +F_0$, anch'esso scelto
in modo pseudorandom uniforme nell'intervallo $\left[ b_-, b_+ \right] = \left[50, 9000\right]$.
Si noti che tale raggruppamento è possibile solo per i compartimenti ``analoghi''.

La soglia di popolazione $b_{thres} = 20$ sotto la quale una colonia si considera ``irreversibilmente perduta''
è utilizzata nell'esperimento della sottosezione~\ref{sec:fdod}.

\paragraph{}
Il tempo di simulazione è $T=4~\text{anni}$ ed il passo d'integrazione $\Delta t = 1~\text{giorno}$.

Nello spazio dei parametri rappresentato in figura~\ref{img:param3D}, ogni punto di coordinate $(m,w,b)$
è la simulazione di una colonia con parametri $m,w$ e condizioni iniziali $b=H_0+F_0$.

Nella proiezione sul piano $mb$ (figura~\ref{img:param2D}) è evidente la scelta equispaziata del parametro $m$ per
le diverse simulazioni.
\begin{figure}[hb]
    \centering
    \includegraphics[keepaspectratio,width=\textwidth]{img/k11EA2-parameterSpace2D}

    \caption[Esperimento A2, proiezione dello spazio dei parametri.]{Il diagramma a dispersione della
        figura~\ref{img:param3D} proiettato sul piano $mb$.}
    \label{img:param2D}
\end{figure}

\paragraph{}
Nelle sottosezioni che seguono sono esaminati i dati risultanti dalle simulazioni appena descritte.

\subsection{Rapidità di estinzione e mortalità}
\label{sec:fdod}
Per valutare la rapidità e la gravità con cui una colonia si estingue in presenza di un alto tasso di
mortalità nel comparto bottinatrici, definiamo il \emph{FDOD (First Day Of (colony) Death)} come
$$\text{FDOD} \coloneq \min \left\{ t \geq 0 \st H(t) + F(t) < b_{thres} \right\} \; .$$

Il risultato di questa analisi è illustrato in Figura~\ref{img:kh11expA21}: ogni punto blu di coordinate $(x,y)$
rappresenta una simulazione del modello con mortalità $m=x$: la corrispondente popolazione
della colonia è scesa sotto la soglia $b_{thres}$ al giorno $y$ per la prima volta.
\begin{figure}[hb]
    \centering
    \includegraphics[keepaspectratio,width=0.98\textwidth]{img/k11EA2-fdodVSm}

    \caption[Esperimento A2, \emph{FDOD} vs. mortalità.]{\emph{FDOD} degli alveari che muoiono entro i due anni,
        contro il tasso di mortalità. Il fit lineare (in rosso) riguarda soltanto i \emph{datapoint}
        delle colonie che \emph{effettivamente muoiono} entro i due anni (punti blu).}

    \label{img:kh11expA21}
\end{figure}

\paragraph{}
Si può osservare che se il tasso di mortalità è contenuto ($m< \hat{m}$ con $\hat{m} \approx 0.4$),
la colonia sopravvive con successo.

Quando il tasso di mortalità è eccessivo, le famiglie di \emph{Apis} muoiono ad un ritmo proporzionalmente accelerato, ed
il tempo di sopravvivenza dell'alveare si riduce in funzione della mortalità (che nel modello innesca un reclutamento
precoce delle operaie di nido verso il compartimento delle bottinatrici), e neanche le famiglie (inizialmente) più numerose
riescono a superare una o due stagioni di raccolta.


\subsection{Popolazione iniziale e finale}
In figura~\ref{img:kh11expA22} è mostrato su scala logaritmica il risultato delle simulazioni, con la popolazione
iniziale in ascissa le la popolazione a $t=T$ in ordinata.

Ogni punto di coordinate $(x,y)$ rappresenta una soluzione del modello\footcite{khoury2011} con popolazione iniziale
$N(0) = x$ che all'istante finale $T=2~\text{anni}$ presenta una popolazione totale $N(T) = y$.

\begin{figure}[hbt]
% TODO: colora in base a $m$. % TODO
    \centering
    \includegraphics[keepaspectratio,width=0.98\textwidth]{img/k11EA2-fpopVSipop}
    % TODO aggiustare! togliere quelle morte diocan ci sono le frazioni di miliardesimi.

    \caption[Esperimento A2, popolazione finale vs. Popolazione iniziale.]{Esperimento A2, popolazione finale vs.
    Popolazione iniziale. In blu le simulazioni per cui la condizione~\eqref{eq:cond6b} (p.~\pageref{eq:cond6b}) è vera,
    rosso le simulazioni per cui è falsa.}

    \label{img:kh11expA22}
\end{figure}

Il modello considerato è abbastanza predittivo ed in linea con le osservazioni sugli apiari in campo, purché si
utilizzino misure accurate o quantomeno stime ragionevoli dei parametri da impiegare.
I \emph{datapoint} in rosso indicano le simulazioni per cui la condizione~\eqref{eq:cond6b} è falsa: si
osservi che la popolazione corrispondentemente diverge irragionevolmente, anche per condizioni iniziali
relativamente sfavorevoli (\ie basse numerosità iniziali).
\footnote{Si veda il paragrafo dedicato all'interpretazione della condizione~\eqref{eq:cond6b}
a p.~\pageref{par:interpretationCond6b}.}

\paragraph{}
Come confermano le apicoltrici e gli apicoltori\footcite{privFDL,privFPan,meccanica},
la ``forza'' di un alveare (genericamente intesa come la numerosità della popolazione)
sostiene meglio e più a lungo la famiglia attraverso gli stress ambientali;
ma alveare -- seppur molto popoloso -- che incontri continuativamente avversità troppo intense (ad es. attacchi da
molteplici patogeni) necessita di un intervento umano per non soccombere.

\subsection{Popolazione finale e w}
Per studiare la relazione tra la ``efficienza'' di una colonia nella combinazione delle attività foretiche e di
cura della covata (ingresso di nuovi individui nel comparto delle ``nascite'') si illustra il risultato delle
simulazioni nel grafico in figura~\ref{img:kh11expA23}, che riporta ascissa il parametro $w$ ed in ordinata
la popolazione totale $N(t)$ al termine della simulazione ($t=T$).
\begin{figure}[hbt]
    \centering
    \includegraphics[keepaspectratio,width=0.98\textwidth]{img/k11EA2-finpopVSw}
    % TODO filtrare anche qui la roba sottozero diocan

    \caption[Esperimento A2, popolazione finale contro w.]{Esperimento A2, popolazione finale contro $w$.
    In rosso il fit quadratico, ed in verde i \emph{datapoint} per cui la condizione~\eqref{eq:cond6b} non
    è verificata.}
    \label{img:kh11expA23}
\end{figure}

Il parametro $w$ compare nell'equazione~\eqref{eq:eclos} e nella risposta funzionale di Holling tipo II
\footnote{Cfr. sezioni~\ref{sec:rispFunz} e~\ref{sec:eclos}}
modula la schiusa delle api adulte con il tasso di ovideposizione $L$ della regina; come
affermato dagli stessi autori del modello\footcite[2]{khoury2011}, senza informazioni aggiuntive è ragionevole
proporre che il tasso di schiusa delle uova deposte 21 giorni prima approcci $L$ dal basso per $N(t) \to \infty$,
in modo liscio.

\paragraph{}
Come già discusso nelle sezioni~\ref{sec:rispFunz} e~\ref{sec:eclos}, il parametro $w$ è l'inverso del tasso di
attacco nella risposta funzionale della schiusa al numero di operaie (di casa e di campo) disponibili nell'alveare.
Nella figura~\ref{img:kh11expA23} si può osservare che il parametro $w$ impatta in ogni caso il destino di un
alveare simulato, e che per alti valori di $w$ (\ie famiglie ``inefficienti'' nella cura della prole) gli altri
fattori ambientali (qui incorporati nelle condizioni iniziali) influiscono progressivamente di più sulla
numerosità della popolazione finale.

\subsection{Soglie di mortalità}
Il risultato fondamentale che accompagna il modello di~\citeauthor{khoury2011}, precisato e dimostrato nella
Proposizione~\ref{teo:esistenzPosF} (p.~\pageref{teo:esistenzPosF}), fornisce le condizioni affinché esista un
equilibrio positivo. Nella pratica, mantenere i parametri calcolati a partire dalle misure dirette sul campo
entro i limiti ivi stabiliti, dovrebbe consentire di garantire la sopravvivenza di una colonia in apiario.

Nella figura~\ref{img:kh11expA24} è mostrato il grafico delle soglie di mortalità $m_i^*$ stabilite nella Proposizione.
\begin{figure}[hbt]
    \centering
    \includegraphics[keepaspectratio,width=0.98\textwidth]{img/k11EA2-mistarVSw}

    \caption[Esperimento A2, soglie di mortalità]{Previsione del modello per le soglie di mortalità $m_i^*$, $i=1,2,3$.}

    \label{img:kh11expA24}
\end{figure}

La linea tratteggiata delimita la regione del parametro $w$ per cui la condizione~\eqref{eq:cond6b} è vera (a destra),
in cui la popolazione di api si stabilizza ad un equilibrio positivo per mortalità $m$ superiori ad $m_1^*$ (in blu)
ed inferiori ad $m_2^*$ (in rosso). Si noti che quando la~\eqref{eq:cond6b} diventa falsa, la soglia $m_2^*$ funge
da limite \emph{inferiore} alla mortalità, prevenendo la divergenza della popolazione complessiva.

\paragraph{}
In verde è mostrata nel grafico la soglia $m_3^*$, che di nuovo funge da limite inferiore alla mortalità.

Questa conferma sperimentale ci consente di assimilare definitivamente i casi 2 e 3 della
Proposizione~\ref{teo:esistenzPosF}, e di considerare solamente il caso 1 come regione di parametri ammissibile
per questo modello -- e successivi, più raffinati -- per descrivere efficacemente le reali colonie di \emph{A.~mellifera},
sia allo stato selvatico che in allevamento.
