\documentclass[12pt,a4paper,oneside,hidelinks]{book} % opz. draft per vedere dove sfora i margini
\usepackage[utf8]{inputenc}
\usepackage[T1]{fontenc}
\usepackage[english,italian]{babel}
\usepackage{csquotes} % rivorrei << e >>
\usepackage{amsmath}
\usepackage{amsfonts}
\usepackage{amssymb}
\usepackage{amsthm}
\usepackage{mathrsfs}
\usepackage{mdframed}
% \usepackage{dsfont} % per l'operatore identità (unità) (v. \ident)
\usepackage{makeidx}
\usepackage{graphicx} % figure
\usepackage{booktabs} % filetti per le tabelle: [top|mid|bottom]rule
\usepackage{cancel}

\usepackage{caption}
\captionsetup{tableposition=top,figureposition=bottom,font=small}

\usepackage{subcaption} % spezzare figure in sottofigure
\usepackage{esdiff} % comandi per le derivate: \diff, \diffp, c'è anche la valutazione in un punto!
\usepackage{mathtools}  % norma, modulo, := bellino, etc.
\usepackage[left=2.6cm,right=2.6cm,top=4cm,bottom=4cm,heightrounded]{geometry}
\usepackage{microtype}
\usepackage{mdframed} % per bordare gli statement, e.g. nei teoremi, definizioni, etc.

\usepackage[bibstyle=alphabetic,citestyle=alphabetic,backend=biber,backref]{biblatex}
\bibliography{api}

% \usepackage[bookmarks,hidelinks]{hyperref} % va caricato per ultimo
\usepackage[]{bookmark} % carica da solo hyperref

% contributor: from PQC2015
% AMBIENTI PER TEOREMI, DEFINIZIONI ET CETERA (DA AMSTHM)
\theoremstyle{plain}
\newtheorem{teorema}{Teorema}
\newtheorem{lemma}{Lemma}
\newtheorem{corollario}{Corollario}
\newtheorem*{proposizione}{Proposizione}
\newmdtheoremenv{postulato}{Postulato}

\theoremstyle{definition}
\newtheorem{definizione}{Definizione}
\newtheorem{esempio}{Esempio}

\theoremstyle{remark}
\newtheorem{osservazione}{Osservazione}


% ]----------------------------------------------------------------[ %


% COMANDI e RINNOVI per la tesis.

\newcommand{\pcite}[1]{(\cite{#1})}

\newcommand{\omissis}{[\textellipsis\unkern] }                          % produce "[...] "

\DeclarePairedDelimiter\abs{\lvert}{\rvert}                             % modulo
\DeclarePairedDelimiter\norm{\lVert}{\rVert}                            % norma
% Swap the definition of \abs* and \norm*, so that \abs
% and \norm resizes the size of the brackets, and the
% starred version does not.
\makeatletter
    \let\oldabs\abs
    \def\abs{\@ifstar{\oldabs}{\oldabs*}}

    \let\oldnorm\norm
    \def\norm{\@ifstar{\oldnorm}{\oldnorm*}}
\makeatother


\newcommand{\N}{\mathbb{N}}                                             % numeri naturali
\newcommand{\Z}{\mathbb{Z}}                                             % interi
\newcommand{\R}{\mathbb{R}}                                             % campo reali
\newcommand{\Rplus}{\R_{\geq 0}}                                        % reali non negativi
\newcommand{\C}{\mathbb{C}}                                             % campo complessi
\newcommand{\K}{\mathbb{K}}                                             % campo generico
\newcommand{\F}{\mathbb{F}}                                             % campo finito; metti a pedice p^q
\renewcommand{\P}[1]{\mathbb{P} \left( #1 \right) }                     % spazio proiettivo (su qualcosa!)

\newcommand{\prob}{\mathds{P}}                                          % probabilità
\newcommand{\E}{\mathds{E}}                                             % valore atteso

\DeclareMathOperator{\Tr}{Tr}                                       % traccia
\DeclareMathOperator{\Sp}{Sp}                                       % spettro
\DeclareMathOperator{\Imm}{Imm}                                     % immagine di una applicazione
\DeclareMathOperator{\Span}{Span}                                   % chiusura lineare


%-----------------------------------------------------------%

\newcommand{\conj}[1]{\overline{#1}}

\newcommand{\eg}{\emph{e.g.~}}                                      % per esempio
\newcommand{\ie}{\emph{i.e.~}}                                      % id est
\newcommand{\st}{\,\mid\;}                                          % tali che, da usare in mathmode all'interno di \Set




% ]-------------- VARIABILI -----------------[ %

\renewcommand{\epsilon}{\varepsilon}
\renewcommand{\theta}{\vartheta}
\renewcommand{\phi}{\varphi}
% \renewcommand{\rho}{\varrho}


% ]-------------- AMBIENTI -----------------[ %

\newenvironment{sistema}%
{\left\lbrace\begin{array}{@{}l@{}}}%
{\end{array}\right.}
 % COMANDI E AMBIENTI

% TODO rivedere!
\author{De Leo, Martino}
\title{Dinamica delle popolazioni di \emph{Apis mellifera} }

\begin{document}
\maketitle % TODO togliere p. via del frontespizio

\frontmatter

\section*{Sommario}
\input{lorem}

\section*{Notazione}
Da inserire certamente: tutti i nuovi comandi definiti

\cleardoublepage

\tableofcontents

\listoffigures
\listoftables


% <MAIN>
\mainmatter

\clearpage
\part{Dinamica delle popolazioni}

\chapter{Biologia dell'ape europea}
\label{chap:bio}
Questo studio si concentra sulla specie \emph{Apis mellifera} (Linneo, 1758) del genere \emph{Apis},
comunemente nota come ``ape europea'' o ``ape occidentale''.
Questa specie ha avuto origine in Africa ed in Medio Oriente, diffondendosi poi su ogni continente (esclusa l'Antartide) durante gli scorsi 2 milioni di anni.

Il rapporto tra \emph{H. sapiens} e \emph{A. mellifera} ha origine nel Pleistocene con le prime rudimentali interazioni
di tipo puramente predatorio (figura~\ref{img:arana}), evolvendosi nell'arco dei millenni in forme complesse e reciprocanti.

\begin{displayquote}[\cite{honeyreligion}]
``Humans have eaten, and fixed their wounds with honey and traded with it since history has been recorded.''
\end{displayquote}

L'interazione simbiotica uomo--ape è oggi articolata in forme complesse, che comprendono:
\begin{itemize}
    \item La selezione genetica: l'uomo tenta di rinforzare i tratti comportamentali vantaggiosi che incidono sugli aspetti sociali delle colonie, \footcite{behavGenetics}
    di aumentarne l'efficienza nella produzione di miele, cera, pappa reale, \footcite{algerianHoney}
    oppure di migliorarne la resistenza alle malattie.

    Il genoma di \emph{A. mellifera} è stato interamente sequenziato nel 2006. \footcite{genomeSeq}
    \item La dispersione antropocora: l'espansione di \emph{A. mellifera} nell'Eurasia continentale è avvenuta in almeno due momenti distinti, ed è molto probabile che gli eventi di diffusione più recenti siano stati catalizzati dall'uomo.
    \footcite{antropocora}
    \footnote{Vedasi la figura~\ref{img:europ}.}

        L'introduzione di svariate sottospecie di \emph{A. mellifera} in America Latina (XVI sec.), in America del Nord (XVII sec.) ed in Australia (XIX sec.) è interamente dovuta all'opera umana.
    \item La dimensione economica: il valore di mercato ``diretto'' dei prodotti dell'apicoltura è stimato intorno agli 8 miliardi di \$ annui. \footcite{honeymarket1}

        Il valore ``indiretto'' dovuto all'impollinazione delle colture nel settore agricolo è di gran lunga superiore, stimato tra i 235 e i 570 miliardi di \$ all'anno. \footcite{honeymarket2,honeymarket3}
    \item La dimensione ecologica: la presenza del genere \emph{Apis} contribuisce -- assieme ad altri impollinatori -- al mantenimento della biodiversità negli ecosistemi non disturbati dall'attività umana, alla stabilizzazione degli ambienti antropizzati; il declino degli impollinatori aggrava gli effetti negativi del cambiamento climatico e dell'industrializzazione dell'agricoltura. \footcite{decline}
\end{itemize}

\begin{figure}
    \centering
    \includegraphics[keepaspectratio,width=0.98\textwidth]{img/tesiFDL-distribuzioneRazzeEuropaClean}

    \caption[Distribuzione delle razze d'api in Europa]{``Distribuzione
    delle razze d'api in Europa'', da \cite[24]{tesiFDL}.}
    \label{img:europ}
\end{figure}

L'ape occidentale presenta una organizzazione \emph{eusociale}, ossia la più elevata forma di organizzazione sociale realizzata da una popolazione animale. L'eusocialità è definita dalle seguenti caratteristiche:
\begin{itemize}
    \item La cura della covata è una attività \emph{cooperativa}; gli elementi della colonia dediti a tale attività si occupano anche della progenie degli altri individui.
    \item Sovrapposizione tra le generazioni di individui adulti in una colonia completamente sviluppata.
    \item Suddivisione del lavoro che include la separazione tra i compiti riproduttivi e non riproduttivi.
\end{itemize}

L'organizzazione eusociale di \emph{A.~mellifera} ci permette di concettualizzare l'intero alveare come un macro--organismo, le cui capacità trofiche, riproduttive e di interazione con l'ambiente sono il risultato emergente dalla collettività dei comportamenti degli individui che costituiscono la colonia.

La suddivisione del lavoro in un alveare riflette la partizione in caste fertili (regina, fuchi) e non fertili (operaie) della popolazione.

\begin{figure}[hbp]
    \centering
    \includegraphics[keepaspectratio,width=0.76\textwidth]{img/uovaLarve}

    \caption[Covata giovane.]{Porzione di favo con covata: le celle in alto a sinistra contengono larve di 5--8 giorni; nelle altre celle è presente un uovo (< 3 giorni). \\ (Waugsberg -- 22 Aprile 2007 -- CC BY-SA 3.0)}
    \label{img:uovaLarve}
\end{figure}

L'allevamento della covata (figure~\ref{img:uovaLarve},~\ref{img:lifecycle}) avviene in modo collettivo: la regina depone le uova fecondate, mentre le operaie nutrici si occupano della pulizia, nutrimento e cura delle uova e delle larve.

Al termine dello stadio larvale le operaie nutrici chiudono la cella con un opercolo di cera; all'interno della cella la pupa svolge una metamorfosi da cui emerge dopo 10--11 giorni un'ape adulta.

\begin{figure}[hbp]
    \centering
    \includegraphics[keepaspectratio,width=0.76\textwidth]{img/honeybeeLifecycle}

    \caption[Ciclo vitale dell'ape.]{Schema della riproduzione di \emph{A. mellifera}.
        \\ (illustrazione di Jennifer Sartell (dett.) -- 22 Aprile 2017 -- ©)}
    \label{img:lifecycle}
\end{figure}

\paragraph{}
La popolazione di un alveare varia notevolmente col ciclo delle stagioni e mediamente comprende un'ape regina (unica femmina fertile), da 20000 a 100000 operaie (femmine sterili) e qualche centinaia di fuchi (maschi fertili) in primavera/estate.

Le tre caste presentano notevoli differenze morfologiche, che riflettono le mansioni di ognuna all'interno dell'alveare.
Ad esempio nei fuchi e nella regina le ghiandole ceripare e faringee sono decisamente meno sviluppate che nelle operaie; anche il loro apparato boccale è ridotto e per questo motivo sia la regina che i maschi vengono alimentati dalle operaie nutrici.

\paragraph{}
È importante precisare che un'ape operaia svolge ruoli differenti nella colonia durante l'arco della sua vita.

Una giovane adulta rimane perlopiù all'interno del nido:
\begin{itemize}
    \item come \emph{nutrice} curando la covata, la regina e i fuchi,
    \item come \emph{ceraiola} modellando la cera secreta da speciali ghiandole per costruire il favo,
    \item come \emph{spazzina} che pulisce il nido e cura l'igiene della matrice cerea,
    \item come \emph{guardiana} difendendo l'ingresso da animali estranei.
\end{itemize}

Un'altra funzione importante per il mantenimento della covata è la termoregolazione dell'alveare (figura~\ref{img:fanning}):
le operaie di nido contrastano il freddo formando il \emph{glomere}, ed il caldo eccessivo sfruttando il calore latente di evaporazione dell'acqua.

\begin{figure}[hbp]
    \centering
    \includegraphics[keepaspectratio,width=0.76\textwidth]{img/fanning}

    \caption[Ventilazione forzata per raffrescare l'interno dell'arnia.]{Gruppo di operaie in
        prossimità dell'ingresso dell'alveare ``ventilano'' per abbassare la
        temperatura interna dell'arnia.
        \\ (Ken Thomas -- 26 Maggio 2008 -- Pubblico dominio)
    }
    \label{img:fanning}
\end{figure}

\paragraph{}
Soltanto in età avanzata, con un bagaglio di esperienze che informa le abilità cognitive avanzate di cui è capace, l'ape operaia si avventura fuori dal nido per esplorare, individuare e discriminare le fonti di cibo, raccogliere polline e nettare (``bottinare'') dalle fioriture, importare acqua nell'arnia e scambiare informazioni con le compagne.

Le capacità cognitive per svolgere tali compiti richiedono la coordinazione di svariate abilità, tra cui la navigazione, il conteggio, la misura di distanze ed angoli.
È dimostrato che le bottinatrici più anziane sono più efficienti nelle attività all'esterno del nido, rispetto alle operaie reclutate in età precoce che sono più lente e imprecise nel volo.

\paragraph{}
Dunque le operaie giovani svolgono ruoli fondamentali all'interno dell'alveare come la cura della covata e dell'igiene, e attraverso il reclutamento diventano abili bottinatrici in volo in età più avanzata.
È importante notare la crucialità di entrambe le dinamiche, perché riguardano due aspetti fondamentali di ogni organismo vivente: la \emph{riproduzione} (la cura della covata) e l'approvvigionamento di cibo.

La colonia deve costantemente perseguire un delicato equilibrio nell'allocazione di risorse per soddisfare questi due bisogni fondamentali.
Ne segue che ogni modello matematico che voglia risultare sufficientemente accurato nel riassumere la dinamica di una colonia di \emph{A. mellifera} debba comprendere un effetto di tipo Allee perlomeno sulla popolazione di operaie giovani
\emph{ed anche} sulla popolazione di operaie bottinatrici più anziane.

Nella tabella~\ref{tab:FDLruoliEta} sono illustrati alcuni ruoli svolti dalle api operaie ed il loro periodo di insorgenza.
Per una disamina più dettagliata dei meccanismi di transizione tra le caste di operaie in \emph{A.~mellifera},
si veda \cite{meccanica}.

\begin{table}[pbh]
    \centering
    \begin{tabular}{rcc}
        \toprule
        funzione & durata (gg) & età (gg) \\
        \midrule
        riposo e inizio lavori interni & 3 & 0--3 \\
        pulizia dell'arnia e dei favi  & 3 & 4--6 \\
        nutrice                        & 7 & 7--13 \\
        secrezione cera, guardiania, ventilazione & 6 & 14--19 \\
        bottinatrice                            & 20--25 & 20--45 \\
        \bottomrule
    \end{tabular}
    \caption{Demografia media per un alveare sano. Dati~\cite{balio}.}
    \label{tab:FDLruoliEta}
\end{table}




\section{Organismi patogeni per l'ape}
L'ape europea presenta una varietà di antagonisti naturali, appartenenti a diversi regni.

Nell'ambito dei virus troviamo il ``virus della paralisi''\footnote{Acute Bee Paralysis Virus -- ABPV.}, il ``virus della regina nera'' (Black Queen Cell Virus) ed altri (Israeli Acute Paralysis Virus, Kashmir Bee Virus, Cloudy Wing Virus, \dots) nella famiglia dei \emph{Dicistroviridae}.

\paragraph{}
Il nosema (\emph{Nosema ceranae)} è un microsporidio (regno \emph{Fungi}) che nasce come parassita di \emph{Apis cerana} (la specie asiatica di ape mellifera) ma che attacca anche \emph{A. mellifera}.

Le api sono predate abitualmente da alcune specie di uccelli (ad es. il gruccione, \emph{Merops apiaster}) e da alcuni insetti (ad es. il calabrone, v.~figura~\ref{img:calabron}).

\begin{figure}[hbt]
    \centering
    \includegraphics[keepaspectratio,width=0.76\textwidth]{img/calabron}

    \caption[Ape predata da un calabrone.]{Una calabrone operaia (\emph{Vespa Crabro}) si
        nutre di una ape operaia (\emph{Apis mellifera}) appena catturata.
        \\ (Böhringer Friedrich -- 20 Agosto 2004 -- CC BY-SA 2.5)
    }
    \label{img:calabron}
\end{figure}

Alcune specie non parassitano direttamente le api ma altre componenti dell'alveare, come la falena \emph{Galleria mellonella} che si nutre degli esoscheletri delle pupe di ape sfarfallate, di cera e di tracce di polline che trova nelle celle abbandonate dalle api adulte.

\subsection{\emph{Varroa}}
Il patogeno di gran lunga più studiato è la varroa, acaro appartenente alle due specie \emph{Varroa destructor} e \emph{V.~jacobsoni}. La varroa parassita le larve, le pupe e le api adulte, nutrendosi delle componenti grasse e dell'emolinfa.

\paragraph{}
Il ciclo vitale della varroa è suddiviso nella fase riproduttiva e nella fase foretica:
\begin{itemize}
    \item Durante la fase riproduttiva, una femmina feconda di varroa si introduce in una cella contenente una larva, prima dell'opercolazione. Dopo l'opercolazione depone le uova da cui fuoriescono gli individui fertili, che si accoppiano immediatamente e si nutrono sulla larva.

    Con lo sfarfallamento dell'ape adulta, la varroa madre e le giovani femmine fecondate fuoriescono dalla cella, eventualmente attaccandosi al corpo dell'ape.
    I maschi, terminata la loro funzione riproduttiva, muoiono dentro la cella.
    \item La fase foretica avviene nelle varroe adulte che si attaccano sulle api operaie, nascondendosi negli spazi interstiziali tra i segmenti dell'addome.
\end{itemize}

La varroa accorcia l'aspettativa di vita e indebolisce le api adulte che attacca; determina inoltre malformazioni (ad es. alle ali) nelle operaie che vengono parassitate allo stadio larvale, compromettendone le abilità al volo.

È dimostrato inoltre che la varroa funge da vettore per molti patogeni delle api, sia batterici che virali, tra cui l'ABPV.

\paragraph{}
Il sistema immunitario di una colonia, che comprende le barriere fisiologiche a livello di individuo ed i comportamenti sociali di tipo igienico, viene compromesso dal parassitaggio di \emph{V.~destructor} e dalle patologie che essa trasmette.

Questo meccanismo di indebolimento di un sistema di difesa della colonia aggiunge un ulteriore livello di complessità ai processi di interazione tra api e varroe.

\paragraph{}
Una assunzione ubiquitaria nei modelli matematici apistici che considerano l'infestazione da \emph{V.~destructor} è
che le attività sostanziali della regina all'interno della colonia non sono disturbate dalla presenza di varroa e dei patogeni ad essa correlati.

Ciò corrisponde all'esperienza sul campo\footcite{privFDL,privFPan} ed alla letteratura disponibile, secondo le quali l'acaro non si trova \emph{praticamente mai} sul corpo della regina. Non è chiaro  se ciò sia dovuto alla particolare attenzione con cui le operaie attorno alla regina svolgono attività di \emph{grooming}.

\subsection{Declino dovuto all'attività umana}
Molte classi di pesticidi, erbicidi ed altri antiparassitari di sintesi impiegati in agricoltura danneggiano le api: le bottinatrici entrano in contatto con queste sostanze chimiche posandosi sulle colture trattate, importano nel nido polline e nettare contaminati, dove vengono manipolati anche dalle operaie di alveare.

Le molecole \emph{imidacloprid}, \emph{clothianidin}, e \emph{thiamethoxam} (nella classe dei neonicotinoidi) sono state correlate alla sindrome di spopolamento (Colony Collapse Disorder -- CCD) in cui si verifica un crollo repentino della popolazione in alveari apparentemente sani, che rende impossibile la sopravvivenza della famiglia.
Queste molecole vengono considerate anche causa o concausa di patologie cognitive delle api adulte, tra cui una riduzione della memoria a breve termine ed il deterioramento delle facoltà di navigazione: le bottinatrici affette perdono l'orientamento e non riescono a tornare al nido, morendo in breve tempo (per il sopraggiungere della notte o di un predatore).

\paragraph{}
Gli studi più recenti ridimensionano il ruolo dei pesticidi chimici nel CCD ed indicano piuttosto \emph{la combinazione di una molteplicità di fattori} come causa scatenante del declino delle colonie di ape europea.

Tali fattori includono l'uso di pesticidi ed erbicidi sulle colture, la perdita di biodiversità, il cambiamento climatico, l'alterazione delle fioriture (disponibilità di cibo per la colonia) e la proliferazione di patologie, soprattutto legate a \emph{V.~destructor}.

Questa combinazione di concause produce effetti ramificati sul comportamento degli alveari, che a loro volta influenzano il processo di declino della popolazione di api.

\paragraph{}
Ad esempio, sia \emph{imidacloprid} che il parassitaggio da \emph{V.~destructor} inibiscono la memoria spaziale e le abilità di volo delle bottinatrici esperte, che ``si perdono'' più spesso e non riescono a tornare all'alveare (\emph{homing failure}), circostanza quasi sempre fatale. Ciò determina un deficit di operaie anziane rispetto all'equilibrio della popolazione, a cui la colonia reagisce innescando meccanismi sociali che accelerano il processo di reclutamento, per cui le operaie vengono ``promosse'' all'esterno del nido in età precoce.

Le bottinatrici più giovani sono quindi più inesperte e meno efficienti nel bottinare, l'importazione di cibo nella colonia diminuisce; inoltre si perdono più spesso, aumentando così il ``costo unitario'' che la colonia sostiene per una unità di miele importato.

Il reclutamento precoce causa anche una diminuzione del numero di operaie di alveare e più precisamente di nutrici:
con meno cure, la frazione di covata che raggiunge l'età adulta diminuisce per l'aumento di morti precoci allo stadio larvale.

\paragraph{}
Il meccanismo di regolazione sociale del reclutamento è un comportamento naturale di \emph{A.~mellifera}
di risposta allo stress su due elementi vitali principali della colonia -- la \emph{riproduzione} e l'\emph{alimentazione} -- ed in condizioni tipiche garantisce la sopravvivenza della stessa.

Nello scenario d'esempio qui sopra, le attività dannose dell'uomo e di altre specie si intersecano ai molti piani delle attività biologiche di una colonia o di un apiario, i cui parametri vitali sembrano abbastanza stabili nel tempo, ma che improvvisamente collassano per l'occorrenza contemporanea di danni irreversibili a \emph{molti} processi fondamentali che sostengono la vita eusociale dell'ape.



\chapter[Strumenti per i modelli biologici]{Strumenti per lo studio dei modelli di tipo biologico}
\label{sec:ingredientiBioMat}

Esiste una vasta gamma di fenomeni biologici analizzabili matematicamente, come la dinamica delle popolazioni
o l'epidemiologia, e all'aumentare del numero di processi interconnessi presi in considerazione cresce
invariabilmente la complessità matematica dei modelli da analizzare.

In questo capitolo sono riassunti gli strumenti matematici che si incontrano più spesso nella letteratura
specifica per quanto riguarda la dinamica di popolazioni di \emph{Apis}, ed in particolare le sue interazioni
con i patogeni ed altri fattori di stress a cui possono essere esposte le colonie, quali i pesticidi di
origine antropica o la scarsa disponibilità di cibo.

\paragraph{}
I modelli matematici sono dati generalmente sotto forma di sistemi dinamici, \ie di sistemi di equazioni
differenziali del tipo
\begin{equation}
    \diff{}{t} \mathbf{x} = F( \mathbf{x}, t ) \; ,
    \label{eq:sistDin}
\end{equation}
data $F: W \to \R^n$ con $W \subseteq \R^{n+1}$ aperto ed $F$ supposta abbastanza regolare.%
\footnote{Ovvero $F$ localmente lipschitziana rispetto a $\mathbf{x}$, uniformemente rispetto a $t$.
(Cfr.~appendice~\ref{chap:teoria})}

Le equazioni differenziali date sopra, unitamente con le condizioni iniziali $\mathbf{x} (t_0) = \mathbf{x}_0$,
costituiscono un \emph{problema di Cauchy}, ed il Teorema di Esistenza e Unicità garantisce che le sue soluzioni
esistano per ogni condizione iniziale $\mathbf{x}_0$, definite in almeno un intervallo.
\footnote{Cfr.~p.~\pageref{eq:sdcGenerale} in appendice.}

\paragraph{}
Risultati generali sui sistemi dinamici sono ampiamente disponibili in letteratura e ne diamo un riassunto
nelle appendici~\ref{chap:teoria} e~\ref{chap:floquetTheory}.

In questa sezione ci soffermeremo piuttosto su alcuni strumenti matematici che trovano applicazione nel campo
più specifico della biomatematica, ossia nella modellazione di popolazioni delle specie animali, vegetali
o microbiche.

\paragraph{}
Intorno al XIX secolo le scienze matematiche iniziano a rivolgersi al campo biologico con rinnovato interesse:
Pierre--François Verhulst propone di utilizzare la funzione logistica nel 1838, dopo aver letto il famoso
modello di Malthus.\footcite{malthus1986essay}
La derivazione di Verhulst è correlata anche al modello preda--predatore, pubblicato da Alfred
Lotka nel 1920, che estende la sua applicazione precedente alle reazioni autocatalitiche.

Lo stesso sistema di equazioni viene pubblicato da Vito Volterra nel 1926
\footcite{vito},
che si avvicina all'applicazione della matematica alle scienze biologiche grazie alle interazioni col
biologo marino Umberto D'Ancona.

Negli ultimi due secoli e mezzo sono stati fatti molti progressi in campo matematico e nelle scienze biologiche;
tuttavia, gli studi più recenti lamentano ancora una compartimentazione troppo netta ed auspicano una maggiore
collaborazione tra le scienze matematiche e la ricerca bio-etologica sul campo.

\paragraph{}
Presentiamo in questa sezione alcuni utili strumenti per i sistemi attinenti alla ricerca biomatematica.
Vedremo alcuni oggetti definiti all'epoca di Verhulst, ed argomentazioni più moderne che attorno al principio
del terzo millennio hanno iniziato a combinare lo sguardo matematico con la simulazione numerica,
sfruttando la forza delle macchine di calcolo.

% sections
\section{Funzione Logistica}
\label{sec:logistic}
La \emph{funzione logistica} è una famiglia di sigmoidi che
\blockquote[\cite{WENlogistic}]{``\omissis trova applicazione in molti campi,
compresi: biologia (specialmente ecologia), biomatematica, chimica, demografia, economia,
geoscienze, psicomatematica, probabilità, sociologia, scienze politiche, linguistica,
statistica e reti neurali.

Ne esistono varie generalizzazioni, a seconda del campo di applicazione.''}

\paragraph{}
Consideriamo dapprima il modello malthusiano\footcite{malthus1986essay} in cui il tasso di crescita di una popolazione $P$
è proporzionale alla popolazione medesima:
$$\diff{P}{t} = rP \; , $$
ove il tasso di riproduzione $r$ è costante. L'equazione differenziale è facilmente integrabile e fornisce la soluzione
$P(t) = P_0 e^{rt}$, in cui $P_0$ rappresenta la popolazione iniziale all'istante $t=0$.

L'assunzione di Malthus è verosimile soltanto nelle fasi iniziali dello sviluppo, ossia
per popolazioni ``relativamente basse'': ad esempio un embrione umano cresce secondo la sequenza
2, 4, 8, 16 cellule etc. subito dopo la fecondazione, ed anche lo sviluppo delle colture batteriche appena inoculate
segue un'andamento esponenziale.

Il feto umano però rallenta il suo sviluppo quando raggiunge dimensioni comparabili con l'utero che lo contiene, così
come la crescita delle popolazioni batteriche satura in vicinanza dei limiti fisici imposti dalla piastra o dall'esaurimento
delle sostanze nutrienti.

La soluzione esponenziale non cattura questi fenomeni reali, e quindi descrive propriamente l'evoluzione di una popolazione soltanto nelle fasi iniziali.

Infatti, in una crescita di tipo esponenziale il tasso di crescita \emph{pro~capite} è costante, quindi la popolazione
aumenta tanto più velocemente quanto è grande il numero di individui.

\paragraph{}
Per risolvere questo problema, Verhulst\footcite{verhulst} introduce l'equazione logistica,
ottenuta supponendo che il tasso di crescita sia proporzionale alla popolazione\footnote{Come già nel modello malthusiano.}
\emph{ma anche} alle risorse disponibili:

\begin{equation}
    \diff{P}{t} = r P \left( 1 - \frac{P}{K} \right) \; ,
    \label{eq:verhulstLogistic}
\end{equation}
dove $K$ rappresenta la \emph{capacità portante} dell'ambiente, ossia il massimo teorico della popolazione
che è possibile sostenere in base alle risorse disponibili.
Il fattore $\frac{K-P}{K}$ nella crescita logistica rappresenta il fatto che il tasso di crescita \emph{pro~capite}
diminuisce quando numero di individui si avvicina al limite ambientale imposto dalla disponibilità di risorse.

Il parametro adimensionale $\frac{P}{K}$ si chiama \emph{competizione intraspecifica} e costituisce la modifica
cruciale di Verhulst al modello malthusiano: rappresenta infatti la diminuzione del tasso di crescita per popolazioni
vicine alla capacità portante, rispetto a popolazioni ``piccole'' (a parità di tasso di riproduzione).

Tramite semplici manipolazioni algebriche possiamo risolvere l'equazione~\eqref{eq:verhulstLogistic} ed ottenere
l'espressione $P=P(t)$ che descrive l'evoluzione temporale della popolazione:
\begin{multline*}
\int_{P_0}^{P(t)} \frac{1}{\pi} + \frac{1}{K-\pi} \de \pi = \int_0^t r \de \tau
\quad \implies \quad
\log \left( \frac{K-P_0}{P_0} \cdot \frac{P}{K-P} \right) = rt
\quad \implies \\
\implies \quad
\frac{K}{P} - 1 = \frac{K-P_0}{P_0} e^{-rt}
\quad \implies \quad
P = \frac{K P_0 e^{rt}}{K + P_0 (e^{rt} -1)} \; ,
\end{multline*}

la quale può essere riscritta come
\begin{equation}
    P(t) = \frac{K}{1 + q e^{-rt}} \; ,
    \label{eq:verhulstLogistic2}
\end{equation}
avendo posto
$$ q \coloneq \frac{K- P_0}{P_0} \; .$$

Si osservi che dalla~\eqref{eq:verhulstLogistic2} segue
$$\lim_{t \to \infty} P(t) = K \; ,$$
ovvero che la popolazione tende alla capacità portante, anziché divergere all'infinito come nel modello esponenziale.

\paragraph{}
Conviene adimensionalizzare la~\eqref{eq:verhulstLogistic2} in modo da ottenere la cosiddetta funzione logistica standard:
\begin{equation}
    f(x) \coloneq \frac{1}{1 + e^{-x}} = \frac{e^x}{1+e^x} \; ,
    \label{eq:logisticF}
\end{equation}
la quale è definita per ogni $x \in \R$. Nella pratica $f(x)$ si può considerare satura intorno a $\abs{x} > 6$,
come si può vedere in figura~\ref{img:logisticF}.

\begin{figure}[pbh]
    \centering
    \includegraphics[keepaspectratio,width=0.86\textwidth]{img/logisticF}

    \caption[Funzione logistica]{Funzione logistica standard.}
    \label{img:logisticF}
\end{figure}

La funzione logistica è dispari rispetto a $(0, \frac{1}{2})$, e vale $f(0) = \frac{1}{2}$.
Queste osservazioni si applicano anche alla logistica generalizzata, con parametri $L$, $k$ ed $x_0$:
\begin{equation}
    \hat{f}(x) \coloneq \frac{L}{1+e^{-k(x -x_0)}} \; ,
    \label{eq:logisticFgen}
\end{equation}
che è una versione traslata lungo $x$ e riscalata su entrambi gli assi:
\begin{itemize}
    \item $x_0$ adesso è la retroimmagine del semimassimo $\frac{L}{2}$;
    \item è applicata un'omotetia di fattore $\frac{1}{k}$ lungo $x$ ed $L$ lungo $y$.
\end{itemize}

\paragraph{}
Nelle applicazioni è utile conoscere le derivate della $f$ logistica, che è di classe $C^{\infty}$:
nelle figg.~\ref{img:logisticV}, \ref{img:logisticA}, \ref{img:logisticJ} si osservi che
la $f'$ è sempre positiva, e le derivate sono alternativamente pari e dispari, secondo la
facile generalizzazione di un noto risultato.\footnote{Ossia che $f \text{ pari} \implies
    f' \text{ dispari}$ ed analogamente al contrario,
producendo una catena di alternanze.}

\begin{figure}[pbh]
    \centering
    \includegraphics[keepaspectratio,width=0.86\textwidth]{img/logisticV}

    \caption{Derivata logistica prima.}
    \label{img:logisticV}
\end{figure}
\begin{figure}[pbh]
    \centering
    \includegraphics[keepaspectratio,width=0.86\textwidth]{img/logisticA}

    \caption{Derivata logistica seconda.}
    \label{img:logisticA}
\end{figure}
\begin{figure}[pbh]
    \centering
    \includegraphics[keepaspectratio,width=0.86\textwidth]{img/logisticJ}

    \caption{Derivata logistica terza.}
    \label{img:logisticJ}
\end{figure}

Il calcolo esplicito delle derivate della $f$ logistica è facilitato parecchio osservando che
$$\diff{}{x} f(x) = f(x) \left( {1-f(x)} \right) \; ,$$
da cui tutte le derivate successive possono ricavarsi algebricamente.

\paragraph{}
La proprietà di simmetria della funzione logistica
$$1 -f(x) = f(-x)$$
riflette il fatto che la crescita sopra $0$ intorno a valori piccoli di $x$ è
simmetrica rispetto al decadimento al raggiungere il valore--limite di saturazione,
per grandi valori di $x$.

\paragraph{}
Un esempio di applicazione nella dinamica delle popolazioni sarà esaminato nel modello della sezione~\ref{sec:ratti17}:
nelle equazioni \eqref{eq:r17m} e \eqref{eq:r17n} la crescita della popolazione di varroe è di tipo logistico,
e la capacità portante dell'ambiente è proporzionale alla popolazione di api adulte.

\paragraph{}
Concludiamo questa sezione osservando che altre formulazioni sono state proposte ed utilizzate per ottenere
delle sigmoidi nei modelli, ad esempio $\tanh (x)$, $\arctan (x)$, $\erf (x)$ e la funzione di Hill.\footcite{hill}


\section{Risposta funzionale}

\section{Modelli a compartimenti}
I modelli a compartimenti costituiscono una tecnica generale e molto potente per la modellazione, soprattutto nelle
applicazioni di modellazione matematica delle malattie infettive, ambito in cui furono sviluppati nella prima metà
del '900.\footcite{kmk,kdg}

Sono molto utilizzati anche nella modellazione di più popolazioni diverse: l'interazione biologica tra le diverse specie
prese in considerazione può avere carattere predatorio, parassitario, simbiotico, etc.
Questo ultimo ambito è quello di maggior interesse ai fini dei modelli esaminati in questa Tesi, il cui scopo è
la modellazione della popolazione di \emph{Apis~mellifera} tenendo conto della suddivisione in caste del nido, della
presenza di macropatogeni (come la \emph{Varroa}) e di altre infezioni batteriche o virali.

\paragraph{}
In questo tipo di modelli l'intera popolazione è suddivisa in compartimenti etichettati -- ad esempio ``S'' per gli
individui suscettibili ed ``I'' per gli individui infetti -- e le etichette sono da interpretare come variabili del sistema:
ad esempio $I$ è il numero degli individui infetti presenti nella popolazione in un dato momento.

Per specificare completamente il modello a compartimenti occorre anche fornire i tassi di transizione da e verso ogni
compartimento, i quali dipendono dalle variabili e dai parametri del sistema.

\subsection{Modello SIR e sue varianti}
Nonostante la relativa semplicità di questo modello che adotta soltanto tre compartimenti, esso ha un potere predittivo
piuttosto ragionevole\footcite{chinaVirus} nel contesto delle infezioni trasmesse infra--specie in \emph{H.~sapiens}
per le quali la guarigione comporta una resistenza di lungo periodo.\footnote{Ad esempio il morbillo, la parotite,
la rosolia e -- con le dovute cautele -- anche la sindrome respiratoria COVID-19 data dal virus SARS-CoV-2.}









\chapter{Mathematical models}
In questo capitolo discuteremo alcuni modelli proposti in letteratura per descrivere la dinamica della popolazione di una colonia di \emph{Apis Mellifera}.

[[Per una panoramica di modelli si rimanda alla review di Kang, [tabella caratteristiche], che fa anche un lavoro di omogeneizzare la notazione]]

\paragraph{}
Il primo modello preso in considerazione è \parencite{khoury2011} BLABLA

\paragraph{}
Il secondo modello che prendiamo in considerazione \parencite{ratti2017} suddivide la popolazione di operaie sane in api ``di alveare'' -- che svolgono mansioni di cura e manutenzione all'interno del nido -- e in bottinatrici.
Gli altri compartimenti del modello riguardano la popolazione di \emph{V.~destructor} e le api infette dal virus della paralisi (ABPV), trasmesso dagli acari.

\section[Modello a due compartimenti con reclutamento]{Modello a due compartimenti con reclutamento sociale}
\label{sec:kh11}

Nel \citeyear{khoury2011} è stato proposto un modello\footcite{khoury2011} per indagare le dinamiche sociali
tra le caste della popolazione operaia in \emph{Apis~mellifera}.

Nello studio si ottengono due equazioni, in cui le variabili reali e non negative $H$ ed $F$ rappresentano
il numero delle operaie di nido e delle foraggiatrici, rispettivamente:

\begin{align}
    \diff{H}{t} &= \overbrace{E(H,F)}^{\text{eclosion}}- H \cdot R(H,F) \label{eq:kh11h} \; , \\
    \diff{F}{t} &= H \cdot R(H,F)  - m \cdot F \label{eq:kh11f} \; ,
\end{align}
dove $m$ è il tasso di mortalità, ed $E$ il tasso di schiusa.
$R$ rappresenta la funzione di \emph{reclutamento} nel termine di
trasferimento $H \to F$.

Un diagramma sintetico di questo modello è illustrato in figura~\ref{img:kh11diagram}.

\begin{figure}[hbp]
    \centering
    \includegraphics[keepaspectratio,width=0.76\textwidth]{img/pone.0018491.g001}

    \caption[\figurename~1 da \parencite{khoury2011}]{Diagramma del modello che illustra le mutue interazioni
        tra i compartimenti. Si noti che la mortalità delle api di nido è trascurabile.
        \\
        { \footnotesize \ttfamily doi: 10.1371/journal.pone.0018491.g001 }
        }
    \label{img:kh11diagram}
\end{figure}


\paragraph{}
Alcuni oggetti proposti in questo modello si sono dimostrati molto interessanti e sono stati incorporati
in seguito da altri modelli, con variazioni minime;
uno di questi modelli\footcite{ratti2017} sarà esaminato nella prossima sezione a p.~\pageref{sec:ratti17}.


\subsection{Reclutamento con inibizione sociale.}
Nello studio\footcite{khoury2011} la funzione di reclutamento è rappresentata dalla funzione
\begin{equation}
    \label{eq:Recr}
    R(H,F) = \alpha - \sigma \frac{F}{H+F} \; ,
\end{equation}
in cui $\alpha$ rappresenta il tasso massimo di reclutamento.\footnote{Raggiunto quando nella colonia non è
presente nessuna bottinatrice.}

Il secondo termine rappresenta l'inibizione sociale del processo di reclutamento.
Gli autori assumono
\textquote[{\footcite[3]{khoury2011}}]{\omissis that social inhibition is directly proportional to the fraction of the total number of adult bees that are foragers, such that a high fraction of foragers in the hive results in low recruitment.}

\paragraph{}
La ricerca biologica suggerisce che l'ethyl~oleate scambiato tra le api operaie sia
il principale modulatore nel processo di reclutamento sociale.
\footcite{ethyloleate,ethyloleate2,meccanica}

Ad esempio una colonia risponderà ad una elevata mortalità di bottinatrici (più anziane) attraverso
la riduzione dell'inibizione del reclutamento (dati i livelli più bassi di feromone) che dunque risulta
\textquote[\footcite{khoury2011}]{\omissis in a precocious onset of foraging behaviour in young bees}.

Nella situazione opposta -- \ie quando vi sono molte bottinatrici e la loro mortalità è bassa -- la colonia
risponde \emph{aumentando} l'inibizione sociale: un maggior numero di bottinatrici si astiene dunque dal volo e
dalle attività all'esterno, ripiegando sui lavori di cura all'interno del nido. Questa risposta inibitoria
è probabilmente dovuta alla maggiore concentrazione di ethyl~oleate:
\textquote[{\footcite[1]{khoury2011}}]{Old forager bees transfer ethyl oleate to young hive bees via trophallaxis, which delays the age at which they begin foraging.}

\paragraph{}
Tramite questo meccanismo mediato dai segnali feromonici, l'alveare tenta di mantenere un equilibrio
demografico tra le due classi, sufficiente a soddisfare le necessità della covata e dell'importazione
di polline, nettare, acqua.

In questo scenario la risposta collettiva -- che riduce il tasso di reclutamento -- aumenta con
l'intensità degli scambi di ethyl oleate, che a loro volta sono correlati al rapporto $\frac{F}{H+F}$:
queste assunzioni motivano la presenza di tale fattore nell'equazione~\eqref{eq:Recr}.


\paragraph{Tassi di mortalità.}
Come si è notato in precedenza,\footnote{Nel capitolo~\ref{chap:bio}.}
le operaie adulte -- esploratrici e bottinatrici -- sono soggette ad una gamma più ampia di fattori di stress
nell'ambiente esterno, rispetto alle operaie di nido. Di conseguenza l'aspettativa di vita per le operaie di
nido è più lunga rispetto quella delle foraggiatrici:
\textquote[{\footcite[1]{khoury2011}}]{Survival of bees in the protected hive environment is high, but the
survival of forager bees is much lower. \omissis The average foraging life of a bee has been estimated as less
than seven days, because of the many risks and severe metabolic costs associated with foraging.}

Il termine di pozzo $-mF$ nell'equazione~\eqref{eq:kh11f} rappresenta la mortalità delle foraggiatrici;
il tasso parametrico $m$ imposta la frazione del comparto $F$ che muore ogni giorno.\footnote{Cfr. %
p.~\pageref{par:mInverseFlightspan} e tab.~\ref{tab:FDLruoliEta}.}


\paragraph{}
L'assenza di un fattore di mortalità nel comparto delle operaie di nido nell'equazione~\eqref{eq:kh11h} è
una assunzione piuttosto severa, e comporta una netta semplificazione del modello.

Tuttavia, essa è giustificata dall'obiettivo principale dell'articolo, ossia modellare la relazione tra le varie
forme di collasso nella popolazione di colonie di \emph{A.~mellifera} e gli eccezionali tassi di mortalità
tra le foraggiatrici.
\footcite[1,2,3,5]{khoury2011}

\subsection{Tasso di schiusa.}
Il termine $E(H,F)$ tiene conto della schiusa della covata che emerge come giovani operaie adulte nel
comparto $H$ delle api di nido.
\blockquote[{\footcite[2]{khoury2011}}]{\omissis the number of eggs reared in a colony (and hence the eclosion rate) is related to the number of bees in the hive. Big colonies raise more brood.}

Come ricordato nel capitolo~\ref{chap:bio}, il sostentamento di una popolazione numerosa dipende non soltanto
dal tasso di deposizione della regina, ma anche dal numero di operaie giovani che rimangono nel nido per
accudire uova e larve, al fine di massimizzare il numero di esse che giunge sana al termine di tutti gli stadi
della metamorfosi.

\paragraph{}
Gli autori scelgono la formulazione
\begin{equation}
    \label{eq:eclos}
    E(H,F) = L \frac{H+F}{w + H + F} \; ,
\end{equation}
ove $L$ rappresenta il tasso di deposizione della regina e $w$ è la rapidità con cui $E$ cresce verso $L$ quando
la popolazione totale $H+F$ aumenta.\footnote{In questa sezione useremo spesso l'abbreviazione
$N=H+F$ per il numero totale di operaie, di nido e di campo, presenti nella colonia ad un dato istante $t$.}

Infatti, sommando le equazioni \eqref{eq:kh11h}--\eqref{eq:kh11f} e trascurando il termine di mortalità otteniamo
il tasso di crescita asintotico della popolazione totale
$$\diff{N}{t} \simeq \frac{L}{ \frac{w}{N} + 1} \approx L \; , \quad \text{quando } \frac{w}{N} \ll 1 \; ,$$
ovvero per popolazioni $N$ molto grandi.

La $E$ è una funzione Holling di tipo II, in cui il tasso di attacco è $a=\frac{1}{w}$ e $w$
è la retroimmagine del semimassimo nella risposta. In figura~\ref{img:eclos} è riportata $E(N)$ in funzione di $N$.
Possiamo vedere come valori
crescenti di $w$ ``schiacciano'' la curva mentre la saturazione è raggiunta per valori sempre maggiori di $N$,
e il grafico di $E(N)$ si ``inclina'' verso la destra.

\begin{figure}[hbp]
    \centering
    \includegraphics[keepaspectratio,width=0.86\textwidth]{img/hollingTypeII-eclosion}

    \caption[Schiusa, Holling tipo II]{$E$ rappresenta la quantità di uova deposte dalla regina che raggiunge lo
    stadio adulto, come risposta funzionale al numero totale di operaie $N=H+F$.
    \\
    Sono illustrati tre diversi valori di $w$, mantenendo costante il tasso di deposizione $L=2000$.}
    \label{img:eclos}
\end{figure}

\paragraph{}
Modelli successivi\footcite{ratti2017} hanno incorporato questa proposta, modificandola in una funzione Holling
di tipo III e aggiungendo certi fattori di modulazione ulteriori, ma tutto sommato hanno mantenuto il nucleo
della formulazione molto simile a quanto~\citeauthor{khoury2011} propongono nell'equazione~\eqref{eq:eclos}.


\subsection[Configurazioni di equilibrio]{Configurazioni di equilibrio e analisi della stabilità}
Rivediamo adesso i risultati forniti nell'articolo, a proposito dell'esistenza e della stabilità
dei punti di equilibrio nel modello~\eqref{eq:kh11h}--\eqref{eq:kh11f}.

Gli autori hanno usato \textquote[{\footcite[3]{khoury2011}}]{standard linear stability analysis and phase
plane analysis} e ripercorrendo i loro passi forniremo criteri per l'esistenza
di equilibri banali e non, \ie zeri del membro destro delle equazioni del modello.

\paragraph{}
Una volta stabilita l'esistenza di un punto di equilibrio, calcoliamo la matrice jacobiana del sistema e
la valutiamo sull'equilibrio stesso; un semplice criterio sufficiente per la stabilità asintotica
\footnote{Vedasi il Teorema~\ref{teo:pozzoNonLineare} a p.~\pageref{teo:pozzoNonLineare}.}
è che $\Re (\lambda) < 0$ per ogni autovalore $\lambda$ della matrice.

Un'altra tecnica, dovuta a Lyapounov, consiste nell'esibire una funzione che abbia un minimo
locale forte sul punto di equilibrio e che sia localmente decrescente lungo le
soluzioni.\footnote{Vedasi il Teorema~\ref{teo:lyapounovFunc} a p.~\pageref{teo:lyapounovFunc}.}

\paragraph{}
Si noti da subito che $H, F$ sono non--negative, \ie $H(t), F(t) \geq 0$ per ogni $t$.
Inoltre, tutti i parametri del modello debbono essere strettamente positivi.

Il prossimo risultato stabilisce una condizione necessaria per tutti i punti d'equilibrio non banali.

\begin{lemma}
    \label{lem:necessJ}
    Tutti i punti di equilibrio $(H^*, F^*) \neq (0,0)$
    del modello~\eqref{eq:kh11h}--\eqref{eq:kh11f} verificano
    \begin{equation}
    F^* = J H^* \; ,
    \label{eq:FeqJH}
    \end{equation}
    dove la costante adimensionale $J$ vale
    \begin{equation}
        J \coloneq \frac{1}{2} \left[
            \frac{\alpha}{m} - \frac{\sigma}{m} - 1 +
            \sqrt{ {\left( \frac{\alpha}{m} - \frac{\sigma}{m} - 1 \right)}^2
                + 4 \frac{\alpha}{m}
            }
        \right].
        \label{eq:kh11posEqJ}
    \end{equation}
\end{lemma}

Vale a dire, i punti di equilibrio giacciono sulla semiretta $F= JH$ nello spazio delle fasi, quando esistono.

\begin{proof}
    Annullando il membro destro dell'equazione~\eqref{eq:kh11f} otteniamo
    $$F^2 - \left( \frac{\alpha}{m} - \frac{\sigma}{m} -1
        \right) H F - \frac{\alpha}{m} H^2 = 0 \; .$$
    Risolvendo rispetto ad $F$ otteniamo il discriminante
    $$\Delta = \left[ {\left( \frac{\alpha}{m} - \frac{\sigma}{m} -1
        \right)}^2 + 4 \frac{\alpha}{m}
        \right] H^2 \; , $$
    che è sempre non negativo.

    \paragraph{}
    Adottando la definizione di $J$ dell'equazione~\eqref{eq:kh11posEqJ}, possiamo vedere
    che ogni punto d'equilibrio $(H^*, F^*)$ deve necessariamente verificare la~\eqref{eq:FeqJH}.
\end{proof}


Calcoliamo adesso la matrice jacobiana del sistema nella configurazione generale $(H, F)$:
\begin{equation}
    D(H, F) =
    \begin{pmatrix}
        L \frac{w}{{(w + H + F)}^2} - \alpha + \sigma \frac{F^2}{{(H+F)}^2} &
        L \frac{w}{{(w + H + F)}^2} + \sigma \frac{H^2}{{(H+F)}^2}
        \\
        \alpha - \sigma \frac{F^2}{{(H+F)}^2} &
        - \sigma \frac{H^2}{{(H+F)}^2} - m
    \end{pmatrix} \; .
    \label{eq:jacoGeneral}
\end{equation}

Se assumiamo che $(H, F) = (H^*, F^*)$ sia un punto d'equilibrio, sotto la condizione ~\eqref{eq:kh11posEqJ}
la matrice jacobiana~\eqref{eq:jacoGeneral} si riduce all'espressione
\begin{equation}
    D \left( F/J, F \right) =
    \begin{pmatrix}
        L \frac{w}{ {\left[ w + \left( \frac{1}{J} +1 \right) F \right]}^2 } - \alpha
        + \frac{ \sigma }{ {\left( \frac{1}{J} + 1 \right)}^2 } &
        %
        L \frac{w}{ {\left[ w + \left( \frac{1}{J} +1 \right) F \right]}^2 }
        + \frac{ \sigma }{ {\left( J + 1 \right)}^2 } \\
        %
        \alpha - \frac{ \sigma }{ {\left( \frac{1}{J} + 1 \right)}^2 } &
        %
        - \frac{ \sigma }{ {\left( J + 1 \right)}^2 } - m
    \end{pmatrix} \; ,
    \label{eq:jacoOnJFH}
\end{equation}
in cui la seconda riga dipende soltanto dai parametri, e non dalle variabili $H$ ed $F$.

% \begin{gather}
%     \diffp*{ \dot{H} }{H}{H= \frac{1}{J} F} =
%     L \frac{w}{ {\left[ w + \left( \frac{1}{J} +1 \right) F \right]}^2 } - \alpha
%     + \frac{ \sigma }{ {\left( \frac{1}{J} + 1 \right)}^2 } \; ,
%     \\
%     \diffp*{ \dot{H} }{F}{H= \frac{1}{J} F} =
%     L \frac{w}{ {\left[ w + \left( \frac{1}{J} +1 \right) F \right]}^2 }
%     + \frac{ \sigma }{ {\left( J + 1 \right)}^2 } \; ,
%     \\
%     \diffp*{ \dot{F} }{H}{H= \frac{1}{J} F} =
%     \alpha - \frac{ \sigma }{ {\left( \frac{1}{J} + 1 \right)}^2 } \; ,
%     \\
%     \diffp*{ \dot{F} }{F}{H= \frac{1}{J} F} =
%     - \frac{ \sigma }{ {\left( J + 1 \right)}^2 } - m \; ,
%     \label{eq:jacoOnJFH}
% \end{gather}
% e le ultime due equazioni dipendono soltanto dai parametri, non dalle variabili $H$ ed $F$.

\subsubsection{Equilibrio banale}
L'equilibrio banale $(0,0)$ è chiaramente sempre soluzione\footnote{Dunque il sistema ammette
\emph{sempre} l'equilibrio banale.}
di $$\dot{H}=\dot{F}=0 \; ,$$ e lo studio della sua stabilità è cruciale.
Infatti, nelle regioni di parametri per cui questo equilibrio è stabile e attrattivo generalmente si ha il declino
inevitabile dell'intera colonia.

\paragraph{}
La matrice jacobiana~\eqref{eq:jacoGeneral} del modello valutata nell'origine è:
\begin{equation}
    D_{00} \coloneq
    \begin{pmatrix}
        \frac{L}{w} -\alpha &
        \frac{L}{w}
        \\
        \alpha & -m
    \end{pmatrix} \; .
    \label{eq:jacoEquZero}
\end{equation}

% Con il Teorema del pozzo non lineare\footnote{Cfr. p.~\pageref{teo:pozzoNonLineare}}
% possiamo vedere il motivo per cui la condizione~\ref{eq:cond6b}
% $\alpha -\frac{L}{w}>0$
% sia utile per caratterizzare la stabilità dell'origine.\footcite[3]{khoury2011}

Abbiamo
$$\det  D_{00} > 0 \quad \iff \quad
m > \frac{\alpha L}{w} {\left( \alpha - \frac{L}{w} \right)}^{-1} \; ,$$
e
$$ \alpha - \frac{L}{w} > 0  \quad \implies \quad
\tr D_{00} = - \left( \alpha - \frac{L}{w} \right) - m < 0 \; ,$$

da cui ricaviamo che se $\alpha -\frac{L}{w}>0$ (condizione~\eqref{eq:cond6b}), allora $D_{00}$ ha
determinante positivo e traccia negativa, cioè gli autovalori sono concordi e negativi;
dunque\footnote{Cfr. p.~\pageref{teo:pozzoNonLineare}} l'origine è stabile.

\paragraph{}
Riassumiamo quanto appena discusso nel seguente
\begin{lemma}[Condizione sufficiente per la stabilità dell'equilibrio banale]
    Se $\alpha - \frac{L}{w} > 0$ e
    $$m > \frac{\alpha L}{w \alpha -L} \; ,$$
    allora l'equilibrio banale del modello \eqref{eq:kh11h}--\eqref{eq:kh11f} è asintoticamente stabile.
    \label{lem:eqBanKh}
\end{lemma}


\subsubsection{Equilibrio positivo}
Indaghiamo adesso nel dettaglio le condizioni per l'esistenza e la stabilità di un equilibrio non banale, ossia
con popolazione strettamente positiva. Tali punti dovrebbero rappresentare l'esito favorevole di una colonia che
sopravvive alle avversità.

\paragraph{}
Presentiamo una successione di risultati più semplici, al fine di costituire un criterio per l'esistenza di
un punto d'equilibrio positivo.

\begin{proposizione}[Unicità dell'equilibrio positivo]
    Se il modello~\eqref{eq:kh11h}--\eqref{eq:kh11f} ammette un equilibrio positivo $(H^*, F^*)$, allora
    questi è unico.

    Esso è determinato dall'equazione~\eqref{eq:kh11posEqJ}, insieme con la seguente:
    \begin{equation}
        F^* = \frac{L}{m} - w \frac{J}{1+J} \; .
        \label{eq:kh11posEqF}
    \end{equation}

    \label{teo:exUniqFstarPos} % WARNING prima era teorema, ho cambiato idea dopo.
\end{proposizione}

La possibilità della non--esistenza dell'equilibrio positivo è implicitamente
catturata nell'equazione~\eqref{eq:kh11posEqF}, laddove si abbia $F^* \leq 0$.

\begin{proof}
    Otteniamo la dimostrazione con alcuni calcoli elementari.

    Dal momento che stiamo cercando un equilibrio non banale $(H^*, F^*)$, per il
    Lemma~\ref{lem:necessJ} possiamo supporre ambo le componenti strettamente positive.

    Ancora per il Lemma~\ref{lem:necessJ}, possiamo sostituire $H=\frac{1}{J} F$ nel membro destro della
    prima equazione~\eqref{eq:kh11h} del modello, ed usare la condizione di equilibrio per uguagliare
    a zero quanto ottenuto. Da
    $$L \frac{ \left( \frac{1}{J} +1 \right) F }{ w +\left( \frac{1}{J} +1 \right) F }
    - \frac{1}{J} F \left( \alpha - \sigma \frac{\cancel{F}}{ \left( \frac{1}{J} +1 \right) \cancel{F} } \right) = 0$$
    possiamo semplificare $F$ perché non nulla, così da ottenere
    $$L \left( \frac{1}{J} +1 \right) \cancel{F} - \frac{1}{J} \left( \alpha - \frac{\sigma}{ \frac{1}{J} +1 } \right)
    \left[ w + \left( \frac{1}{J} +1 \right) F \right] \cancel{F} = 0 \; ,$$
    dove la semplificazione di $F$ procede con lo stesso argomento di sopra; si noti che ciò ammonta a scartare la
    soluzione nulla, che non ci interessa per ipotesi.

    L'unicità dell'equilibrio positivo si ottiene manipolando ulteriormente l'equazione sopra, ottenendo infine
    $$F^* = \frac{LJ}{ \alpha - \frac{\sigma}{ 1 + \frac{1}{J} } } - w \frac{J}{1+J} \; ,$$
    in cui possiamo osservare che il membro destro dipende esclusivamente dai parametri.

    \paragraph{}
    Avendo a disposizione l'ultima equazione, per concludere la dimostrazione -- \ie provare la~\eqref{eq:kh11posEqF}
    -- resta soltanto da dimostrare che
    $$J m = \alpha - \sigma \frac{J}{1+J} \; ,$$
    dove abbiamo semplificato il fattore $L$ ad ambo i membri, essendo non nullo.

    Per stabilire quest'ultima equazione può essere utile razionalizzare l'espressione $\frac{J}{1+J}$ come
    $$\frac{J}{1+J} = \frac{ \alpha + \sigma + m - \sqrt{ {(\alpha - \sigma -m)}^2 - 4 \alpha m } }{2 \sigma}$$
    e raggiungere l'obiettivo tramite semplici manipolazioni algebriche.
\end{proof}

Il modello~\eqref{eq:kh11h}--\eqref{eq:kh11f} ammette quindi sempre l'equilibrio banale, ed un ulteriore
configurazione di equilibrio positivo in alcune circostanze, riassunte nella seguente
\begin{proposizione}[Criterio di esistenza]
    Una condizione necessaria e sufficiente affinché il modello~\eqref{eq:kh11h}--\eqref{eq:kh11f} abbia
    un equilibrio strettamente positivo $P^* \coloneq (H^*, F^*)$ è data per casi:
    \begin{itemize}
        \item[1.] se $\alpha - \frac{L}{w} >0$, allora $P^*$ esiste $\iff m_1^* < m < m_2^*$;
        \item[2.] se $\alpha - \frac{L}{w} <0$, allora $P^*$ esiste $\iff m > m_2^*$;
        \item[3.] se $\alpha - \frac{L}{w} =0$, allora $P^*$ esiste $\iff m > m_3^*$;
    \end{itemize}
    dove
    \begin{equation}
        m_3^* = \frac{ \sigma L}{ w (\alpha + \sigma) } \; ,
        \label{eq:FstarPosM3}
    \end{equation}
    e $m_1^*, m_2^*$ sono sempre prese nell'ordine $m_1^* < m_2^*$ dalla seguente
    \begin{equation}
        m_{1,2}^* = \frac{L}{2w} \cdot \frac{ \alpha + \sigma \pm \sqrt{ {(\alpha - \sigma)}^2 +\frac{4 \sigma L}{w}  } }
        {\alpha -\frac{L}{w}} \; .
        \label{eq:FstarPosM12}
    \end{equation}

    \label{teo:esistenzPosF} % WARNING prima era teorema, ho cambiato idea dopo.
\end{proposizione}

\begin{proof}
Possiamo usare l'equazione~\eqref{eq:kh11posEqF} dalla Proposizione~\ref{teo:exUniqFstarPos} e studiare
la disequazione $F^* >0$ per dimostrare il criterio di esistenza:
$$F^* > 0 \; \iff \;
\left( \frac{2 \alpha w}{L} -1 \right) m - \alpha - \sigma < \sqrt{ {(\alpha - \sigma -m)}^2 + 4 \alpha m } \; .$$

Elevando al quadrato ambo i membri otteniamo la seguente disuguaglianza in $m$:
\begin{equation}
    \left( \frac{\alpha w}{L} -1 \right) m^2 -( \alpha + \sigma) m + \frac{\sigma L}{w} < 0
    \label{eq:mDiseq}
\end{equation}
la quale ha discriminante sempre positivo
$$\Delta = {(\alpha -\sigma)}^2 +4 \frac{\sigma L}{w} \; .$$

L'equazione~\eqref{eq:FstarPosM12} si ottiene applicando la formula delle radici quadratiche.
Nel seguito sceglieremo la notazione per avere sempre $m_1^* < m_2^*$ fino al termine della dimostrazione.

\paragraph{Caso 1.}
Se $\alpha - \frac{L}{w} >0$, allora $F^*>0$ esiste se e soltanto se
$$m_1^* < m < m_2^* \; ,$$
ed anche il limite inferiore ha senso, dal momento che
$m_1^* > 0$ si riduce a $\alpha - \frac{L}{w}>0$, vera per ipotesi del caso.

\paragraph{Caso 2.}
Se $\alpha -\frac{L}{w} <0$, allora dobbiamo prendere $m < m_1^*$ oppure $m > m_2^*$ affinché $F^* >0$ esista.
La prima disequazione si può scartare dal momento che $m_1^* <0$ per ipotesi del caso.

La seconda va mantenuta dal momento che $m_2^* >0$ si riduce alla ipotesi del caso $\alpha - \frac{L}{w} < 0$.

\paragraph{Caso 3.}
Quando $\alpha - \frac{L}{w} =0$, la nostra disequazione~\eqref{eq:mDiseq} diviene lineare e la sua soluzione è
$$m > \frac{\sigma L}{w (\alpha +\sigma)} \; ,$$
ciò che conclude l'ultimo caso restante della dimostrazione.
\end{proof}

\paragraph{}
Un criterio sufficiente per la stabilità dell'equilibrio positivo è dato nella seguente Proposizione
\footcite[3]{khoury2011}; la enunciamo qui senza dimostrazione poiché essa richiede calcoli lunghi e tediosi,
seppure non sia particolarmente complicata sotto il profilo teorico.

\begin{proposizione}[Criterio di stabilità dell'equilibrio positivo]
L'equilibrio positivo $(H^*, F^*)$ del modello \eqref{eq:kh11h}--\eqref{eq:kh11f}, determinato
dalla Proposizione~\ref{teo:exUniqFstarPos},
è asintoticamente stabile se $\alpha -\frac{L}{w} >0$ e
$$m < \frac{L}{2w} \cdot \frac{ \alpha + \sigma + \sqrt{ {(\alpha - \sigma)}^2 +\frac{4 \sigma L}{w}  } }
    {\alpha -\frac{L}{w}} \; .$$
    \label{prop:stabPosKh}
\end{proposizione}

\paragraph{Interpretazione.}
\label{par:interpretationCond6b}
Notiamo che tutte le situazioni che forniscono un limite inferiore ad $m$ potrebbero essere attribuite alle
semplificazioni intrinseche al modello, dal momento che potrebbero non descrivere accuratamente un alveare
in apiario sul campo. La non esistenza di un equilibrio positivo nelle situazioni in cui
$m< \hat{m}$ per qualche $\hat{m}$ potrebbe semplicemente essere dovuta alla divergenza verso l'infinito
della popolazione in tali casi.

Al contrario, la condizione che ritroviamo nel Lemma~\ref{lem:eqBanKh} e nelle Proposizioni~\ref{teo:esistenzPosF}
e \ref{prop:stabPosKh} data dalla disuguaglianza
\begin{equation}
\alpha - \frac{L}{w} > 0
    \label{eq:cond6b}
\end{equation}
si può leggere altrettanto semplicemente come $w \alpha >L$, dove entrambi i membri hanno unità di api al giorno.
Si noti anche che il membro destro cattura l'``efficienza'' dei processi sociali di reclutamento e cura della
covata, mentre il membro sinistro misura l'``efficienza'' della regina nel deporre uova fresche.

Vale a dire, la~\eqref{eq:cond6b} significa che il processo di reclutamento $R$ è capace
di sostenere un livello di deposizione $L$, bilanciando le necessità dell'alveare che richiedono api sia nel
comparto $H$ (per la cura della covata), che nel comparto $F$ (per l'importazione di cibo).

\paragraph{}
Condizioni della forma $m< \hat{m}$ tipicamente servono ad assicurare la sopravvivenza della colonia;
in questa regione dei parametri, il modello può manifestare la non esistenza di un equilibrio positivo,
per via della crescita incontrastata dei numeri nella popolazione totale.

Come notato in precedenza, questo modello non ha un fattore di mortalità nel compartimento $H$: alla luce
di questo fatto, possiamo spiegare la condizione~\eqref{eq:cond6b} come limite inferiore per il tasso
di trasferimento dal comparto delle operaie di casa al comparto delle operaie di campo, dove
le api hanno una aspettativa di vita finita.
Questa equazione esprime la reale necessità di coordinare efficientemente i passaggi tra le suddivisioni
del lavoro per la buona riuscita della colonia; per contro, le limitazioni puramente matematiche che
prevengono la divergenza della popolazione sono conseguenze puramente teoriche e non trovano applicazione
reale sul campo.

% Nel limite asintotico, $\frac{L}{w} H$ bees enter the hive compartment through eclosion and $\alpha H$ bees
% are withdrawn to the $F$ compartment: condition~\eqref{eq:cond6b} then says that the maximal recruitment rate
% must be sufficiently large to unload the excess hive bees towards the foragers compartment.
%  moar interpr? TODO
% cfr. condiz per 0

\paragraph{}
La ramificazione degli equilibri in questo modello è illustrata in figura~\ref{img:kh11phasePlane}.

\begin{figure}[hpb]
    \centering
    \includegraphics[keepaspectratio,width=0.76\textwidth]{img/pone.0018491.g003}

    \caption[\figurename~3 da \parencite{khoury2011}]{Diagramma delle fasi per le
        soluzioni del modello con differenti valori di mortalità $m$.
        Ogni curva del diagramma rappresenta la traiettoria di una soluzione, fornendo il numero di
        operaie di nido $H$ e di foraggiatrici $F$.
        All'aumentare del tempo $t$ le soluzioni variano nel verso delle frecce.

        In (a) abbiamo $m = 0.24$ e la popolazione tende ad un equilibrio positivo stabile,
        segnalato da un punto.

        Nella (b) abbiamo $m = 0.40$ e l'unico equilibrio del sistema è quello banale: la popolazione
        collassa verso lo zero.

        I valori degli altri parametri sono $L = 2000$,  $\alpha= 0.25$,  $\sigma= 0.75$ e $w = 27 000$.
        \\
        { \footnotesize \ttfamily doi: 10.1371/journal.pone.0018491.g003 }
        }

    \label{img:kh11phasePlane}
\end{figure}

\paragraph{}
\label{par:mInverseFlightspan}
Tra i contributi più utili di questo semplice modello, oltre al ruolo sociale delle funzioni $E$ ed
$R$, rileviamo una concentrazione particolare degli autori sul parametro $m$:
ciò riflette in parte l'attenzione dedicata dalla ricerca alle varie sindromi da
collasso di popolazione in \emph{Apis} rilevate in tutto il mondo.
D'altro canto, è importante per la validazione di ogni modello matematico
un confronto coi dati sperimentali rilevati sul campo:\footcite{rueppell2009honey}
in questo senso $m$ è facilmente derivabile come l'inverso
di una misura diretta, l'aspettativa di vita media per le bottinatrici (\emph{flightspan}).

Se ogni bottinatrice ha mediamente probabilità $1-m$ di sopravvivere al primo giorno in tale ruolo, $(1-m)^2$
probabilità di sopravvivere al secondo, e così di seguito, la sua aspettativa di vita media è
$$ \sum_{k=1}^{\infty} k m {(1-m)}^k = m \underbrace{ \sum_{k=1}^{\infty} k {(1-m)}^k }_{m^{-2}}
= \frac{1}{m} \; .$$


% TODO stabilità dell'equilibrio non banale?


% \section{Modello a cinque compartimenti operaie, con \emph{Varroa} e ABPV}
Esaminiamo il modello proposto da \citeauthor{ratti2017} nel \citeyear{ratti2017}, che estende alcune proposte precedenti con lo scopo di studiare gli effetti diretti e indiretti di \emph{Varroa destructor} sulla popolazione di un alveare.

\subsection{Derivazione del modello}
Per ottenere una formulazione del modello, gli autori di \cite{ratti2017} propongono le seguenti assunzioni preliminari:
\begin{enumerate}
    \item La popolazione delle operaie sane (non infette da ABPV) è suddivisa in due comparti: le operaie più giovani che svolgono mansioni all'interno dell'alveare ($x_h$) e le bottinatrici/esploratrici ($x_f$).

    Questa impostazione ricalca lo studio \cite{khoury2011}, in cui si propone un modello a due compartimenti per esaminare gli effetti del tasso di schiusa delle pupe e del tasso di reclutamento da operaie d'alveare a bottinatrici sulla dinamica di una popolazione sana.
    \item La popolazione degli acari è suddivisa in due comparti: le varroe portatrici di ABPV ($m$) e le varroe libere dal virus ($n$).
    \item Le operaie infette con ABPV ($y$) perdono rapidamente l'abilità al volo, le loro capacità di orientamento degradano e muoiono in breve tempo. Per questo motivo le api infette sono considerate ``di alveare'', e non vengono reclutate per bottinare.
\end{enumerate}

I compartimenti del modello sono dunque cinque:
\begin{itemize}
    \item il numero di operaie di nido sane, $x_h$
    \item il numero di operaie bottinatrici sane, $x_f$
    \item il numero di operaie infette, $y$
    \item il numero di varroe portatrici di ABPV, $m$
    \item il numero di varroe senza virus, $n$
\end{itemize}

Abbiamo inoltre
$$\text{Pop. totale api operaie} = x_h + x_f + y,$$
e
$$\text{Pop. totale varroe} = m + n.$$

\paragraph{}
Proseguiamo con l'esame delle assunzioni del modello:
\begin{enumerate}
    \setcounter{enumi}{3}
    \item Il tasso massimale di riproduzione delle varroe è lo stesso, che siano portatrici di virus o meno.
    Ciò si riflette nelle equazioni \eqref{eq:r17m} e \eqref{eq:r17n}, in cui il primo termine descrive le nuove nascite: esso è proporzionale alla popolazione del compartimento ed il fattore di proporzionalità è lo stesso in entrambe le equazioni. In esso, $r$ è il tasso massimo di riproduzione del parassita \emph{Varroa}, mentre $\alpha$ è la capacità portante dell'ospite \emph{Apis}.
    \item Sappiamo che la covata necessita di cure da parte delle operaie nutrici, quindi il tasso di schiusa è dipendente dal numero di operaie sane nella popolazione, che seguendo \cite{khoury2011} assumiamo pari alla somma $x_h + x_f$.
    \item L'ape regina non subisce gli effetti avversi di ABPV e dell'infestazione da varroe; di conseguenza, il tasso di schiusa non dipende dalla preminenza nell'alveare di varroe o di virus.
    \item La trasmissione di ABPV è esclusivamente orizzontale: ciò significa che l'infezione di una operaia avviene per trasmissione del virus da un'altra operaia infetta.\footnote{La trasmissione \emph{verticale} di un patogeno riguarda invece l'infezione della prole a causa dei genitori già infetti.}

    Le varroe sono un vettore puramente meccanico per il virus: la carica virale non è incentivata né inibita dalle varroe, che per ABPV fungono meramente da ``mezzo di trasporto'' tra le api.
    \item Poiché le pupe infette con ABPV muoiono rapidamente prima della schiusa, tutte le giovani operaie che giungono all'età adulta si assumono libere dal virus. L'effetto limitante sul tasso di schiusa dovuto alla presenza di varroe infette viene incorporato in un fattore di modulazione $h(m)$.
    \item Per studiare l'effetto dei trattamenti acaricidi, introduciamo un ulteriore termine di pozzo $\delta_i$ in ogni equazione. Assumiamo che l'impatto dei trattamenti sia di gran lunga maggiore sulle varroe che sulle api: $\delta_4, \delta_5 > \delta_1, \delta_2, \delta_3$.
    \item In tutti e tre i compartimenti delle api, si assume un tasso di mortalità naturale $d_i$ in aggiunta alla mortalità dovuta alle varroe $\gamma_i$ (a sua volta dipendente sia dal parassitismo che dalla trasmissione del virus). Nelle api indebolite dall'infezione da ABPV entrambi questi tassi sono maggiori rispetto alle operaie sane: $d_3 > d_1, d_2$ e $\gamma_3 > \gamma_1, \gamma_2$.

    Inoltre per le bottinatrici si tiene conto anche della mortalità da disorientamento (fattore $p$), dovuto principalmente al degrado delle facoltà cognitive e di navigazione per effetto dei pesticidi sintetici sulle colture.
\end{enumerate}


\subsubsection{Equazione del modello}
Il modello a compartimenti proposto in \cite{ratti2017} è il seguente:
\begin{align}
    % hive sane
    \diff{x_h}{t} &= \mu g( x_h + x_f ) h(m) - \beta_1 m \frac{x_h}{x_h + x_f + y} - \left( d_1 + \delta_1 \right) x_h \notag \\ % barbatrucco
    \label{eq:r17xh}
        &{\;} - \gamma_1 (m+n) x_h - x_h \cdot R(x_h, x_f) \\ % barbatrucco
    % foragers sane
    \label{eq:r17xf}
    \diff{x_f}{t} &= x_h \cdot R(x_h, x_f) - \beta_1 m \frac{x_f}{x_h + x_f + y}
    - \left( p + d_2 + \delta_2 \right) x_f - \gamma_2 (m+n) x_f \\
    % (hive) infette
    \label{eq:r17y}
    \diff{y}{t} &= \beta_1 m \frac{x_h + x_f}{x_h + x_f + y} - \left( d_3 + \delta_3 \right) y - \gamma_3 (m+n) y \\
    % varroe con ABPV
    \label{eq:r17m}
    \diff{m}{t} &= rm \left( 1 - \frac{m+n}{ \alpha (x_h + x_f + y) } \right) + \beta_2 n \frac{y}{x_h + x_f + y}
    - \beta_3 m \frac{x_h + x_f}{x_h + x_f + y} - \delta_4 m \\
    % varroe senza ABPV
    \label{eq:r17n}
    \diff{n}{t} &= rn \left( 1 - \frac{m+n}{ \alpha (x_h + x_f + y) } \right) - \beta_2 n \frac{y}{x_h + x_f + y}
    + \beta_3 m \frac{x_h + x_f}{x_h + x_f + y} - \delta_5 n
\end{align}

% ELIMINO RIPETIZIONE
% dove
% \begin{itemize}
%     \item $x_h$ è il numero di operaie sane che vivono all'interno dell'alveare (\emph{hive})
%     \item $x_f$ è il numero di operaie bottinatrici (\emph{foragers})
%     \item $y$ è il numero di operaie infette da ABPV
%     \item $m$ è il numero di varroe portatrici di ABPV
%     \item $n$ è il numero di varroe non portatrici di virus.
% \end{itemize}

\paragraph{}
Il termine ``di reclutamento'' $x_h \cdot R(x_h, x_f)$ nelle equazioni \eqref{eq:r17xh} e \eqref{eq:r17xf} rappresenta appunto il passaggio di ruolo delle operaie, da impiegate \emph{all'interno} dell'alveare a \emph{bottinatrici} che escono dall'arnia per esplorare e raccogliere il cibo.

Come nello studio precedente \cite{khoury2011}, il fattore $R$ rappresenta gli effetti \emph{sociali}
della colonia sul tasso di reclutamento, con la seguente formulazione:
\begin{equation}
    \label{eq:reclutam}
    R(x_h, x_f) = \sigma_1 - \sigma_2 \frac{x_f}{x_h + x_f}
\end{equation}
in cui $\sigma_1$ è il tasso massimo di reclutamento delle operaie, quando non ci sono bottinatrici nella colonia.
Il termine $\sigma_2 \frac{x_f}{x_h + x_f}$ descrive l'inibizione sociale del reclutamento, ossia il rallentamento del processo di ``promozione'' all'esterno del nido, proporzionale alla frazione di popolazione impiegata nella bottinatura; quando tale frazione è in surplus, le bottinatrici vengono ``riallocate'' in attività interne all'alveare.

\paragraph{}
Il tasso di nuove nascite è descritto dal primo termine nella equazione \eqref{eq:r17xh}, in cui troviamo il parametro $\mu$ che è determinato principalmente dalla regina e dalla stagione, e rappresenta il tasso massimo di schiusa (in unità di numero di uova deposte al giorno).

La funzione $g$ descrive il bisogno di un numero minimo di operaie sane nella colonia per allevare la covata; se questo numero scende sotto una certa soglia -- che può presentare variazioni stagionali -- la covata non produce più api adulte.
La $g$ rappresenta la risposta funzionale dell'allevamento della covata; nelle parole degli autori:
\begin{displayquote}[\cite{ratti2017}]
``We think of $g(x_h + x_f)$ as a switch function.''
\end{displayquote}

Abbiamo dunque $g(0)=0$, $\diff{g}{x_h} \geq 0$ e $\diff{g}{x_f} \geq 0$. Inoltre $\lim_{x_h+x_f \to \infty} g(x_h+x_f)=1$, e dunque un'utile formulazione per $g$ è data dalla sigmoide di Hill (figura~\ref{img:hillSigmoid}):
\begin{equation}
    g(x_h + x_f) = \frac{ (x_h+x_f)^i }{ K^i + (x_h+x_f)^i }
    \label{eq:hillSigmoid}
\end{equation}

\begin{figure}
    \centering
    \includegraphics[keepaspectratio,width=0.86\textwidth]{img/hillSigmoid}

    \caption[Sigmoidi di Hill.]{Sigmoide di Hill per diversi valori dell'esponente $i$.
        \\ Il parametro $K$ rappresenta la retroimmagine del semimassimo.}
    \label{img:hillSigmoid}
\end{figure}

dove $K$ è la dimensione della popolazione di operaie nell'alveare corrispondente al tasso di schiusa semimassimo, e l'esponente $i>1$ si assume intero per semplicità di analisi.

\paragraph{}
Sempre nel termine delle nascite di api adulte, un altro fattore modulante tiene conto degli effetti deleteri per la covata dovuti alla presenza degli acari portanti ABPV nell'alveare: le larve e le pupe infette dal virus muoiono rapidamente prima di raggiungere lo stadio di adulte, quindi la funzione $h(m)$ nell'equazione~\ref{eq:r17xh} verifica le condizioni $h(0)=1$, $\diff{h}{m} <0$ e $\lim_{m \to \infty} h(m) = 0$.

In \cite{sumMar04} si suggerisce $h(m) \approx e^{-km}$ con $k\geq0$, e questa è la forma utilizzata in \cite{ratti2017} per le simulazioni numeriche.


\subsection{Analisi del modello con parametri costanti}
Seguiamo l'impostazione dello studio \cite{ratti2017} per l'analisi degli equilibri, iniziando con l'esame di sotto-casi più semplici ed estendendo progressivamente i risultati fino al modello completo.

\paragraph{}
Dapprima assumiamo l'ipotesi di lavoro di coefficienti costanti; utilizzeremo in seguito la teoria di Floquet per analizzare il caso di coefficienti periodici (sottosezione~\ref{sez:paramPeriodici}).

% \paragraph{}
% Per le dimostrazioni dei risultati esposti in questa sottosezione, si rimanda all'articolo originale \cite[11--15]{ratti2017}.
Una volta stabilite le condizioni per l'esistenza dei punti di equilibrio, la metodologia standard per analizzarne la stabilità consiste nel linearizzare il sistema in un intorno di un punto di equilibrio: il carattere attrattivo o repulsivo dell'equilibrio è determinato dagli \emph{esponenti di Lyapounov locali}, ossia dalle parti reali degli autovalori della matrice jacobiana.

Questo approccio metodologico costituisce la base delle dimostrazioni in questa sottosezione e nella successiva.


\subsubsection{Modello bidimensionale senza patogeni}
In assenza di \emph{Varroa} e di virus, il modello \eqref{eq:r17xh}--\eqref{eq:r17n} si riduce alle due sole equazioni per le api sane:
\begin{align}
    \diff{x_h}{t} &= \mu g( x_h + x_f ) - d_1 x_h - x_h \left( \sigma_1 - \sigma_2 \frac{x_f}{x_h + x_f} \right)
    \label{eq:rattiRidotto1prima}
    \\
    \diff{x_f}{t} &= x_h \left( \sigma_1 - \sigma_2 \frac{x_f}{x_h + x_f} \right) - (p + d_2) x_f
    \label{eq:rattiRidotto1seconda}
\end{align}
dove $g(x_h + x_f) = \frac{ (x_h+x_f)^i }{ K^i + (x_h+x_f)^i }$ come nella \eqref{eq:hillSigmoid}.

\paragraph{}
L'equilibrio banale $(x_h^*, x_f^*) = (0,0)$ esiste sempre, ed è asintoticamente stabile; questo risultato è contenuto nelle due proposizioni che seguono.

Nelle parole degli autori, la stabilità asintotica dell'equilibrio banale significa che
\begin{displayquote}[\cite{ratti2017}]
``\omissis in order to establish itself as a properly working colony, a sufficiently large healthy adult bee population is required to take care of the brood.''
\end{displayquote}

E viceversa, se il numero di operaie sane scende sotto un valore critico, la colonia si avvia verso una morte inesorabile.
Si noti che questo effetto di tipo Allee si osserva già nel modello senza patogeni; l'introduzione di \emph{Varroa} e virus non può che esacerbare tale fenomeno.

\paragraph{}
Proponiamo adesso una analisi dei punti di equilibrio del modello \eqref{eq:rattiRidotto1prima}--\eqref{eq:rattiRidotto1seconda}, introducendo per comodità di notazione le due costanti
$$
    F \coloneq \frac{1}{2} \left( \frac{ \sigma_1 - \sigma_2 - p - d_2 }{p+d_2} +
    \sqrt{ {\left( \frac{ \sigma_1 - \sigma_2 - p - d_2 }{p+d_2} \right)}^2 + \frac{4 \sigma_1}{p+d_2} } \, \right),
$$
ed
$$ a \coloneq - \frac{ \mu }{ \frac{\sigma_2 F}{1+F} - d_1 - \sigma_1 }.$$

\begin{proposizione}[3.1, Existence of equilibria {\cite[11]{ratti2017}}]
    Il modello ridotto alle sole api \eqref{eq:rattiRidotto1prima}--\eqref{eq:rattiRidotto1seconda} ammette sempre l'equilibrio banale $(0,0)$, e se $\frac{\sigma_2 F}{1+F} - d_1 - \sigma_1 >0$ allora esso è unico.

    Viceversa, se $\frac{\sigma_2 F}{1+F} - d_1 - \sigma_1 <0$ allora il sistema ammette due ulteriori equilibri strettamente positivi, a patto che sia verificata l'ulteriore condizione
    $$a > \frac{Ki}{1+F} (i-1)^{- (1+ 1/i)}.$$
\end{proposizione}

\begin{proof}
    Le equazioni del modello sono omogenee, dunque l'equilibrio banale $(0,0)$ esiste sempre.
    Dalla \eqref{eq:rattiRidotto1seconda} segue che ogni punto di equilibrio strettamente positivo $(x_h^*, x_f^*)$ verifica
    $$x_f^2 + \frac{\sigma_2 - \sigma_1 + p + d_2}{p +d_2} x_h x_f - \frac{\sigma_1}{p+d_2} x_h^2 = 0,$$
    da cui ricaviamo la relazione tra le componenti $x_f^* = F x_h^*$.
\end{proof}




Viceversa, se tale disequazione è invertita, possiamo formulare una condizione sufficiente per l'esistenza di due equilibri strettamente positivi.
Poniamo per comodità

ed osserviamo che la \eqref{eq:rattiRidotto1seconda} implica che ogni punto di equilibrio positivo $(x_h^*, x_f^*)$ verifica la relazione $x_f^* = F x_h^*$.

Definiamo inoltre la quantità

che ci permette di stabilire il seguente criterio: se $\frac{\sigma_2 F}{1+F} - d_1 - \sigma_1 <0$ ed inoltre

allora esistono due equilibri non banali (positivi).

\paragraph{}
Se esistono, gli equilibri positivi sono stabili se
\begin{equation}
\frac{i \mu K^i {(x_h^*)}^{i-1} {(1+F)}^{i-1}}{ {\left( K^i + {(x_h^*)}^i {(1+F)}^i \right)}^2 }
<
\frac{d_1 + (p+d_2) F}{{(1+F)}^2}.
\label{eq:rattiRidotto1stability}
\end{equation}

\subsubsection{Modello tridimensionale api--acaro}
Introduciamo adesso nella colonia le varroe senza virus, ed indaghiamo la conseguente alterazione della stabilità degli equilibri individuati nel caso precedente.

Si eliminano quindi dal modello \eqref{eq:r17xh}--\eqref{eq:r17n} le equazioni per le api infette da ABPV e per le varroe che trasportano il virus; notiamo che da $m \equiv 0$ segue che $h(m) \equiv 1$.

\paragraph{}
Il modello ridotto che si ottiene è il seguente:

\begin{align}
    \diff{x_h}{t} &= \mu g( x_h + x_f ) - (d_1 + \delta_1) x_h - \gamma_1 n x_h
        - x_h \left( \sigma_1 - \sigma_2 \frac{x_f}{x_h + x_f} \right)
        \label{eq:rattiRidotto2prima}
    \\
    \diff{x_f}{t} &= x_h \left( \sigma_1 - \sigma_2 \frac{x_f}{x_h + x_f} \right) - (p + d_2 + \delta_2) x_f
        - \gamma_2 n x_f
        \label{eq:rattiRidotto2seconda}
    \\
    \diff{n}{t} &= rn \left( 1 - \frac{n}{ \alpha (x_h + x_f) } \right) - \delta_5 n
        \label{eq:rattiRidotto2terza}
\end{align}

Anche in questo caso, l'equilibrio banale $(x_h, x_f, n)=(0,0,0)$ esiste ed è sempre asintoticamente stabile, con la medesima interpretazione pratica del caso precedente (effetto Allee forte).

\paragraph{Equilibrio senza varroe. } Consideriamo adesso un equilibrio non banale $(x_h^*, x_f^*)$ per il modello \eqref{eq:rattiRidotto1prima}--\eqref{eq:rattiRidotto1seconda}, secondo le condizioni di esistenza date sopra.
Il punto $(x_h^*, x_f^*, 0)$ è un equilibrio per il modello \eqref{eq:rattiRidotto2prima}--\eqref{eq:rattiRidotto2terza}.

Se $r > \delta_5$ allora è instabile; se viceversa $r< \delta_5$ allora il punto $(x_h^*, x_f^*, 0)$ eredita la stabilità di $(x_h^*, x_f^*)$ sotto le \eqref{eq:rattiRidotto1prima}--\eqref{eq:rattiRidotto1seconda}.

\paragraph{}
Ciò significa che l'equilibrio senza varroe è stabile se la \eqref{eq:rattiRidotto1stability} è soddisfatta
ed il tasso massimo di natalità delle varroe è inferiore alla loro mortalità dovuta ai trattamenti.
Si osservi inoltre che in assenza di trattamenti varroacidi l'alveare non riesce a debellare l'infestazione dell'acaro.

Nel caso instabile, non è chiaro se il sistema converge all'equilibrio banale oppure ad un equilibrio endemico. \cite[15]{ratti2017}


\subsubsection{Modello completo api--\emph{Varroa}--ABPV}
Consideriamo infine il modello completo \eqref{eq:r17xh}--\eqref{eq:r17n} e studiamo la stabilità dell'equilibrio senza patogeni.

Sia $(x_h^*, x_f^*)$ un equilibrio positivo per il modello ridotto \eqref{eq:rattiRidotto1prima}--\eqref{eq:rattiRidotto1seconda}, sempre secondo le condizioni di esistenza di cui sopra; che $(x_h^*, x_f^*, 0,0,0)$ sia un equilibrio per \eqref{eq:r17xh}--\eqref{eq:r17n} è ovvio.

Se valgono entrambe le disuguaglianze
$$
\begin{sistema}
    r - \beta_3 - \delta_4 < 0 \\
    r - \delta_5 < 0,
\end{sistema}
$$
allora $(x_h^*, x_f^*, 0,0,0)$ eredita la stabilità di $(x_h^*, x_f^*)$ sotto le \eqref{eq:rattiRidotto1prima}--\eqref{eq:rattiRidotto1seconda}.
Se invece (almeno) una delle disuguaglianze è invertita, l'equilibrio è instabile.

\paragraph{}
Quest'ultimo risultato si può interpetare con la seguente condizione: affinché l'alveare sconfigga completamente la \emph{Varroa} ed il virus, è necessario e sufficiente che
\begin{itemize}
    \item i trattamenti varroacidi siano abbastanza potenti da eradicare gli acari che non portano ABPV;
    \item le varroe portatrici perdano la loro carica virale ad un tasso più rapido della loro natalità.
\end{itemize}

L'esperienza sul campo dell'apicoltura, sia sperimentale che in produzione, indica come estremamente improbabile l'eradicazione completa della \emph{Varroa}.~\cite{privFPan}

Piuttosto, è più ragionevole che \textquote[\cite{privFDL}]{a livello di alveare o di apiario, si riesca a mantenere per un certo periodo (almeno 3 o 4 stagioni) un equilibrio in cui api e varroe coesistono senza causare il collasso delle colonie.}

Ciò corrisponderebbe alla presenza di un equilibrio endemico nel modello, la cui esistenza tuttavia non è ancora dimostrata.
Notiamo inoltre che abbiamo condizioni sufficienti per la stabilità degli equilibri positivi, ma non per la stabilità asintotica: un punto di equilibrio stabile ma non attrattivo potrebbe presentare dei cicli-limite.


\paragraph{}
Le attività in apiario che promuovono la sopravvivenza dell'alveare richiedono un monitoraggio costante e diversi tipi di interventi, che includono i trattamenti acaricidi, ma che non possono limitarsi al contrasto della \emph{Varroa}: è infatti necessario il mantenimento di \textquote[\cite{privFDL}]{\omissis uno stato di salute complessivamente buono nella colonia, che riguarda anche l'efficienza del sistema immunitario, la ``forza'' della famiglia in termini di operaie sane, le scorte di cibo, la salute della regina, e molti altri fattori.}

Questo elemento di complessità è catturato anche nel modello relativamente semplice di \cite{ratti2017}, in quanto le condizioni di esistenza e stabilità dei punti di equilibrio includono una molteplicità di parametri, che nel modello rappresentano non solo gli effetti ``diretti'' dovuti all'acaro e al virus, ma anche lo stato di salute dell'alveare, compresa la sua capacità di allevare covata e di approvvigionarsi di cibo.


\subsection{Analisi del modello con parametri periodici}
\label{sez:paramPeriodici}
Questo ultimo passaggio è fondamentale per incorporare nel modello le variazioni stagionali nel tasso di deposizione da parte della regina, l'arresto della raccolta di cibo nel periodo invernale e di molte altre attività dell'alveare.

Il periodo dei coefficienti è dunque $T= 1 \text{ anno}$.

suloz non banali, Floquet

\begin{proposizione}[4.1, Stability of mite--free periodic solution {\cite[19]{ratti2017}} ]
    Sia $\left( x_h^*(t), x_f^*(t) \right)$ sia una soluzione periodica per il modello ridotto alle sole api \eqref{eq:rattiRidotto1prima}--\eqref{eq:rattiRidotto1seconda}.

    Allora $\left( x_h^*(t), x_f^*(t), 0 \right)$ è una soluzione periodica per il modello ridotto api--acaro \eqref{eq:rattiRidotto2prima}--\eqref{eq:rattiRidotto2terza}; è stabile se
    \begin{enumerate}
        \item $\left( x_h^*(t), x_f^*(t) \right)$ è stabile per il modello ridotto alle sole api;
        \item $\int_0^T (r - \delta_5) dt \leq 0$.
    \end{enumerate}
\end{proposizione}


\begin{proposizione}[4.2, Stability of the disease--free periodic solution {\cite[21]{ratti2017}} ]
    Sia $\left( x_h^*(t), x_f^*(t) \right)$ sia una soluzione periodica per il modello ridotto alle sole api \eqref{eq:rattiRidotto1prima}--\eqref{eq:rattiRidotto1seconda}.

    Allora $\left( x_h^*(t), x_f^*(t), 0, 0, 0 \right)$ è una soluzione periodica per il modello completo \eqref{eq:r17xh}--\eqref{eq:r17n}; è stabile se
    \begin{enumerate}
        \item $\left( x_h^*(t), x_f^*(t) \right)$ è stabile per il modello ridotto alle sole api;
        \item $\int_0^T (r - \delta_5) dt \leq 0$;
        \item $\int_0^T (r -\beta_3 - \delta_4) dt \leq 0$.
    \end{enumerate}
\end{proposizione}



\subsection{Riduzioni ai modelli precedenti}
Se nell'espressione di $g$ (equazione~\eqref{eq:hillSigmoid}) prendiamo $K=0$ ci riduciamo al modello di \cite{sumMar04}





\part{Simulazioni numeriche}

\chapter{Metodi di propagazione delle soluzioni}
Il supporto del calcolatore elettronico ha un valore inestimabile nella ricerca Matematica.

Nella applicazione d'interesse -- i modelli matematici per le api -- viene utilizzato
per manipolare le equazioni dei modelli, sia tramite il calcolo simbolico che svolgendo simulazioni numeriche.

\paragraph{}
L'Analisi Numerica offre una vasta classe di metodi per l'approssimazione di soluzioni di un sistema di equazioni
non lineari, tra cui i metodi Runge--Kutta che includono il metodo di Eulero.

Alcuni sistemi presentano equazioni cosiddette \emph{rigide} poiché alcuni metodi numerici si rivelano
molto instabili numericamente a meno di prendere un passo d'approssimazione esageratamente breve.

Per i due modelli esaminati in questa tesi, una analisi dettagliata delle simulazioni numeriche
si trova nei rispettivi articoli.\footcite{khoury2011,ratti2017}

Presentiamo adesso le simulazioni svolte per il modello a due compartimenti della Sezione~\ref{sec:kh11},
che confermano ed estendono alcuni risultati degli autori.

\paragraph{}
È stato usato il software libero \texttt{GNU Octave}, versione \texttt{9.2.0} e nello specifico
la routine \texttt{ode45} che implementa il metodo di Dormand--Prince al quarto ordine, un solutore
di equazioni differenziali ordinarie della famiglia Runge--Kutta.

% Link al codice? Licens? % TODO


\section{Esperimento A}
\label{sec:esperimentoA}
La simulazione per gli esperimenti descritti in questa sezione utilizza il modello a due compartimenti\footcite{khoury2011}
esaminato nella sezione~\ref{sec:kh11},
definito dalle equazioni~\eqref{eq:kh11h}--\eqref{eq:kh11f} a p.~\pageref{eq:kh11h}, che per comodità
di lettura riportiamo qui sotto:
$$\begin{sistema}
    \dot{H} = E(H,F)- H \cdot R(H,F)\; , \\
    \dot{F} = H \cdot R(H,F)  - m \cdot F\; ,
\end{sistema}$$
ove $E(H,F) = L \frac{H+F}{w + H + F}$ rappresenta la schiusa di nuove operaie di nido,
mentre il reclutamento sociale è dato dalla $R(H,F) = \alpha - \sigma \frac{F}{H+F}$.
La dinamica dei compartimenti è illustrata nel diagramma  riportato
in figura~\ref{img:kh11diagram} a p.~\pageref{img:kh11diagram}.

\paragraph{}
Tutte le simulazioni sono state svolte mantenendo costanti i seguenti parametri:
\begin{itemize}
    \item il tasso di deposizione della regina $L=2000$;
    \item il tasso massimo di reclutamento $\alpha = \frac{1}{4}$;
    \item il tasso di inibizione sociale $\sigma = \frac{3}{4}$;
\end{itemize}

Per esplorare lo spazio dei restanti parametri, sono state svolte 2304 simulazioni, risolvendo
il problema di Cauchy
con parametri e condizioni iniziali scelti come descritto di seguito.

I parametri $m,w$ e la popolazione iniziale $b=H_0 + F_0$ (condizioni iniziali) vengono scelti prima di ogni simulazione,
in modo pseudorandom uniforme in un intervallo specificato dall'utente. La distribuzione uniforme è stata scelta per esplorare estensivamente
lo spazio dei parametri: si noti che invece i dati sperimentali si distribuiscono secondo una normale gaussiana.

La mortalità $m \in \left[ m_-, m_+ \right]$ è stata scelta tra $m_-=0,016$ ed $m_+ =0,7$.

Il parametro $w \in \left[ w_-, w_+ \right]$, che regola la risposta funzionale nella schiusa $E$, è
compreso nell'intervallo tra $w_- = 3000$ e $w_+ = 30000$.

\paragraph{}
Le condizioni iniziali sono state riassunte in un solo parametro $b = H_0 +F_0$, anch'esso scelto
in modo pseudorandom uniforme nell'intervallo $\left[ b_-, b_+ \right] = \left[50, 9000\right]$.
Si noti che tale raggruppamento è concettualmente sensato solo per i compartimenti omologhi di un modello.

La soglia di popolazione $b_{thres} = 20$ sotto la quale una colonia si considera ``irreversibilmente perduta''
è utilizzata nell'esperimento descritto nella sottosezione~\ref{sec:fdod}.

\paragraph{}
Il tempo di simulazione è $T=2~\text{anni}$ ed il passo d'integrazione $\Delta t = 1~\text{giorno}$.

Nello spazio dei parametri in figura~\ref{img:param3D}, ogni punto di coordinate $(m,w,b)$
rappresenta la simulazione di una colonia con parametri $m,w$ e condizioni iniziali $b=H_0+F_0$; la dimensione di ogni
punto è proporzionale alla popolazione finale per esprimere visivamente l'esito della simulazione.

\paragraph{}
\label{par:colore}
Il colore di ogni punto rappresenta la quantità $\alpha w - L$, che compare nel criterio di esistenza per gli equilibri
del modello (proposizione~\ref{teo:esistenzPosF} a p.~\pageref{teo:esistenzPosF}).

Le regioni dei parametri per cui $\alpha w -L$ è negativo (colore rosso) non rappresentano realisticamente un
alveare di campo, poiché l'equilibrio tra i processi interni di ovideposizione ($L$), reclutamento ($\alpha$) ed
allevamento sociale della covata ($w$) è fortemente sbilanciato e determina un eccesso insensato di api di casa.

Inoltre, per le caratteristiche algebriche del sistema (mortalità $m$ soltanto nel comparto $F$),
per $\alpha w -L<0$ si ha quasi sempre la \emph{divergenza} della popolazione nel comparto $H$: questo è un artefatto
dovuto alla semplicità del modello, che tuttavia si verifica \emph{a priori} soltanto in siutazioni inverosimili.

Le simulazioni con $\alpha w -L >0$ rappresentano più realisticamente le colonie di api (colore verde: valori positivi,
colore blu: valori ``molto grandi''), in cui il tasso di deposizione della regina $L$ viene forzatamente
regolato dalle nutrici per rispondere alle variazioni dei ritmi fisiologici della famiglia. %
\footnote{Cfr. la discussione a p.~\pageref{par:interpretationCond6b} per una interpretazione dettagliata
della condizione~\eqref{eq:cond6b} $\alpha w -L>0$.}

Per tutti i grafici in questa sezione escluso l'ultimo (cioè per le figg.~\ref{img:param3D}--\ref{img:kh11expA23}),
l'interpretazione del colore dei
\emph{datapoint} è la stessa della figura~\ref{img:param3D}.

\begin{figure}[!h]
    \centering
    \includegraphics[keepaspectratio,width=\textwidth]{img/k11EA2-parameterSpace3D}

    \caption[Esperimento A2, spazio dei parametri.]{Ogni punto $(m,w,b)$ corrisponde ad una simulazione effettuata
        coi parametri $m,w$ e popolazione iniziale $b$. La dimensione del punto rappresenta la popolazione finale
        $H(T)+F(T)$ raggiunta al termine della simulazione, compresa tra 100 e $10^5$ operaie.

        Il colore rappresenta la quantità $\alpha w - L$, che è una misura di quanto la condizione~\eqref{eq:cond6b}
        sia soddisfatta o meno nella simulazione corrente (cfr.~p.~\pageref{par:interpretationCond6b} per una interpretazione quantitativa).
    }
    \label{img:param3D}
\end{figure}

\paragraph{}
Nella proiezione dello spazio dei parametri sul piano $(m, \, b)$ (figura~\ref{img:param2D}) possiamo confermare
empiricamente
che in questo semplice modello il parametro $m$ influenza definitivamente le capacità di sopravvivenza della colonia;
gli altri parametri determinano il valore del punto di equilibrio nella popolazione.
\begin{figure}[!h]
    \centering
    \includegraphics[keepaspectratio,width=\textwidth]{img/k11EA2-parameterSpace2D}

    \caption[Esperimento A, proiezione dello spazio dei parametri.]{Il diagramma a dispersione della
        figura~\ref{img:param3D} proiettato sul piano $(m, \, b)$.}
    \label{img:param2D}
\end{figure}

Dalla figura~\ref{img:param2D} emerge un altro interessante aspetto di questo semplice modello:
la popolazione iniziale \emph{non influenza} l'esito finale della colonia;
in effetti il punto di equilibrio (cfr. sezione~\ref{sec:kh11}, Proposizione~\ref{teo:exUniqFstarPos}) è determinato
soltanto dai parametri.

\paragraph{}
Nelle sottosezioni che seguono sono esaminati i dati risultanti dalle simulazioni appena descritte.

\subsection{Rapidità di estinzione e mortalità}
\label{sec:fdod}
Per valutare la rapidità e la gravità con cui una colonia si estingue in presenza di un alto tasso di
mortalità nel comparto bottinatrici, definiamo il \emph{FDOD (First Day Of (colony) Death)} come
$$\text{FDOD} \coloneq \min \left\{ t \geq 0 \st H(t) + F(t) < b_{thres} \right\} \; .$$

Il risultato di questa analisi è illustrato in Figura~\ref{img:kh11expA21}: ogni punto blu di coordinate $(x,y)$
rappresenta una simulazione del modello con mortalità $m=x$: la corrispondente popolazione
della colonia è scesa sotto la soglia $b_{thres}=20$ al giorno $y$ per la prima volta.
\begin{figure}[!h]
    \centering
    \includegraphics[keepaspectratio,width=0.98\textwidth]{img/k11EA2-fdodVSm}

    \caption[Esperimento A, \emph{FDOD} vs. mortalità.]{Il \emph{FDOD} (in giorni) degli alveari che
        muoiono entro i due anni,
        contro il tasso di mortalità. Con fit lineare (retta celeste).

        La dimensione del punto rappresenta qui la popolazione iniziale $b$.
    }

    \label{img:kh11expA21}
\end{figure}

\paragraph{}
Si può osservare che se il tasso di mortalità è contenuto ($m< \hat{m}$ con $\hat{m} \approx 0,4$),
la colonia sopravvive con successo. % pché mancano i datapoint a sinistra
Quando invece il tasso di mortalità è eccessivo ($m \gtrapprox 0,4$) le famiglie di \emph{Apis} muoiono
ad un ritmo proporzionalmente accelerato, ed
il tempo di sopravvivenza dell'alveare si riduce in funzione della mortalità (che nel modello innesca un reclutamento
precoce delle operaie di nido verso il compartimento delle bottinatrici), e neanche le famiglie (inizialmente) più numerose
riescono a superare una o due stagioni di raccolta.

\paragraph{}
Se da un lato la popolazione iniziale non determina affatto il destino della colonia, dall'altro un effetto
catturato con successo da questo semplice modello è che la numerosità iniziale delle api permette alle colonie
di resistere più a lungo, anche negli scenari in cui la morte è inevitabile.

\paragraph{}
Si osservi infine che nella figura~\ref{img:kh11expA21} mancano del tutto punti di colore rosso,
per cui la condizione~\eqref{eq:cond6b} è falsa: in tali colonie la popolazione diverge all'infinito (irrealisticamente)
e perciò non scende mai sotto la soglia $b_{thres} =20$.

Questa è una ulteriore conferma delle considerazioni circa la condizione $\alpha w -L >0$ fatte durante l'analisi
della stabilità del modello (p.~\pageref{par:interpretationCond6b} e segg.),
e già in questo esperimento (p.~\pageref{par:colore} e segg.).


\subsection{Popolazione iniziale e finale}
In figura~\ref{img:kh11expA22} è mostrato su scala logaritmica il risultato delle simulazioni, con la popolazione
iniziale $b=H(0)+F(0)$ in ascissa e la popolazione finale $H(T)+F(T)$ in ordinata.

Ricordiamo che il periodo di ogni simulazione è $T = 2~\text{anni}$.
\begin{figure}[!h]
    \centering
    \includegraphics[keepaspectratio,width=0.98\textwidth]{img/k11EA2-fpopVSipop}

    \caption[Esperimento A, popolazione finale vs. Popolazione iniziale.]{Esperimento A, popolazione finale $H(T)+F(T)$
        vs. Popolazione iniziale $H(0)+F(0)$. I triangoli rappresentano le simulazioni per cui la condizione~\eqref{eq:cond6b} è falsa,
        mentre per i cerchi è vera.

        La dimensione di ogni punto è proporzionale alla mortalità $m$, compresa nell'intervallo
        $\left[ 0,016, \, 0,7 \right]$.
    }

    \label{img:kh11expA22}
\end{figure}

Il modello considerato è abbastanza predittivo ed in linea con le osservazioni sugli apiari in campo, purché si
utilizzino misure accurate o quantomeno stime ragionevoli dei parametri da impiegare.%
\footnote{Cfr. p.~\pageref{par:interpretationCond6b} e segg. a chiusura della sezione~\ref{sec:kh11}.}

I \emph{datapoint} col triangolo indicano le simulazioni per cui la condizione~\eqref{eq:cond6b} è falsa: si
osservi che la popolazione corrispondentemente diverge irragionevolmente, anche per condizioni iniziali
relativamente sfavorevoli (\ie basse numerosità iniziali).%
\footnote{Si veda il paragrafo dedicato all'interpretazione della condizione~\eqref{eq:cond6b}
a p.~\pageref{par:interpretationCond6b}.}

I triangoli compaiono soltanto nella parte alta del grafico (\ie popolazioni finali superiori a $10^4$ individui);
popolazioni finali irrealisticamente alte ($>10^5$ api) si trovano sempre più in alto in corrispondenza di
mortalità sempre più basse (punti più piccoli).

\paragraph{}
Come confermano gli apicoltori\footcite{privFDL,privFPan,meccanica},
la ``forza'' di un alveare (genericamente intesa come la numerosità della popolazione)
sostiene meglio e più a lungo la famiglia attraverso gli stress ambientali;
ma alveare -- seppur molto popoloso -- che incontri continuativamente avversità troppo intense (ad es. attacchi da
molteplici patogeni) necessita di un intervento umano per non soccombere.

\subsection[Popolazione finale e w]{Popolazione finale e $w$}
Per studiare la relazione tra la ``efficienza'' di una colonia nella combinazione delle attività foretiche e di
cura della covata (ingresso di nuovi individui nel comparto $H$ delle operaie di nido),
si consideri il grafico in figura~\ref{img:kh11expA23}: ogni \emph{datapoint} alle coordinate $(x,y)$ rappresenta
una simulazione con parametro $w=x$ e popolazione finale $N(T)=y$; ricordiamo che in questo esperimento il
tempo di ogni simulazione è $T=2~\text{anni}$, mentre la popolazione iniziale $b=H(0)+F(0)$ è scelta
casualmente per ogni simulazione tra $b_-=50$ e $b_+ =9000$ individui.
\begin{figure}[!h]
    \centering
    \includegraphics[keepaspectratio,width=0.98\textwidth]{img/k11EA2-finpopVSw}

    \caption[Esperimento A, popolazione finale vs $w$.]{Esperimento A, popolazione finale contro $w$.
        I triangoli rappresentano le simulazioni per cui la condizione~\eqref{eq:cond6b} è falsa,
        mentre per i cerchi è vera.

        La dimensione di ogni punto è proporzionale alla mortalità $m \in \left[0,016, \, 0,7 \right]$.
    }
    \label{img:kh11expA23}
\end{figure}

Il gradiente di colore orizzontale nella figura~\ref{img:kh11expA23} è dovuto al fatto che in $\alpha w - L$ i
parametri $\alpha, L$ vengono mantenuti costanti attraverso le simulazioni.

\paragraph{}
Il parametro $w$ compare nell'equazione~\eqref{eq:eclos} e nella risposta funzionale
(Holling di tipo~II\footnote{Cfr. sezioni~\ref{sec:rispFunz} e~\ref{sec:eclos}.})
modula la schiusa delle api adulte con il tasso di ovideposizione $L$ della regina; come
affermato dagli stessi autori del modello\footcite[2]{khoury2011}, senza informazioni aggiuntive è ragionevole
proporre che il tasso di schiusa delle uova (deposte 21 giorni prima) approcci $L$ dal basso per $N(t) \to \infty$,
in modo liscio.

Come già discusso nelle sezioni~\ref{sec:rispFunz} e~\ref{sec:eclos}, il parametro $w$ è l'inverso del tasso di
attacco nella risposta funzionale della schiusa al numero di operaie (di casa e di campo) disponibili nell'alveare.

Nella figura~\ref{img:kh11expA23} si può osservare che il parametro $w$ impatta in ogni caso il destino di un
alveare simulato, e che per alti valori di $w$ (\ie famiglie ``inefficienti'' nella cura della prole) gli altri
fattori ambientali (\ie la mortalità $m$) influiscono progressivamente sempre di più sulla
numerosità della popolazione finale.


\subsection{Soglie di mortalità}
\label{ssec:simSoglie}
Il risultato fondamentale che accompagna il modello di~\citeauthor{khoury2011}\footcite{khoury2011},
precisato e dimostrato nella
Proposizione~\ref{teo:esistenzPosF} (p.~\pageref{teo:esistenzPosF}), fornisce le condizioni affinché esista un
equilibrio positivo.

In teoria, mantenere i parametri calcolati a partire dalle misure dirette sul campo
entro i limiti ivi stabiliti, dovrebbe consentire di garantire la sopravvivenza di una colonia in apiario.

\paragraph{}
Nella figura~\ref{img:kh11expA24} è mostrato il grafico delle soglie di mortalità $m_1^*$ (in blu) ed
$m_2^*$ (in verde) stabilite nella proposizione~\ref{teo:esistenzPosF}.
Gli altri punti alle coordinate $(x,y)$ sono simulazioni con parametri $w=x$ ed $m=y$, in cui il
colore rappresenta la popolazione finale $N(T)$ raggiunta dalla colonia.

Ricordiamo che in questo esperimento il
tempo di ogni simulazione è $T=2~\text{anni}$, mentre la popolazione iniziale $b=H(0)+F(0)$ è scelta
casualmente per ogni simulazione tra $b_-=50$ e $b_+ =9000$ individui.

\begin{figure}[!h]
    \centering
    \includegraphics[keepaspectratio,width=0.98\textwidth]{img/k11EA2-mistarVSw}

    \caption[Esperimento A, soglie di mortalità]{Previsioni del modello per $m_1^*$ (in blu) ed $m_2^*$ (in verde).
        Gli altri punti sono le simulazioni con parametri $(w,m)$ ed il colore rappresenta la popolazione
        finale raggiunta dalla colonia.
    }

    \label{img:kh11expA24}
\end{figure}

La linea tratteggiata delimita la regione del parametro $w$ per cui la condizione~\eqref{eq:cond6b} è vera (a destra),
in cui la popolazione di api si stabilizza ad un equilibrio positivo per mortalità $m$ superiori ad $m_1^*$
(altrimenti la popolazione diverge) ed inferiori ad $m_2^*$ (altrimenti la colonia si estingue).
Si noti che quando la~\eqref{eq:cond6b} diventa falsa, la soglia $m_2^*$ funge
da limite \emph{inferiore} alla mortalità, come affermato nella proposizione~\ref{teo:esistenzPosF}.

\paragraph{}
Questa conferma sperimentale ci consente di assimilare definitivamente i casi 2 e 3 della
proposizione~\ref{teo:esistenzPosF}, e di considerare solamente il caso 1 come regione di parametri ammissibile
per questo modello -- e successivi, più raffinati -- per descrivere efficacemente le reali colonie di \emph{A.~mellifera},
sia allo stato selvatico che in allevamento.


\section{Experiment C, iteration 2}
Two iterations were made of this simulation with constant parameters $\alpha = 0.25$, $\sigma=0.75$, $L=2500$:
data \texttt{A} was obtained with $w=27000$, while the value $w=9800$ was used for data \texttt{B}.

In both experiments a random value for the parameter $m$ was obtained from a uniform distribution on $[0.006, \, 0.6]$
with a resolution of 30. For each $m$ we selected at random initial values $1 \leq H_0 \leq 15000$ and $1 \leq F_0 \leq 6000$ and ran 30 such simulations for a total of 900 days each.

\subsection{Data A}
\label{sec:kh11expC2A}
Note that with $w=27000$ the condition~\eqref{eq:cond6b} from Theorem~\ref{teo:esistenzPosF} is true: $\alpha - \frac{L}{w} > 0$, which is case 1 of the Theorem,
when a positive equilibrium exists for mortalities $m$ bound from above and from below:
$$ 0.0804 \simeq m_1^* < m < m_2^* \simeq 0.5078 \; .$$

In the remainder of this section, we will call ``weak'' those colonies which have final total
population $N(900) < N_- = 100$ bees. We will say that a colony is ``strong'' if
its final population is more than $N_+ = 40000$ workers.

Results are plotted in Figure~\ref{img:kh11expC2A}.

All colonies which eventually die off have a high rate of forager mortality: $m > m_2^*$.
Some of them don't collapse in 2.5 years and we don't call them ``weak'' in the sense above.

\paragraph{}
For mortalities above a reasonable value and below a fatal threshold, \ie for $m_1^* < m < m_2^*$,
we get steadily close to a positive equilibrium $N^*$: final workers number can get even into
the ``strong'' classification for hives with lower mortality rates, and become lower
(in the hundreds) near the $m_2^*$ end of the spectrum.

\paragraph{}
Outside the shown plot range were final populations in the unreasonable order of
magnitude of $10^6$ and beyond, clearly indicating a diverging behaviour, even with low values of $N(0)$.
All these points lie on the left of $m_1^* \simeq 0.0804$.\footnote{Cfr.~Interpretation at page~\pageref{par:interpretationCond6b}.}

\begin{figure}[pbh]
    \centering
    \includegraphics[keepaspectratio,width=0.98\textwidth]{img/kh11expC2A}

    \caption[Experiment C2A]{Experiment C2A: total number of worker bees $N(t)= H(t)+F(t)$ is shown at $t=0$ and $t=900$ for different levels of forager mortality $m$.

    With the terminology estabilished in page \pageref{sec:kh11expC2A}, we see that
    initial conditions of colonies that become ``weak'' (blue squares) lie all in
    the half--plane $m>0.5078$ (dotted vertical line).
    Similarly, initial conditions of colonies that eventually become ``strong''
    (red x's) all lie around the line $m=0.0804$ (dashed).

    In black we see the initial conditions (+) and final populations (triangles) for colonies that
    don't get to be ``strong'' nor ``weak'' at day $t=900$.

    Final population numbers for ``strong'' hives (purple circles) all happen in the left--hand side
    of the spectrum $0.0804 < m < 0.5078$.
    }

    \label{img:kh11expC2A}
\end{figure}

\subsection{Data B}
In this experiment we ran 900 simulations for 900 days each, within the same setting as above,
where we changed the value of the parameter $w$ only.

Note that in this case inequality~\eqref{eq:cond6b} is reversed, since
$$ \alpha - \frac{L}{w} = \frac{1}{4} - \frac{2500}{9800} < 0 \; .$$

Results are plotted in Figure~\ref{img:kh11expC2B}: for each simulation the fraction $\frac{N(900)}{N(0)}$ is plotted against its level of forager mortality.
This is a rough measure of performance of the hive, for it depends on the colony ability to secure
trophic and reproductive capabilities, within the limits of its given initial conditions.

\paragraph{}
We can see that initial population levels definitely increased in a span of 2.5 years for
almost all initial conditions. A slight decreasing trend in final population levels is
noted for higher mortality rates; still, these higher values for $m$ cannot keep the colony
population at reasonably stable levels and only lower the speed at which population is diverging.

\begin{figure}[pbh]
    \centering
    \includegraphics[keepaspectratio,width=0.98\textwidth]{img/kh11expC2B}

    \caption[Experiment C2B]{Experiment C2B: the ratio $\frac{N(900)}{N(0)}$ of final and initial
    total populations is plotted against $m$ in abscissa.}

    \label{img:kh11expC2B}
\end{figure}






\appendix
% \part{\appendixname}
\chapter{Richiami di teoria dei sistemi dinamici}
\label{chap:teoria}
In questo capitolo si richiamano alcuni risultati nell'ambito dei sistemi dinamici, utili strumenti ai fini
di questa tesi.
Un'ampia letteratura è disponibile in materia, anche in lingua italiana.
\footnote{Ad es. \cite{ricciSistDin} Cap.~2, \cite{introSD} Capp.~1--4.}

Presentiamo qui soltanto i teoremi e le definizioni che si applicano all'ambito generale; alcuni oggetti
e risultati riguardanti più nello specifico i sistemi biomatematici, di modellazione ecologica delle popolazioni
e delle patologie sono esaminati nel sezione~\ref{sec:ingredientiBioMat}.

\paragraph{}
È utile definire un po' di notazione preliminare, prima di procedere oltre.
Sia
\begin{equation}
\begin{sistema}
\dot{\mathbf{x}} = F ( \mathbf{x}, t ) \\
\mathbf{x} (t_0) = \mathbf{x}_0
\end{sistema}
\label{eq:sdcGenerale}
\end{equation}
un sistema dinamico con condizione iniziale $\mathbf{x}_0$. Supporremo nel seguito del capitolo che $F$ sia
definita su qualche aperto $W \subseteq \R^{n+1}$ e che sia di classe $C^1$ almeno.

Per una classe abbastanza ampia\footnote{Tra le molte varianti del Teorema, scegliamo come ipotesi $F$
localmente lipschitziana rispetto a $\mathbf{x}$, uniformemente rispetto a $t$.}
di sistemi nella forma~\eqref{eq:sdcGenerale} vale il Teorema di Esistenza e Unicità locale delle soluzioni;
le soluzioni dipendono con continuità dal dato iniziale $\mathbf{x}_0$.

\paragraph{}
Se
$$\diffp{F}{t} = 0 \; ,$$
\ie il sistema non dipende esplicitamente dal tempo, esso si dice \emph{autonomo}.
Per tali sistemi non è restrittivo supporre la condizione iniziale data all'istante zero: $t_0=0$.

Lo spazio delle soluzioni è sempre uno spazio vettoriale, ma nel caso dei sistemi autonomi abbiamo
la possibilità di esprimere -- almeno localmente -- in forma canonica la soluzione generale come
combinazione lineare delle colonne di una matrice fondamentale che si ottiene direttamente
dalla $F$ del sistema.

Vedremo nella prossima Appendice~\ref{chap:floquetTheory} una forma canonica che è possibile
ottenere nei sistemi non autonomi a coefficienti periodici.

\paragraph{}
Possiamo descrivere sinteticamente le soluzioni di un sistema dinamico tramite la
\emph{soluzione generale}\footnote{Talvolta anche \emph{``flusso integrale''}.}
$$\Phi \, : \, D \to W, \quad (t, \mathbf{x}_0 ) \mapsto \Phi_t ( \mathbf{x}_0 ) \; ,$$
che mappa $(t, \mathbf{x}_0 )$ nella soluzione $\mathbf{x} (t)$ con condizioni iniziali $\mathbf{x}_0$,
valutata al tempo $t$. Questa è una applicazione ben definita grazie al Teorema di esistenza e unicità delle
soluzioni; purtroppo non abbiamo garanzie a priori sulla forma del dominio di ogni soluzione, che dipende
comunque dalle condizioni iniziali.
Dunque la forma di $D \subseteq \R \times W$ non ci è nota a priori.

Se indichiamo con $p$ la proiezione $D \to W$, sappiamo che ogni fibra
$p^{-1} ( \mathbf{x}_0 )$ è un intervallo aperto contenente $t_0$.

Per definizione, la mappa $t \mapsto \Phi_t ( \mathbf{x}_0 )$ è la soluzione del problema di Cauchy
$$\begin{sistema}
\diff{ }{t} \Phi_t ( \mathbf{x}_0 ) = F \left( \Phi_t ( \mathbf{x}_0 ) \right) \\
\Phi_{t_0} ( \mathbf{x}_0 ) = \mathbf{x}_0
\end{sistema}$$

Richiamiamo la \emph{proprietà di semigruppo} del flusso integrale:
$$\Phi_h \circ \Phi_t = \Phi_{h+t}$$
per ogni $h, t \geq t_0$.

Inoltre, la dipendenza continua delle soluzioni di~\eqref{eq:sdcGenerale} dal dato iniziale si riassume nel
Teorema di Continuità del Flusso (\ie $\Phi_t$ è continua in $\mathbf{x}_0$).

\begin{definizione}[Punto di Equilibrio]
    Diciamo che $\mathbf{x}_S \in W$ è un \emph{punto di equilibrio} per il sistema dinamico,
    se $F( \mathbf{x}_s ) = \mathbf{0}$.
\end{definizione}

Sopra ogni punto di equilibrio abbiamo la soluzione costante $\mathbf{x} (t) \equiv \mathbf{x}_S$, che è un'orbita,
ma ogni altra soluzione -- comunque vicina a $\mathbf{x}_S$ -- non può raggiungere $\mathbf{x}_S$
per nessun $t$ finito.

\begin{definizione}[Stabilità]
    Un punto $\mathbf{x}_S \in W$ si dice \emph{stabile} se per ogni intorno $U$ di $\mathbf{x}_S$
    esiste un intorno $V$ di $\mathbf{x}_S$
    tale che
    \footnote{Senza perdita di generalità $U, V, \subseteq W$.}
    $$(\forall \mathbf{x}_0 \in V) \; \Phi_t ( \mathbf{x}_0 ) \; ,$$
    per ogni $t \geq 0$.
\end{definizione}

Per definizione, le soluzioni possono essere ``costrette'' a restare arbitrariamente vicine ad un punto
stabile, purché si scelgano condizioni iniziali sufficiente vicine ad esso.

Si osservi che ogni punto stabile è anche di equilibrio.

\begin{definizione}[Attrazione e repulsione]
    Diciamo che un punto $\mathbf{x}_S \in W$ è \emph{attrattivo} se esiste un intorno
    $U$ di $\mathbf{x}_S$ tale che
    $$(\forall \mathbf{x}_0 \in U) \; \exists \lim_{t \to +\infty} \Phi_t (\mathbf{x}_0) = \mathbf{x}_S \; .$$

    Quando otteniamo la stessa cosa con $t \to -\infty$, diciamo che $\mathbf{x}_S$ è \emph{repulsivo}.
\end{definizione}

Si noti che essendo un punto limite, ogni punto attrattivo o repulsivo è anche un punto di equilibrio.

\begin{definizione}[Stabilità asintotica]
    Un punto che è sia attrattivo che stabile in avanti si dice \emph{asintoticamente stabile}.
\end{definizione}

\begin{definizione}[Esponenti di Lyapounov]
    Sia $J = DF (\mathbf{x}_S)$ la matrice jacobiana di $F$ valutata sul punto di equilibrio $\mathbf{x}_S$.

    Gli \emph{esponenti di Lyapounov} di questo equilibrio sono le parti reali degli autovalori di $J$.

    Diciamo che $\mathbf{x}_S$ è un \emph{pozzo} se tutti gli esponenti di Lyapounov sono strettamente negativi.
\end{definizione}

\begin{teorema}[del pozzo non lineare]
    Sia $\mathbf{x}_S \in W$ un pozzo per il sistema dinamico continuo $\dot{ \mathbf{x} } = F(\mathbf{x})$,
    sia $c>0$ tale che $\Re ( \lambda ) < -c$ per ogni autovalore $\lambda$ di $DF (\mathbf{x}_S)$.

    Allora esiste un intorno $U$ di $\mathbf{x}_S$ tale che:
    \begin{enumerate}
        \item la mappa $\Phi_t (\mathbf{x}_0)$ è definita per ogni $t \geq 0$, per ogni $\mathbf{x}_0 \in U$;
        \item $(\exists B>0) \, (\forall \mathbf{x}_0 \in U) \;
        \norm{ \Phi_t (\mathbf{x}_0) - \mathbf{x}_S } \leq
        B e^{-ct} \norm{ \mathbf{x}_0 - \mathbf{x}_S }.$
    \end{enumerate}
    \label{teo:pozzoNonLineare}
\end{teorema}


\begin{definizione}[Funzione di Lyapounov]
    Sia $\dot{\mathbf{x}} = F(\mathbf{x})$ un sistema dinamico continuo con $F \in C^1(W,\R^n)$, dove
    $W \subseteq \R^n$ è aperto.

    Diciamo che una funzione $V \in C^1(B, \R)$, per qualche aperto $B \subseteq W$, è una
    \emph{funzione di Lyapounov} per l'equilibrio $\mathbf{x}_S \in B$ se:
    \begin{itemize}
        \item $\dot{V} ( \mathbf{x} ) \leq 0 \; (\forall \mathbf{x} \in B),$
        \item $V( \mathbf{x}_S ) \lneq V( \mathbf{x} ) \; (\forall \mathbf{x} \in B, \mathbf{x} \neq \mathbf{x}_S).$
    \end{itemize}

    Diciamo inoltre che una funzione di Lyapounov $V$ è \emph{stretta} se
    $$\dot{V}(\mathbf{x}) \lneq 0 \; (\forall \mathbf{x} \in B, \mathbf{x} \neq \mathbf{x}_S) \; .$$
\end{definizione}

In pratica, $V$ ha un minimo locale forte sul punto di equilibrio, ed è non crescente
lungo le soluzioni vicine.

Possiamo vedere come $\dot{V} ( \mathbf{x} )$ sia una misura della crescita
di $V$ \emph{lungo le soluzioni} del sistema
dinamico definito dal campo vettoriale $F$, dal momento che
$$\dot{V} ( \mathbf{x} ) = \diff{}{t} \left( \dot{V} \left( \mathbf{x} (t) \right) \right) =
\sum_{i=1}^n \diffp{V}{{\mathbf{x}_i}} \left( \mathbf{x} (t) \right) \cdot \diff{ \mathbf{x}_i }{t} (t) =
\langle \nabla V ( \mathbf{x} ) , \dot{ \mathbf{x} } \rangle \; .$$

\begin{teorema}[della funzione di Lyapounov]
    Se un punto di equilibrio ammette una funzione di Lyapounov, allora esso è stabile.
    \label{teo:lyapounovFunc}
\end{teorema}

\begin{corollario}
    Se un punto di equilibrio ammette una funzione di Lyapounov stretta, allora esso è asintoticamente stabile.
\end{corollario}

\chapter{Teoria di Floquet}
\label{chap:floquetTheory}
Gaston Floquet (1847, Épinal -- 1920, Nancy)
appartiene al novero ``glorioso'' dei matematici francesi che a cavallo del XIX~secolo rinnovarono
le fondamenta e le prospettive dell'Analisi e di altri campi scientifici.
\footnote{Senza pretesa di esaustività citiamo tra i connazionali di Floquet, all'incirca suoi coevi: Sophie Germain,
Jacques Charles François Sturm, Henri Poincaré, Henri Padé, Gaspard Monge, Joseph Liouville,
Jules Antoine Lissajous, Adrien-Marie Legendre, Joseph-Louis Lagrange (n.~Giuseppe Luigi Lagrangia),
Camille Jordan, Charles Hermite, Henri Lebesgue, Pierre-Simon Laplace,
Évariste Galois, Jean Frédéric Frenet, Joseph Fourier, Jean Gaston Darboux, Gaspard-Gustave de Coriolis,
Auguste Comte, André-Louis Cholesky, Augustin-Louis Cauchy, Lazare Carnot, Émile Borel.}

La Teoria di Floquet è la branca dello studio delle equazioni differenziali a coefficienti periodici,
fondata all'epoca dello studio seminale\footcite{gFloquet} di Floquet, in cui è enunciato il
risultato cruciale che ci permette di scrivere la matrice fondamentale di sistemi del tipo~\eqref{eq:sdcGenerale}
in cui la $F$ è periodica in $t$.

\paragraph{}
Consideriamo un sistema omogeneo, lineare e periodico della forma
\begin{equation}
    \diff{}{t} \mathbf{x} = A(t) \cdot \mathbf{x} \; ,
    \label{eq:sdcPeriodico}
\end{equation}
in cui la mappa $t \mapsto A(t) \in \R^{n \times n}$ è periodica: esiste ben definito $T >0$ tale che
$$A(t+T) =A(t) \quad \forall t \in \R \; .$$

\begin{definizione}[Matrice fondamentale]
    Una mappa
    $$\Psi (t) : \; \R \rightarrow \C^{n \times n}$$
    è una \emph{matrice fondamentale} per il sistema~\eqref{eq:sdcPeriodico} se le sue colonne sono soluzioni
    linearmente indipendenti per ogni $t \in \R$.
\end{definizione}
Equivalentemente, $\Psi$ è una matrice fondamentale se $\Imm \Psi \subseteq GL(n)$ e
$$ \dot{\Psi} (t) = A(t) \Psi (t) \quad (\forall t \in \R) \; .$$

Osserviamo che la matrice fondamentale non è unica, ma lo diventa se imponiamo $\Psi(0)=I_n$, ciò
che è sempre possibile.
\footnote{A meno di premoltiplicare per $\Psi^{-1} (0)$.}

\begin{definizione}[Moltiplicatori di Floquet]
    Una matrice
    $$C= \Psi^{-1} (0) \Psi (T) $$
    si chiama \emph{matrice di monodromia} del sistema.

    Gli autovalori $\mu_j$ della matrice di monodromia $C$ si chiamano \emph{moltiplicatori di Floquet}.
\end{definizione}

Se assumiamo senza perdita di generalità che $\Psi (0) = I_n$, la matrice di monodromia si riduce a $C= \Psi (T)$.

La non unicità della matrice di monodromia è conseguenza delle ramificazioni del logaritmo complesso
di matrici, la cui esistenza è necessaria per la dimostrazione del Teorema di Floquet, assieme ai
seguenti Lemmi:
\begin{itemize}
    \item Se $\det A \neq 0$ allora $A \cdot \Psi (t)$ è fondamentale $(\forall t)$.
    \item $\Psi (t)$ fondamentale $\implies \Psi (t+T)$ fondamentale $(\forall t)$, ove $T$ è il periodo.
    \item $\exists C$ tale che $\Psi (t+T) = C \Psi (t) \; (\forall t)$.\footnote{Una $C$ siffatta è la matrice di monodromia.}
\end{itemize}

\paragraph{}
Il seguente risultato è noto come Teorema di Bloch nelle applicazioni di Fisica della materia condensata.

\begin{teorema}[di Floquet]
    Consideriamo un sistema omogeneo a coefficienti periodici della forma~\eqref{eq:sdcPeriodico}, e
    sia $\Psi (t)$ la sua matrice fondamentale con $\Psi (0) = I_n$.

    Esiste una decomposizione in \emph{Forma normale di Floquet} della matrice fondamentale data da
    $$\Psi (t) = Q(t) e^{tB} \; ,$$
    dove la mappa $Q : \R \rightarrow GL(n)$ è $T$--periodica, di classe $C^1$ e verifica $Q(0) = I_n$.
    \label{teo:floquet}
\end{teorema}
Si osservi che la matrice di monodromia si ottiene quindi dalla seguente:
$$C= \Psi (T) = e^{TB} \; .$$

\begin{definizione}[Esponenti di Floquet]
    Gli autovalori $\phi_j$ di $B$ si chiamano \emph{esponenti di Floquet} del sistema.
\end{definizione}

Anche la $B$ non è univocamente determinata,\footnote{Per lo stesso motivo per cui non lo è la matrice di monodromia.}
dunque non lo sono neanche gli esponenti di Floquet. Si osservi però che per via della periodicità vale
la seguente relazione tra gli
esponenti ed i moltiplicatori di Floquet:
$$\mu_j = e^{\phi_j T} \; .$$

\paragraph{}
Grazie alla forma normale di Floquet siamo in grado di formulare il seguente risultato,
analogo al Teorema~\ref{teo:pozzoNonLineare} visto nel caso dei sistemi autonomi.

\begin{teorema}
    Siano $\mu_j$ i moltiplicatori di Floquet di un sistema omogeneo a coefficienti periodici.
    Sia $D \subset \C$ il disco unitario $D= { \abs{z} \leq 1 }$.

    L'equilibrio $\mathbf{0}$ del sistema è
    \begin{itemize}
        \item asintoticamente stabile se e soltanto se tutti i moltiplicatori di Floquet stanno nella
        parte interna di $D$;
        \item stabile se tutti i $\mu_j \in D$, e quelli sul bordo sono semisemplici;
        \item instabile altrimenti.
    \end{itemize}
    \label{teo:stabPeriod}
\end{teorema}



% </MAIN>
\backmatter

\cleardoublepage
\addcontentsline{toc}{chapter}{\bibname}
\nocite{*} % per fargli stampare anche i riferimenti non citati nella tesi
\printbibliography
\end{document}
